\section{Descrizione generale}
\subsection{Obbiettivi del prodotto}
L'obbiettivo del progetto è quello di realizzare un prototipo di un servizio di e-commerce con architettura server-less. Basato principalmente su tecnologie \textit{AWS Cloud}\textsubscript{\textbf{G}}. Il sito dovrà permettere a un ipotetico venditore di vendere e gestire i proprio prodotti all'interno della piattaforma. Deve permettere ai clienti dell'e-commerce di registrarsi, avendo così un account personale, di ricercare i prodotti e, una volta in possesso di un'account, di acquistare i prodotti in vendita all'interno della piattaforma. I clienti possono inoltre visualizzare lo storico dei propri ordini ed eventualmente restituire uno di questi attraverso l'apposita funzionalità.
\subsection{Funzioni del prodotto}
Il prodotto propone diverse funzionalità a seconda del utente in utilizzo della piattaforma.
\begin{itemize}
    \item \textit{Cliente}: un cliente può ricercare e filtrare prodotti di suo interesse, aggiungerli al carrello e, se autenticato attraverso un servizio di autenticazione esterno, può procedere al checkout e quindi all'acquisto, può inoltre visualizzare lo storico degli ordini;
    \item \textit{Venditore}: può gestire tutti i prodotti del catalogo, aggiungendone di nuovi, modificando o rimuovendo quelli già presenti. Per ogni articolo fornisce inoltre delle  informazioni:
    \begin{itemize}
        \item nome;
        \item prezzo;
        \item descrizione e caratteristiche;
        \item categoria;
        \item altri attributi in relazione al tipo di prodotto;
        \item tag di ricerca, per aiutare i clienti a trovarlo.
    \end{itemize}
    \item \textit{Admin}: L'admin è colui che va a rilasciare e gestire la piattaforma software. Ha inoltre il compito di gestire le integrazioni con sistemi di terze parti. Questa gestione sarà permessa dall'utilizzo della piattaforma \textit{Amazon CloudWatch}\textsubscript{\textbf{G}}.
\end{itemize}
\subsection{Caratteristiche degli utenti}
I tipi di utenti sono divisi in tre categorie:
\begin{itemize}
    \item clienti;
    \item venditore;
    \item admin.
\end{itemize}
I clienti e il venditore devono avere conoscenze minime, ossia l'utilizzo di un browser web. Il venditore inoltre deve avere una competenza minima sulla gestione della vendita sulla piattaforma e una buona conoscenza del funzionamento dell'IVA. L'admin deve avere delle sviluppate competenze per gestire al meglio l'intera piattaforma, come la relazione con i servizi di terze parti.
\subsection{Macro architetture del progetto}
Il progetto è caratterizzato da alcune macro architetture:
\subsubsection{EmporioLambda - Backend}
Il backend si occupa d'implementare la business-logic della piattaforma, andando a gestire i dati del sistema, come i dati relativi agli utenti, i prodotti gli ordini e lo stato del carrello. 
\subsubsection{EmporioLambda - Frontend}
Il frontend si occupa di gestire la UI\textsubscript{\textbf{G}} che consiste in una serie di pagine web visualizzabili attraverso i vari browser. Le pagine devono essere precaricate lato server per permettere una renderizzazione più veloce.
\subsubsection{EmporioLambda - Integration}
È la parte di sistema dedicata alla gestione di componenti di terze parti, come i servizi utilizzati per il pagamento e per l'accesso al proprio account.
\subsubsection{EmporioLambda - Monitoring}
Sono gli strumenti impiegati degli admin per gestire la piattaforma e la correlazione con gli strumenti di terze parti.






