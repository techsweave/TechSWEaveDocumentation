\section{Casi d'uso}
\subsection{Attori dei casi d'uso}
\subsubsection{Attori primari}
Gli attori che il gruppo ha ritenuto essere i più adeguati sono:
\begin{itemize}
    \item \textbf{Utente generico:} utente che può navigare nell'e-commerce e può usufruire di alcune funzionalità, come la visualizzazione e la ricerca dei prodotti, che può aggiungere al proprio carrello, la applicazione di filtri e categorie per la ricerca, e infine di potersi autenticare.\\ Inoltre si divide in:
          \begin{itemize}
              \item \textbf{Utente autenticato:} un utente autenticato può a sua volta essere:
                    \begin{itemize}
                        \item \textbf{Cliente autenticato:} cliente che ha effettuato il login, può accedere a tutte le funzionalità riservate ai clienti, come l'aggiunta di prodotti al carrello, l'acquisto, la visualizzazione della lista degli ordini, la possibilità di contattare il venditore;
                        \item \textbf{Venditore autenticato:} venditore che ha effettuato il login, può accedere alle funzionalità relative all'amministrazione dei prodotti e delle categorie.
                    \end{itemize}
                    \begin{figure}[!ht]
                        \caption{Attori primari}
                        \vspace{5px}
                        \includegraphics[scale=0.6]{../../../Images/AnalisiRequisiti/attori}
                        \centering
                    \end{figure}
          \end{itemize}
\end{itemize}
\subsubsection{Attori secondari}
\begin{itemize}
    \item \textbf{Servizio di autenticazione esterno:} servizio esterno per il login e per la gestione delle credenziali;
    \item \textbf{Stripe:} servizio esterno per la gestione dei pagamenti.
\end{itemize}
\newpage
\subsection{Elenco dei casi d'uso}
\textbf{Utilizzo della piattaforma:}
\vspace{-10px}
\begin{figure}[!ht]
    \caption{Casi d'uso che interessano gli utenti}
    \vspace{10px}
    \includegraphics[scale=0.42]{../../../Images/AnalisiRequisiti/casiUso}
    \centering
\end{figure}
\newpage
\subsection{UC1: Autenticazione}
\label{sec:UC1}
\begin{itemize}
    \item \textbf{Descrizione:} permette l'autenticazione di un utente;
    \item \textbf{Attore Primario:} utente non autenticato;
    \item \textbf{Attore Secondario:} servizio di autenticazione esterno;
    \item \textbf{Precondizione:} l'utente non si è ancora autenticato;
    \item \textbf{Input:} pressione bottone \textit{login};
    \item \textbf{Postcondizione:} l'utente è autenticato;
    \item \textbf{Scenario Principale:}
          \begin{itemize}
              \item l'utente entra nella pagina;
              \item preme il bottone per il login;
              \item viene indirizzato alla pagina di login fornita dal servizio di autenticazione.
          \end{itemize}
\end{itemize}
\newpage
\subsection{UC2: Autenticazione}
    \label{sec:UC2}
    \begin{itemize}
        \item \textbf{Descrizione:} permette l'autenticazione di un utente;
        \item \textbf{Attore Primario:} utente generico;
        \item \textbf{Attore Secondario:} servizio di autenticazione esterno;
        \item \textbf{Precondizione:} l'utente generico non si è ancora autenticato
        \item \textbf{Input:} pressione bottone \textit{login}\textsubscript{\textbf{G}};
        \item \textbf{Postcondizione:} l'utente è autenticato;
        \item \textbf{Scenario Principale:} 
        \begin{itemize}
            \item l'utente entra nella pagina
            \item preme il bottone per il \textit{login}\textsubscript{\textbf{G}}
            \item viene inderizzato alla pagina di \textit{login}\textsubscript{\textbf{G}} fornita dal servizio di autenticazione.
        \end{itemize}
    \end{itemize}
\subsection{UC3: Autenticazione standard}
\begin{figure}[!ht]
    \caption{Diagramma di UC3: Autenticazione standard}
    \vspace{10px}
    \includegraphics[scale=0.5]{../../../Images/AnalisiRequisiti/UC03}
    \centering
\end{figure}
\label{sec:UC3}
\begin{itemize}
    \item \textbf{Descrizione:} permette l'autenticazione di un utente;
    \item \textbf{Attore Primario:} utente generico;
    \item \textbf{Attore Secondario:} servizio di autenticazione esterno;
    \item \textbf{Precondizione:} l'utente è all'interno della pagina di autenticazione
    \item \textbf{Input:} inserimento dati;
    \item \textbf{Postcondizione:} l'utente è autenticato;
    \item \textbf{Scenario Principale:} 
    \begin{itemize}
        \item l'utente è nella pagina dedicata all'autenticazione;
        \item inserisce i propri dati;
        \item preme l'apposito bottone;
        \item l'utente è autenticato.
    \end{itemize}
    \textbf{Estensione:}
        Se la l'username o la password non sono corretti:
        \begin{itemize}
            \item viene visualizzato un messaggio di errore (\hyperref[sec:UC28]{\underline{UC28}});
            \item si può ritentare l'inserimento.
        \end{itemize}
\end{itemize}
\subsubsection{UC3.1: Inseriemtno username}
\begin{itemize}
    \item \textbf{Descrizione:} permette l'inserimento dell'username;
    \item \textbf{Attore Primario:} utente generico;
    \item \textbf{Attore Secondario:} servizio di autenticazione esterno;
    \item \textbf{Precondizione:} l'utente è all'interno della pagina di autenticazione
    \item \textbf{Input:} inserimento username;
    \item \textbf{Postcondizione:} l'username è stata inserita;
\end{itemize}
\subsubsection{UC3.2: Inseriemtno password}
\begin{itemize}
    \item \textbf{Descrizione:} permette l'inserimento della password;
    \item \textbf{Attore Primario:} utente generico;
    \item \textbf{Attore Secondario:} servizio di autenticazione esterno;
    \item \textbf{Precondizione:} l'utente è all'interno della pagina di autenticazione
    \item \textbf{Input:} inserimento password;
    \item \textbf{Postcondizione:} la password è stata inserita;
\end{itemize}

\subsection{UC4: Autenticazione esterna}
\label{sec:UC4}
\begin{itemize}
    \item \textbf{Descrizione:} permette l'autenticazione di un utente attraverso un servizio esterno (Google, Amazon, Facebook, ecc.);
    \item \textbf{Attore Primario:} utente generico;
    \item \textbf{Attore Secondario:} servizio di autenticazione esterno;
    \item \textbf{Precondizione:} l'utente generico è nella pagina di autenticazione;
    \item \textbf{Input:} sceglie il servizio con cui autenticarsi;
    \item \textbf{Postcondizione:} l'utente è autenticato;
    \item \textbf{Scenario Principale:}
          \begin{itemize}
              \item l'utente entra nella pagina;
              \item preme il bottone del servizio scelto;
              \item viene indirizzato alla pagina di login del servizio;
              \item l'utente è autenticato.
          \end{itemize}
\end{itemize}
\subsection{UC5: Aggiunta di un prodotto al carrello}
\label{sec:UC5}
\begin{itemize}
    \item \textbf{Descrizione:} l'utente aggiunge al carrello un prodotto che intende acquistare;
    \item \textbf{Attore Primario:} utente generico;
    \item \textbf{Precondizione:} l'utente generico di trova nella pagina dettagliata del prodotto (\hyperref[sec:UC4]{\underline{UC4}});
    \item \textbf{Input:} click sul bottone di aggiunta al carrello;
    \item \textbf{Postcondizione:} il prodotto è stato aggiunto al carrello;
    \item \textbf{Scenario Principale:}
          \begin{itemize}
              \item l'utente si trova nella pagina dei dettagli del prodotto;
              \item l'utente clicca il bottone per aggiungere il prodotto al carrello;
              \item il prodotto viene aggiunto al carrello.
          \end{itemize}
\end{itemize}
\newpage
\subsection{UC6: Ricerca dei prodotti}
\label{sec:UC6}
\begin{figure}[!ht]
    \caption{Diagramma di UC6: Ricerca dei prodotti}
    \vspace{10px}
    \includegraphics[scale=0.5]{../../../Images/AnalisiRequisiti/UC06}
    \centering
\end{figure}
\begin{itemize}
    \item \textbf{Descrizione:} l'utente vuole ricercare un prodotto in base ad una parola chiave;
    \item \textbf{Attore Primario:} utente generico;
    \item \textbf{Precondizione:} l'utente si trova in una pagina dedicata alla ricerca;
    \item \textbf{Input:} stringa relativa a ciò che si vuole cercare;
    \item \textbf{Postcondizione:} l'utente visualizza i prodotti corrispondenti alla parola chiave (\hyperref[sec:UC8]{\underline{UC8}});
    \item \textbf{Scenario Principale:}
          \begin{itemize}
              \item l'utente vuole ricercare un prodotto secondo una determinata parola chiave;
              \item gli vengono mostrati tutti i risultati della ricerca nella lista dei prodotti.
          \end{itemize}
    \item \textbf{Inclusioni:}
          \begin{itemize}
              \item viene visualizzata una pagina con tutti i risultati della ricerca.
          \end{itemize}
\end{itemize}

\subsection{UC7: Ricerca dei prodotti}
\label{sec:UC7}
\begin{itemize}
    \item \textbf{Descrizione:} l'utente vuole ricercare un prodotto in base ad una parola chiave;
    \item \textbf{Attore Primario:} utente generico;
    \item \textbf{Precondizione:} l'utente si trova in una pagina dedicata alla ricerca;
    \item \textbf{Input:} stringa relativa a ciò che si vuole cercare;
    \item \textbf{Postcondizione:} l'utente visualizza i prodotti corrispondenti alla parola chiave (\hyperref[sec:UC8]{\underline{UC8}});
    \item \textbf{Scenario Principale:}
          \begin{itemize}
              \item l'utente vuole ricercare un prodotto secondo una determinata parola chiave;
              \item gli vengono mostrati tutti i risultati della ricerca nella lista dei prodotti.
          \end{itemize}
    \item \textbf{Inclusioni:}
          \begin{itemize}
              \item viene visualizzata una pagina con tutti i risultati della ricerca.
          \end{itemize}
\end{itemize}
\subsection{UC6: Visualizzazione della lista dei prodotti}
        \label{sec:UC8}
        \begin{figure}[!ht]
            \caption{Diagramma di UC6: Visualizzazione della lista dei prodotti}
            \vspace{10px}
            \includegraphics[scale=0.5]{../../../Images/AnalisiRequisiti/UC6.png}
            \centering
        \end{figure}
        \begin{itemize}
            \item \textbf{Descrizione:} l'utente visualizza la pagina della lista dei prodotti;
            \item \textbf{Attore Primario:} utente generico;
            \item \textbf{Precondizione:} l'utente si trova in una qualsiasi pagina della piattaforma;
            \item \textbf{Postcondizione:} l'utente visualizza la pagina con la lista dei prodotti;
            \item \textbf{Scenario Principale:}
            \begin{itemize}
                \item l'utente è in una pagina della piattaforma;
                \item cerca un prodotto o sceglie una categoria;
                \item l'utente visualizza la lista dei prodott.
            \end{itemize}
        \end{itemize}
        \subsubsection{UC8.1: Ordinamento prodotti}
        \label{sec:UC8.1}
        \begin{itemize}
            \item \textbf{Descrizione:} l'utente seleziona la modalità in cui visualizzare i file;
            \item \textbf{Attore Primario:} utente generico;
            \item \textbf{Precondizione:} l'utente si trova all'interno della pagina con la lista dei prodotti(\hyperref[sec:UC6]{\underline{UC6}});
            \item \textbf{Input:} selezione del parametro scelto;
            \item \textbf{Postcondizione:} la lista dei prodotti si aggiorna secondo l'ordinamento scelto;
            \item \textbf{Scenario Principale:}
            \begin{itemize}
                \item l'utente si trova all'interno della pagina con la lista dei prodotti;
                \item inserisce i parametri per ordinare i prodotti;
                \item la pagina con la lista dei prodotti si aggiorna con i prodotti ordinati.
            \end{itemize}
        \end{itemize}
        \subsubsection{UC8.2: Applicazione filtri}
        \label{sec:UC8.2}
        \begin{itemize}
            \item \textbf{Descrizione:} l'utente filtra i prodotti in base alle loro caratteristiche;
            \item \textbf{Attore Primarilo:} utente generico
            \item \textbf{Precondizione:} l'utente si trova all'interno della pagina con la lista dei prodotti (\hyperref[sec:UC6]{\underline{UC6}});
            \item \textbf{Input:} selezione del filtro desiderato;
            \item \textbf{Postcondizione:} la pagina si aggiorna secondo i filtri selezionati;
            \item \textbf{Scenario Principale:}
            \begin{itemize}
                \item l'utente si trova all'interno della pagina con la lista dei prodotti;
                \item inserisce i parametri per filtrare i prodotti;
                \item la pagina con la lista dei prodotti si aggiorna con i prodotti filtrati.
            \end{itemize}
        \end{itemize}
\newpage
\subsection{UC9: Registrazione standard}
\label{sec:UC9}
\begin{figure}[!ht]
    \caption{Diagramma di UC9: Registrazione standard}
    \vspace{10px}
    \includegraphics[scale=0.5]{../../../Images/AnalisiRequisiti/UC09.png}
    \centering
\end{figure}
\begin{itemize}
    \item \textbf{Descrizione:} l'utente vuole registrarsi nella piattaforma;
    \item \textbf{Attore Primario:} utente non autenticato;
    \item \textbf{Attore Secondario:} Amazon Cognito;
    \item \textbf{Precondizione:} l'utente non ha un profilo nella piattaforma;
    \item \textbf{Input:} pressione apposito bottone;
    \item \textbf{Postcondizione:} l'utente ha effettuato la registrazione tramite Amazon Cognito.
    \item \textbf{Scenario Principale:}
          \begin{itemize}
              \item un utente arriva per la prima volta nella piattaforma;
              \item si registra attraverso Amazon Cognito, inserendo i seguenti dati:
                    \begin{itemize}
                        \item nome (\hyperref[sec:UC9.1]{\underline{UC9.1}});
                        \item cognome (\hyperref[sec:UC9.2]{\underline{UC9.2}});
                        \item username (\hyperref[sec:UC9.3]{\underline{UC9.3}});
                        \item password (\hyperref[sec:UC9.4]{\underline{UC9.4}});
                        \item ripetizione password.
                    \end{itemize}
              \item l'utente è registrato.
          \end{itemize}
\end{itemize}

\subsubsection{UC9.1: Inserimento nome}
\label{sec:UC9.1}
\begin{itemize}
    \item \textbf{Descrizione:} l'utente inserisce il nome per registrarsi;
    \item \textbf{Attore Primario:} utente generico;
    \item \textbf{Attore Secondario:} Amazon Cognito;
    \item \textbf{Precondizione:} l'utente ha iniziato la registrazione;
    \item \textbf{Input:} stringa con il nome;
    \item \textbf{Postcondizione:} l'utente ha inserito il nome.
\end{itemize}

\subsubsection{UC9.2: Inserimento cognome}
\label{sec:UC9.2}
\begin{itemize}
    \item \textbf{Descrizione:} l'utente inserisce il cognome per registrarsi;
    \item \textbf{Attore Primario:} utente generico;
    \item \textbf{Attore Secondario:} Amazon Cognito;
    \item \textbf{Precondizione:} l'utente ha inserito il nome (\hyperref[sec:UC9.1]{\underline{UC9.1}});
    \item \textbf{Input:} stringa con il cognome;
    \item \textbf{Postcondizione:} l'utente ha inserito il cognome.
\end{itemize}

\subsubsection{UC9.3: Inserimento username}
\label{sec:UC9.3}
\begin{itemize}
    \item \textbf{Descrizione:} l'utente inserisce lo username per registrarsi;
    \item \textbf{Attore Primario:} utente generico;
    \item \textbf{Attore Secondario:} Amazon Cognito;
    \item \textbf{Precondizione:} l'utente ha inserito il cognome (\hyperref[sec:UC9.2]{\underline{UC9.2}});
    \item \textbf{Input:} stringa con lo username;
    \item \textbf{Postcondizione:} l'utente ha inserito lo username.
    \item \textbf{Estensione:}
          \begin{itemize}
              \item Se lo username è già utilizzato da un altro utente viene visualizzato un errore. (\hyperref[sec:UC9.5]{\underline{UC9.5}})
          \end{itemize}
\end{itemize}

\subsubsection{UC9.4: Inserimento password}
\label{sec:UC9.4}
\begin{itemize}
    \item \textbf{Descrizione:} l'utente inserisce la password per registrarsi;
    \item \textbf{Attore Primario:} utente generico;
    \item \textbf{Attore Secondario:} Amazon Cognito;
    \item \textbf{Precondizione:} l'utente ha inserito lo username (\hyperref[sec:UC9.3]{\underline{UC9.3}});
    \item \textbf{Input:} stringa con la password;
    \item \textbf{Postcondizione:} l'utente ha inserito sia la password che la conferma.
    \item \textbf{Scenario Principale:}
          \begin{itemize}
              \item l'utente ha inserito tutti gli altri dati;
              \item l'utente inserisce la propria password;
              \item l'utente inserisce la conferma della password;
              \item l'utente termina la registrazione.
          \end{itemize}
    \item \textbf{Estensione:}
          \begin{itemize}
              \item La password inserita non è conforme ai requisiti:
                    \begin{itemize}
                        \item L'utente può inserire la password rispettando i requisiti.
                    \end{itemize}
          \end{itemize}
\end{itemize}


\subsubsection{UC9.5: Visualizzazione username già presente}
\label{sec:UC9.5}
\begin{itemize}
    \item \textbf{Descrizione:} Visualizzazione di un errore se lo username è già in uso;
    \item \textbf{Attore Primario:} utente generico;
    \item \textbf{Attore Secondario:} Amazon Cognito;
    \item \textbf{Precondizione:} l'utente ha inserito un'username non disponibile perchè già usato;
    \item \textbf{Postcondizione:} l'utente visualizza un messaggio di errore.
\end{itemize}

\subsubsection{UC9.6: Password non conforme}
\label{sec:UC9.6}
\begin{itemize}
    \item \textbf{Descrizione:} Visualizzazione di un errore se la password non è conforme ai requisiti;
    \item \textbf{Attore Primario:} utente generico;
    \item \textbf{Attore Secondario:} Amazon Cognito;
    \item \textbf{Precondizione:} l'utente ha inserito una password non conforme ai requisiti, che sono:
          \begin{itemize}
              \item lunghezza almeno di 8 caratteri;
              \item contenga almeno una lettera maiuscola;
              \item contenga almeno una lettera minuscola;
              \item contenga almeno un numero;
              \item contenga almeno un carattere speciale.
          \end{itemize}
    \item \textbf{Postcondizione:} l'utente visualizza un messaggio di errore.
\end{itemize}
\subsection{UC10: Registrazione standard}
\label{sec:UC10}
\begin{figure}[!ht]
    \caption{Diagramma di UC10: Registrazione standard}
    \vspace{10px}
    \includegraphics[scale=0.5]{../../../Images/AnalisiRequisiti/UC10}
    \centering
\end{figure}
\begin{itemize}
    \item \textbf{Descrizione:} l'utente vuole registrarsi nella piattaforma;
    \item \textbf{Attore Primario:} utente generico;
    \item \textbf{Attore Secondario:} servizio di autenticazione esterno;
    \item \textbf{Precondizione:} l'utente non ha un profilo nella piattaforma;
    \item \textbf{Input:} pressione apposito bottone;
    \item \textbf{Postcondizione:} l'utente ha effettuato la registrazione tramite \textit{servizio di autenticazione esterno}. 
    \item \textbf{Scenario Principale:}
    \begin{itemize}
        \item un utente arriva per la prima volta nella piattaforma;
        \item si registra attraverso la piattaforma servizio di autenticazione esterno dove potrà registrarsi inserendo:
        \begin{itemize}
            \item nome (\hyperref[sec:UC10.1]{\underline{UC10.1}});
            \item cognome (\hyperref[sec:UC10.2]{\underline{UC10.2}});
            \item \textit{username}\textsubscript{\textbf{G}} (\hyperref[sec:UC10.3]{\underline{UC10.3}});
            \item \textit{password}\textsubscript{\textbf{G}} (\hyperref[sec:UC10.4]{\underline{UC10.4}});
            \item ripetizione \textit{password}\textsubscript{\textbf{G}}.
        \end{itemize}
        \item l'utente è registrato.
    \end{itemize} 
\end{itemize}

\subsubsection{UC10.1: Inserimento nome}
\label{sec:UC10.1}
\begin{itemize}
    \item \textbf{Descrizione:} l'utente inserisce il nome per registrarsi;
    \item \textbf{Attore Primario:} utente generico;
    \item \textbf{Attore Secondario:} servizio di autenticazione esterno;
    \item \textbf{Precondizione:} l'utente ha iniziato la registrazione;
    \item \textbf{Input:} stringa con il nome;
    \item \textbf{Postcondizione:} l'utente ha inserito il nome. 
\end{itemize}

\subsubsection{UC10.2: Inserimento cognome}
\label{sec:UC10.2}
\begin{itemize}
    \item \textbf{Descrizione:} l'utente inserisce il cognome per registrarsi;
    \item \textbf{Attore Primario:} utente generico;
    \item \textbf{Attore Secondario:} servizio di autenticazione esterno;
    \item \textbf{Precondizione:} l'utente ha inserito il nome (\hyperref[sec:UC10.1]{\underline{UC10.1}});
    \item \textbf{Input:} stringa con il cognome;
    \item \textbf{Postcondizione:} l'utente ha inserito il cognome. 
\end{itemize}

\subsubsection{UC10.3: Inserimento \textit{username}}
\label{sec:UC10.3}
\begin{itemize}
    \item \textbf{Descrizione:} l'utente inserisce l'\textit{username} per registrarsi;
    \item \textbf{Attore Primario:} utente generico;
    \item \textbf{Attore Secondario:} servizio di autenticazione esterno;
    \item \textbf{Precondizione:} l'utente ha inserito il cognome (\hyperref[sec:UC10.2]{\underline{UC10.2}});
    \item \textbf{Input:} stringa con l'\textit{username};
    \item \textbf{Postcondizione:} l'utente ha inserito l'\textit{username}.
    \item \textbf{Estensione:} 
    \begin{itemize}
        \item Se l'\textit{username} è già utilizzato da un altro utente: viene visualizzato un errore. (\hyperref[sec:UC10.5]{\underline{UC10.5}}) 
    \end{itemize} 
\end{itemize}

\subsubsection{UC10.4: Inserimento \textit{password}}
\label{sec:UC10.4}
\begin{itemize}
    \item \textbf{Descrizione:} l'utente inserisce la \textit{password} per registrarsi;
    \item \textbf{Attore Primario:} utente generico;
    \item \textbf{Attore Secondario:} servizio di autenticazione esterno;
    \item \textbf{Precondizione:} l'utente ha inserito l'username (\hyperref[sec:UC10.3]{\underline{UC10.3}});
    \item \textbf{Input:} stringa con la \textit{password};
    \item \textbf{Postcondizione:} l'utente ha inserito sia la \textit{password} che la conferma.
    \item \textbf{Scenario Principale:}
        \begin{itemize}
            \item l'utente ha inserito tutti gli altri dati;
            \item l'utente inserisce la propria \textit{password};
            \item l'utente inserisce la conferma della \textit{password};
            \item l'utente termina la registrazione.
        \end{itemize} 
\end{itemize}

\subsubsection{UC10.5: Visualizzazione username già presente}
\label{sec:UC10.5}
\begin{itemize}
    \item \textbf{Descrizione:} Visualizzazione di un errore se l'\textit{username} è già in uso;
    \item \textbf{Attore Primario:} utente generico;
    \item \textbf{Attore Secondario:} servizio di autenticazione esterno;
    \item \textbf{Precondizione:} l'utente ha inserito l'\textit{username} già usato;
    \item \textbf{Postcondizione:} l'utente visualizza un messaggio di errore. 
\end{itemize}
%Gestione del carrello
\subsection{UC11: Visualizzazione dei prodotti inseriti nel carrello}
\label{sec:UC11}
\begin{itemize}
    \item \textbf{Descrizione:} sezione per visualizzare tutti i prodotti inseriti nel carrello;
    \item \textbf{Attore Primario:} utente non autenticato;
    \item \textbf{Precondizione:}  l'utente si trova in una qualsiasi pagina della piattaforma;
    \item \textbf{Input:} l'utente clicca il bottone per accedere alla pagina del carrello;
    \item \textbf{Postcondizione:} l'utente visualizza la pagina del carrello e vede tutti i prodotti presenti al suo interno;
    \item \textbf{Scenario Principale:}
          \begin{itemize}
              \item l'utente si trova in una pagina della piattaforma;
              \item l'utente clicca il bottone per accedere alla pagina del carrello;
              \item l'elenco dei prodotti presenti nel carrello è visibile a schermo;
              \item Per ogni prodotto dell'elenco si visualizzano:
              \begin{itemize}
                  \item il nome del prodotto;
                  \item il prezzo di un' unità;
                  \item le tasse da applicare;
                  \item la quantità scelta.
              \end{itemize}
              \item inoltre si visualizza il totale complessivo.
          \end{itemize}
\end{itemize}
\subsection{UC12: Gestione del carrello}
        \begin{figure}[!ht]
            \caption{Diagramma di UC12: Gestione del carrello}
            \vspace{10px}
            \includegraphics[scale=0.5]{../../../Images/AnalisiRequisiti/UC8}
            \centering
        \end{figure}
        \begin{itemize}
            \item \textbf{Descrizione:} questa sezione tratta l'insieme di operazioni che un utente generico fa per gestire i prodotti nel carrello;
            \item \textbf{Attore Primario:} utente generico;
            \item \textbf{Precondizione:} l'utente si trova in una qualsiasi pagina della piattaforma;
            \item \textbf{Input:} l'utente clicca il bottone per entrare nel carrello;
            \item \textbf{Postcondizione:} l'utente ha completato le operazioni per gestire i prodotti nel carrello;
            \item \textbf{Scenario Principale:}
                \begin{itemize}
                    \item l'utente si trova in una qualsiasi pagina della piattaforma;
                    \item clicca il bottone per entrare nel carrello;
                    \item una volta entrato vede la lista di tutti i prodotti all'interno del carrello;
                    \item può decidere se rimuovere qualche prodotto o modificarne le quantità.
                \end{itemize}
        \end{itemize}
        \subsubsection{UC12.1: Visualizzazione dei prodotti inseriti}
        \label{sec:UC12.1}
        \begin{itemize}
            \item \textbf{Descrizione:} sezione per visualizzare tutti i prodotti inseriti nel carrello;
            \item \textbf{Attore Primario:} utente generico;
            \item \textbf{Precondizione:}  l'utente si trova in una qualsiasi pagina della piattaforma;
            \item \textbf{Input:} l'utente clicca il bottone per entrare nel carrello;
            \item \textbf{Postcondizione:} l'utente si trova sulla pagina del carrello e vede tutti i prodotti presenti al suo interno;
            \item \textbf{Scenario Principale:}
                \begin{itemize}
                    \item l'utente si trova in una pagina della piattaforma;
                    \item l'utente clicca il bottone per entrare nel carrello;
                    \item una volta entrato l'elenco dei prodotti è subito visibile.
                \end{itemize}
        \end{itemize}
        \subsubsection{UC12.2: Rimozione di uno o più prodotti}
        \begin{itemize}
            \item \textbf{Descrizione:} sezione per rimuovere uno o più prodotti dal carrello;
            \item \textbf{Attore Primario:} utente generico;
            \item \textbf{Precondizione:} l'utente è nella pagina del carrello (\hyperref[sec:UC8.1]{\underline{UC8.1}});
            \item \textbf{Input:} l'utente seleziona i prodotti da rimuovere dal carrello;
            \item \textbf{Postcondizione:} vengono rimossi i prodotti selezionati;
        \end{itemize}
        \subsubsection{UC12.3: Modifica della quantità di uno o più prodotti}
        \begin{itemize}
            \item \textbf{Descrizione:} sezione per modificare le quantità dei prodotti nel carrello;
            \item \textbf{Attore Primario:} utente generico;
            \item \textbf{Precondizione:} l'utente è nella pagina del carrello (\hyperref[sec:UC8.1]{\underline{UC8.1}});
            \item \textbf{Input:} l'utente inserisce le modifiche desiderate ai prodotti;
            \item \textbf{Postcondizione:} le modifiche sono applicate;
            \item \textbf{Estensioni:} 
                \begin{itemize}
                    \item se la nuova quantità non dovesse essere disponibile per il venditore, l'utente viene avvisato.
                \end{itemize}
        \end{itemize}
        \subsection{UC12.4 Visualizzazione errore quantità superata}
        \begin{itemize}
            \item \textbf{Descrizione:} La quantità inserita dall'utente supera quella disponibile in magazzino;
            \item \textbf{Attore primario:} Utente generico;
            \item \textbf{Precondizione:} L'utente si trova all'interno del carrello;
            \item \textbf{Postcondizione:} Vienie visualizzato il messaggio di errore.
        \end{itemize}
\subsection{UC13: Scelta della categoria}
\label{sec:UC13}
\begin{itemize}
    \item \textbf{Descrizione:} l'utente sceglie una specifica categoria di prodotti da ricercare;
    \item \textbf{Attore Primario:} utente generico;
    \item \textbf{Precondizione:} l'utente si trova in una pagina che consente la divisione in categorie;
    \item \textbf{Input:} selezione della categoria;
    \item \textbf{Postcondizione:} la categoria di ricerca cambia secondo la scelta dell'utente;
    \item \textbf{Scenario Principale:}
    \begin{itemize}
        \item l'utente vuole ricercare un prodotto appartenente ad una determinata categoria.
    \end{itemize} 
    \item \textbf{Inclusioni:}
    \begin{itemize}
        \item viene visualizzata una pagina con tutti i risultati della ricerca \hyperref[sec:UC8]{\underline{UC8}}.
    \end{itemize}
\end{itemize}
\subsection{UC14 Visualizzazione errore quantità superata di un prodotto nel carrello}
\label{sec:UC14}
\begin{itemize}
    \item \textbf{Descrizione:} la quantità inserita dall'utente supera quella disponibile in magazzino;
    \item \textbf{Attore primario:} utente non autenticato, utente autenticato;
    \item \textbf{Precondizione:} l'utente si trova all'interno del carrello;
    \item \textbf{Postcondizione:} viene visualizzato il messaggio di errore.
\end{itemize}
\subsection{UC15: Gestione profilo}
\label{sec:UC15}
\begin{itemize}
    \item \textbf{Descrizione:} l'utente vuole gestire il proprio profilo;
    \item \textbf{Attore Primario:} utente autenticato;
    \item \textbf{Attore Secondario:} servizio di autenticazione esterno;
    \item \textbf{Precondizione:} l'utente si trova all'interno del proprio profilo
    \item \textbf{Postcondizione:} l'utente può modificare il proprio profilo
    \item \textbf{Scenario Principale:}
    \begin{itemize}
        \item  l'utente può modificare:
        \begin{itemize}
            \item nome;
            \item cognome;
            \item username\textsubscript{\textbf{G}};
            \item password\textsubscript{\textbf{G}};
            \item indirizzi.
        \end{itemize}
    \end{itemize}
\end{itemize}
\subsection{UC16: Logout}
\label{sec:UC16}
\begin{itemize}
    \item \textbf{Descrizione:} sezione per effettuare il logout;
    \item \textbf{Attore Primario:} utente autenticato;
    \item \textbf{Attore Secondario:} servizio di autenticazione esterno;
    \item \textbf{Precondizione:} l'utente (cliente o venditore) si trova in una qualsiasi pagina della piattaforma;
    \item \textbf{Input:} l'utente clicca il bottone apposito per effettuare il logout;
    \item \textbf{Postcondizione:} l'utente diventa non autenticato, uscendo dalla sessione.
    \item \textbf{Scenario Principale:}
    \begin{itemize}
        \item l'utente autenticato clicca il bottone per effettuare il logout ed esce dalla sessione corrente.
    \end{itemize}
\end{itemize}
\subsection{UC17: Gestione profilo}
\label{sec:UC17}
\begin{figure}[!ht]
    \caption{Diagramma di UC17: Gestione profilo}
    \vspace{10px}
    \includegraphics[scale=0.5]{../../../Images/AnalisiRequisiti/UC17}
    \centering
\end{figure}

\begin{itemize}
    \item \textbf{Descrizione:} l'utente vuole gestire il proprio profilo;
    \item \textbf{Attore Primario:} utente autenticato;
    \item \textbf{Attore Secondario:} servizio di autenticazione esterno;
    \item \textbf{Precondizione:} l'utente si trova all'interno del proprio profilo
    \item \textbf{Postcondizione:} l'utente può modificare il proprio profilo
    \item \textbf{Scenario Principale:}
          \begin{itemize}
              \item  l'utente decide cosa modificare tra:
                    \begin{itemize}
                        \item nome;
                        \item cognome;
                        \item username;
                        \item password;
                        \item indirizzo.
                    \end{itemize}
          \end{itemize}
\end{itemize}



\subsubsection{UC17.1 Modifica del nome}
\label{sec:UC17.1}
\begin{itemize}
    \item \textbf{Descrizione:} l'utente vuole modificare il proprio nome relativo all'account;
    \item \textbf{Attore Primario:} utente autenticato;
    \item \textbf{Attore Secondario:} servizio di autenticazione esterno;
    \item \textbf{Precondizione:} l'utente si trova all'interno del proprio profilo
    \item \textbf{Postcondizione:} l'utente può modificare il proprio nome.
\end{itemize}

\subsubsection{UC17.2 Modifica del cognome}
\label{sec:UC17.2}
\begin{itemize}
    \item \textbf{Descrizione:} l'utente vuole modificare il proprio cognome relativo all'account;
    \item \textbf{Attore Primario:} utente autenticato;
    \item \textbf{Attore Secondario:} servizio di autenticazione esterno;
    \item \textbf{Precondizione:} l'utente si trova all'interno del proprio profilo
    \item \textbf{Postcondizione:} l'utente può modificare il proprio cognome.
\end{itemize}

\subsubsection{UC17.3 Modifica dell'username}
\label{sec:UC17.3}
\begin{itemize}
    \item \textbf{Descrizione:} l'utente vuole modificare il proprio username;
    \item \textbf{Attore Primario:} utente autenticato;
    \item \textbf{Attore Secondario:} servizio di autenticazione esterno;
    \item \textbf{Precondizione:} l'utente si trova all'interno del proprio profilo;
    \item \textbf{Postcondizione:} l'utente può modificare il proprio username;
    \item \textbf{Estensione}: visualizzazione errore se l'username è già presente nel database \hyperref[sec:UC17.6]{UC17.6}.
\end{itemize}

\subsubsection{UC17.4 Modifica della password}
\label{sec:UC17.4}
\begin{itemize}
    \item \textbf{Descrizione:} l'utente vuole modificare la propria password;
    \item \textbf{Attore Primario:} utente autenticato;
    \item \textbf{Attore Secondario:} servizio di autenticazione esterno;
    \item \textbf{Precondizione:} l'utente si trova all'interno del proprio profilo
    \item \textbf{Postcondizione:} l'utente può modificare la propria password.
\end{itemize}

\subsubsection{UC17.5 Modifica dell'indirizzo}
\label{sec:UC17.5}
\begin{itemize}
    \item \textbf{Descrizione:} l'utente vuole modificare il proprio indirizzo;
    \item \textbf{Attore Primario:} utente autenticato;
    \item \textbf{Attore Secondario:} servizio di autenticazione esterno;
    \item \textbf{Precondizione:} l'utente si trova all'interno del proprio profilo
    \item \textbf{Postcondizione:} l'utente può modificare il proprio indirizzo.
\end{itemize}

\subsubsection{UC17.6: Visualizzazione username già presente}
\label{sec:UC17.6}
\begin{itemize}
    \item \textbf{Descrizione:} Visualizzazione di un errore se lo username è già in uso;
    \item \textbf{Attore Primario:} utente autenticato;
    \item \textbf{Attore Secondario:} servizio di autenticazione esterno;
    \item \textbf{Precondizione:} l'utente ha inserito uno username non disponibile perchè già usato;
    \item \textbf{Postcondizione:} l'utente visualizza un messaggio di errore.
\end{itemize}

\subsubsection{UC17.7: Password non conforme}
\label{sec:UC17.7}
\begin{itemize}
    \item \textbf{Descrizione:} Visualizzazione di un errore se la password non è conforme ai requisiti;
    \item \textbf{Attore Primario:} utente autenticato;
    \item \textbf{Attore Secondario:} Amazon Cognito;
    \item \textbf{Precondizione:} l'utente ha inserito una password non conforme ai requisiti, che sono:
          \begin{itemize}
              \item lunghezza almeno di 8 caratteri;
              \item contenga almeno una lettera maiuscola;
              \item contenga almeno una lettera minuscola;
              \item contenga almeno un numero
              \item contenga almeno un carattere speciale;
          \end{itemize}
    \item \textbf{Postcondizione:} l'utente visualizza un messaggio di errore.
\end{itemize}
\newpage
\subsection{UC18: Checkout}
\label{sec:UC18}
\begin{figure}[!ht]
    \caption{Diagramma di UC18: Checkout}
    \vspace{10px}
    \includegraphics[scale=0.5]{../../../Images/AnalisiRequisiti/UC18}
    \centering
\end{figure}
\begin{itemize}
    \item \textbf{Descrizione:} caso d'uso per la creazione di un nuovo ordine e l'acquisto dei prodotti;
    \item \textbf{Attore Primario:} cliente autenticato;
    \item \textbf{Attore Secondario:} Stripe\textsubscript{\textbf{G}}, gestore di pagamenti di terze parti;
    \item \textbf{Precondizione:} il cliente si trova nel carrello (\hyperref[sec:UC11]{\underline{UC11}}) e ha già inserito almeno un prodotto oppure si trova nella pagina del prodotto (\hyperref[sec:UC5]{\underline{UC5}});
    \item \textbf{Input:} il cliente clicca il bottone per iniziare il checkout;
    \item \textbf{Postcondizione:} l'ordine viene emesso e aggiunto alla lista degli ordini di quel cliente; i prodotti acquistati vengono rimossi dal carrello e viene diminuita la rispettiva quantità dal deposito del venditore; il cliente viene reindirizzato alla pagina di riepilogo dell'ordine (\hyperref[sec:UC21]{\underline{UC21}})
    \item \textbf{Scenario Principale:}
          \begin{itemize}
              \item il cliente clicca il bottone per effettuare il checkout;
              \item il cliente inserisce i dati di fatturazione (\hyperref[sec:UC18.1]{\underline{UC18.1}}) e, se diversi, i dati di spedizione (\hyperref[sec:UC18.2]{\underline{UC18.2}});
              \item vengono inseriti eventuali costi di spedizione;
              \item il cliente viene reindirizzato al servizio di pagamento esterno, dove inserisce i dati di pagamento;
              \item l'ordine è emesso e segnato come completato.
          \end{itemize}
    \item \textbf{Estensioni:}
    \item Il cliente decide di non completare il checkout:
          \begin{itemize}
              \item  il cliente esce dalla pagina senza causare modifiche al carrello;
          \end{itemize}
\end{itemize}
\subsubsection{UC18.1: Inserimento dell'indirizzo di fatturazione}
\label{sec:UC18.1}
\begin{itemize}
    \item \textbf{Descrizione:} Sezione per l'inserimento dell'indirizzo fatturazione;
    \item \textbf{Attore Primario:} cliente autenticato;
    \item \textbf{Precondizione:} il cliente si trova nella fase di checkout;
    \item \textbf{Input:} il cliente inserisce i dati richiesti dalla pagina;
    \item \textbf{Postcondizione:} si procede con la fase successiva del checkout (\hyperref[sec:UC18.2]{\underline{UC18.2}})
    \item \textbf{Scenario Principale:} il cliente inserisce negli appositi spazi i dati per completare l'indirizzo di fatturazione.
\end{itemize}
\subsubsection{UC18.2: Inserimento dell'indirizzo di spedizione}
\label{sec:UC18.2}
\begin{itemize}
    \item \textbf{Descrizione:} Sezione per l'inserimento dell'indirizzo spedizione;
    \item \textbf{Attore Primario:} cliente autenticato;
    \item \textbf{Precondizione:} il cliente si trova nella fase di checkout;
    \item \textbf{Input:} il cliente inserisce i dati richiesti oppure clicca un apposito bottone se l'indirizzo di spedizione è lo stesso di quello di fatturazione;
    \item \textbf{Postcondizione:} si procede con la fase successiva del checkout;
    \item \textbf{Scenario Principale:} il cliente inserisce negli appositi spazi i dati per completare l'indirizzo di spedizione oppure clicca il pulsante per autocompletarli se è il medesimo di quello di fatturazione.
\end{itemize}

\subsection{UC19: Effettua un reso}
\label{sec:UC19}
\begin{itemize}
    \item \textbf{Descrizione:} sezione per effettuare il reso di un ordine;
    \item \textbf{Attore Primario:} cliente autenticato;
    \item \textbf{Precondizione:} il cliente si trova nella pagina per la visualizzazione dell'elenco degli ordini (\hyperref[sec:UC6]{\underline{UC6}});
    \item \textbf{Input:} il cliente clicca il bottone per entrare in questa sezione;
    \item \textbf{Postcondizione:} la richiesta di reso è andata a buon fine e i prodotti dell'ordine torneranno al venditore;
    \item \textbf{Scenario Principale:}
        \begin{itemize}
            \item il cliente si trova nella sezione per visualizzare gli ordini effettuati;
            \item clicca il bottone per eseguire il reso di un ordine;
            \item il cliente conferma e il reso è confermato.
        \end{itemize}
\end{itemize}
\subsection{UC20: Contatta il venditore}
\label{sec:UC20}
\begin{itemize}
    \item \textbf{Descrizione:} l'utente vuole contattare il venditore;
    \item \textbf{Attore Primario:} utente autenticato;
    \item \textbf{Precondizione:} l'utente si trova in una pagina che permette di contattare il venditore;
    \item \textbf{Input:} pressione dell'apposito bottone;
    \item \textbf{Postcondizione:} l'utente può contattare il venditore;
    \item \textbf{Scenario Principale:}
          \begin{itemize}
              \item l'utente contatta il venditore tramite questa sezione.
          \end{itemize}
\end{itemize}

\subsection{UC21: Checkout}
\label{sec:UC21}
\begin{figure}[!ht]
    \caption{Diagramma di UC21: Checkout}
    \vspace{10px}
    \includegraphics[scale=0.5]{../../../Images/AnalisiRequisiti/UC21}
    \centering
\end{figure}
\begin{itemize}
    \item \textbf{Descrizione:} caso d'uso per la creazione di un nuovo ordine e l'acquisto dei prodotti;
    \item \textbf{Attore Primario:} cliente autenticato;
    \item \textbf{Attore Secondario:} Stripe\textsubscript{\textbf{G}}, gestore di pagamenti di terze parti;
    \item \textbf{Precondizione:} il cliente si trova nel carrello (\hyperref[sec:UC11]{\underline{UC11}}) e ha già inserito almeno un prodotto oppure si trova nella pagina del prodotto (\hyperref[sec:UC4]{\underline{UC4}});
    \item \textbf{Input:} il cliente clicca il bottone per iniziare il checkout;
    \item \textbf{Postcondizione:} l'ordine viene emesso e aggiunto alla lista degli ordini di quel cliente; i prodotti acquistati vengono rimossi dal carrello e viene diminuita la rispettiva quantità dal deposito del venditore; il cliente viene reindirizzato alla pagina di riepilogo dell'ordine (\hyperref[sec:UC21]{\underline{UC21}});
    \item \textbf{Scenario Principale:}
          \begin{itemize}
              \item il cliente clicca il bottone per effettuare il checkout;
              \item il cliente inserisce i dati di fatturazione (\hyperref[sec:UC21.1]{\underline{UC21.1}}) e, se diversi, i dati di spedizione (\hyperref[sec:UC21.2]{\underline{UC21.2}});
              \item vengono inseriti eventuali costi di spedizione;
              \item il cliente viene reindirizzato al servizio di pagamento esterno, dove inserisce i dati di pagamento;
              \item l'ordine è emesso e segnato come completato.
          \end{itemize}
    \item \textbf{Estensioni:}
    \item Il cliente decide di non completare il checkout:
          \begin{itemize}
              \item  il cliente esce dalla pagina senza causare modifiche al carrello;
          \end{itemize}
    \item Il pagamento non è andato a buon fine:
          \begin{itemize}
              \item L'utente ritorna al carrello \hyperref[sec:UC36]{\underline{UC36}}.
          \end{itemize}
\end{itemize}

\subsubsection{UC21.1: Inserimento dell'indirizzo di fatturazione}
\label{sec:UC21.1}
\begin{itemize}
    \item \textbf{Descrizione:} Sezione per l'inserimento dell'indirizzo fatturazione;
    \item \textbf{Attore Primario:} cliente autenticato;
    \item \textbf{Precondizione:} il cliente si trova nella fase di checkout;
    \item \textbf{Input:} il cliente inserisce i dati richiesti dalla pagina;
    \item \textbf{Postcondizione:} si procede con la fase successiva del checkout (\hyperref[sec:UC21.2]{\underline{UC21.2}});
    \item \textbf{Scenario Principale:} il cliente inserisce negli appositi spazi i dati per completare l'indirizzo di fatturazione.
\end{itemize}
\subsubsection{UC21.2: Inserimento dell'indirizzo di spedizione}
\label{sec:UC21.2}
\begin{itemize}
    \item \textbf{Descrizione:} Sezione per l'inserimento dell'indirizzo spedizione;
    \item \textbf{Attore Primario:} cliente autenticato;
    \item \textbf{Precondizione:} il cliente si trova nella fase di checkout;
    \item \textbf{Input:} il cliente inserisce i dati richiesti oppure clicca un apposito bottone se l'indirizzo di spedizione è lo stesso di quello di fatturazione;
    \item \textbf{Postcondizione:} si procede con la fase successiva del checkout;
    \item \textbf{Scenario Principale:} il cliente inserisce negli appositi spazi i dati per completare l'indirizzo di spedizione oppure clicca il pulsante per autocompletarli se è il medesimo di quello di fatturazione.
\end{itemize}
\subsubsection{UC21.3: Inserimento dei dati del pagamento}
\label{sec:UC21.3}
\begin{itemize}
    \item \textbf{Descrizione:} Sezione per l'inserimento dei dati di pagamento;
    \item \textbf{Attore Primario:} cliente autenticato;
    \item \textbf{Attore Secondario:} Stripe;
    \item \textbf{Precondizione:} il cliente si trova nella piattaforma di Stripe per completare il checkout;
    \item \textbf{Input:} il cliente inserisce i dati richiesti dalla pagina;
    \item \textbf{Postcondizione:} il cliente ha inserito i dati per il pagamento;
    \item \textbf{Scenario Principale:} il cliente inserisce negli appositi spazi i dati della propria carta per il pagamento.
\end{itemize}
\subsection{UC22: Checkout}
        \label{sec:UC22}
            \begin{figure}[!ht]
                \caption{Diagramma di UC22: Checkout}
                \vspace{10px}
                \includegraphics[scale=0.5]{../../../Images/AnalisiRequisiti/UC22}
                \centering
            \end{figure}
                \begin{itemize}
                \item \textbf{Descrizione:} caso d'uso per la creazione di un nuovo ordine e l'acquisto dei prodotti inseriti nel carrello;
                \item \textbf{Attore Primario:} cliente autenticato;
                \item \textbf{Attore Secondario:} Stripe, gestore di pagamenti di terze parti;
                \item \textbf{Precondizione:} il cliente si trova sul carrello e ha già inserito almeno un prodotto;
                \item \textbf{Input:} il cliente clicca il bottone per iniziare il checkout;
                \item \textbf{Postcondizione:} l'ordine viene emesso e aggiunto alla lista degli ordini di quel cliente; i prodotti acquistati vengono rimossi dal carrello e viene diminuita la quantità dal deposito del venditore;
                \item \textbf{Scenario Principale:} 
                    \begin{itemize}
                        \item il cliente preme sul pulsante per entrare nella sezione del checkout;
                        \item il cliente inserisce i dati di fatturazione (\hyperref[sec:UC22.1]{\underline{UC22.1}}) e, se diversi, i dati di spedizione (\hyperref[sec:UC22.2]{\underline{UC22.2}});
                        \item vengono inseriti eventuali costi di spedizione;
                        \item il cliente viene reindirizzato al servizio di pagamento esterno, dove inserisce i dati di pagamento;
                        \item l'ordine è emesso e segnato come completato.
                    \end{itemize}
                \item \textbf{Estensioni:}
                    \begin{itemize}
                        \item se il cliente decide di non completare il checkout, può uscire dalla pagina senza causare modifiche al carrello;
                        \item se il pagamento ha avuto esito negativo, l'ordine non viene emesso e quindi è necessario ricominciare il pagamento.
                    \end{itemize}
            \end{itemize}
            \subsubsection{UC22.1: Inserimento dell'indirizzo di fatturazione}
            \label{sec:UC22.1}
                \begin{itemize}
                    \item \textbf{Descrizione:} Sezione per l'inserimento dell'indirizzo fatturazione;
                    \item \textbf{Attore Primario:} cliente autenticato;
                    \item \textbf{Precondizione:} il cliente si trova nella fase di checkout;
                    \item \textbf{Input:} il cliente inserisce i dati richiesti dalla pagina;
                    \item \textbf{Postcondizione:} si procede con la fase successiva del checkout (\hyperref[sec:UC22.2]{\underline{UC22.2}})
                    \item \textbf{Scenario Principale:} il cliente inserisce negli appositi spazi i dati per completare l'indirizzo di fatturazione.
                \end{itemize}
            \subsubsection{UC22.2: Inserimento dell'indirizzo di spedizione}
            \label{sec:UC22.2}
                \begin{itemize}
                    \item \textbf{Descrizione:} Sezione per l'inserimento dell'indirizzo spedizione;
                    \item \textbf{Attore Primario:} cliente autenticato;
                    \item \textbf{Precondizione:} il cliente si trova nella fase di checkout;
                    \item \textbf{Input:} il cliente inserisce i dati richiesti dalla pagina, oppure clicca un pulsante se l'indirizzo di spedizione è lo stesso di quello di fatturazione;
                    \item \textbf{Postcondizione:} si procede con la fase successiva del checkout (\hyperref[sec:UC22.3]{\underline{UC22.3}})
                    \item \textbf{Scenario Principale:} il cliente inserisce negli appositi spazi i dati per completare l'indirizzo di spedizione oppure clicca il pulsante per autocompletarli se è il medesimo di quello di fatturazione.
                \end{itemize}
            \subsubsection{UC22.3: Reindirizzamento al servizio di pagamento esterno}
            \label{sec:UC22.3}
                \begin{itemize}
                    \item \textbf{Descrizione:} Sezione per il pagamento dell'ordine da effettuare;
                    \item \textbf{Attore Primario:} cliente autenticato;
                    \item \textbf{Attore Secondario:} Stripe, gestore di pagamenti di terze parti;
                    \item \textbf{Precondizione:} il cliente ha inserito i dati per la fatturazione e la spedizione nella sezione del checkout;
                    \item \textbf{Input:} il cliente preme il pulsante per il pagamento;
                    \item \textbf{Postcondizione:} il pagamento ha avuto esito positivo, l'ordine viene confermato e viene decrementata la quantità dei prodotti disponibili al venditore, in base al numero di prodotti acquistati dal cliente;
                    \item \textbf{Scenario Principale:}
                    \begin{itemize}
                        \item il cliente preme sul pulsante e viene reindirizzato al servizio esterno per eseguire il pagamento;
                        \item il cliente inserisce i propri dati ed esegue il pagamento;
                        \item il pagamento è riuscito e l'ordine viene confermato.
                    \end{itemize}
                    \item \textbf{Estensioni:}
                    \begin{itemize}
                        \item il pagamento non è riuscito, viene visualizzato un errore di pagamento (\hyperref[sec:UC22.4]{\underline{UC22.4}}) e si viene reindirizzati alla pagina del checkout per riprovare.
                    \end{itemize}
                \end{itemize}
            \subsubsection{UC22.4: Errore di pagamento}
            \label{sec:UC22.4}
                \begin{itemize}
                    \item \textbf{Descrizione:} visualizzazione di un errore per un fallimento nella fase di pagamento;
                    \item \textbf{Attore Primario:} Stripe
                    \item \textbf{Precondizione:} il cliente ha inserito i dati del pagamento;
                    \item \textbf{Input:} Stripe ritorna un errore nel risultato del pagamento;
                    \item \textbf{Postcondizione:} viene visualizzato un messaggio di errore, successivamente il cliente viene reindirizzato alla sezione del checkout;
                \end{itemize}
\subsection{UC20: Annullamento checkout}
\label{sec:UC20}
\begin{itemize}
    \item \textbf{Descrizione:} visualizzazione di un errore per un fallimento nella fase di pagamento;
    \item \textbf{Attore Primario:} Stripe
    \item \textbf{Precondizione:} il cliente ha inserito i dati del pagamento;
    \item \textbf{Input:} Stripe ritorna un errore nel risultato del pagamento;
    \item \textbf{Postcondizione:} viene visualizzato un messaggio di errore, successivamente il cliente viene reindirizzato alla sezione del checkout;
\end{itemize}

\subsection{UC24: Visualizzazione della lista dei clienti}
\label{sec:UC24}
\begin{itemize}
    \item \textbf{Descrizione:} sezione per visualizzare l'elenco dei clienti del proprio negozio;
    \item \textbf{Attore Primario:} venditore autenticato; 
    \item \textbf{Precondizione:} il venditore si trova in una qualsiasi pagina;
    \item \textbf{Input:} il venditore clicca il bottone per entrare in questa sezione; 
    \item \textbf{Postcondizione:} il venditore visualizza la lista di tutti i clienti del negozio.
\end{itemize}
\subsection{UC25: Contatta un cliente}
\label{sec:UC25}
\begin{itemize}
    \item \textbf{Descrizione:} sezione per il venditore che permette di contattare un cliente;
    \item \textbf{Attore Primario:} venditore autenticato;
    \item \textbf{Precondizione:} il venditore si trova nella lista dei suoi clienti (\hyperref[sec:UC24]{\underline{UC24}});
    \item \textbf{Input} il venditore clicca il bottone per entrare in questa sezione; 
    \item \textbf{Postcondizione:} il venditore ha inviato una e-mail ad un suo cliente;
    \item \textbf{Scenario Principale}
        \begin{itemize}
            \item il venditore si trova nella lista dei suoi clienti;
            \item sceglie uno dei suoi clienti per poterlo contattare;
            \item compila il campo di testo scrivendo ciò che deve inviare;
            \item preme il pulsante per inviare, una e-mail verrà inviata al cliente;
        \end{itemize}
\end{itemize}
\subsubsection{UC26: Gestione delle categorie (\textbf{Aggiungere info specifiche})}
\label{sec:UC26}
\begin{figure}[!ht]
    \caption{Diagramma di UC26: Gestione delle categorie}
    \vspace{10px}
    \includegraphics[scale=0.5]{../../../Images/AnalisiRequisiti/UC26}
    \centering
\end{figure}
\begin{itemize}
    \item \textbf{Descrizione:} sezione dove il venditore può gestire le categorie di prodotti del negozio;
    \item \textbf{Attore Primario:} venditore autenticato;
    \item \textbf{Precondizione:} il venditore si trova in una qualsiasi pagina della piattaforma;
    \item \textbf{Input:} il venditore clicca il bottone per entrare in questa sezione;
    \item \textbf{Postcondizione:} il venditore ha completato le operazioni sulle categorie;
    \item \textbf{Scenario Principale:}
          \begin{itemize}
              \item il venditore si trova nella pagina per la gestione del negozio;
              \item può decidere se aggiungere una nuova categoria o se rimuoverne una già presente;
              \item una volta terminato applica le modifiche.
          \end{itemize}
\end{itemize}
\subsubsubsection{UC26.1: Aggiunta di una categoria}
\label{sec:UC26.1}
\begin{itemize}
    \item \textbf{Descrizione:} sezione per aggiungere una categoria di prodotti nel negozio;
    \item \textbf{Attore Primario:} venditore autenticato;
    \item \textbf{Precondizione:} il venditore si trova nella sezione per gestire le categorie (\hyperref[sec:UC26]{\underline{UC26}});
    \item \textbf{Input:} il venditore inserisce tutti i dati relativi alla categoria da inserire;
    \item \textbf{Postcondizione:} la categoria è aggiunta e i clienti possono trovarla nel negozio;
    \item \textbf{Scenario Principale:}
          \begin{itemize}
              \item il venditore si trova nella sezione per gestire le categorie;
              \item il venditore decide di aggiungere una nuova categoria ed entra in questa sezione;
              \item inserisce tutti i dati richiesti;
              \item conferma l'aggiunta e la categoria è visibile nel negozio.
          \end{itemize}
\end{itemize}
\subsubsubsection{UC26.2: Rimozione di una categoria}
\label{sec:UC26.2}
\begin{itemize}
    \item \textbf{Descrizione:} sezione per rimuovere una categoria dal negozio;
    \item \textbf{Attore Primario:} venditore autenticato;
    \item \textbf{Precondizione:} il venditore si trova nella sezione per gestire le categorie (\hyperref[sec:UC26]{\underline{UC26}});
    \item \textbf{Input:} il venditore sceglie la categoria da rimuovere;
    \item \textbf{Postcondizione:} la categoria è stata rimossa dal negozio;
    \item \textbf{Scenario Principale:}
          \begin{itemize}
              \item il venditore si trova nella sezione per gestire le categorie;
              \item il venditore decide di rimuovere una categoria dal negozio;
              \item il venditore conferma la rimozione della categoria selezionata;
              \item la categoria non è più visibile nel negozio.
          \end{itemize}
\end{itemize}


\subsection{UC27: Gestione dei prodotti}
        \label{sec:UC27}
        \begin{figure}[!ht]
            \caption{Diagramma di UC21.1: Gestione dei prodotti}
            \vspace{10px}
            \includegraphics[scale=0.5]{../../../Images/AnalisiRequisiti/UC27}
            \centering
        \end{figure}
        \begin{itemize}
            \item \textbf{Descrizione:} sezione per il venditore, dove può gestire i prodotti del negozio;
            \item \textbf{Attore Primario:} venditore autenticato;
            \item \textbf{Precondizione:} il venditore si trova in una qualsiasi pagina del negozio;
            \item \textbf{Input:} il venditore preme il bottone per entrare in questa sezione;
            \item \textbf{Postcondizione:} il venditore ha completato le modifiche ai prodotti desiderati;
            \item \textbf{Scenario Principale:} 
                \begin{itemize}
                    \item il venditore si trova nella pagina per la gestione del negozio;
                    \item può decidere se aggiungere un nuovo prodotto, modificarne o rimuoverne uno già presente;
                    \item una volta terminato applica le modifiche
                \end{itemize}
        \end{itemize}
        \subsubsection{UC27.1: Aggiunta di un prodotto}
        \begin{itemize}
            \item \textbf{Descrizione:} sezione per aggiungere un prodotto nel negozio;
            \item \textbf{Attore Primario:} venditore autenticato;
            \item \textbf{Precondizione:} il venditore si trova nella sezione per gestire i prodotti (\hyperref[sec:UC27]{\underline{UC27}});
            \item \textbf{Input:} il venditore inserisce tutti i dati relativi al prodotto da inserire;
            \item \textbf{Postcondizione:} il prodotto è aggiunto e i clienti possono trovarlo nel negozio;
            \item \textbf{Scenario Principale:} 
                \begin{itemize}
                    \item il venditore si trova nella sezione per gestire i prodotti;
                    \item il venditore decide di aggiungere un nuovo prodotto ed entra in questa sezione;
                    \item inserisce tutti i dati richiesti (nome, descrizione, prezzo ecc.);
                    \item conferma l'aggiunta e il prodotto è disponibile nel negozio.
                \end{itemize}
        \end{itemize}
        \subsubsection{UC27.2: Modifica di un prodotto}
        \begin{itemize}
            \item \textbf{Descrizione:} sezione per modificare un prodotto del negozio;
            \item \textbf{Attore Primario:} venditore autenticato;
            \item \textbf{Precondizione:} il venditore si trova nella sezione per gestire i prodotti (\hyperref[sec:UC27]{\underline{UC27}});
            \item \textbf{Input:} il venditore modifica i dati che ritiene opportuni relativi al prodotto da modificare;
            \item \textbf{Postcondizione:} il prodotto è stato modificato;
            \item \textbf{Scenario Principale:} 
                \begin{itemize}
                    \item il venditore si trova nella sezione per gestire i prodotti;
                    \item il venditore decide di modificare un prodotto ed entra in questa sezione;
                    \item modifica i dati che preferisce;
                    \item conferma le modifiche.
                \end{itemize}
        \end{itemize}
        \subsubsubsection{UC27.3: Rimozione di un prodotto}
        \begin{itemize}
            \item \textbf{Descrizione:} sezione per rimuovere un prodotto dal negozio;
            \item \textbf{Attore Primario:} venditore autenticato;
            \item \textbf{Precondizione:} il venditore si trova nella sezione per gestire i prodotti (\hyperref[sec:UC27]{\underline{UC27}});
            \item \textbf{Input:} il venditore sceglie il prodotto da eliminare;
            \item \textbf{Postcondizione:} il prodotto è stato rimosso dal negozio;
            \item \textbf{Scenario Principale:} 
                \begin{itemize}
                    \item il venditore si trova nella sezione per gestire i prodotti;
                    \item il venditore decide di rimuovere un prodotto dal negozio;
                    \item conferma;
                    \item il prodotto non è più visibile nel negozio.
                \end{itemize}
        \end{itemize}
\subsection{UC28: Aggiunta di un prodotto}
\label{sec:UC28}
\begin{itemize}
    \item \textbf{Descrizione:} sezione per aggiungere un prodotto nel negozio;
    \item \textbf{Attore Primario:} venditore autenticato;
    \item \textbf{Precondizione:} il venditore si trova nella sezione per gestire i prodotti;
    \item \textbf{Input:} il venditore inserisce tutti i dati relativi al prodotto da inserire;
    \item \textbf{Postcondizione:} il prodotto viene aggiunto e i clienti possono visualizzarlo nel negozio;
    \item \textbf{Scenario Principale:}
          \begin{itemize}
              \item il venditore si trova nella sezione per gestire i prodotti;
              \item il venditore decide di aggiungere un nuovo prodotto ed entra in questa sezione;
              \item il venditore inserisce tutti i dati richiesti:
                    \begin{itemize}
                        \item il nome del prodotto;
                        \item il prezzo di vendita esentasse;
                        \item la descrizione del prodotto;
                        \item la quantità disponibile in magazzino;
                        \item l'immagine che lo descrive;
                        \item la categoria del prodotto;
                        \item l'eventuale sconto.
                    \end{itemize}
              \item il venditore conferma l'aggiunta e il prodotto diviene disponibile nel negozio.
          \end{itemize}
\end{itemize}