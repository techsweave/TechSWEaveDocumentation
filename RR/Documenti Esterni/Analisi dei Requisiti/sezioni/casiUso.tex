\section{Casi d'uso}
    \subsection{Attori dei casi d'uso}
    \subsubsection{Attori primari}
    Gli attori che il gruppo ha ritenuto essere i più adeguati sono:
        \begin{itemize}
            \item \textbf{Utente generico:} si divide in:
                \begin{itemize}
                    \item \textbf{Utente non autenticato:} utente che può navigare nell'e-commerce e può usufruire di alcune funzionalità, come la visualizzazione e la ricerca dei prodotti, che può aggiungere al proprio carrello, la applicazione di filtri e categorie per la ricerca, e infine di potersi autenticare.
                    \item \textbf{Utente autenticato:} un utente autenticato può a sua volta essere:
                        \begin{itemize}
                            \item \textbf{Cliente autenticato:} cliente che ha effettuato il login, può accedere a molte funzionalità, come l'aggiunta di prodotti al carrello, l'acquisto, la visualizzazione della lista degli ordini, la possibilità di contattare il venditore;
                            \item \textbf{Venditore autenticato:} venditore che ha effettuato il login, può accedere alle funzionalità relative all'amministrazione dei prodotti e delle categorie.
                        \end{itemize}
                        \begin{figure}[!ht]
                            \caption{Attori primari}
                            \vspace{5px}
                            \includegraphics[scale=0.59]{../../../Images/AnalisiRequisiti/attori}
                            \centering
                        \end{figure}
                \end{itemize}
        \end{itemize}
        \subsubsection{Attori secondari}
        \begin{itemize}
            \item \textbf{Amazon Cognito:} servizio esterno per il login e per la gestione delle credenziali;
            \item \textbf{Stripe:} servizio esterno per la gestione dei pagamenti.
        \end{itemize}
        \newpage
    \subsection{Elenco dei casi d'uso}
        \textbf{Utilizzo della piattaforma:}
        \begin{figure}[!ht]
            \caption{Casi d'uso che interessano gli utenti}
            \vspace{10px}
            \includegraphics[scale=0.44]{../../../Images/AnalisiRequisiti/casiUso}
            \centering
        \end{figure}
        \subsection{UC1: Primo ingresso alla piattaforma}
        \begin{itemize}
            \item \textbf{Descrizione} L'utente effettua il primo accesso alla piattaforma;
            \item \textbf{Attore Primario} Utente generico;
            \item \textbf{Precondizione} L'utente non sta visualizzando nulla;
            \item \textbf{Postcondizione} L'utente visualizza l'homepage della piattaforma;
            \item \textbf{Scenario Principale} L'utente arriva per la prima volta nell'e-commerce e visualizza l'homepage.
        \end{itemize}
        \subsection{UC2: Login}
        \begin{itemize}
            \item \textbf{Descrizione} Permette l'autenticazione di un utente;
            \item \textbf{Attore Primario} Utente generico;
            \item \textbf{Attore Secondario} Amazon Cognito;
            \item \textbf{Precondizione}L'utente generico non si è ancora autenticato
            \item \textbf{Input} Pressione bottone login;
            \item \textbf{Postcondizione} L'utente è autenticato;
            \item \textbf{Scenario Principale} L'utente entra nella pagina, preme il bottone per il login e viene inderizzato alla pagina di login fornita da \textit{Amazon cognito}.
        \end{itemize}
        \subsection{UC3: visualizzazione dettagli di un prodotto}
        \begin{itemize}
            \item \textbf{Descrizione} L'utente può visualizzare i dettagli di un prodotto di suo interesse;
            \item \textbf{Attore Primario} Utente generico
            \item \textbf{Precondizione} L'utente generico si trova nella pagina della lista dei prodotti;
            \item \textbf{Input} Click sul prodotto (icona, immagine, nome);
            \item \textbf{Postcondizione} L'utente visualizza i dettagli del prodotto d'interesse;
            \item \textbf{Scenario Principale} L'utente all'interno della lista dei prodotti visualizza un prodotto di suo interesse e lo apre e ne visualizza i dettagli.
        \end{itemize}
        \subsection{UC4: Aggiunta di un prodotto al carrello}
        \begin{itemize}
            \item \textbf{Descrizione} L'utente aggiunge al carrello un prodotto che intende acquistare;
            \item \textbf{Attore Primario} Utente generico;
            \item \textbf{Precondizione} L'utente generico di trova nella pagina dettagliata del prodotto;
            \item \textbf{Input} Click sul bottone di aggiunta al carrello;
            \item \textbf{Postcondizione} Il prodotto è stato aggiunto al carrello;
            \item \textbf{Scenario Principale} L'utente una volta deciso il prodotto da acquistare lo aggiunge al proprio carrello tramite l'apposito bottone.
        \end{itemize}
        \subsection{UC5: Ricerca dei prodotti}
        \begin{itemize}
            \item \textbf{Descrizione} L'utente vuole ricercare un prodotto in base ad una parola chiave; 
            \item \textbf{Attore Primario} Utente generico;
            \item \textbf{Precondizione} L'utente si trova in una pagina dedicata alla ricerca; 
            \item \textbf{Input} stringa;
            \item \textbf{Postcondizione} L'utente visualizza i prodotti corrispondenti alla parola chiave;
            \item \textbf{Scenario Principale} Un utente vuole ricercare un prodotto secondo una determinata parola chiave, gli vengo mostrati tutti i risultati della ricerca nella lista dei prodotti;
            \item \textbf{Inclusioni}
            \begin{itemize}
                \item viene visualizzata una pagina con tutti i risultati della ricerca (UC6)
            \end{itemize}
        \end{itemize}
        \subsection{UC6: Visualizzazione della lista dei prodotti}
        \begin{itemize}
            \item \textbf{Descrizione}
            \item \textbf{Attore Primario}
            \item \textbf{Attore Secondario}
            \item \textbf{Precondizione}
            \item \textbf{Input}
            \item \textbf{Postcondizione}
            \item \textbf{Scenario Principale}
            \item \textbf{Inclusioni}
        \end{itemize}
        \subsection{UC7: Contatta il venditore}
        \subsection{UC8: Prima registrazione}
        \subsection{UC9: Scelta della categoria}
        \subsection{UC10: Reimposta password (dimenticata)}
        \subsection{UC11: Gestione credenziali}
        \begin{itemize}
            \item \textbf{Descrizione}
            \item \textbf{Attore Primario}
            \item \textbf{Attore Secondario}
            \item \textbf{Precondizione}
            \item \textbf{Input}
            \item \textbf{Postcondizione}
            \item \textbf{Scenario Principale}
            \item \textbf{Estensioni}
            \item \textbf{Generalizzazioni}
            \item \textbf{Inclusioni}
        \end{itemize}

        
        \subsection{UC12: Logout}
        \begin{itemize}
            \item \textbf{Descrizione:} sezione per effettuare il logout;
            \item \textbf{Attore Primario:} utente autenticato;
            \item \textbf{Precondizione:} l'utente (cliente o venditore) si trova in una qualsiasi pagina della piattaforma;
            \item \textbf{Input:} l'utente clicca il bottone apposito per effettuare il logout;
            \item \textbf{Postcondizione:} l'utente diventa non autenticato, uscendo dalla sessione;
        \end{itemize}


        \subsection{UC13: Visualizza gli ordini effettuati}
        \label{sec:UC13}
        \begin{itemize}
            \item \textbf{Descrizione:} sezione per visualizzare l'elenco degli ordini effettuati;
            \item \textbf{Attore Primario:} cliente autenticato;
            \item \textbf{Precondizione:} il cliente si trova in una qualsiasi pagina della piattaforma;
            \item \textbf{Input:} il cliente clicca il bottone apposito per entrare il questa sezione;
            \item \textbf{Postcondizione:} il cliente visualizza l'elenco degli ordini che ha effettuato;
        \end{itemize}


        \subsection{UC14: Effettua un reso}
        \begin{itemize}
            \item \textbf{Descrizione:} sezione per effettuare il reso di un ordine;
            \item \textbf{Attore Primario:} cliente autenticato;
            \item \textbf{Precondizione:} il cliente si trova nella pagina per la visualizzazione dell'elenco degli ordini (\hyperref[sec:UC13]{UC13});
            \item \textbf{Input:} il cliente clicca il bottone per entrare in questa sezione;
            \item \textbf{Postcondizione:} la richiesta di reso è andata a buon fine e i prodotti dell'ordine torneranno al venditore;
            \item \textbf{Scenario Principale:}
                \begin{itemize}
                    \item il cliente si trova nella sezione per visualizzare gli ordini effettuati;
                    \item clicca il bottone per eseguire il reso di un ordine;
                    \item il cliente conferma e il reso è confermato.
                \end{itemize}
        \end{itemize}


        \subsection{UC15: Gestione del carrello}
        \begin{figure}[!ht]
            \caption{Diagramma di UC15: Gestione del carrello}
            \vspace{10px}
            \includegraphics[scale=0.5]{../../../Images/AnalisiRequisiti/UC15}
            \centering
        \end{figure}
        \begin{itemize}
            \item \textbf{Descrizione:} questa sezione tratta l'insieme di operazioni che un utente generico fa per gestire i prodotti nel carrello;
            \item \textbf{Attore Primario:} utente generico;
            \item \textbf{Precondizione:} l'utente si trova in una pagina della piattaforma;
            \item \textbf{Input:} l'utente clicca il bottone per entrare nel carrello;
            \item \textbf{Postcondizione:} l'utente ha completato le operazioni per gestire i prodotti nel carrello;
            \item \textbf{Scenario Principale:}
                \begin{itemize}
                    \item l'utente si trova in una qualsiasi pagina della piattaforma;
                    \item clicca il bottone per entrare nel carrello;
                    \item una volta entrato vede la lista di tutti i prodotti all'interno del carrello;
                    \item può decidere se rimuovere qualche prodotto o modificarne le quantità.
                \end{itemize}
        \end{itemize}
        \subsubsection{UC15.1: Visualizzazione dei prodotti inseriti}
        \label{sec:UC15.1}
        \begin{itemize}
            \item \textbf{Descrizione:} sezione per visualizzare tutti i prodotti inseriti nel carrello;
            \item \textbf{Attore Primario:} utente generico;
            \item \textbf{Precondizione:}  l'utente si trova in una pagina della piattaforma;
            \item \textbf{Input:} l'utente clicca il bottone per entrare nel carrello;
            \item \textbf{Postcondizione:} l'utente si trova sulla pagina principale del carrello e vede tutti i prodotti presenti al suo interno;
            \item \textbf{Scenario Principale:}
                \begin{itemize}
                    \item l'utente si trova in una pagina della piattaforma;
                    \item l'utente clicca il bottone per entrare nel carrello;
                    \item una volta entrato l'elenco dei prodotti è subito visibile.
                \end{itemize}
        \end{itemize}
        \subsubsection{UC15.2: Rimozione di uno o più prodotti}
        \begin{itemize}
            \item \textbf{Descrizione:} sezione per rimuovere uno o più prodotti dal carrello;
            \item \textbf{Attore Primario:} utente generico;
            \item \textbf{Precondizione:} l'utente è nella pagina del carrello (\hyperref[sec:UC15.1]{UC15.1});
            \item \textbf{Input:} l'utente seleziona i prodotti da rimuovere dal carrello;
            \item \textbf{Postcondizione:} vengono rimossi i prodotti selezionati;
        \end{itemize}
        \subsubsection{UC15.3: Modifica della quantità di uno o più prodotti}
        \begin{itemize}
            \item \textbf{Descrizione:} sezione per modificare le quantità dei prodotti nel carrello;
            \item \textbf{Attore Primario:} utente generico;
            \item \textbf{Precondizione:} l'utente è nella pagina del carrello (\hyperref[sec:UC15.1]{UC15.1});
            \item \textbf{Input:} l'utente inserisce le modifiche desiderate ai prodotti;
            \item \textbf{Postcondizione:} le modifiche sono applicate;
            \item \textbf{Estensioni:} 
                \begin{itemize}
                    \item se la nuova quantità non dovesse essere disponibile per il venditore, l'utente viene avvisato.
                \end{itemize}
        \end{itemize}

        
        \subsection{UC16: Checkout}
            \begin{figure}[!ht]
                \caption{Diagramma di UC16: Checkout}
                \vspace{10px}
                \includegraphics[scale=0.5]{../../../Images/AnalisiRequisiti/UC16}
                \centering
            \end{figure}
                \begin{itemize}
                \item \textbf{Descrizione:} caso d'uso per la creazione di un nuovo ordine e l'acquisto dei prodotti inseriti nel carrello;
                \item \textbf{Attore Primario:} cliente autenticato;
                \item \textbf{Attore Secondario:} Stripe, gestore di pagamenti di terze parti;
                \item \textbf{Precondizione:} il cliente si trova sul carrello e ha già inserito almeno un prodotto;
                \item \textbf{Input:} il cliente clicca il bottone per iniziare il checkout;
                \item \textbf{Postcondizione:} l'ordine viene emesso e aggiunto alla lista degli ordini di quel cliente; i prodotti acquistati vengono rimossi dal carrello e viene diminuita la quantità dal deposito del venditore;
                \item \textbf{Scenario Principale:} 
                    \begin{itemize}
                        \item il cliente preme sul pulsante per entrare nella sezione del checkout;
                        \item il cliente inserisce i dati di fatturazione (\hyperref[sec:UC16.1]{UC16.1}) e, se diversi, i dati di spedizione (\hyperref[sec:UC16.2]{UC16.2});
                        \item vengono inseriti eventuali costi di spedizione;
                        \item il cliente viene reindirizzato al servizio di pagamento esterno, dove inserisce i dati di pagamento;
                        \item l'ordine è emesso e segnato come completato.
                    \end{itemize}
                \item \textbf{Estensioni:}
                    \begin{itemize}
                        \item se il cliente decide di non completare il checkout, può uscire dalla pagina senza causare modifiche al carrello;
                        \item se il pagamento ha avuto esito negativo, l'ordine non viene emesso e quindi è necessario ricominciare il pagamento.
                    \end{itemize}
            \end{itemize}
            \subsubsection{UC16.1: Inserimento dell'indirizzo di fatturazione}
            \label{sec:UC16.1}
                \begin{itemize}
                    \item \textbf{Descrizione:} Sezione per l'inserimento dell'indirizzo fatturazione;
                    \item \textbf{Attore Primario:} cliente autenticato;
                    \item \textbf{Precondizione:} il cliente si trova nella fase di checkout;
                    \item \textbf{Input:} il cliente inserisce i dati richiesti dalla pagina;
                    \item \textbf{Postcondizione:} si procede con la fase successiva del checkout (\hyperref[sec:UC16.2]{UC16.2})
                    \item \textbf{Scenario Principale:} il cliente inserisce negli appositi spazi i dati per completare l'indirizzo di fatturazione.
                \end{itemize}
            \subsubsection{UC16.2: Inserimento dell'indirizzo di spedizione}
            \label{sec:UC16.2}
                \begin{itemize}
                    \item \textbf{Descrizione:} Sezione per l'inserimento dell'indirizzo spedizione;
                    \item \textbf{Attore Primario:} cliente autenticato;
                    \item \textbf{Precondizione:} il cliente si trova nella fase di checkout;
                    \item \textbf{Input:} il cliente inserisce i dati richiesti dalla pagina, oppure clicca un pulsante se l'indirizzo di spedizione è lo stesso di quello di fatturazione;
                    \item \textbf{Postcondizione:} si procede con la fase successiva del checkout (\hyperref[sec:UC16.3]{UC16.3})
                    \item \textbf{Scenario Principale:} il cliente inserisce negli appositi spazi i dati per completare l'indirizzo di spedizione oppure clicca il pulsante per autocompletarli se è il medesimo di quello di fatturazione.
                \end{itemize}
            \subsubsection{UC16.3: Reindirizzamento al servizio di pagamento esterno}
            \label{sec:UC16.3}
                \begin{itemize}
                    \item \textbf{Descrizione:} Sezione per il pagamento dell'ordine da effettuare;
                    \item \textbf{Attore Primario:} cliente autenticato;
                    \item \textbf{Attore Secondario:} Stripe, gestore di pagamenti di terze parti;
                    \item \textbf{Precondizione:} il cliente ha inserito i dati per la fatturazione e la spedizione nella sezione del checkout;
                    \item \textbf{Input:} il cliente preme il pulsante per il pagamento;
                    \item \textbf{Postcondizione:} il pagamento ha avuto esito positivo, l'ordine viene confermato e viene decrementata la quantità dei prodotti disponibili al venditore, in base al numero di prodotti acquistati dal cliente;
                    \item \textbf{Scenario Principale:}
                    \begin{itemize}
                        \item il cliente preme sul pulsante e viene reindirizzato al servizio esterno per eseguire il pagamento;
                        \item il cliente inserisce i propri dati ed esegue il pagamento;
                        \item il pagamento è riuscito e l'ordine viene confermato.
                    \end{itemize}
                    \item \textbf{Estensioni:}
                    \begin{itemize}
                        \item il pagamento non è riuscito, viene visualizzato un errore di pagamento (\hyperref[sec:UC16.4]{UC16.4}) e si viene reindirizzati alla pagina del checkout per riprovare.
                    \end{itemize}
                \end{itemize}
            \subsubsection{UC16.4: Errore di pagamento}
            \label{sec:UC16.4}
                \begin{itemize}
                    \item \textbf{Descrizione:} visualizzazione di un errore per un fallimento nella fase di pagamento;
                    \item \textbf{Attore Primario:} Stripe
                    \item \textbf{Precondizione:} il cliente ha inserito i dati del pagamento;
                    \item \textbf{Input:} Stripe ritorna un errore nel risultato del pagamento;
                    \item \textbf{Postcondizione:} viene visualizzato un messaggio di errore, successivamente il cliente viene reindirizzato alla sezione del checkout;
                \end{itemize}


        \subsection{UC17: Visualizzazione dei resi effettuati}
            \begin{itemize}
                \item \textbf{Descrizione:} sezione per visualizzare l'elenco dei resi effettuati;
                \item \textbf{Attore Primario:} cliente autenticato;
                \item \textbf{Precondizione:} l'utente si trova nella sezione degli ordini effettuati (\hyperref[sec:UC18]{UC18})
                \item \textbf{Input:} l'utente clicca il bottone apposito per entrare il questa sezione;
                \item \textbf{Postcondizione:} l'utente visualizza l'elenco dei resi che ha effettuato;
            \end{itemize}


        \subsection{UC18: Visualizzazione della lista degli ordini}
        \label{sec:UC18}
        \begin{itemize}
            \item \textbf{Descrizione:} sezione per visualizzare l'elenco degli ordini effettuati dai clienti;
            \item \textbf{Attore Primario:} venditore autenticato; 
            \item \textbf{Precondizione:} il venditore si trova in una qualsiasi pagina;
            \item \textbf{Input:} il venditore clicca il bottone per entrare in questa sezione; 
            \item \textbf{Postcondizione:} il venditore visualizza la lista di tutti gli ordini effettuati dai suoi clienti.
        \end{itemize}


        \subsection{UC19: Visualizzazione della lista dei clienti}
        \label{sec:UC19}
            \begin{itemize}
                \item \textbf{Descrizione:} sezione per visualizzare l'elenco dei clienti del proprio negozio;
                \item \textbf{Attore Primario:} venditore autenticato; 
                \item \textbf{Precondizione:} il venditore si trova in una qualsiasi pagina;
                \item \textbf{Input:} il venditore clicca il bottone per entrare in questa sezione; 
                \item \textbf{Postcondizione:} il venditore visualizza la lista di tutti i clienti del negozio.
                \end{itemize}


        \subsection{UC20: Contatta un cliente}
        \begin{itemize}
            \item \textbf{Descrizione:}
            \item \textbf{Attore Primario:} venditore autenticato;
            \item \textbf{Precondizione:} il venditore si trova nella lista dei suoi clienti (\hyperref[sec:UC19]{UC19});
            \item \textbf{Input} il venditore clicca il bottone per entrare in questa sezione; 
            \item \textbf{Postcondizione:} il venditore ha inviato una e-mail ad un suo cliente;
            \item \textbf{Scenario Principale}
                \begin{itemize}
                    \item il venditore si trova nella lista dei suoi clienti;
                    \item sceglie uno dei suoi clienti per poterlo contattare;
                    \item compila il campo di testo scrivendo ciò che deve inviare;
                    \item preme il pulsante per inviare, una e-mail verrà inviata al cliente;
                \end{itemize}
        \end{itemize}


        \subsection{UC21: Gestione del negozio}
        \label{sec:UC21}
        \begin{figure}[!ht]
            \caption{Diagramma di UC21: Gestione del negozio}
            \vspace{10px}
            \includegraphics[scale=0.5]{../../../Images/AnalisiRequisiti/UC21}
            \centering
        \end{figure}
        \begin{itemize}
            \item \textbf{Descrizione:} sezione per il venditore, dove può gestire le parti relative al negozio;
            \item \textbf{Attore Primario:} venditore autenticato;
            \item \textbf{Precondizione:} il venditore si trova in una qualsiasi pagina;
            \item \textbf{Input:} il venditore preme il bottone per entrare in questa sezione;
            \item \textbf{Postcondizione:} il venditore ha completato le modifiche desiderate;
            \item \textbf{Scenario Principale:} 
                \begin{itemize}
                    \item il venditore si trova in una pagina della piattaforma e preme il bottone per entrare in questa sezione;
                    \item può gestire i prodotti presenti nel negozio;
                    \item può gestire le categorie per permettere ai clienti di individuare meglio i prodotti;
                    \item una volta terminato applica le modifiche.
                \end{itemize}
        \end{itemize}
        \subsubsection{UC21.1: Gestione dei prodotti}
        \label{sec:UC21.1}
        \begin{figure}[!ht]
            \caption{Diagramma di UC21.1: Gestione dei prodotti}
            \vspace{10px}
            \includegraphics[scale=0.5]{../../../Images/AnalisiRequisiti/UC21.1}
            \centering
        \end{figure}
        \begin{itemize}
            \item \textbf{Descrizione:} sezione per il venditore, dove può gestire i prodotti del negozio;
            \item \textbf{Attore Primario:} venditore autenticato;
            \item \textbf{Precondizione:} il venditore si trova nella pagina per la gestione del negozio (\hyperref[sec:UC21]{UC21});
            \item \textbf{Input:} il venditore preme il bottone per entrare in questa sezione;
            \item \textbf{Postcondizione:} il venditore ha completato le modifiche ai prodotti desiderati;
            \item \textbf{Scenario Principale:} 
                \begin{itemize}
                    \item il venditore si trova nella pagina per la gestione del negozio;
                    \item può decidere se aggiungere un nuovo prodotto, modificarne o rimuoverne uno già presente;
                    \item una volta terminato applica le modifiche
                \end{itemize}
        \end{itemize}
        \subsubsubsection{UC21.1.1: Aggiunta di un prodotto}
        \begin{itemize}
            \item \textbf{Descrizione:} sezione per aggiungere un prodotto nel negozio;
            \item \textbf{Attore Primario:} venditore autenticato;
            \item \textbf{Precondizione:} il venditore si trova nella sezione per gestire i prodotti (\hyperref[sec:UC21.1]{UC21.1});
            \item \textbf{Input:} il venditore inserisce tutti i dati relativi al prodotto da inserire;
            \item \textbf{Postcondizione:} il prodotto è aggiunto e i clienti possono trovarlo nel negozio;
            \item \textbf{Scenario Principale:} 
                \begin{itemize}
                    \item il venditore si trova nella sezione per gestire i prodotti;
                    \item il venditore decide di aggiungere un nuovo prodotto ed entra in questa sezione;
                    \item inserisce tutti i dati richiesti (nome, descrizione, prezzo ecc.);
                    \item conferma l'aggiunta e il prodotto è disponibile nel negozio.
                \end{itemize}
        \end{itemize}
        \subsubsubsection{UC21.1.2: Modifica di un prodotto}
        \begin{itemize}
            \item \textbf{Descrizione:} sezione per modificare un prodotto del negozio;
            \item \textbf{Attore Primario:} venditore autenticato;
            \item \textbf{Precondizione:} il venditore si trova nella sezione per gestire i prodotti (\hyperref[sec:UC21.1]{UC21.1});
            \item \textbf{Input:} il venditore modifica i dati che ritiene opportuni relativi al prodotto da modificare;
            \item \textbf{Postcondizione:} il prodotto è stato modificato;
            \item \textbf{Scenario Principale:} 
                \begin{itemize}
                    \item il venditore si trova nella sezione per gestire i prodotti;
                    \item il venditore decide di modificare un prodotto ed entra in questa sezione;
                    \item modifica i dati che preferisce;
                    \item conferma le modifiche.
                \end{itemize}
        \end{itemize}
        \subsubsubsection{UC21.1.3: Rimozione di un prodotto}
        \begin{itemize}
            \item \textbf{Descrizione:} sezione per rimuovere un prodotto dal negozio;
            \item \textbf{Attore Primario:} venditore autenticato;
            \item \textbf{Precondizione:} il venditore si trova nella sezione per gestire i prodotti (\hyperref[sec:UC21.1]{UC21.1});
            \item \textbf{Input:} il venditore sceglie il prodotto da eliminare;
            \item \textbf{Postcondizione:} il prodotto è stato rimosso dal negozio;
            \item \textbf{Scenario Principale:} 
                \begin{itemize}
                    \item il venditore si trova nella sezione per gestire i prodotti;
                    \item il venditore decide di rimuovere un prodotto dal negozio;
                    \item conferma;
                    \item il prodotto non è più visibile nel negozio.
                \end{itemize}
        \end{itemize}


        \subsubsection{UC21.2: Gestione delle categorie}
        \label{sec:UC21.2}
        \begin{figure}[!ht]
            \caption{Diagramma di UC21.2: Gestione delle categorie}
            \vspace{10px}
            \includegraphics[scale=0.5]{../../../Images/AnalisiRequisiti/UC21.2}
            \centering
        \end{figure}
        \begin{itemize}
            \item \textbf{Descrizione:} sezione per il venditore, dove può gestire le categorie del negozio;
            \item \textbf{Attore Primario:} venditore autenticato;
            \item \textbf{Precondizione:} il venditore si trova nella pagina per la gestione del negozio (\hyperref[sec:UC21]{UC21});
            \item \textbf{Input:} il venditore preme il bottone per entrare in questa sezione;
            \item \textbf{Postcondizione:} il venditore ha completato le operazioni sulle categorie;
            \item \textbf{Scenario Principale:} 
                \begin{itemize}
                    \item il venditore si trova nella pagina per la gestione del negozio;
                    \item può decidere se aggiungere una nuova categoria o se rimuoverne una già presente;
                    \item una volta terminato applica le modifiche.
                \end{itemize}
        \end{itemize}
        \subsubsubsection{UC21.2.1: Aggiunta di una categoria}
        \begin{itemize}
            \item \textbf{Descrizione:} sezione per aggiungere una categoria di prodotti nel negozio;
            \item \textbf{Attore Primario:} venditore autenticato;
            \item \textbf{Precondizione:} il venditore si trova nella sezione per gestire le categorie (\hyperref[sec:UC21.2]{UC21.2});
            \item \textbf{Input:} il venditore inserisce tutti i dati relativi alla categoria da inserire;
            \item \textbf{Postcondizione:} la categoria è aggiunta e i clienti possono trovarlo nel negozio;
            \item \textbf{Scenario Principale:} 
                \begin{itemize}
                    \item il venditore si trova nella sezione per gestire le categorie;
                    \item il venditore decide di aggiungere una nuova categoria ed entra in questa sezione;
                    \item inserisce tutti i dati richiesti;
                    \item conferma l'aggiunta e la categoria è visibile nel negozio.
                \end{itemize}
        \end{itemize}
        \subsubsubsection{UC21.2.2: Rimozione di una categoria}
        \begin{itemize}
            \item \textbf{Descrizione:} sezione per rimuovere una categoria dal negozio;
            \item \textbf{Attore Primario:} venditore autenticato;
            \item \textbf{Precondizione:} il venditore si trova nella sezione per gestire le categorie (\hyperref[sec:UC21.2]{UC21.2});
            \item \textbf{Input:} il venditore sceglie la categoria da eliminare;
            \item \textbf{Postcondizione:} la categoria è stata rimossa dal negozio;
            \item \textbf{Scenario Principale:} 
                \begin{itemize}
                    \item il venditore si trova nella sezione per gestire le categorie;
                    \item il venditore decide di rimuovere una categoria dal negozio;
                    \item conferma;
                    \item la categoria non è più visibile nel negozio.
                \end{itemize}
        \end{itemize}

