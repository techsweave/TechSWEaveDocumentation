\section{Introduzione}
    \subsection{Scopo del documento}
    Il documento ha lo scopo di descrivere in maniera dettagliata i requisiti individuati per il prodotto. Tali requisiti sono stati identificati dall'analisi del capitolato\textsubscript{\textbf{G}} \textbf{C2} e degli incontri con il proponente \textit{Red Babel}.
    \subsection{Scopo del prodotto}
    Lo scopo del prodotto è lo sviluppo di una piattaforma e-commerce basata sulle tecnologie \textit{server-less}\textsubscript{\textbf{G}}. L'interfaccia web sarà sviluppata con il framework\textsubscript{\textbf{G}} Next.js\textsubscript{\textbf{G}}, la parte di back-end\textsubscript{\textbf{G}} sarà sviluppata attraverso la piattaforma AWSLambda\textsubscript{\textbf{G}}, con l'utilizzo del Server-less framework\textsubscript{\textbf{G}}. Il linguaggio utilizzato per lo sviluppo dell'intera piattaforma è il Typescript\textsubscript{\textbf{G}}. Il pagamento sarà interamente gestito da un applicazione terza, Stripe\textsubscript{\textbf{G}}.
    \subsection{Glossario}
    Con l'intento di evitare possibili ambiguità dovute al linguaggio utilizzato nei documenti formali, viene fornito il \textit{Glossario v1.0.0}. In questo documento vengono definiti e descritti tutti i termini con un significato particolare. Per facilitarne l'individuazione, i termini saranno contrassegnati da una '\textbf{G}' a pedice.
    \subsection{Riferimenti}
    \subsubsection{Normativi}
    \begin{itemize}
        \item \textit{Norme di progetto v1.0.0};
        \item \textbf{Capitolato d'appalto C2 - EmporioLambda}:\\ \href{https://www.math.unipd.it/~tullio/IS-1/2020/Progetto/C2.pdf}{https://www.math.unipd.it/~tullio/IS-1/2020/Progetto/C2.pdf};
        \item \textit{Verbale esterno 2021-03-17}
        \item \textit{Verbale esterno 2021-03-25}
    \end{itemize}
    \subsubsection{Informativi}
    \begin{itemize}
        \item \textit{Studio di fattibilità v1.0.0};
        \item \textbf{Capitolato d'appalto C2 - EmporioLambda}:\\ \href{https://www.math.unipd.it/~tullio/IS-1/2020/Progetto/C2.pdf}{https://www.math.unipd.it/~tullio/IS-1/2020/Progetto/C2.pdf};
        \item \textbf{Slide del corso Ingegneria del Software, A.A. 2020/21}: \\ \href{https://www.math.unipd.it/~tullio/IS-1/2020/Dispense/L07.pdf}{Analisi dei Requisiti}; \\ \href{https://www.math.unipd.it/~rcardin/swea/2021/Diagrammi%20Use%20Case_4x4.pdf}{Diagrammi dei casi d'uso};
        \item \textbf{Software Engineering - Ian Sommerville - 10th Edition 2014}\\
        Chapter 4: Requirements engineering;
        \item \textbf{Sito informativo sull'utilizzo delle tecnologie server-less}:\\ \href{https://aws.amazon.com/training/ramp-up-guides/}{https://aws.amazon.com/training/ramp-up-guides/};
        \item \textbf{Sito informativo sull'utilizzo e l'implementazione di Amazon Cognito}:\\ \href{https://docs.aws.amazon.com/cognito/latest/developerguide/cognito-user-identity-pools.html}{https://docs.aws.amazon.com/cognito/latest/developerguide/cognito-user-identity-pools.html}.
        %Da aggiornare con i siti che useremo per comprendere al meglio le tecnologie usate
    \end{itemize}