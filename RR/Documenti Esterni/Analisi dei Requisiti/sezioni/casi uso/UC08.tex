\subsection{UC6: Visualizzazione della lista dei prodotti}
        \label{sec:UC8}
        \begin{figure}[!ht]
            \caption{Diagramma di UC6: Visualizzazione della lista dei prodotti}
            \vspace{10px}
            \includegraphics[scale=0.5]{../../../Images/AnalisiRequisiti/UC6.png}
            \centering
        \end{figure}
        \begin{itemize}
            \item \textbf{Descrizione:} l'utente visualizza la pagina della lista dei prodotti;
            \item \textbf{Attore Primario:} utente generico;
            \item \textbf{Precondizione:} l'utente si trova in una qualsiasi pagina della piattaforma;
            \item \textbf{Postcondizione:} l'utente visualizza la pagina con la lista dei prodotti;
            \item \textbf{Scenario Principale:}
            \begin{itemize}
                \item l'utente è in una pagina della piattaforma;
                \item cerca un prodotto o sceglie una categoria;
                \item l'utente visualizza la lista dei prodott.
            \end{itemize}
        \end{itemize}
        \subsubsection{UC8.1: Ordinamento prodotti}
        \label{sec:UC8.1}
        \begin{itemize}
            \item \textbf{Descrizione:} l'utente seleziona la modalità in cui visualizzare i file;
            \item \textbf{Attore Primario:} utente generico;
            \item \textbf{Precondizione:} l'utente si trova all'interno della pagina con la lista dei prodotti(\hyperref[sec:UC6]{\underline{UC6}});
            \item \textbf{Input:} selezione del parametro scelto;
            \item \textbf{Postcondizione:} la lista dei prodotti si aggiorna secondo l'ordinamento scelto;
            \item \textbf{Scenario Principale:}
            \begin{itemize}
                \item l'utente si trova all'interno della pagina con la lista dei prodotti;
                \item inserisce i parametri per ordinare i prodotti;
                \item la pagina con la lista dei prodotti si aggiorna con i prodotti ordinati.
            \end{itemize}
        \end{itemize}
        \subsubsection{UC8.2: Applicazione filtri}
        \label{sec:UC8.2}
        \begin{itemize}
            \item \textbf{Descrizione:} l'utente filtra i prodotti in base alle loro caratteristiche;
            \item \textbf{Attore Primarilo:} utente generico
            \item \textbf{Precondizione:} l'utente si trova all'interno della pagina con la lista dei prodotti (\hyperref[sec:UC6]{\underline{UC6}});
            \item \textbf{Input:} selezione del filtro desiderato;
            \item \textbf{Postcondizione:} la pagina si aggiorna secondo i filtri selezionati;
            \item \textbf{Scenario Principale:}
            \begin{itemize}
                \item l'utente si trova all'interno della pagina con la lista dei prodotti;
                \item inserisce i parametri per filtrare i prodotti;
                \item la pagina con la lista dei prodotti si aggiorna con i prodotti filtrati.
            \end{itemize}
        \end{itemize}