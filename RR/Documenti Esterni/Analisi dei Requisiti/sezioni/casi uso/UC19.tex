\newpage
\subsection{UC19: Checkout}
\label{sec:UC19}
\begin{figure}[!ht]
    \caption{Diagramma di UC19: Checkout}
    \vspace{10px}
    \includegraphics[scale=0.5]{../../../Images/AnalisiRequisiti/UC19}
    \centering
\end{figure}
\begin{itemize}
    \item \textbf{Descrizione:} caso d'uso per la creazione di un nuovo ordine e l'acquisto dei prodotti;
    \item \textbf{Attore Primario:} cliente autenticato;
    \item \textbf{Attore Secondario:} Stripe\textsubscript{\textbf{G}}, gestore di pagamenti di terze parti;
    \item \textbf{Precondizione:} il cliente si trova nel carrello (\hyperref[sec:UC12]{\underline{UC12}}) e ha già inserito almeno un prodotto oppure si trova nella pagina del prodotto (\hyperref[sec:UC5]{\underline{UC5}});
    \item \textbf{Input:} il cliente clicca il bottone per iniziare il checkout;
    \item \textbf{Postcondizione:} l'ordine viene emesso e aggiunto alla lista degli ordini di quel cliente; i prodotti acquistati vengono rimossi dal carrello e viene diminuita la rispettiva quantità dal deposito del venditore; il cliente viene reindirizzato alla pagina di riepilogo dell'ordine (\hyperref[sec:UC21]{\underline{UC21}})
    \item \textbf{Scenario Principale:}
          \begin{itemize}
              \item il cliente clicca il bottone per effettuare il checkout;
              \item il cliente inserisce i dati di fatturazione (\hyperref[sec:UC19.1]{\underline{UC19.1}}) e, se diversi, i dati di spedizione (\hyperref[sec:UC19.2]{\underline{UC19.2}});
              \item vengono inseriti eventuali costi di spedizione;
              \item il cliente viene reindirizzato al servizio di pagamento esterno, dove inserisce i dati di pagamento;
              \item l'ordine è emesso e segnato come completato.
          \end{itemize}
    \item \textbf{Estensioni:}
    \item Il cliente decide di non completare il checkout:
          \begin{itemize}
              \item  il cliente esce dalla pagina senza causare modifiche al carrello;
          \end{itemize}
\end{itemize}
\subsubsection{UC19.1: Inserimento dell'indirizzo di fatturazione}
\label{sec:UC19.1}
\begin{itemize}
    \item \textbf{Descrizione:} Sezione per l'inserimento dell'indirizzo fatturazione;
    \item \textbf{Attore Primario:} cliente autenticato;
    \item \textbf{Precondizione:} il cliente si trova nella fase di checkout;
    \item \textbf{Input:} il cliente inserisce i dati richiesti dalla pagina;
    \item \textbf{Postcondizione:} si procede con la fase successiva del checkout (\hyperref[sec:UC19.2]{\underline{UC19.2}})
    \item \textbf{Scenario Principale:} il cliente inserisce negli appositi spazi i dati per completare l'indirizzo di fatturazione.
\end{itemize}
\subsubsection{UC19.2: Inserimento dell'indirizzo di spedizione}
\label{sec:UC19.2}
\begin{itemize}
    \item \textbf{Descrizione:} Sezione per l'inserimento dell'indirizzo spedizione;
    \item \textbf{Attore Primario:} cliente autenticato;
    \item \textbf{Precondizione:} il cliente si trova nella fase di checkout;
    \item \textbf{Input:} il cliente inserisce i dati richiesti oppure clicca un apposito bottone se l'indirizzo di spedizione è lo stesso di quello di fatturazione;
    \item \textbf{Postcondizione:} si procede con la fase successiva del checkout (\hyperref[sec:UC19.3]{\underline{UC19.3}})
    \item \textbf{Scenario Principale:} il cliente inserisce negli appositi spazi i dati per completare l'indirizzo di spedizione oppure clicca il pulsante per autocompletarli se è il medesimo di quello di fatturazione.
\end{itemize}
\subsubsection{UC19.3: Reindirizzamento al servizio di pagamento esterno}
\label{sec:UC19.3}
\begin{itemize}
    \item \textbf{Descrizione:} sezione per il pagamento dell'ordine da effettuare;
    \item \textbf{Attore Primario:} cliente autenticato;
    \item \textbf{Attore Secondario:} Stripe, gestore di pagamenti di terze parti;
    \item \textbf{Precondizione:} il cliente ha inserito i dati per la fatturazione e la spedizione nella sezione di checkout;
    \item \textbf{Input:} il cliente preme il pulsante per il pagamento;
    \item \textbf{Postcondizione:} il pagamento ha avuto esito positivo, l'ordine viene confermato e viene decrementata la rispettiva quantità dei prodotti disponibili del venditore, in base al numero di prodotti acquistati dal cliente;
    \item \textbf{Scenario Principale:}
          \begin{itemize}
              \item il cliente preme sul pulsante e viene reindirizzato al servizio esterno per eseguire il pagamento;
              \item il cliente inserisce i propri dati ed esegue il pagamento;
              \item il pagamento è riuscito e l'ordine viene confermato tramite la visualizzazione del riepilogo dell'ordine (\hyperref[sec:UC21]{\underline{UC21}}).
          \end{itemize}
    \item \textbf{Estensioni:}\\
          Il pagamento non è andato a buon fine:
          \begin{itemize}
              \item viene visualizzato un errore di pagamento (\hyperref[sec:UC19.4]{\underline{UC19.4}});
              \item il cliente viene reindirizzato alla pagina di checkout per riprovare l'acquisto.
          \end{itemize}
\end{itemize}
\subsubsection{UC19.4: Errore di pagamento}
\label{sec:UC19.4}
\begin{itemize}
    \item \textbf{Descrizione:} visualizzazione di un errore per un fallimento nella fase di pagamento;
    \item \textbf{Attore Primario:} Stripe;
    \item \textbf{Precondizione:} il cliente ha inserito i dati del pagamento;
    \item \textbf{Input:} Stripe ritorna un errore nel risultato del pagamento;
    \item \textbf{Postcondizione:} viene visualizzato un messaggio di errore, successivamente il cliente viene reindirizzato alla pagina di checkout.
\end{itemize}