\subsection{UC3: Autenticazione standard}
\begin{figure}[!ht]
    \caption{Diagramma di UC3: Autenticazione standard}
    \vspace{10px}
    \includegraphics[scale=0.5]{../../../Images/AnalisiRequisiti/UC03}
    \centering
\end{figure}
\label{sec:UC3}
\begin{itemize}
    \item \textbf{Descrizione:} permette l'autenticazione di un utente;
    \item \textbf{Attore Primario:} utente generico;
    \item \textbf{Attore Secondario:} servizio di autenticazione esterno;
    \item \textbf{Precondizione:} l'utente è all'interno della pagina di autenticazione
    \item \textbf{Input:} inserimento dati;
    \item \textbf{Postcondizione:} l'utente è autenticato;
    \item \textbf{Scenario Principale:} 
    \begin{itemize}
        \item l'utente è nella pagina dedicata all'autenticazione;
        \item inserisce i propri dati;
        \item preme l'apposito bottone;
        \item l'utente è autenticato.
    \end{itemize}
    \textbf{Estensione:}
        Se la l'username o la password non sono corretti:
        \begin{itemize}
            \item viene visualizzato un messaggio di errore (\hyperref[sec:UC28]{\underline{UC28}});
            \item si può ritentare l'inserimento.
        \end{itemize}
\end{itemize}
\subsubsection{UC3.1: Inseriemtno username}
\begin{itemize}
    \item \textbf{Descrizione:} permette l'inserimento dell'username;
    \item \textbf{Attore Primario:} utente generico;
    \item \textbf{Attore Secondario:} servizio di autenticazione esterno;
    \item \textbf{Precondizione:} l'utente è all'interno della pagina di autenticazione
    \item \textbf{Input:} inserimento username;
    \item \textbf{Postcondizione:} l'username è stata inserita;
\end{itemize}
\subsubsection{UC3.2: Inseriemtno password}
\begin{itemize}
    \item \textbf{Descrizione:} permette l'inserimento della password;
    \item \textbf{Attore Primario:} utente generico;
    \item \textbf{Attore Secondario:} servizio di autenticazione esterno;
    \item \textbf{Precondizione:} l'utente è all'interno della pagina di autenticazione
    \item \textbf{Input:} inserimento password;
    \item \textbf{Postcondizione:} la password è stata inserita;
\end{itemize}
