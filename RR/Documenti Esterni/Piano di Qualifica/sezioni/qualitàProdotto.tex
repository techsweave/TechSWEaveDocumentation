\section{Qualità di prodotto}
Per valutare la qualità di prodotto il gruppo  \textit{TechSWEave} ha selezionato lo standard ISO/IEC 9126\textsubscript{\textbf{G}}.\\
Il modello di qualità scelto espone le caratteristiche di qualità del prodotto con una lista di attributi. Di seguito sono riportati quelli che il 
gruppo ha deciso di selezionare, omettendo quelli ritenuti meno importatanti nel contesto del progetto.
\subsection{Funzionalità}
Una funzionalità è la capacità di un prodotto software di offire gli strumenti necessari allo svolgimento di funzioni prestabilite, definite 
nell'\textit{Analisi dei requisiti v1.0.0.}
\subsubsection{Obiettivi}
\begin{itemize}
    \item \textbf{Appropriatezza:} capacità del prodotto software di fornire un insieme di funzioni in grado
    di soddisfare gli obiettttivi esposti nell’\textit{Analisi dei Requisiti 1.0.0};
    \item \textbf{Accuratezza:} capacità del prodotto software di fornire i risultati desiderati con la precisione richiesta;
    \item \textbf{Conformità:} il prodotto deve aderire a determinati standard.
\end{itemize}
\subsubsection{Metriche}
\textbf{Completezza dell'implementazione:} misura la completezza del prodotto in percentuale.
\begin{itemize}
    \item misurazione: valore percentuale: C = $1-\frac{N\textsubscript{FNI}}{N\textsubscript{FI}}*100$ \\
    \\dove N\textsubscript{FNI} indica il numero di funzionalità non implementate e N\textsubscript{FI} indica il numero di funzionalità 
    individuate dall'analisi;
    \item valore preferibile: 100\%;
    \item valore accettabile: 100\%.
\end{itemize}
\subsection{Affidabilità}
È la capacità del prodotto software di mantenere il livello di prestazioni
desiderato anche quando operante in un contesto e per un arco di tempo dati. Possibili
limitazioni all’affidabilità sono causate da errori nei requisiti, nella progettazione e nello
sviluppo del codice.
\subsubsection{Obiettivi}
\begin{itemize}
    \item \textbf{Maturità:} capacità del prodotto software di evitare errori e risultati non corretti durante l’esecuzione;
    \item \textbf{Tolleranza agli errori:} capacità del prodotto software di conservare il livello di prestazioni 
    anche in caso di malfunzionamenti o di uso inappropriato del prodotto.
\end{itemize}
\subsubsection{Metriche}
\textbf{Densità errori:} indica la capacità del prodotto di resistere a malfunzionamenti.
\begin{itemize}
    \item misurazione: valore percentuale: M = $\frac{N\textsubscript{ER}}{N\textsubscript{TE}}*100$ \\
    \\dove N\textsubscript{ER} indica il numero di errori rilevati e N\textsubscript{TE} indica il numero di test eseguiti;
    \item valore preferibile: 0\%;
    \item valore accettabile: $\leq$ 10\%.
\end{itemize}
\subsubsection{Usabilità}
È la capacità del prodotto software di essere compreso, appreso, usato e gradito dall’utente quando usato nel contesto per cui è stato progettato.
\subsubsection{Obiettivi}
\begin{itemize}
    \item \textbf{Comprensibilità:} l'utente deve poter comprendere facilmente i concetti base del prodotto software;
    \item \textbf{Apprendibilità:} l'utente deve poter imparare facilmente ad usare il software;
    \item \textbf{Attrattività:} il prodotto deve essere piacevole da usare.
\end{itemize}
\subsubsection{Metriche}
\textbf{Facilità di utilizzo:} indica la facilità con cui l'utente riesce ad ottenere l'informazione che sta cercando.
\begin{itemize}
    \item misurazione: numero di click necessari per arrivare alla pagina di checkout;
    \item valore preferibile: $\leq$ 10;
    \item valore accettabile: $\leq$ 15.
\end{itemize}
\textbf{Facilità di apprendimento:} indica la facilità con cui l'utente riesce ad imparare ad usare le funzionalità del prodotto.
\begin{itemize}
    \item misurazione: minuti necessari a raggiungere la pagina di checkout;
    \item valore preferibile: $\leq$ 3;
    \item valore accettabile: $\leq$ 5.
\end{itemize}
\textbf{Profondità della gerarchia:} indica la profondità del sito.
\begin{itemize}
    \item misurazione: livello di profondità delle pagine;
    \item valore preferibile: $\leq$ 4;
    \item valore accettabile: $\leq$ 7.
\end{itemize}
\subsection{Manutenibilità}
È la capacità del software di essere modificato mediante correzioni, miglioramenti o adattamenti.
\subsubsection{Obiettivi}
\begin{itemize}
    \item \textbf{Analizzabilità:} indica la difficoltà incontrata nel diagnosticare un errore nel prodotto;
    \item \textbf{Modificabilità:} indica la facilità di apportare modifiche al prodotto.
\end{itemize}
\subsubsection{Metriche}
\textbf{Facilità di comprensione:} la facilità con cui l'utente riesce a comprendere il codice può essere rappresentata dal numero di linee di 
commento nel codice.
\begin{itemize}
    \item misurazione: R = $\frac{N\textsubscript{LCOM}}{N\textsubscript{LCOD}}$ \\
    \\dove N\textsubscript{LCOM} indica il numero di linee di commento e N\textsubscript{LCOD} indica le linee di codice;
    \item valore preferibile: $\geq$ 0.20;
    \item valore accettabile: $\geq$ 0.10.
\end{itemize}
\textbf{Semplicità delle funzioni:} indica la semplicità di un metodo in base al numero di parametri passati allo stesso.
\begin{itemize}
    \item misurazione: numero di parametri per metodo;
    \item valore preferibile: $\leq$ 3;
    \item valore accettabile: $\leq$ 6.
\end{itemize}
\textbf{Semplicità delle classi:} indica la semplicità di una classe in base al numero di metodi della stessa.
\begin{itemize}
    \item misurazione: numero di metodi per classe;
    \item valore preferibile: $\leq$ 8;
    \item valore accettabile: $\leq$ 15.
\end{itemize}

