\section{Specifica dei test}
Per garantire qualità del prodotto è necessario stabilire delle metriche per l’esecuzione e per il soddisfacimento dei test. 
Il gruppo \textit{TechSWEave} ha optato per l'adozione del \textit{Modello a V}\textsubscript{\textbf{G}} di sviluppo del software, che 
prevede lo sviluppo dei test in parallelo alle attività di analisi e progettazione.\\
Per definire in modo conciso lo stato dei test, vengono utilizzate le seguenti diciture:
\begin{itemize}
    \item \textbf{I:} indica che il test è stato implementato;
    \item \textbf{NI:} indica che il test non è stato implementato;
    \item \textbf{S:} indica che il test soddisfa la richiesta;
    \item \textbf{NS:} indica che il test non soddisfa la richiesta.
\end{itemize}
\subsection{Tipi di test}
Ci sono quattro tipologie di test:
\begin{itemize}
    \item \textbf{Test di Accettazione [TA]:} eseguiti per verificare che il prodotto soddisfi le richieste del proponente;
    \item \textbf{Test di Sistema [TS]:} eseguiti quando il sistema viene installato per verificare che raggiunga gli obiettivi fissati;
    \item \textbf{Test di Integrazione [TI]:} eseguiti per verificare il funzionamento di un gruppo di moduli;
    \item \textbf{Test di Unit\`a [TU]:} eseguiti per verificare il funzionamento delle singole unit\`a del software.
\end{itemize}
\subsubsection{Test di Accettazione}
I test di accettazione vengono svolti per verificare che il programma soddisfi i requisiti, i quali sono divisi nelle seguenti categorie:
\begin{itemize}
    \item Obbligatorio (O)
    \item Desiderabile (D)
    \item Facoltativo (F)
\end{itemize}
E assumono le seguenti importanze:
\begin{itemize}
    \item Funzionale (F)
    \item Vincolo (V)
    \item Qualit\`a (Q)
    \item Prestazionale (P)
\end{itemize}
La seguente tabella illustra i test di accettazione da eseguire sul sistema, ogni test ha un nome che identifica tipologia, categoria e importanza e un numero che lo associa a un requisito,

\begin{center}
    \centering
    \rowcolors{2}{logo!10}{logo!40}
    \renewcommand{\arraystretch}{1.8}
    \label{tab:TestAccettazione}
    \begin{longtable}[!h]{p{50px} p{245px} p{75px} p{50px}}
        \caption{Test di accettazione} \\           
        \rowcolor{logo!70} \textbf{Test} & \textbf{Descrizione} & \textbf{Esito} \\
        TSOF1   & Visualizzazione homepage: all'utente viene chiesto di: \begin{itemize} \item accedere a barra di ricerca e menu \item accedere al carrello \end{itemize} & NI \\
        TSOF2   & Autenticazione: all'utente vien richiesto di visualizzare la pagina di autenticazione & NI \\
        TSOF3   & Autenticazione: all'utente viene richiesto di autenticarsi con username e password. & NI \\
        TSOF4   & Autenticazione: all'utente viene richiesto di autenticarsi tramite un servizio esterno & NI \\
        TSOF5   & Visualizzazione dettagli: all'utente viene richiesto di visualizzare i dettagli di un prodotto, nello specifico: \begin{itemize} \item Nome \item Foto \item Descrizione \item Prezzo\item IVA \item Disponibilit\`a \item Carrello \item Acquisto diretto \end{itemize} & NI \\
        TSOF6   & Aggiungi a carrello: all'utente viene richiesto di aggiungere un prodotto al carrello & NI \\
        TSOF7   & Ricerca: all'utente viene richiesto di cercare dei prodotti tramite parola chiave & NI \\
        TSOF8   & Lista: all'utente viene richiesto di visualizzare la lista dei prodotti e selezionarne pi\`u di uno & NI \\
        TSDF9   & Ordinamento lista: all'utente viene richiesto di: \begin{itemize} \item Ordinare i prodotti per prezzo \item Ordinare i prodotti per nome \end{itemize}& NI \\
        TSOF10  & Filtri lista: all'utente viene chiesto di: \begin{itemize} \item Impostare un prezzo minimo \item Impostare un prezzo massimo \item Scegliere una o pi\`u marche \item Scegliere una categoria \item Modificare i filtri impostati \item Rimuovere i filtri impostati \end{itemize} & NI \\
        TSOF11  & Registrazione: all'utente viene richiesto di registrarsi inserendo i propri dati& NI \\
        TSOF12  & Registrazione: all'utente viene richiesto di registrarsi tramite un servizio esterno & NI \\
        TSOF13  & Gestione carrello: all'utente viene richiesto di: \begin{itemize} \item Visualizzare i prodotti nel carrello \item Rimuovere i prodotti nel carrello \item Modificare la quantit\`a dei prodotti nel carrello \item Visualizzare totale, iva e avvisi circa la disponibilit\`a \end{itemize}  & NI \\
        TSOF14  & Reimpostare password: all'utente viene richiesto di reimpostare la password & NI \\
        TSOF15  & Modificare dati: all'utente viene richiesto di modificare i dati del proprio profilo, nel dettaglio: \begin{itemize} \item Email \item Username \item Password \item Nominativo \end{itemize} & NI \\
        TSOF16  & Logout: all'utente viene richiesto di effettuare il logout  & NI \\
        TSOF17  & Visualizzare profilo: all'utente viene richiesto di visualizzare il proprio profilo & NI \\
        TSOF18  & Contatta venditore: all'utente viene richiesto di contattare il venditore & NI \\
        TSOF19  & Checkout: all'utente viene richiesto di effettuare il checkout di un ordine, nel dettaglio: \begin{itemize} \item Inserire i dati di pagamento \item Inserire l'indirizzo di fatturazione \item Inserire l'indirizzo di spedizione \item Essere indirizzato al servizio di pagamento \item Visualizzare risultato del pagamento \end{itemize} & NI \\
        TSOF20  & Checkout: all'utente viene richiesto di annullare il checkout & NI \\
        TSOF21  & Riepilogo ordine: all'utente viene richiesto di visualizzare il riepilogo dell'ordine & NI \\
        TSOF22  & Lista ordini: all'utente viene richiesto di visualizzare la lista degli ordini effettuati  & NI \\
        TSOF23  & Lista clienti: al venditore viene richiesto di visualizzare la lista dei clienti & NI \\
        TSOF24  & Lista ordini: al venditore viene richiesto di visualizzare la lista degli ordini ricevuti & NI \\
        TSOF25  & Contatta cliente: al venditore viene richiesto di contattare il cliente & NI \\
        TSOF26  & Gestione prodotto: al venditore viene richiesto di \begin{itemize} \item Aggiungere un prodotto \item Modificare un prodotto \item Rimuovere un prodotto \end{itemize} & NI \\
        TSOF27  & Errore di accesso: all'utente viene richiesto di visualizzare un messaggio di errore a seguito di un tentativo di accesso con credenziali errate & NI \\
    \end{longtable}    
\end{center}