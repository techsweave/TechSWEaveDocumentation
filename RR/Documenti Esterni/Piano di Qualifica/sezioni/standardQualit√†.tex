\section{Standard di qualit\`a}
\subsection{ISO/IEC 9126}
\`E lo standard impiegato ed \`e composto da quattro parti.
\subsection{Metriche per la qualit\`a interna}
Sono metriche che non riguardano alla parte non eseguibile. Vengono rilevate mediante analisi statica. Idealmente determinano la qualit\`a esterna.
\subsection{Metriche per la qualit\`a esterna}
Sono le metriche che riguardano la parte eseguibile. Vengono rilevate tramite l'analisi dinamica. Idealmente determinano la qualit\`a in uso.
\subsection{Metriche per la qualit\`a in uso}
Metriche applicabili al prodotto finito, e utilizzato in condizioni reali.
\subsection{Modello per la qualit\`a software}
Questo modello suddivide la quali\`a in 6 categorie, ognuna con le sue varie sottocategorie cos\`i strutturate:
\subsubsection{Funzionalit\`a}
\`E la capacit\`a del software di soddisfare i requisiti. In maniera pi\`u specifica queste sono le caratteristiche che il software deve avere:
\begin{itemize}
    \item \textbf{Appropriatezza:} capacit\`a di fornire funzioni appropriate per svolgere i compiti previsti dagli obiettivi prefissati;
    \item \textbf{Accuratezza:} capacit\`a di fornire i risultati concordati oppure la precisione richiesta;
    \item \textbf{Interoperabilit\`a:} capacit\`a  di interagire con altri sistema;
    \item \textbf{Conformit\`a:} capacit\`a di aderire agli standard;
    \item \textbf{Sicurezza:} capacit\`a di garantire protezione a informazioni e dati.
\end{itemize}
\subsubsection{Affidabilit\`a}
\`E la capacit\`a del software di mantenere prestazioni specifiche in condizioni specificate.
Si compone di:
\begin{itemize}
    \item \textbf{Maturit\'a: }capacit\`a di evitare errori, malfunzionamenti e risultati non corretti;
    \item \textbf{Tolleranza agli errori: }capacit\`a di mantenere i livelli prefissati di prestazioni anche in caso di errori o di usi scorretti del prodotto;
    \item \textbf{Recuperabilit\'a: }capacit\`a, a seguito di un malfunzionamento o di un uso scorretto, di ripristinare i livelli di prestazioni;
    \item \textbf{Aderenza: }capacit\`a cdi aderire a standard e regole riguardanti l'affidabilit\`a.
\end{itemize}
\subsubsection{Efficienza}
\'E la capacita del software di svolgere le proprie funzioni ottimizzando l'uso delle risorse e minimizzando i tempi. Si compone di:
\begin{itemize}
    \item  \textbf{Nello spazio: }capacit\`a di utilizzo di quantit\`a e tipologia di risorse in maniera appropriata;
    \item  \textbf{Nel tempo: }capacit\`a di fornire tempi di risposta ed elaborazione adeguati.
\end{itemize}
\subsubsection{Usabilit\`a}
Capacit\`a del prodotto di essere utilizzato e compreso dall'utente, sotto determinate condizioni. Si compone di:
\begin{itemize}
    \item \textbf{Comprensibilit\`a: }capacit\`a di essere chiaro riguardo le proprie funzionalit\`a e il proprio utilizzo:
    \item \textbf{Apprendibilit\`a: }capacit\`a di essere facilmente appreso dall'utente;
    \item \textbf{Operabilit\`a: }capacit\`a di eseguire gli scopi dell'utente e di essere controllato dall'utente;
    \item \textbf{Attrattivit\`a: }capacit\`a di essere piacevole per l'utente.
\end{itemize}
\subsection{Manutenibilit\`a}
Capacit\`a del prodotto di essere modificato, corretto e adattato. Si compone di:
\begin{itemize}
    \item \textbf{Analizzabilit\`a: }capacit\`a di essere analizzato per identificare errori;
    \item \textbf{Modificabilit\`a: }capacit\`a di poter essere modificato nel codice, nella documentazione o nella progettazione;
    \item \textbf{Stabilit\`a: }capacit\`a di evitare effetti collaterali indesiderati a seguito di modifiche;
    \item \textbf{Testabilit\`a: }capacit\`a di essere testato per validare le modifiche.
\end{itemize}
\subsubsection{Portabilit\`a}
\begin{itemize}
    \item \textbf{Adattabilt\`a: }capacit\`a di essere adattato a vari ambienti operativi senza modifiche.
    \item \textbf{Installabilit\`a: }capacit\`a di essere installato in un ambiente.
    \item \textbf{Conformit\`a: }capacit\`a di coesistere con altre applicazione, condividendo le risorse;
    \item \textbf{Sostituibilit\`a: }capacit\`a di essere impiegato al posto di altre applicazioni per svolgere gli stessi compiti.
\end{itemize}