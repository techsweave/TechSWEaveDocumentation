\section{Introduzione}
\subsection{Scopo del documento}
Il \textit{Piano di Qualifica} è un documento sul quale si prevede di lavorare per l'intera durata del progetto seguendo una
logica di tipo incrementale: i suoi contenuti iniziali sono incompleti e soggetti ad aggiunte significative e possibili modifiche 
effettuate in istanti temporali successivi nello svolgimento del progetto.
Alcune delle metriche scelte non sono applicabili nella fase iniziale e solo con il loro utilizzo pratico nel tempo
si può valutarne l'utilità. Anche i processi selezionati potranno subire dei cambiamenti qualora dovessero rivelarsi
inadeguati o inefficaci agli scopi del progetto e al modo di operare del gruppo.\\
Il presente documento ha l'obiettivo di descrivere le strategie di verifica e validazione che il gruppo \textit{TechSWEave} intende
adottare per garantire la qualità di prodotto e di processo. Per il raggiungimento di questo scopo viene effettuata
un'attività di verifica continua sulle attività svolte al fine di individuare e correggere quanto prima 
eventuali problemi insorti.
\subsection{Scopo del prodotto}
Lo scopo del prodotto, descritto nel capitolato C2 - \textit{Emporio-Lambda}, nasce dall'esigenza del proponente \textit{RedBabel}
di disporre di una piattaforma di e-commerce generica, basato sulla tecnologia serverless, da utilizzare come prototipo.
\subsection{Glossario}
All'interno del documento sono presenti termini che potrebbero risultare ambigui a seconda del contesto. Al fine di evitare possibili incomprensioni 
e rendere chiare agli stakeholders\textsubscript{\textbf{G}} i termini utilizzati, viene fornito un \textit{Glossario v1.0.0.} contente i suddetti termini 
e la loro spiegazione. Nella seguente documentazione tali termini saranno individuabili tramite una 'G' a pedice.
\subsection{Riferimenti}
\subsubsection{Riferimenti normativi}
\begin{itemize}
    \item \textit{Norme di Progetto v1.0.0}
    \item \textbf{Capitolato d'appalto C2 - Emporio-Lambda: piattaforma di e-commerce in stile Serverless:} \\ \href{https://www.math.unipd.it/~tullio/IS-1/2020/Progetto/C2.pdf}{https://www.math.unipd.it/~tullio/IS-1/2020/Progetto/C2.pdf}
\end{itemize}
\subsubsection{Riferimenti informativi}
\begin{itemize}
    \item \textbf{Qualità di processo:} \\ \href{https://www.math.unipd.it/~tullio/IS-1/2020/Dispense/L13.pdf}{https://www.math.unipd.it/~tullio/IS-1/2020/Dispense/L13.pdf}
    \item \textbf{Qualità di prodotto:} \\ \href{https://www.math.unipd.it/~tullio/IS-1/2020/Dispense/L12.pdf}{https://www.math.unipd.it/~tullio/IS-1/2020/Dispense/L12.pdf}
    \item \textbf{Verifica e validazione:} \\ \href{https://www.math.unipd.it/~tullio/IS-1/2020/Dispense/L14.pdf}{https://www.math.unipd.it/~tullio/IS-1/2020/Dispense/L14.pdf}
    \item \textbf{Indice di Gulpease:} \\ \href{https://it.wikipedia.org/wiki/Indice\_Gulpease}{https://it.wikipedia.org/wiki/Indice\_Gulpease}
    \item \textbf{ISO/IEC 9126:} \\ \href{https://it.wikipedia.org/wiki/ISO/IEC\_9126}{https://it.wikipedia.org/wiki/ISO/IEC\_9126}
    \item \textbf{ISO/IEC 12207:1995:} \\ \href{https://www.math.unipd.it/~tullio/IS-1/2009/Approfondimenti/ISO\_12207-1995.pdf}{https://www.math.unipd.it/~tullio/IS-1/2009/Approfondimenti/ISO\_12207-1995.pdf}
    \item \textbf{Schedule Variance e metriche correlate:} \\ \href{https://www.smartsheet.com/hacking-pmp-how-calculate-schedule-variance}{https://www.smartsheet.com/hacking-pmp-how-calculate-schedule-variance}
\end{itemize}
\newpage