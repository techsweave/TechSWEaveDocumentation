\section{Introduzione}
    \subsection{Scopo del documento}
    Questo documento ha come fine la stesura delle modalità attraverso le quali il gruppo \emph{TechSWEave} svilupperà il progetto \emph{EmporioLambda}. In particolare il documento espone i seguenti argomenti:
    \begin{itemize}
        \item analisi dei rischi;
        \item descrizione del modello di sviluppo utilizzato;
        \item Divisione e assegnazione dei compiti tra i membri del gruppo;
        \item stima dei costi e delle risorse necessarie allo sviluppo del progetto.
    \end{itemize}
    \subsection{Scopo del prodotto}
    L'obiettivo del progetto è quello di realizzare una piattaforma e-commerce basata su teconologie serverless. Il prodotto dovrà poter essere usufruito da un generico commerciante con la minima interazione tecnica tramite account AWS Merchant\textsubscript{\textbf{G}}. Inoltre dovranno essere implementate delle funzioni irrinunciabili per tutte le categorie di utenti che ne faranno uso:
    \begin{itemize}
        \item clienti;
        \item commercianti;
        \item admin.
    \end{itemize}
    \subsection{Glossario}
    Al fine di evitare possibili ambiguità relative al linguaggio utilizzato nei documenti formali, vienefornito ilGlossario v1.0.0. In questo documento vengono definiti e descritti tutti i termini conun significato particolare. Per facilitare la lettura, i termini saranno contrassegnati da una ’G’ apedice.
    \subsection{Riferimenti}
        \subsubsection{Normativi}
        \begin{itemize}
            \item \textbf{Norme di Progetto: } \emph{Norme di Progetto v1.0.0};
            \item \textbf{Regolamento organigramma e specifiche tecnico-economiche: } \\ \href{https://www.math.unipd.it/~tullio/IS-1/2020/Progetto/RO.html}{https://www.math.unipd.it/~tullio/IS-1/2020/Progetto/RO.html}
        \end{itemize}
        \subsubsection{Informativi}
        \begin{itemize}
            \item \textbf{EmporioLambda: piattaforma di e-commerce in stile Serverless: } \\ \href{https://www.math.unipd.it/~tullio/IS-1/2020/Progetto/C2.pdf}{https://www.math.unipd.it/~tullio/IS-1/2020/Progetto/C2.pdf};
            \item \textbf{Slide L05 del corso Ingegneria del Software - Ciclo di vita del software: } \\ \href{https://www.math.unipd.it/~tullio/IS-1/2020/Dispense/L05.pdf}{https://www.math.unipd.it/~tullio/IS-1/2020/Dispense/L05.pdf};
            \item \textbf{Slide L06 del corso Ingegneria del Software - Gestione di Progetto: } \\ \href{https://www.math.unipd.it/~tullio/IS-1/2020/Dispense/L06.pdf}{https://www.math.unipd.it/~tullio/IS-1/2020/Dispense/L06.pdf}.
            %\item \textbf{Software Engineering - Ian Sommerville - 9thEdition, 2010}.
        \end{itemize}
    \subsection{Scadenze}
    Il gruppo \emph{TechSWEave} si impegnera a rispettare le seguenti scadenze per lo sviluppo del progetto \emph{EmporioLambda}:
    \begin{itemize}
        \item \textbf{Revisione dei Requisiti: }??
        \item \textbf{Revisione di Progettazione: }??
        \item \textbf{Revisione di Qualifica: }??
        \item \textbf{Revisione di Accettazione: }??
    \end{itemize}
