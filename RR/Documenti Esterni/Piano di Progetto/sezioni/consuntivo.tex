\section{Consuntivo}
    Di seguito vengono indicate le spese sostenute per ogni ruolo confrontandole con quanto preventivato. Il bilancio potrà essere:
    \begin{itemize}
        \item \textbf{Positivo:} se la spesa effettiva è minore di quanto preventivato;
        \item \textbf{Pari:} se la spesa effettiva è uguale a quanto preventivato;
        \item \textbf{Negativo:} se la spesa effettiva è maggiore di quanto preventivato.
    \end{itemize}
    \subsection{Periodo di analisi}
    Le ore di lavoro sostenute in questa fase sono da considerarsi come ore di investimento per l’approfondimento personale, non vengono quindi rendicontate.

    \begin{center}
        \begin{table}[!ht]
            \centering
            \caption{Consuntivo della fase di analisi}
            \vspace{5px}
            \rowcolors{2}{logo!10}{logo!40}
            \renewcommand{\arraystretch}{1.8}
            \begin{tabular}{p{150px} p{110px} p{110px}}
                \rowcolor{logo!70} \textbf{Ruolo} & \textbf{Ore} & \textbf{Costo}\\
                Responsabile & 21 (+0) & 630,00\EURdig (+0,00 \EURdig) \\
                Amministratore (+0) & 35 & 700,00\EURdig (+0,00 \EURdig) \\
                Analista & 98 (+0) & 2.450,00\EURdig (+0,00 \EURdig) \\
                Progettista & 0 (+0) & 0(+0,00 \EURdig) \\
                Programmatore & 0 (+0) & 0(+0,00 \EURdig) \\
                Verificatore & 91 (+0) & 1.365,00\EURdig (+0,00 \EURdig) \\
                \textbf{Totale preventivo} & 245 & 5.145,00\EURdig \\
                \textbf{Totale consuntivo} & 245 & 5.145,00\EURdig \\
                \textbf{Differenza} & 0 & (+0,00 \EURdig) \\
            \end{tabular}
        \end{table}
    \end{center}
    \subsubsection{Conclusioni}
    Avendo deciso coi componenti del gruppo di rispettare la scadenza per la prima consegna disponibile, si è lavorato per rispettare i tempi e si è riusciti a non superare le tempistiche prestabilite.