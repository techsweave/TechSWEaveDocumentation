\section{Analisi dei Rischi}
Durante lo sviluppo del progetto è molto facile incorrere in problemi che possono essere evitati dopo un'attenta analisi dei principali fattori rischi.
Per ognuna delle voci nella tabella sottostante è stata utilizzata la seguente procedura di identificazione e risoluzione:

\begin{itemize}
    \item \textbf{Individuazione dei rischi: }attività svolta allo scopo di identificare i vari elementi problematici che possono rallentare o impedire il normale proseguimento del progetto;
    \item \textbf{Analisi dei rischi: }attività di studio dei fattori di rischio ai cui successivamente viene assegnata una probabilità con cui si verificano e un indice di gravità, in modo da poterne determinare l'impatto che avrebbe sul progetto;
    \item \textbf{Pianificazione di controllo: } attività volta alla pianificazione di una metodologia per evitare che si verifichino i rischi individuati e venga stabilita una modalità d'intervento qualora si verificassero;
    \item \textbf{Monitoraggio dei rischi: }attività continua svolta al fine di prevenire l'incontro con queste complicazioni o, nel peggiore dei casi, permetta di arginarle tempestivamente.
\end{itemize} 

Sono stati inoltre definiti i seguenti codici per raggruppare le varie tipologie di fattori dei rischio:
\begin{itemize}
    \item \textbf{RT: }Rischi Tecnologici;
    \item \textbf{RO: }Richi Orgnainzzativi;
    \item \textbf{RI: }Rischi Interpersonali.
\end{itemize}

\newcommand{\acapo}[1]{%
  \begin{tabular}{@{}c@{}}\strut#1\strut\end{tabular}%
}


%Tabella analisi dei rischi
\begin{center}
    \vspace{10px}
    \begin{table}[h!]
    \centering
    \caption{Tabella dei Rischi di Progetto}
    \rowcolors{2}{logo!10}{logo!40}
    \renewcommand{\arraystretch}{1.8}
    \begin{tabular}{p{60px} p{150px} p{150px} p{90px}}
        \rowcolor{logo!70} \acapo{\textbf{Nome}\\\textbf{Codice}} & \textbf{Descrizione} & \textbf{Rilevamento} & \textbf{Grado di  rischio}\\
        Inesperienza Tecnologica RT1 & Molte delle tecnologie adottate per lo sviluppo del progetto richiesto sono nuove per i componenti del team, è quindi molto probabile incappare in problemi operativi. & Il \emph{responsabile} dovrà constatare le conoscenze ed eventuali lacune dei vari componenti del team. Ogni componente del gruppo inoltre provverà a comunicare in assoluta trasparenza eventuali difficoltà. & \acapo {Occorrenza:\textbf{Alta} \\  Pericolosità: \textbf{Alta} }\\
        
        \textbf{Piano di contingenza}& \multicolumn{3}{c}{\acapo{I compiti più onerosi, o che richiedono maggiori conoscenze tecnologiche, verranno assegnati\\  a più persone favorendone l'assisstenza reciproca.}} \\  
    \end{tabular}
\end{table}
\end{center}
