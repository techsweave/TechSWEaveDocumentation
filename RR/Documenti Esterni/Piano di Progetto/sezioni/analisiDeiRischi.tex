\section{Analisi dei Rischi}
Durante lo sviluppo del progetto è molto facile incorrere in problemi che possono essere evitati dopo un'attenta analisi dei principali fattori rischi.
Per ognuna delle voci nella tabella sottostante è stata utilizzata la seguente procedura di identificazione e risoluzione:

\begin{itemize}
    \item \textbf{Individuazione dei rischi: }attività svolta allo scopo di identificare i vari elementi problematici che possono rallentare o impedire il normale proseguimento del progetto;
    \item \textbf{Analisi dei rischi: }attività di studio dei fattori di rischio ai cui successivamente viene assegnata una probabilità con cui si verificano e un indice di gravità, in modo da poterne determinare l'impatto che avrebbe sul progetto;
    \item \textbf{Pianificazione di controllo: } attività volta alla pianificazione di una metodologia per evitare che si verifichino i rischi individuati e venga stabilita una modalità d'intervento qualora si verificassero;
    \item \textbf{Monitoraggio dei rischi: }attività continua svolta al fine di prevenire l'incontro con queste complicazioni o, nel peggiore dei casi, permetta di arginarle tempestivamente.
\end{itemize} 

Sono stati inoltre definiti i seguenti codici per raggruppare le varie tipologie di fattori dei rischio:
\begin{itemize}
    \item \textbf{RT: }Rischi Tecnologici;
    \item \textbf{RO: }Richi Orgnainzzativi;
    \item \textbf{RI: }Rischi Interpersonali.
\end{itemize}

%Tabella analisi dei rischi
  \renewcommand{\arraystretch}{1.5}
\rowcolors{2}{logo!40}{logo!10}
	\arrayrulecolor{white}
	\begin{longtable}{ 
			>{\centering}p{0.17\textwidth} 
			>{\raggedright}p{0.28\textwidth}
			>{\raggedright}p{0.29\textwidth} 
			>{\centering}p{0.15\textwidth}
		}

	
	\caption{Tabella dei Rischi di Progetto}\\
	\rowcolor{white}\\
	\rowcolor{logo!70}
	\textbf{Nome \\ Codice} & \centering\textbf{Descrizione} & 
	\centering\textbf{Rilevamento} & 
	\textbf{Grado di rischio} 
	\tabularnewline
	\endfirsthead
	\rowcolor{white}\caption[]{(continua)}\\
	\rowcolor{logo!70}
	\textbf{Nome \\ Codice} & \centering\textbf{Descrizione} & 
	\centering\textbf{Rilevamento} & 
	\textbf{Grado di rischio} 
	\tabularnewline
	\endhead
	
	%RT1---------------------------------------------------------
	Inesperienza Tecnologica \\ RT1 & Molte delle tecnologie adottate per lo sviluppo del progetto richiesto sono nuove per i componenti del team, è quindi molto probabile incappare in problemi operativi. & Il \emph{responsabile} dovrà constatare le conoscenze ed eventuali lacune dei vari componenti del team. Ogni componente del gruppo inoltre provverà a comunicare in assoluta trasparenza eventuali difficoltà. &
	Occorrenza: \textbf{Alta} \\
	Pericolosità: \textbf{Alta} 
	\tabularnewline
	\multicolumn{1}{p{0.17\textwidth}}{\centering\textbf{Piano di contingenza}}& 
	\multicolumn{3}{p{0.7700\textwidth}}{I compiti più onerosi, o che 
	richiedono maggiori conoscenze tecnologiche, verranno assegnati a più 
	persone favorendo così l'assistenza reciproca. }
	\tabularnewline 
	 	
	%RO1---------------------------------------------------------
	Calcolo Tempistiche \\ RO1 & L'utilizzo di molte tecnologie nuove per molti componenti, può causare imprecisioni e variazioni nel calcolo delle tempistiche. & Durante lo sviluppo verranno suddivise delle mansioni alle quali sarà assegnata una scadenza; sarà compito del possessore comunicare problematiche, nel rispettare le scadenze delle task.&	
	Occorrenza: \textbf{Alta} \\
	Pericolosità: \textbf{Alta}
	\tabularnewline
	\multicolumn{1}{p{0.17\textwidth}}{\centering\textbf{Piano di contingenza}}& 
	\multicolumn{3}{p{0.7700\textwidth}}{All'avvennimento di tali problematiche, il \emph{responsabile} in accordo con il possessore della mansione, provvederà all'assegnazione di maggiori risorse o allo spostamento delle scadenza.}
	\tabularnewline	
	
	%R02------------------------------------------------------------
	
	Calcolo costi \\ RO2 &
	L'imprecisioni nelle valutazioni economiche sono dovute dall'inesperienza del team in tale ambito. &
	Per permettere al \textit{responsabile} di monitorare le ore di lavoro di ciascun 
	componente il team ha organizzato specifiche tabelle condivise.&
	Occorrenza: \textbf{Media} \\
	Pericolosità: \textbf{Alta}
	\tabularnewline
	\multicolumn{1}{p{0.17\textwidth}}{\centering\textbf{Piano di contingenza}}& 
	\multicolumn{3}{p{0.7700\textwidth}}{All'insorgere di rilevanti variazioni 
	orarie rispetto al preventivo iniziale, verranno comunicati tempestivamente 
	al committente tali mutamenti.}
	\tabularnewline	
	
	%R03------------------------------------------------------------*
	Impegni Universitari \\ RO3 & 
	È possibile che si verificano periodi nei quali uno o più componenti risultino non disponibili a causa degli impegni accademici. &
	È stato creato un calendario  condiviso nel quale indicare gli impegni accademici al fine di evitare eventuali rallentamenti.&
	Occorrenza: \textbf{Alta} \\
	Pericolosità: \textbf{Bassa}
	\tabularnewline
	\multicolumn{1}{p{0.17\textwidth}}{\centering\textbf{Piano di contingenza}}& 
	\multicolumn{3}{p{0.7700\textwidth}}{ L'assegnazione di incarchi e scadenze 
	avverrà nel rispetto degli impegni segnalati nel calendario.}
	\tabularnewline	
	
	%R04------------------------------------------------------------
	
    Impegni Personali \\ RO4 &
	È possibile che si verificano periodi nei quali uno o più componenti risultino non disponibili a causa di impegni personali.&
	Per informare il team dei propri impegni personali ciascun componente utilizzerà 
	il calendario già descritto nel caso precedente. Eventuali impegni imprevisti verranno tempestivamente 
	comunicati al \textit{responsabile}.&
	Occorrenza: \textbf{Media} \\
	Pericolosità: \textbf{Bassa}
	\tabularnewline
	\multicolumn{1}{p{0.17\textwidth}}{\centering\textbf{Piano di contingenza}}& 
	\multicolumn{3}{p{0.7700\textwidth}}{L'assegnazione di incarchi e scadenze 
		avverrà nel rispetto degli impegni segnalati nel calendario. 
		All'insorgere di imprevisti, il \emph{reponsabile} valuterà una riallocazione 
		di risorse oppure una riassegnazione del task.}
	\tabularnewline	
	
	%R05------------------------------------------------------------
	
	Ritardi \\ RO5 &
	Le problematiche sopracitate (RO1, RO3, RO4) possono 
	comportare a ritardi nello svolgimento delle mansioni.&
	Il possessore di ciascuna mansione comunicherà in modo tempestivo l'impossibiltà di 
	rispettare le proprie scadenze.&
	Occorrenza: \textbf{Media} \\
	Pericolosità: \textbf{Bassa}
	\tabularnewline
	\multicolumn{1}{p{0.17\textwidth}}{\centering\textbf{Piano di contingenza}}& 
	\multicolumn{3}{p{0.7700\textwidth}}{ Il \textit{responsabile}, se 
	necessario, 
	riassegnerà le risorse allo scopo di evitare rallentamenti.}
	\tabularnewline	
	
	%RI1------------------------------------------------------------
	
	Comunicazione Interna \\ RI1 & 
	Si potranno verificare dei momenti in cui uno più membri del gruppo siano irreperibili. &
	I componenti sono tenuti a comunicare eventuali momenti di irreperibilità e organizzare i propri impegni al fine di poter presenziare alle riunioni del gruppo. &
	Occorrenza: \textbf{Bassa} \\
	Pericolosità: \textbf{Alta}
	\tabularnewline
	\multicolumn{1}{p{0.17\textwidth}}{\centering\textbf{Piano di contingenza}}& 
	\multicolumn{3}{p{0.7700\textwidth}}{ Il gruppo ha predisposto molteplici 
	vie per la comunicazione interna. Inoltre verranno organizzati incontri a 
	scadenze fissa per discutere dell'avanzamento del progetto.}
	\tabularnewline	
	
	%RI2------------------------------------------------------------
	
	Comunicazione Esterna \\ RI2 &
	Il proponente ha la propria 
	sede all'estero, di conseguenza le comunicazioni saranno più difficili. &
	Come per le comunicazioni interne, sono stati predisposti più canali di 
	comunicazione; le video conferenze con il proponente saranno organizzare 
	con il dovuto preavviso.&
	Occorrenza: \textbf{Bassa} \\
	Pericolosità: \textbf{Media}
	\tabularnewline
	\multicolumn{1}{p{0.17\textwidth}}{\centering\textbf{Piano di contingenza}}& 
	\multicolumn{3}{p{0.7700\textwidth}}{Il gruppo provvederà a raggruppare 
	quesiti e segnalazioni per il proponente.}
	\tabularnewline	
	
	%RI3------------------------------------------------------------
	
	Conflitti interni \\ RI3 &
	Come in qualsiasi gruppo è possibile che si verifichino conflitti e tensioni tra i vari componenti, nel corso delle varie attività. &
	Ciascun membro si dorvrà impegnare a limitare tali tensioni e fare in modo che esse non influiscano sul normale svolgersi delle attività. &
	Occorrenza: \textbf{Bassa} \\
	Pericolosità: \textbf{Media}
	\tabularnewline
	\multicolumn{1}{p{0.17\textwidth}}{\centering\textbf{Piano di contingenza}}& 
	\multicolumn{3}{p{0.7700\textwidth}}{Il \textit{responsabile} avrà il compito di essere il mediatore in tali controversie.}
	\tabularnewline	
		
	\end{longtable}
\counterwithin{table}{subsection}	
\renewcommand{\arraystretch}{1}