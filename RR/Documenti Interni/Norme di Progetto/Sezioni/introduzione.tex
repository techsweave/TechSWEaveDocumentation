\section{Introduzione}
    \subsection{Scopo del documento}
    Questo documento ha lo scopo di stabilire tutte le regole per uniformare il modo di lavorare di tutto il team, per ottenere un'organizzazione efficiente dei file prodotti. \\Durante lo svolgimento del progetto tutti i membri sono tenuti a visionare questo documento per rispettare le norme definite.
    Le norme possono subire modifiche, che dovranno essere tempestivamente comunicate a ciascun membro del gruppo.

    \subsection{Scopo del prodotto}
    Il capitolato richiede la realizzazione di una piattaforma e-commerce basata interamente sulle tecnologie serverless che possa essere utilizzata dal proponente come prototipo da mostrare ad altre aziende. EmporioLambda deve poter essere utilizzato da un ipotetico commerciante con il minimo quantitativo di configurazione manuale tramite account AWS Merchant. Il capitolato prevede che vengano fornite alcune funzioni irrinunciabili per tutte le categorie di utenti che ne faranno uso: clienti, commercianti, admin.
    \subsection{Glossario}
    All'interno del documento sono presenti termini che possono risultare ambigui o che necessitano di una definizione precisa: questi termini vengono identificati con una 'G' a pedice, almeno alla prima occorrenza, e vengono definiti all'interno del Glossario.
    \subsection{Riferimenti}
        \subsubsection{Riferimenti normativi}
        \begin{itemize}
            \item \href{https://www.math.unipd.it/~tullio/IS-1/2020/Progetto/C2.pdf}{Capitolato d'appalto C2: EmporioLambda}
        \end{itemize}

        \subsubsection{Riferimenti informativi}
        \begin{itemize}
            \item \href{https://www.math.unipd.it/~tullio/IS-1/2009/Approfondimenti/ISO_12207-1995.pdf}{ISO 12207:1995}
            \item \href{https://www.math.unipd.it/~tullio/IS-1/2020/Dispense/L03.pdf}{Slide lezione L03}
            \item Libro di testo: Ingegneria del Software - 10\textsuperscript{A} edizione - Ian Sommerville
        \end{itemize}