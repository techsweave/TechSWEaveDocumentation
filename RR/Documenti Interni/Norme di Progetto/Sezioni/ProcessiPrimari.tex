\section{Processi Primari}
\subsection{Fornitura}
\subsubsection{Scopo}
Il processo di Fornitura, come stabilito dallo standard ISO/IEC 12207:1995\textsubscript{\textbf{G}}, descrive le attività e i compiti che il fornitore deve rispettare per soddisfare le richieste del proponente. Dopo una prima attività di analisi per comprendere le richieste del proponente, il fornitore redige lo \textit{Studio di Fattibilità} al fine di individuare gli aspetti positivi e le criticità in base alle informazioni acquisite. Occorre quindi stipulare un contratto con cui regolare i rapporti e le scadenze con il proponente e definire un \textit{Piano di Progetto} da seguire fino alla consegna del prodotto finale.
\\Il processo di fornitura è composto dalle fasi seguenti:
\begin{itemize}
    \item avvio;
    \item approntamento di risposte alle richieste;
    \item contrattazione;
    \item pianificazione;
    \item esecuzione e controllo;
    \item revisione e valutazione;
    \item consegna e completamento.
\end{itemize}
\subsubsection{Descrizione}
L’obiettivo del processo di Fornitura consiste nell’individuare e formalizzare norme e procedure a cui il gruppo \textit{TechSWEave} deve attenersi durante tutte le fasi di realizzazione del prodotto per diventare fornitore del proponente \textit{RedBabel} e dei committenti Prof. Tullio Vardanega e Prof. Riccardo Cardin.
\subsubsection{Aspettative}
Durante lo svolgimento del progetto è intenzione del gruppo \textit{TechSWEave} instaurare una collaborazione continua e costante con lo scopo di provvedere alle seguenti esigenze:
\begin{itemize}
    \item determinare i bisogni che il prodotto finale dovrà soddisfare;
    \item fissare i vincoli e i requisiti sui processi;
    \item stimare le tempistiche di lavoro;
    \item chiarire eventuali dubbi;
    \item assicurare una verifica continua;
    \item accordarsi sulla qualifica del prodotto.
\end{itemize}
\subsubsection{Studio di fattibilità}
Il \textit{Responsabile di Progetto}\textsubscript{\textbf{G}} ha il compito di convocare i membri del gruppo per discutere dei capitolati d’appalto disponibili, effettuando una prima analisi del materiale proposto. Per ogni capitolato gli \textit{Analisti}\textsubscript{\textbf{G}} provvedono poi a redigere uno \textit{Studio di Fattibilità} in cui viene analizzato ulteriormente il materiale disponibile.
\\Lo Studio di Fattibiltà comprende i seguenti punti:
\begin{itemize}
    \item \textbf{Informazioni generali}: informazioni di base quali titolo del capitolato, nome del proponente e del committente;
    \item \textbf{Descrizione del capitolato}: breve sintesi delle caratteristiche del prodotto da sviluppare;
    \item \textbf{Finalità del progetto}: le finalità richieste dal capitolato d’appalto;
    \item \textbf{Tecnologie interessate}: elenco delle tecnologie da utilizzare per lo sviluppo del progetto;
    \item \textbf{Aspetti positivi}: elenco di motivi per cui il gruppo reputa interessante il progetto proposto;
    \item \textbf{Aspetti critici}: elenco di motivi per cui il gruppo reputa rischioso o di difficile attuazione il progetto proposto;
    \item \textbf{Conclusioni}: valutazione finale con cui viene dichiarata l’intenzione del gruppo di accettare o rifiutare il capitolato.
\end{itemize}
\subsubsection{Altra documentazione fornita}
Al fine di conseguire un metodo di tracciamento in modo da assicurarne la trasparenza durante tutto il ciclo di vita del progetto è necessario fornire al proponente \textit{RedBabel} e ai committenti Prof. Tullio Vardanega e Prof. Riccardo Cardin i documenti a seguire.
\subsubsubsection{Piano di Progetto}
Il \textit{Responsabile}, con il supporto degli \textit{Amministratori}, redige il \textit{Piano di Progetto} da seguire durante il progetto.
\\Questo documento è strutturato nel modo seguente:
\begin{itemize}
    \item \textbf{Analisi dei rischi}: vengono analizzati i possibili rischi riscontrabili durante la realizzazione del progetto, la probabilità con cui possono presentarsi e il rispettivo livello di gravità;
    \item \textbf{Modello di sviluppo}: descrizione del modello di sviluppo scelto, indispensabile per la progettazione;
    \item \textbf{Pianificazione}: vengono pianificate le attività e le rispettive scadenze temporali che si susseguiranno durante il progetto;
    \item \textbf{Preventivo e consuntivo}: viene calcolato un preventivo per il costo totale del progetto ed esposto un consuntivo di periodo per evidenziare eventuali discostamenti da quanto preventivato.
\end{itemize}
\subsubsubsection{Piano di Qualità}
I \textit{Verificatori} redigono un documento chiamato \textit{Piano di Qualifica} che stabilisce le strategie da rispettare per garantire la qualità dei processi\textsubscript{\textbf{G}} attuati e del materiale del prodotto dal gruppo.
\\Questo documento è strutturato nel modo seguente:
\begin{itemize}
    \item \textbf{Qualità di processo}: vengono individuati dei processi dagli standard, stabiliti degli obiettivi, fissate delle procedure per attuarli e delle metriche per misurarli;
    \item \textbf{Qualità di prodotto}: vengono individuati gli attributi principali del prodotto, definiti degli obiettivi per raggiungerli e delle metriche per misurarli;
    \item \textbf{Specifiche dei test}: definiscono dei test che il prodotto deve passare per garantire il soddisfacimento dei requisiti;
    \item \textbf{Standard di qualità}: vengono determinati gli standard di qualità;
    \item \textbf{Resoconto delle attività di verifica}: vengono descritti i risultati delle metriche calcolate in forma di resoconto;
    \item \textbf{Valutazioni per il miglioramento}: vengono riportate le difficoltà riscontrate e le possibili soluzioni attuabili in futuro.
\end{itemize}
\subsubsection{Strumenti}
Di seguito è riportato l’elenco degli strumenti impiegati dal gruppo per il processo di fornitura.
\subsubsubsection{Gantt Project}
Software utile alla realizzazione dei diagrammi di Gantt\textsubscript{\textbf{G}}, utili nella pianificazione, gestione e assegnazione delle risorse;
\subsubsubsection{Microsoft Excel}
Software utile alla produzione e alla gestione di fogli elettronici. Utilizzato per la creazione di diagrammi, tabelle e per il calcolo matematico.
\subsection{Sviluppo}
\subsubsection{Scopo}
Il processo di Sviluppo, come stabilito dallo standard ISO/IEC 12207:1995\textsubscript{\textbf{G}}, descrive le attività di analisi, progettazione, codifica, integrazione, test, installazione ed accettazione utili alla realizzazione di un prodotto finale conforme ai requisiti richiesti dal proponente.
\subsubsection{Descrizione}
Di seguito sono elencate le attività che compongono il processo di sviluppo:
\begin{itemize}
    \item Analisi dei requisiti;
    \item Progettazione;
    \item Codifica del software.
\end{itemize}
\subsubsection{Aspettative}
Le aspettative sono le seguenti:
\begin{itemize}
    \item determinare gli obiettivi di sviluppo;
    \item determinare i vincoli tecnologici;
    \item determinare i vincoli di design;
    \item realizzare un prodotto finale che supera i test, che soddisfa i requisiti e le richieste del proponente.
\end{itemize}
\subsubsection{Analisi dei requisiti}
\subsubsubsection{Scopo}
L’attività di analisi precede quella di sviluppo. Il documento di \textit{Analisi dei Requisiti} è redatto dagli \textit{Analisti}\textsubscript{\textbf{G}} ed ha come scopo quello di individuare i requisiti diretti e indiretti, espliciti ed impliciti ed ha come scopo quello di:
\begin{itemize}
    \item definire gli obiettivi di sviluppo;
    \item definire le funzionalità del prodotto concordate con il proponente;
    \item definire i vincoli tecnologici;
    \item definire i vincoli di design;
    \item fornire ai verificatori dei riferimenti per l’attività di controllo;
    \item fornire una stima del quantitativo di lavoro da svolgere per tracciare una stima dei costi;
    \item realizzare un prodotto finale che superi i test, che soddisfi i requisiti e le richieste del proponente.
\end{itemize}
\subsubsubsection{Descrizione}
I requisiti si possono trarre da diverse fonti:
\begin{itemize}
    \item \textbf{Capitolati d'Appalto}: il requisito è esplicitato nel capitolato del proponente;
    \item \textbf{Verbali Interni}: il requisito è individuato durante le riunioni del gruppo \textit{TechSWEave};
    \item \textbf{Verbali Esterni}: il requisito è individuato durante le riunioni o a seguito di comunicazioni con il proponente;
    \item \textbf{Casi d'Uso}: il requisito è ricavato da uno o più casi d’uso.
\end{itemize}
\subsubsubsection{Aspettative}
L’obiettivo è quello di creare un documento formale contenente tutti i requisiti richiesti e concordati con il proponente.
\subsubsubsection{Classificazione dei requisiti}
Per la classificazione dei requisiti si è deciso per la seguente convenzione:

\begin{itemize}
    \item \textbf{Codice identificativo:} ogni codice identificativo è univoco e conforme alla seguente codifica:
          \begin{center}
              \textbf{R[ID][Tipologia]}
          \end{center}
          Il significato delle voci è:
          \begin{itemize}
              \item \textbf{ID:} identificatore univoco del requisito in forma gerarchica.
              \item \textbf{Tipologia:} ogni requisito può assumere uno dei seguenti valori:
                    \begin{itemize}
                        \item \textit{F}: funzionale;
                        \item \textit{Q}: qualitativo;
                        \item \textit{V}: vincolo;
                        \item \textit{P}: prestazionale.
                    \end{itemize}
          \end{itemize}

    \item \textbf{Importanza:} riporta l'importanza del requisito:
          \begin{itemize}
              \item obbligatorio: requisito irrinunciabile per gli stakeholder;
              \item desiderabile: requisito non strettamente necessario che però apporta un valore aggiunto al prodotto;
              \item opzionale: requisito relativamente utile oppure contrabile in un secondo momento durante lo sviluppo;
          \end{itemize}
    \item \textbf{Descrizione:} descrizione del requisito, meno ambigua possibile;
    \item \textbf{Fonti:} ogni requisito deriva da una di queste fonti:
          \begin{itemize}
              \item \textit{capitolato:} è un requisito individuato dalle condizione imposte dal capitolato;
              \item \textit{interno:} è un requisito aggiunto dagli analisti;
              \item \textit{caso d'uso:} è un requisito estratto da uno o più casi d'uso, è riportato il codice del caso d'uso a cui ci si riferisce;
              \item \textit{verbale:} è un requisito individuato attraverso una riunione con i proponenti.
          \end{itemize}
\end{itemize}
\subsubsubsection{Classificazione dei casi d'uso}
Un caso d’uso è un diagramma, accompagnato da una descrizione testuale aggiuntiva, che rappresenta un comportamento, offerto o desiderato, sulla base di risultati osservabili.
\\La struttura dei casi d’uso è così suddivisa:
\begin{itemize}
    \item \textbf{Codice identificativo:}
          \begin{center}
              \textbf{UC[CodiceBase](.[CodiceSottoCaso])}
          \end{center}
          dove:
          \begin{itemize}
              \item \textbf{UC}: acronimo di “use case”;
              \item \textbf{CodiceBase}: codice che identifica il caso d’uso generico;
              \item \textbf{CodiceSottoCaso}: codice opzionale che identifica gli eventuali sottocasi.
          \end{itemize}
    \item \textbf{Titolo}: titolo testuale del caso d’uso;
    \item \textbf{Diagramma UML}: rappresentazione grafica del caso d’uso realizzata impiegando il formalismo dell’UML 2.0\textsubscript{\textbf{G}}.
    \item \textbf{Attori}: rappresentano qualsiasi entità esterna al sistema e con il quale interagiscono. Esistono due tipologie di attori:\begin{itemize}
              \item \textbf{Attori primari}: interagiscono attivamente con il sistema per soddisfare le proprie esigenze;
              \item \textbf{Attori secondari}: attori il cui scopo è aiutare l’attore primario.
          \end{itemize}
    \item \textbf{Descrizione}: breve descrizione testuale del caso d’uso;
    \item \textbf{Scenario principale}: elenco puntato dei passi che compongono il caso d’uso;
    \item \textbf{Scenario alternativo (se presente)}: elenco puntato degli eventi che possono manifestarsi in seguito ad un evento imprevisto che causa una deviazione dallo scenario principale;
    \item \textbf{Inclusioni (se presenti)}: utilizzate quando due casi d’uso sono collegati e il secondo è incluso nel primo;
    \item \textbf{Estensioni (se presenti)}: utilizzate per modellare scenari alternativi. Al verificarsi di una determinata condizione il caso d’uso ad essa collegata viene interrotto;
    \item \textbf{Generalizzazioni (se presenti)}: utilizzate per modellare specializzazioni dei casi d’uso;
    \item \textbf{Precondizione}: espone le condizioni in cui si trova il sistema prima del verificarsi dell’evento degli eventi del caso d’uso;
    \item \textbf{Postcondizione}: espone le condizioni in cui si trova il sistema al termine degli eventi del caso d’uso;
    \item \textbf{Input (se presenti)}: valore o oggetto che l’attore porta nel sistema;
    \item \textbf{Output (se presenti)}: valore o oggetto risultato di un’azione di un attore.
\end{itemize}
\subsection{Progettazione}
\subsubsection{Scopo}
La fase di progettazione ha come scopo la ricerca di una soluzione architetturale in grado di soddisfare i requisiti degli \textit{stackeholders}, esposti nel documneto \textit{Analisi dei Requisiti}.
I \textit{progettisti} devono tenere conto che questa fase deve garantire un'approccio sistematico\textsubscript{\textbf{G}} ai problemi, suddividendoli, se necessario, in sotto-problemi più semplici da risolvere.
In questo modo la parte di codifica del codice è semplificata e ottimizzata.
\subsubsection{Descrizione}
La fase di proggettazione si suddivide in due parti:
\begin{itemize}
    \item\textbf{Technology Baseline}: rappresenta la linea base per lo sviluppo e indica tutte le specifiche di alto livello dell'architettura. Deve contenere:
          \begin{itemize}
              \item le tecnologie utilizzate;
              \item le componenti del prodotto (sempre ad alto livello);
              \item diagrammi UML utilizati;
              \item test di verifica.
          \end{itemize}
    \item\textbf{Product Baseline}: integra la \textit{Technology Baseline} entrando più nello specifico. Deve includere:
          \begin{itemize}
              \item diagrammi delle classi;
              \item ulteriori test di verifica.
          \end{itemize}
\end{itemize}
\subsubsection{Aspettative}
Il lavoro dei \textit{progettisti} deve produrre una soluzione architetturale che presenti le seguenti caratteristische:
\begin{itemize}
    \item soddisfare a pieno i requisiti degli \textit{stackeholders};
    \item identificare varie componenti distinte, facilmente comprensibili e coese;
    \item essere affidabile, garantendo il corretto funzionamento del prodotto;
    \item avere una gestione degli errori robusta;
    \item avere una gestione delle risorse ottimizzata;
    \item essere facilmente manutenibile.
\end{itemize}
\subsubsection{Design Pattern}
Una volta stilata l'\textit{Analisi dei Requisiti}, ai \textit{progettisti} verrà affidato il compiti di trovare soluzioni architetturali ai problemi più frequenti.
Tutti i design pattern\textsubscript{\textbf{G}} non devono essere troppo specifici, in modo da permettere ai \textit{programmatori} di essere abbastanza flessibili nel momento della stesura del codice.
Il significato e la struttura di un design pattern\textsubscript{\textbf{G}} vengono rappresentati con diagrammi e una descrizione.
\subsubsection{Diagrammi UML 2.0}
Per facilitare la comprensione delle soluzioni architetturali pensate dai progettisti, questi devono utilizzare diagrammi UML 2.0.
Nello specifico:
\begin{itemize}
    \item \textbf{Diagrammi di classi}: illustra le classi, comprensive di metodi e membri;
    \item \textbf{Diagrammi delle attività}: illustra il flusso di un attività;
    \item \textbf{Diagrammi di sequenza}: illustra sequenza di azioni attraverso l'impego di scelte definite;
    \item \textbf{Diagrammi dei moduli}: illustra il raggruppamento di classi mei moduli dell'applicazione.
\end{itemize}
\subsubsection{Test}
Durante la fase di progettazione devono essere definiti test opportuni, in coerenza con le richieste dei proponenti.
Tali test devono poter individuare errori o anomalie nell'architettura, logica o codice dell'applicazione.

\subsection{Codifica}
\subsubsection{Scopo}
La fase di codifica ha come scopo la stesura del codice stesso, e quindi dell'effettiva realizzazione del prodotto.
Vengono quindi implementate le specifiche esposte nella fase di progettazione, trasformandole in codice eseguibile.
\subsubsection{Descrizione}
Ogni \textit{programmatore} deve rispettare un'insieme di regole stilistiche di stesura del codice, in modo da renderlo uniforme e fruibile.
\subsubsection{Aspettative}
L'obiettivo della fase di codifica è la scrittura del codice del prodotto in modo che questo sia conforme alle richieste del proponente.
Stabilire delle regole stilistiche in questa fase ha lo scopo di migliorare la qualità del prodotto finito, nonché di renderene più semplice la manutenzione e verifica.
\subsubsection{Regole di Stile}
Ogni \textit{programmatore} deve attenersi a queste regole stilistiche:
\begin{itemize}
    \item \textbf{Lingua}: ogni riga di codice deve obbligatoriamente essere scritta in inglese. Quindi anche commenti, nomi di varibili e classi.
    \item \textbf{Classi}: le classi devono evere queste caratteristiche:
          \begin{itemize}
              \item nome in PascalCase\textsubscript{\textbf{G}}.
              \item gli attributi vengono scritti prima dei metodi, e gli attributi privati prima di quelli pubblici;
              \item i metodi privati devono venire dichiarati prima di quelli pubblici;
              \item i nomi dei metodi e attributi non devono contenere un riferimento alla classe alla quale appartengono. Esempio sbagliato: nome classe: Persona, nome attributo: CognomePersona.
          \end{itemize}
    \item \textbf{Variabili}: le variabili vanno scritte in camelCase\textsubscript{\textbf{G}}, precedute da un \_ se sono attributi di una classe;
    \item \textbf{Costanti}: le costanti vanno scritte in maiuscoletto, precedute da un \_ se sono attributi di una classe;
    \item \textbf{Metodi}: i metodi vanno scritti in camelCase\textsubscript{\textbf{G}}, con l'apertura della prima parentesi graffa in linea con la firma. Il codice all'interno deve avere una sola tabulazione.
    \item Altre regole:
          \begin{itemize}
              \item \textbf{Un solo livello di indentazione per metodo}: ogni metodo deve contenere righe di codice con massimo una tabulazione. Se il codice ne richiede più di una, allora tale parte va estratta in un nuovo metodo;
              \item \textbf{Non abbreviare}: non abbreviare il nome di variabili, metodi e classi;
              \item \textbf{Callback}: non utilizzare mai metodi di callback\textsubscript{\textbf{G}};
              \item \textbf{Un solo punto per linea}: se possibile, ogni riga di codice deve contenere al massimo un punto;
              \item \textbf{Ricorsione}: se possibile, evitare di fare metodi ricorsivi\textsubscript{\textbf{G}}.
          \end{itemize}
\end{itemize}
\subsubsection{Strumenti}
Di seguito gli strumenti utilizzati:
\begin{itemize}
    \item \textbf{Visual Studio Code}: IDE\textsubscript{\textbf{G}} (Integrated Development Environment) adatto alla codifica in diversi linguaggi di programmazione;
    \item \textbf{Draw.io}: piattaforma online utile alla creazione di diagrammi UML;
    \item \textbf{Next.js}: framework web di sviluppo front-end\textsubscript{\textbf{G}} React\textsubscript{\textbf{G}} open source\textsubscript{\textbf{G}} che abilita funzionalità come il rendering lato server e la generazione di siti web statici per applicazioni web basate su React. Fornisce possibilità di integrazione con Typescript;
    \item \textbf{TypeScript}: typeScript è un Super-set di JavaScript. Il linguaggio estende la sintassi di JavaScript in modo che qualunque programma scritto in JavaScript sia anche in grado di funzionare con TypeScript senza nessuna modifica;
    \item \textbf{ESLint}: strumento che permette di verificare che il codice TypeScript\textsubscript{\textbf{G}} scritto rispetti correttamente la sintassi di linguaggio;
    \item \textbf{Serverless Framework}: framework Web gratuito e open source scritto utilizzando Node.js\textsubscript{\textbf{G}}. Serverless è il primo framework sviluppato per la creazione di applicazioni su AWS Lambda\textsubscript{\textbf{G}};
    \item \textbf{AWS Lambda}: servizio serverless\textsubscript{\textbf{G}} offerto da Amazon\textsubscript{\textbf{G}} che permette di eseguire codice senza effettuare il provisioning\textsubscript{\textbf{G}} o gestire i server;
    \item \textbf{Amazon CoudWatch}: offre un servizio gestito da Amazon di monitoraggio e osservabilità in grado di monitorare le applicazioni, rispondere ai cambiamenti di prestazioni a livello di sistema, ottimizzare l'utilizzo delle risorse e ottenere una visualizzazione unificata dello stato di integrità operativa;
    \item \textbf{Amazon DynamoDB}: servizio di database NoSQL\textsubscript{\textbf{G}} facente parte dei servizi offerti da Amazon Web Services;
    \item \textbf{Stripe}: servizio che fornisce un'infrastruttura software che permette a privati e aziende di inviare e ricevere pagamenti via Internet.
\end{itemize}