\section{Processi Organizzativi}
\subsection{Coordinamento}
\subsubsection{Comunicazione Interna}
discord
telegram
\subsubsection{Comunicazione Esterna}
email
slack
\subsubsection{strumenti aggiuntivi}
asana
\subsubsection{Riunioni}
\subsection{Pianificazione}
\subsubsection{Scopo}
Lo scopo della seguente sezione è quello di chiarire come il gruppo TechSWEave intende
pianificare il lavoro, a partire dalla scelta dei ruoli, sino alla concreta assegnazione dei
compiti ai vari partecipanti.
Il processo di pianificazione, in accordo con lo standard ISO/IEC 12207\textsubscript{\textbf{G}}, è strutturato nella seguente maniera:
\begin {itemize}
    \item ruoli di progetto;
    \item assegnazione dei ruoli;
    \item ciclo di vita del ticket;
\end {itemize}
\subsubsection{Ruoli di progetto}
Nel corso del progetto, i componenti del gruppo ricopriranno i seguenti ruoli:
\begin {itemize}
    \item Responsabile di progetto;
    \item Amministratore di progetto;
    \item Analista;
    \item Progettista;
    \item Programmatore;
    \item Verificatore;
\end {itemize}
Essi corrispondono alle rispettive figure aziendali, e sarà stato stabilito un calendario che permetterà ad ogni membro di ricoprire almeno una volta ciascun ruolo per un periodo di tempo omogeneo, senza peró gravare sullo svolgimento delle attività. L’assegnazione di un ruolo comporta lo svolgimento di determinati compiti, così come previsto dal Piano di progetto\textsubscript{\textbf{G}}. Inoltre si cercherà di eliminare eventuali conflitti di interesse: per esempio, un componente non potrà redigere e poi verificare ciò che ha prodotto lui stesso.
\subsubsubsection{Responsabile di progetto}
Il responsabile di progetto\textsubscript{\textbf{G}} è una figura importante in quanto ricadono su di lui le responsabilità di pianificazione, gestione, controllo e coordinamento. Un altro suo compito è quello di fare da intermediario nella comuncazione tra il gruppo e i soggetti esterni: sono quindi di sua competenza le comunicazioni con committente e proponente.
Le mansioni legate a questo ruolo sono le seguenti:
\begin {itemize}
    \item Assegnare i ruoli ai componenti del gruppo;
    \item Gestire il coordinamento dei membri del gruppo;
    \item Gestire la pianificazione, intesa come attività da svolgere e scadenze da rispettare;
    \item Essere responsabile della stima dei costi e dell’analisi dei rischi;
    \item Analizzare e gestire le criticità;
    \item Approvare la documentazione;
    \item Curare le relazioni tra il gruppo e i soggetti esterni;
\end {itemize}
\subsubsubsection{Amministratore di progetto}
L’Amministratore di Progetto\textsubscript{\textbf{G}} è incaricato di gestire, controllare e curare gli strumenti che il gruppo utilizza per svolgere il proprio lavoro. È la figura che garantisce l’affidabilità e l’efficacia dei mezzi scelti dal gruppo.\\
I suoi compiti sono i seguenti:
\begin {itemize}
    \item Gestire il versionamento e la configurazione dei prodotti; 
    \item Gestire e salvaguardare la documentazione, controllando che sia corretta, verificata ed approvata e semplificando il suo reperimento;
    \item correggere eventuali problemi legati alla gestione dei processi;
    \item Amministrare le infrastrutture e i servizi necessari ai processi di supporto;
    \item Individuare strumenti utili all'automazione di processi;
    \item Redigere e manutenere le norme e procedure che regolano il lavoro;
\end {itemize}
\subsubsubsection{Analista}
\subsubsubsection{Progettista}
\subsubsubsection{Programmatore}
\subsubsubsection{Verificatore}
\subsubsection{Assegnazione dei compiti}
\subsubsection{Ciclo di vita del ticket}
