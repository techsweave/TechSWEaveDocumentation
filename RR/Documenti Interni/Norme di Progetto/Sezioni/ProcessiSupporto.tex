\section{Processi di Supporto}
    \subsection{Documentazione}
    Ogni attività significativa per lo sviluppo del progetto viene documentata.
        \subsubsection{Scopo}
        Lo scopo di questa sezione è di annotare le norme che regolano il processo di documentazinoe durante tutto il ciclo di vita del software, in modo che tutti i documenti risultino coerenti e validi.
        \subsubsection{Descrizione}
        Questa sezione contiene le norme per la corretta stesura, verifica e approvazione di tutti i documenti prodotti dal gruppo. Ogni membro è tenuto a rispettare tutto ciò che è esposto.
        \subsubsection{Aspettative}
        Le attese, riguardo il processo in questione, sono le seguenti:
        \begin{itemize}
            \item individuazione di una struttura comune a tutti i prodotti del processo, nell’arco delciclo di vita del software;
            \item collezione di tutte le norme comuni per la stesura di documenti ufficiali.
        \end{itemize}
        \subsubsection{Ciclo di vita di un documento}
        Ogni documento attraversa le seguenti fasi del ciclo di vita:
        \begin{itemize}
            \item \textbf{Creazione:} il documento viene creato utilizzano un template situato nella cartella \textit{Template} del repository remoto, con la prima pagina e l'intestazione della tabella contenente il diario nelle modifiche;
            \item \textbf{Strutturazione:} viene iniziato a popolare il diario delle modifiche nella seconda pagina e viene creato l'indice;
            \item \textbf{Stesura:} il corpo del documento viene scritto da più membri del gruppo usando un metodo incrementale, ad ogni modifica viene aggiornato il diario delle modifiche.
            \item \textbf{Revisione:} ogni singola sezione del corpo del documento viene regolarmente rivistada almeno un membro del gruppo, che deve essere obbligatoriamente diverso dalredattore della parte in verifica; se necessario, la verifica può essere svolta da piùpersone: in questo caso può partecipare anche chi ha scritto la sezione in verifica apatto che non si occupi della parte da esso redatta;
            \item \textbf{Approvazione:} terminata la revisione, ilResponsabile di Progettostabilisce la vali-dità del documento, che solo a questo punto può essere considerato completo e puòessere quindi rilasciato.
        \end{itemize}
        \subsubsection{Template}
        Il gruppo ha creato un Template con \LaTeX per uniformare velocemente la struttura grafica e lo stile di tutti i documenti, in modo che tutti si possano concentrare sulla stesura del corpo del documento. Il template permette di adottare automaticamente le conformità previste dalle \textit{Norme di Progetto}.
        \subsubsection{Struttura di un documento}
            \subsubsubsection{Prima pagina}
            Il frontespizio è la prima pagina del documento e contiene:
            \begin{itemize}
                \item \textbf{Logo del gruppo:} posto in alto in centro
                \item \textbf{Titolo del documento:} in centro alla pagina
                \item \textbf{e-mail del gruppo:} appena sotto al titolo del documento
                \item \textbf{Informazioni sul documento:}
                    \begin{itemize}
                        \item \textbf{Versione:} indica la versione corrente del documento;
                        \item \textbf{Responsabile:} il nome del \textit{Responsabile di Progetto} che deve approvare il documento;
                        \item \textbf{Redattori:} i membri del gruppo che si occupano della stesura del documento; 
                        \item \textbf{Verificatore:} i membri del gruppo che si occupano di verificare l'operato dei Redattori;
                        \item \textbf{Destinatari:} i destinatari del documento;
                        \item \textbf{Stato:} ndica se il documento è stato approvato dal \textit{Responsabile di Progetto};
                        \item \textbf{Uso:} indica se il documento è ad uso interno o esterno.
                    \end{itemize} 
            \end{itemize}
            \subsubsubsection{Diario delle modifiche}
            Ogni documento presenta un registro delle modifiche, sotto forma di tabella, che tienetraccia di tutte le modifiche significative apportate al documento durante le fasi del suo ciclodi vita. Ogni voce della tabella riporta:
            \begin{itemize}
                \item \textbf{Modifica:}una sintetica descrizione della modifica apportata;
                \item \textbf{Autore:} il nome dell’autore della modifica;
                \item \textbf{Ruolo:} il ruolo ricoperto dall’autore quando ha eseguito la modifica;
                \item \textbf{Data:} la data in cui è stata apportata tale modifica;
                \item \textbf{Versione:} la versione del documento dopo la modifica.
            \end{itemize}
            \subsubsubsection{Indice}
            L’indice delle sezioni fornisce al fruitore una visione complessiva della struttura deldocumento, permette di orientarsi tra i contenuti e di individuare la posizione delle varieparti.Ogni documento presenta un indice dei contenuti, subito dopo il registro delle modifiche;dove necessario, sono presenti anche un indice delle illustrazioni e uno delle tabelle presentinel documento.
            \subsubsubsection{Contenuto principale}
            Le pagine sono così strutturate:
            \begin{itemize}
                \item in alto a sinistra è presente il logo del gruppo;
                \item in alto a destra è presente il titolo del documento;
                \item una riga divide il contenuto dalla intestazione;
                \item il contenuto è tra l'intestazione e il piè di pagina;
                \item un'altra rigi divide il contenuto dal piè di pagina;
                \item in basso in centro è presente il numero di pagina corrente.
            \end{itemize}
            \subsubsubsection{Verbali}
            I verbali vengono prodotti da un componente del gruppo in occasione di incontri tra i membri con o senza la presenza di esterni. È prevista un'unica stesura. I verbali seguono la struttura degli altri documenti, ma contengono un'introduzione che contiene:
            \begin{itemize}
                \item \textbf{Data e ora:} data e orario in cui è iniziata la riunione;
                \item \textbf{Luogo:} luogo in cui si è svolta la riunione;
                \item \textbf{Partecipanti interni:} elenco dei partecipanti del gruppo presenti alla riunione;
                \item \textbf{Partecipanti esterni:} elenco dei partecipanti esterni al gruppo presenti alla riunione; 
                \item \textbf{Segretario:} il membro del gruppo che prende appunti sulla riunione e fa un resoconto.
            \end{itemize}
            Ogni verbale viene denominato con il formato:
            \begin{center}
                \textbf{Verbale\_GGMMAAAA} \\
            \end{center}
            e mantenuto nella \textit{Directory} Documenti esterni se è relativo a una riunione esterna, nella \textit{Directory} Documenti interni, se si riferisce a una riunione interna
        \subsubsection{Norme tipografiche}
            \subsubsubsection{Convenzioni sui nomi dei file}
            I nomi dei file utilizzano la convenzione camelCase\textsubscript{\textbf{G}}, sono senza le proposizioni e non ci sono lettere accentate.
            Ad esempio sono corretti:
            \begin{itemize}
                \item analisiRequisiti;
                \item studioFattibilita;
                \item normeProgetto.
            \end{itemize}
            Mentre non sono corretti:
            \begin{itemize}
                \item AnalisiDeiRequisiti;
                \item studio\_fattibilità;
                \item normeprogetto.
            \end{itemize}
            \subsubsubsection{Stili del testo}
            
            \subsubsubsection{Glossario}
            \begin{itemize}
                \item Ogni termine nel glossario viene marcato con una \textbf{G} maiuscola a pedice;
                \item Se la voce è presente ripetutamente nello stesso paragrafo non è necessario marcarla ad ogni occorrenza, basta farlo alla prima occorrenza;
                \item Se la voce è presente come singolo termine in elenchi puntati non è necessario marcarlo. 
            \end{itemize}
            
            \subsubsubsection{Elenchi puntati}
            Ogni voce di un elenco termina con ";", tranne l'ultima che termina con ".". I sottoelenchi rispettano le stesse regole, essendo una funzione analoga. Se le voci sono della forma \textit{termine: descrizione}, i termini vanno in grassetto.
            \subsubsubsection{Formati dei dati}
            Le date vengono indicate usando il formato
            \begin{center}
                \textbf{[YYYY]-[MM]-[DD]}
            \end{center}
            \subsubsubsection{Sigle}
            Tutte le sigle sono indicate con le iniziali di ogni parola maiuscole, tranne per le proposizioni, congiunzioni e articoli.
        \subsubsection{Elementi grafici}
            \subsubsubsection{Imamgini}
            Le figure presenti nei documenti sono tutte centrate rispetto al testo e accompagnate dauna opportuna didascalia.
            \subsubsubsection{Tabelle}
            Ogni tabella è contrassegnata da una didascalia descrittiva del contenuto, posta sotto di essa centrata rispetto alla pagina.  Nella didascalia di ogni tabella viene indicato l’identificativo
            \begin{center}
                \textbf{Tabella [X]}
            \end{center}
            dove \textbf{[X]} indica il numero assoluto della tabella all’interno del documento; a questo il testodella didascalia. A questa prassi fanno eccezione le tabelle del registro delle modifiche, chenon hanno la didascalia.
            \subsubsubsection{Grafici UML}
            I grafici in linguaggio UML, usati per la modellazione dei casi\textsubscript{\textbf{G}} e per i diagrammi della progettazione\textsubscript{\textbf{G}}, sono inseriti come immagini.
        \subsubsection{Strumenti di stesura}
            \subsubsubsection{\LaTeX}
            Per la stesura dei documenti, il gruppoProApesha scelto \LaTeX, un linguaggio compi-lato basato sul programma di composizione tipografica \TeX, al fine di produrre documenticoerenti, ordinati, templatizzati e stesi in modo collaborativo.
            \subsubsubsection{Draw.io}
            Per la produzione di grafici UML è stato scelto il programma \textit{Draw.io}.
            \begin{center}
                \href{https://www.draw.io/}{https://www.draw.io/}\\
            \end{center}
    \subsection{Gestione della configurazione}
        \subsubsection{Scopo}
        Il processo ha lo scopo di andare ad ordinare il software e la documentazione. Tutto quello che è stato configurato si trova in un posto preciso e consciuto, con una denominazione ed uno stato ben definiti. Ogni modifica deve rispettare determinate norme e deve essere sottoposta a versionamento.
        \subsubsection{Descrizione}
        La gestione della configurazione raggruppa ed organizza tutti gli strumenti necessari al organizzazione della produzione di documenti, diagrammi e codice, ma anche di quello necessari al versionamento e al coordinamento del gruppo.
        \subsubsection{Versionamento}
        \subsubsubsection{Codice di versione}
        La storia di un documento è ricostruibile attraverso le sue versioni, consultabili nella tabella presente all'inizio di ogni documento. Le versioni sono identificate da un codice a tre cifre nel fromato: 
        \begin{center}
            \textbf{\large [X].[Y].[Z]}\\             
        \end{center}
        dove: 
        \begin{itemize}
            \item \textbf{\large X:} rappresenta la versione del documento approvata dal \textit{Responsabile di progetto}, che ne autorizza l'incremento. La numerazione parte da 0;
            \item \textbf{\large Y:} indica la versione approvata dal \textit{Verificatore}, che autorizza l’incremento del numero della versione; la numerazione inizia da 0, e riparte da questo valore ad ogni incrementodi \textbf{X};
            \item \textbf{\large Z:} indica una versione in fase di elaborazione da parte dei \textit{redattori}, che ne incrementano il valore ad ogni modifica; la numerazione inizia da 0 e riparte da tale valore adogni incremento di \textbf{X} o \textbf{Y}.
        \end{itemize}
        \subsubsubsection{Tecnologie adottate}
        Il versionamento è stato gestito attraverso il sistema di versionamento distribuito Git, con due repo situate in GitHub.
        \subsubsubsection{Repository}
        Le due repository create dal gruppo TechSWEave sono le seguenti:
        \begin{itemize}
            \item \textbf{\href{https://github.com/techsweave/TechSWEave.git}{TechSWEave}}: per il versionamento del codice;
            \item \textbf{\href{https://github.com/techsweave/TechSWEaveDocumentation.git}{TechSWEaveDocumentation}}: per il versionamento dei documenti redatti;
        \end{itemize}
        Per una maggiore chiarezza nell'organizzazione si è deciso di tenere separato la documentazione dal codice.
        \subsubsubsection{Struttura del repository}
        Per ognuno dei due repository, esisite una versione \textbf{remota} ed una \textbf{locale}, con la medemisa struttura: 
        \begin{itemize}
            \item \textbf{Remoto :} presente su \textit{GitHub}, contiene tutti i prodotti condivisi dai membri del gruppo;
            \item \textbf{Locale :} lanciando da terminale il comando {\fontfamily{qcr}\selectfont git clone [URLrepository]}, ogni componente del gruppo ha in locale sul suo pc una copia del repository su cui lavorare;
        \end{itemize} 
        Il repository \textbf{\href{https://github.com/techsweave/TechSWEave.git}{TechSWEave}} non ha ancora una struttura in quanto la scrittura di codice non è richiesta per la \textit{Revisone dei requisiti}. 
        Il repository  \textbf{\href{https://github.com/techsweave/TechSWEaveDocumentation.git}{TechSWEaveDocumentation}} ha invece la seguente struttura:
        \begin{itemize}
            \item \textbf{Template :} la cartella la cui interno è presente un template del documento che ogni \textit{redattore} può utilizzare per produrre i file;
            \item \textbf{Images :} la cartella contenente immagini e grafici da inserire poi nei documenti;
            \item \textbf{RR :} la cartella contenente tutti i documenti che dovranno essere presentati alla \textit{Revisione dei requisiti} divisi in \textit{esterni} ed \textit{interni}, la stessa divisione verrà adottata anche per le altre revisioni queli \textit{Revisoni di progettazione} \textbf{RP}, \textit{Revisione di qualità} \textbf{RQ} e \textit{Revisione di avanzamento} \textbf{RA}.
        \end{itemize}
        La divisione dei file in base alle revisioni favorisce l'organizzazione dei documenti ed una veloce stesura degli stessi.
        \begin{itemize}
            \item i file {\fontfamily{qcr}\selectfont .tex} ossia i sorgenti dei documenti scritti in \LaTeX;
            \item i file {\fontfamily{qcr}\selectfont .pdf} i documenti che verranno poi consegnati a commitenti e proponenti.
            \item le immagini e i grafici che verrano poi usati nei documenti.
        \end{itemize}
        Il file {\fontfamily{qcr}\selectfont .gitignore} è il file che non versiona determinati file, sono stati tolti dal versionamento i file non necesiari.
        \subsubsubsection{Comandi GitHub}
        L'approccio al verisonameno scelto è stato quello dello sviluppo per \textit{feature}\textsubscript{\textbf{G}}, questo ha fatto si che il lavoro sia stato diviso in vari branch:
        \begin{itemize}
            \item \textbf{Master :} il branch principale in cui sono presenti i documenti finali approvati dal \textit{Responsabile di progetto};
            \item \textbf{Develop :} il branch dove sono presenti i documenti terminati ma che devono essere ancora approvati dal \textit{Responsabile di progetto};
            \item Ogni documento, durante la sua redazione, appartiene ad un singolo branch nominato con lo stesso nome del documento. Il documento rimarrà all'interno del proprio branch fino a che non è stato verificato dal \textit{Verificatore};
        \end{itemize}
        Si è scelto un approccio per \textit{feature}, per facilitare l'organizzazione e lo sviluppo parallelo dei documenti. Per raggiungere quest'obbiettivo ogni componente del gruppo deve:
        \begin{itemize}
            \item Dalla propria repository locale deve posizionarsi nel branch corretto attraverso l'apposito comando;
            \item una volta arrivato nel branch corretto, il componente si deve mettere in pari con le, eventuali, modifiche apportate ai documenti;
            \item svolgere il compito assegnato, come la scrittura di nuovo materiale o la modifica di quello già esisitente, attraverso l'editor predisposto (\textit{Visual Studio Code});
            \item una volta terminato il proprio lavoro si deve aggiungere all'area di staging\textsubscript{\textbf{G}}, attraverso l'apposito comando;
            \item eseguire il comando per andare a fare il commit delle modifiche fatte, corredandole con un messaggio identificativo;
            \item caricare le modifiche nel repository remoto 
        \end{itemize}
        \subsubsubsection{Modifiche}
        Tutti i membri possono modificare tutti i documenti tranne quelli presenti all'interno del ramo master, per i quali è richiesta una \textit{pull request}\textsubscript{\textbf{G}}, che deve essere approvata da almeno un componente del gruppo.
        Per modifiche maggiori sui contenuti o modifiche alla struttura si deve:
        \begin{itemize}
            \item presentare la modifica che si vuole effettuare al responsabile motivando tale modifica;
            \item una volta accetta dal resposabile si può procedre con la modifica; 
        \end{itemize} 
        Per le modifiche minori come correzzioni grammaticali e sintattiche si possono fare modicihe autonome, ma che devono comunque essere tracciate attraverso appositi commit autoesplicativi.
    \subsection{Gestione della Qualità}
        \subsubsection{Scopo}
        Il processo di gestione della qualità ha lo scopo di garantire che il prodotto finale soddisfi gli obbiettivi di qualità e i requisiti richiesti dal proponente, attraverso un corretto controllo dei processi e dei prodotti.
        \subsubsection{Descrizione}
        Per l'esposizione approfondita della gestione della qualità si trova nel \emph{Piano di Qualifica v1.0.0}, nel quale sono descritte le modalità utilizzate per garantire la qualità nello sviluppo del progetto. Tale documento contiente i seguenti punti:
        \begin{itemize}
            \item sono esposti gli standard utilizzati;
            \item sono indicati i processi di interesse negli standard;
            \item sono specificati gli attributi del software.
        \end{itemize}
        Per ogni processo vengono descritti:
        \begin{itemize}
            \item gli obbiettivi da raggiungere;
            \item gli standard da applicare;
            \item le metriche da utilizzare.
        \end{itemize}
        Lo scopo è di controllare e migliorare i processi e i risultati. \\
        Ad ogni prodotto viene associato:
        \begin{itemize}
            \item gli obbiettivi prefissati;
            \item le metriche utilizzate.
        \end{itemize} 
        In questo modo si vuole ottenere software e documentazione di qualità soddisfacente.
        \subsubsection{Attività}
        Le attività principali del processo di gestione della qualità sono:
        \begin{itemize}
            \item \textbf{Pianificazione: } Fissare degli obbiettivi di qualità, stabilire le strategie per raggiungerli e di conseguenza collocare le persone e le risorse nel modo migliore;
            \item \textbf{Valutazione: }attuare quanto pianificato, misurando e monitorando i risultati;
            \item \textbf{Reazione: } adattare le prorpie strategie, criteri e piani sulla base dei risultati otttenuti, comprendendo dove sia necessario apportare miglioramenti.
        \end{itemize}
        \subsubsection{Strumenti}
        Gli strumenti predefiniti per la qualità sono:
        \begin{itemize}
            \item quelli forniti dallo standard \textbf{ISO-12207};
            \item le metriche.
        \end{itemize}
    \subsection{Verifica}
        \subsubsection{Scopo}
        \subsubsection{Descrizione}
        \subsubsection{Attività}
            \subsubsubsection{Analisi}
            \subsubsubsection{Test}
    \subsection{Validazione}
        \subsubsection{Scopo}
        Il processo di validazione consesnte di decidere se il prodotto soffisfa il compito per cui è stato generato. L'esito positivo di tale processo ci garantisce che il software rispetti i requisiti e che soddisfi i bisogni del commitente.
        \subsubsection{Descrizione}
        Tale processo consiste nel ricercare gli oggetti da validare e valutane i risultati rispetto le aspettative previste.
        \subsubsection{Attività}
        Le attività per attuare il processo di validazione sono:
        \begin{itemize}
            \item identificare gli oggetti da validare;
            \item identificare una strategia con delle procedute di validazione in cui le procedure di verifica possano essere riutilizzate;
            \item attuare la strategia;
            \item valutare che i risultati rispettino le aspettative.
        \end{itemize}