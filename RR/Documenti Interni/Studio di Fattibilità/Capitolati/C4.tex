\section{Capitolato C4 - HD Viz}

\subsection{Informazioni generali}

\begin{itemize}
\item Nome : HD Viz;
\item Proponente : Zucchetti S.p.A;
\item Committente : \emph{Prof Tullio Vardanega} e \emph{Prof Riccardo Cardin}.
\end{itemize}

\subsection{Descrizione capitolato}
Il capitolato richiede di realizzare un'applicazione per la
visualizzazione di dati con molte dimensioni a supporto della fase esplorativa
dell'analisi dei dati. L'obbiettivo attraverso l'uso di tale applicazione
è di ottenere visualizzazioni più semplici e comprensibili per l'utente finale. 

\subsection{Finalità del progetto}
Allo scopo di sviluppare modelli semplificati che evidenzino delle anomalie in alcuni casi analizzati sarà necessario l'utilizzo di algoritmi di AI che agiscono sui concetti che variano dal grafico preso in analisi.
 
\subsection{Tecnologie}
Le tecnologie richieste per la realizzazione del prodotto sono 
\begin{itemize}
\item HTML/CSS/JavaScript utilizzando la libreria D3.js per la realizzazione del lato Front-End dell'applicazione;
\item Tomcat o Node.js per lo sviluppo di un server database(SQL o NoSQL), in supporto al lato browser
\end{itemize}

\subsection{Aspetti positivi}
\begin{itemize}
\item Tecnologie come HTML/CSS/JAvaScript sono state già utilizzate in corsi e progetti precedenti.
\item La libreria D3.js fornisce già alcune delle visualizzazioni richieste.
\item il capitolato tratta l'implementazione di un'intelligenza artificiale aspetto molto interessante.
\item Lo sviluppo di applicazioni per l'analisi dei Big Data é un problema molto comune e affrontato in molti settori oggi giorno e può risultare utile in ambito professionale futuro.
\end{itemize}

\subsection{Aspetti negativi}
\begin{itemize}
\item La libreria D3.js é molto ampia e per imparare a padroneggiarla può essere richiesto molto tempo.
\item Lavorando con dati grandi dimensioni comporta si potranno riscontrare difficoltà nei test degli algoritmi.
\item Altri capitolati presentano scenari di lavoro più concreti rispetto all'analisi ed elaborazione dei dati.
\end{itemize}


\subsection{Conclusioni}
Il capitolato non presenta più posti disponibili, ma il gruppo é comunque interessato a lavorare su scenari differenti, non sarebbe quindi stata una delle prime scelte.