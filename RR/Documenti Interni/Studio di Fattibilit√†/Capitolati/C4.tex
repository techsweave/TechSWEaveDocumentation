\section{Capitolato C4 - HD Viz}

\subsection{Informazioni generali}

\begin{itemize}
\item \textbf{Nome:} \textit{HD Viz};
\item \textbf{Proponente:} \textit{Zucchetti S.p.A};
\item \textbf{Committente:} \emph{Prof Tullio Vardanega}, \emph{Prof Riccardo Cardin}.
\end{itemize}

\subsection{Descrizione capitolato}
Il capitolato richiede di realizzare un'applicazione per la
visualizzazione di dati con molte dimensioni a supporto della fase esplorativa
dell'analisi dei dati. L'obbiettivo è di ottenere visualizzazioni più semplici e comprensibili per l'utente finale. 

\subsection{Finalità del progetto}
Allo scopo di sviluppare modelli semplificati che evidenzino delle anomalie in alcuni casi analizzati sarà necessario l'utilizzo di algoritmi di AI\textsubscript{\textbf{G}} che agiscono sui concetti che variano dal grafico preso in analisi.
 
\subsection{Tecnologie}
Le tecnologie richieste per la realizzazione del prodotto sono 
\begin{itemize}
\item HTML/CSS/JavaScript utilizzando la libreria D3.js per la realizzazione del lato Front-End dell'applicazione;
\item Tomcat o Node.js per lo sviluppo di un server database(SQL o NoSQL), in supporto al lato browser.
\end{itemize}

\subsection{Aspetti positivi}
\begin{itemize}
\item Tecnologie come HTML/CSS/JAvaScript sono state già utilizzate in corsi e progetti precedenti;
\item La libreria D3.js fornisce già alcune delle visualizzazioni richieste;
\item Il capitolato tratta l'implementazione di un'intelligenza artificiale, aspetto ritenuto molto interessante dai membri del gruppo;
\item Lo sviluppo di applicazioni per l'analisi dei Big Data é un problema comune e affrontato in molti settori e può risultare utile in ambito professionale in futuro.
\end{itemize}

\subsection{Aspetti negativi}
\begin{itemize}
\item La libreria D3.js é molto ampia e imparare a padroneggiarla potrebbe richiedere molto tempo;
\item Lavorando con dati di grandi dimensioni si potranno riscontrare difficoltà nei test degli algoritmi;
\item Altri capitolati presentano scenari di lavoro più concreti rispetto all'analisi ed elaborazione dei dati.
\end{itemize}


\subsection{Conclusioni}
Il tema proposto nel capitolato è interessante ma il gruppo é maggiormente interessato a lavorare su scenari differenti; 
inoltre il capitolato non presenta più posti disponibili. Per questi motivi il capitolato in esame è stato scartato.