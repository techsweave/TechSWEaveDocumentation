\section{Capitolato C3 - Gathering Detection Platform}
\subsection{Informaizoni genarli}
\begin{itemize}
    \item \textbf{Titolo:} \textit{Gathering Detection Platform};
    \item \textbf{Proponente:} \textit{Sync Lab};
    \item \textbf{Commitenti:} \textit{Prof. Tullio Vardanega, Prof. Riccardo Cardin}.
\end{itemize}
\subsection{Descrizione del capitolato}
Il capitolato richiede un software che, ricevendo dati da dispositivi installati ed operativi in specifiche zone, 
vada ad identificare il possibile formarsi di flussi di utenti, con l'intento di evitare assembramenti per scongiurare il diffondersi di Covid-19.
L'interfaccia grafica dell'applicazione sarà caratterizzata da una presenza di \textit{heat-map}\textsubscript{\textbf{G}} per indicare l'eventuale presenza di assembramenti in una determinata area.
\subsection{Finalità del progetto}
Il progetto consiste nella realizzazione di una web-app collegata ad un server dotato di interfaccia grafica. All'interno di questo server verrano salvati tutti i dati ricevuti dai dispositivi analizzati 
(il numero di persone presenti all'interno di un mezzo di trasporto pubblico, l'affluenza media in un centro commerciale in un determinato orario, flussi prenotazioni Uber, contapersone etc...). 
Con questi dati attraverso algoritmi di machine learning e tecniche di \textit{Predictive Analytics}\textsubscript{\textbf{G}} verranno create delle stime da fornire poi agli utenti dell'applicazione. 
Queste stime potranno essere visualizzate dagli utenti attraverso heat-map, grafici e previsioni che usarano per evitare le zone di maggiore rischio.
\subsection{Tecnologie interessate}
\begin{itemize}
    \item \textbf{Java e Angular\textsubscript{\textbf{G}}:} per sviluppare sia le parti di Back-end che di Front-end della Web Application;
    \item \textbf{Leaflet:} framework per la gestione delle mappe (ad esempio per le heatmap);
    \item \textbf{Protocolli asincroni} per le comunicazioni tra i componenti;
    \item \textbf{Pattern Publisher/Subscriber};
    \item \textbf{MQTT\textsubscript{\textbf{G}}} (Message Queue Telemetry Transport): protocollo di messaggistica open, di facile interpretazione e leggero, molto diffuso in applicazioni M2M\textsubscript{\textbf{G}} e IoT\textsubscript{\textbf{G}}.
\end{itemize}

\subsection{Aspetti positivi}
\begin{itemize}
    \item L'argomento proposto è importante per la situazione attuale nella lotta alla pandemia globale;
    \item Il gruppo ritiene che sarebbe stato interessante sviluppare e studiare tecnologie di machine learning.
\end{itemize}

\subsection{Criticità e fattori di rischio}
\begin{itemize}
    \item Alcune tecnologie proposte dal capitolato non hanno riscosso grande interesse nella maggior parte del gruppo;
    \item La mole di lavoro da svolgere è sembrata molto più grande rispetto ad altri capitolati;
    \item Il capitolato è già stato scelto dal numero massimo di gruppi e quindi risulta più disponibile.
\end{itemize}
\subsection{Conclusione}
Il capitolato non è più disponibile e richiede l'apprendimento di tecnologie e la realizzazione di funzionalità complesse. Per questi motivi, 
nonostante l'argomento trattato sia particolarmente interessante e utile nel contrasto alla pandemia, il gruppo ha dovuto 
desistere dallo scegliere il capitolato in esame.