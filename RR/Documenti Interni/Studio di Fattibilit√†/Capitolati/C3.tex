\section{Capitolato C3 - Gathering Detection Platform}
\subsection{Informaizoni genarli}
\begin{itemize}
    \item \textbf{Titolo:} \textit{Gathering Detection Platform};
    \item \textbf{Proponente:} \textit{Sync Lab};
    \item \textbf{Commitenti:} \textit{Prof. Tullio Vardanega, Prof. Riccardo Cardin};
\end{itemize}
\subsection{Descrizione del capitolato}
Il capitoilato richiede un software che ricevendo dati da dispositivi installati ed operativi in specifiche zone, 
vada ad identificare il possibile formarsi di flussi di utenti, con l'intento di evitare assembramenti per scongiurare il diffondersi di Covid-19.
L'interfaccia grafica dell'applicazione sarà caratterizzata da una presenza di \textit{heat-map}\textsubscript{\textbf{G}} per indicare l'eventuale presenza di assembramenti in una determinata area.
\subsection{Finalità del progetto}
Il progetto consiste nella realizzazione di una web-app collegata ad un server dotato di GUI\textsubscript{\textbf{G}}. All'interno di questo server verrano salvati tutti i dati ricevuti dai dispositivi analizzati (il numero di persone presenti all'interno di un mezzo di trasporto pubblico, l'affluenza media in un centro commerciale in un determinato orario, flussi prenotazioni Uber, contapersone etc...). Con questi dati attraverso algoritmi di machine learning\textsubscript{\textbf{G}} e tecniche di \textit{Predictive Analytics}\textsubscript{\textbf{G}} verranno create delle stime da fornire poi agli utenti dell'applicazione. 
Queste stime potranno essere visualizzate dagli utenti attraverso heat-map, grafici e previsioni che usarano per evitare le zone di maggiore rischio.
\subsection{Tecnologie interessate}
\begin{itemize}
    \item \textbf{Java e Angular:} per sviluppare sia le parti di Back-end\textsubscript{\textbf{G}} che di Front-end\textsubscript{\textbf{G}} della Web Application.
    \item \textbf{Leaflet:} framework per la gestione delle mappe (heatmao ecc.).
    \item \textbf{Protocolli asincroni} per le comunicazioni tra i componenti.
    \item \textbf{Pattern Publisher/Subscriber}.
    \item \textbf{MQTT} (Message Queue Telemetry Transport): protocollo di messaggistica open, di facile interpretazione e leggero, molto diffuso in applicazioni M2M e IoT.
\end{itemize}

\subsection{Aspetti positivi}
\begin{itemize}
    \item L'argomento proposto è importante per la situazione attuale nella lotta alla pandemia globale.
    \item Il gruppo ritiene che sarebbe stato interessante sviluppare e studiare tecnologie di machine learning
\end{itemize}

\subsection{Criticità e fattori di rischio}
\begin{itemize}
    \item Alcune tecnologie proposte dal capitolato non hanno riscosso grande interesse nella maggior parte del gruppo.
    \item Il lavoro da svolgere per completare i requisiti è sembrato molto di più rispetto ad altri capitolati studiati.
    \item Il capitolato era già stato scelto dal numero massimo di gruppi e quindi non risultava più disponibile.
\end{itemize}
\subsection{Conclusione}
Il capitolato, in quanto membri del secondo lotto, non presenta più posti disponibili, inoltre non suscitava l'interesse del gruppo in quanto presenta tecnologie non conosciute, 
quali machine learning\textsubscript{\textbf{G}} e kafka\textsubscript{\textbf{G}}. Inoltre il capitolato ha funzionalità molto estese che avrebbero aumentato il lavoro da svolgere per il gruppo. Tutto questo ha fatto desistere il gruppo dal scegliere il capitolato 
anche se l'argomento trattato rappresentava un grande interesse in quanto ha il proposito di aiutare a contrastare la pandemia Covid-19.