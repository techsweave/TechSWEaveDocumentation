\section{Capitolato C5 - PORTACS}
\subsection{Informaizoni genarli}
\begin{itemize}
    \item \textbf{Titolo:} \textit{PORTACS};
    \item \textbf{Proponente:} \textit{Sanmarco Informatica};
    \item \textbf{Commitenti:} \textit{Prof. Tullio Vardanega, Prof. Riccardo Cardin}.
\end{itemize}
\subsection{Descrizione del capitolato}
Il capitolato richiede un software in grado di gestire, in real-time, una serie di unità con dei punti di interesse da raggiungere evitando le collisioni con altre unità del sistema.
Presenta anche una interfaccia utente che consiste in una mappa in cui sono indicate tutte le unità controllate e le direzioni consigliate alle stesse,
di cui saranno indicati: id, un pulsante stop/start, velocità attuale.
\subsection{Finalità del progetto}
La finalità del progetto è quella di realizzare una web-app che ricevuti degli input, possa creare una mappa, all'interno della quale censire le varie unità operative
(id, velocità massima e di crociera, posizione iniziale, lista dei Point Of Interest da raggiungere) e di consigliare
alle unità la prossima mossa da eseguire tenendo conto delle dimesioni delle stesse e di eventuali collisioni con altre unità presenti all'interno del sistema
\subsection{Tecnologie interessate}
\begin{itemize}
    \item La scelta delle tecnologie da utilizzare è stata lasciata agli sviluppatori del progetto, sono comunque stati indicati alcune tecnologie:
          \begin{itemize}
              \item A seconda dell'approccio scelto per il real-time sono stati consigliati determinati linguaggi derivati del C o linguaggi multi-thread;
              \item Per l'interfaccia utente sono stati consigliati framework come React\textsubscript{\textbf{G}} ed Angular\textsubscript{\textbf{G}};
          \end{itemize}
    \item \textbf{Docker:} progetto open source\textsubscript{\textbf{G}} che automatizza il deployment di applicazioni all’interno di container.
\end{itemize}
\subsection{Aspetti positivi}
\begin{itemize}
    \item L'obbiettivo del progetto è in linea con le problematiche tecnologiche attuali;
    \item Ha una grande valenza formativa per i componenti del gruppo.
\end{itemize}
\subsection{Criticità e fattori di rischio}
\begin{itemize}
    \item Il capitolato non presenta più posti disponibili.
\end{itemize}
\subsection{Conclusioni}
Il capitolato, per quanto di forte interesse e utilità, non presenta posti disponibili e non è stato quindi possibile sceglierlo.