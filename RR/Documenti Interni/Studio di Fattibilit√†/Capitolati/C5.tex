\section{Capitolato C5 - PORTACS}
\subsection{Informaizoni genarli}
\begin{itemize}
    \item \textbf{Titolo:} \textit{PORTACS};
    \item \textbf{Proponente:} \textit{Sanmarco Informatica};
    \item \textbf{Commitenti:} \textit{Prof. Tullio Vardanega, Prof. Riccardo Cardin};
\end{itemize}
\subsection{Descrizione del capitolato}
Il capitolato richiede di un software in grado di gestire un ente centrale che riceve dati 
di posizioni e velocita da varie unità e ne regola gli spostementi. L'applicazione avrà una \textit{UI} che inicherà 
la posizione, la velocita di ogniu unità e la direzione che questa dovrà prendere.
\subsection{Finalità del progetto}
La finalità del progetto è quella di realizzare una applicazione in grado di gestire, lo spostamento di un unità all'interno di un area controllata,
attraverso la consocenza di alcuni dati e con l'obbiettivo di minimizzare costi e tempi di viaggio e di evitare eventuali collisioni
\subsection{Tecnologie interessate}
\begin{itemize}
    \item \textbf{Docker:} progetto open-source che automatizza il deploymentGdi applicazioni all’in-terno di container.
\end{itemize}
\subsection{Aspetti positivi}
\begin{itemize}
    \item L'obbiettivo del progetto è in linea con le problematiche tecnologiche attuali;
    \item Ha una grande valenza formativa per i componenti del gruppo
\end{itemize}
\subsection{Criticità e fattori di rischio}
\begin{itemize}
    \item Il capitolato non presentatava più posti disponibili;
\end{itemize}
\subsection{Conclusioni}
Il capitolato per quanto di forte interesse e utilità non presenta posti disponibili e non è stato quindi possibile sceglierlo.