\section{Capitolato C7 - soluzioni di sincronizzazione desktop}
\subsection{Informazioni generali}
\begin{itemize}
    \item \textbf{Titolo:} \textit{SSD};
    \item \textbf{Proponente:} \textit{Zextras};
    \item \textbf{Commitenti:} \textit{Prof. Tullio Vardanega, Prof. Riccardo Cardin};
\end{itemize}
\subsection{Descrizione del capitolato}
Il capitolato richiede lo sviluppo di un software per la sincronizzazione in cloud di file salvati o modificati dagli utenti, puntando ad un target di utenza di tipo professionale.
\subsection{Finalità del progetto}
\begin{itemize}
    \item Sviluppare un algoritmo in grado di garantire
    il salvataggio in cloud del lavoro e contemporaneamente la sincronizzazione
    delle sue modifiche;
    \item Sviluppare un’interfaccia multipiattaforma per l’uso
    del suddetto algoritmo nei sistemi operativi più utilizzati (MacOS,
    Windows, Linux);
    \item  Utilizzare l’algoritmo sviluppato per richiedere e fornire i
    cambiamenti ai contenuti in sincronizzazione verso il prodotto Zextras
    Drive.
\end{itemize}
\subsection{Tecnologie interessate}
Le tecnologie indicate per lo sviluppo del progetto sono le seguenti:
    \begin{itemize}
        \item \textbf{Qt Framework:} basato sul linguaggio di programmazione C++ e consigliato per lo sviluppo dell'inferfaccia e del controller d'architettura. La scelta di un'architettura con un pattern di tipo MVC è vincolante;
        \item \textbf{Python}: da utilizzare come linguaggio per il lato backend e lo sviluppo della business logic, supporta l'integrazione con la maggior parte dei framework per lo sviluppo dell'interfaccia desktop. Lo sviluppo in Python includerà chiamate API ai sistemi di Zextras Drive.
    \end{itemize}
\subsection{Aspetti positivi}
\begin{itemize}
    \item L'obiettivo del progetto è in linea con le problematiche tecnologiche attuali, in un periodo in cui l'immagazzinamento di file in drive e la loro successiva modifica tramite diversi tipi di device sta prendendendo sempre più piede;
    \item Ha una grande valenza formativa per i componenti del gruppo.
\end{itemize}
\subsection{Criticità e fattori di rischio}
\begin{itemize}
    \item La difficoltà nell'integrazione dei processi di sincronizzazione;
    \item La potenziale mole di lavoro molto grande rispetto alle risorse disponibili per il gruppo.
\end{itemize}
\subsection{Conclusioni}
Il capitolato, per quanto di forte interesse e utilità, è stato scartato perchè non scelto dall'unanimità del gruppo.
