\section{Capitolato C6 - RGP: Realtime Gaming Platform}
\subsection{Informazioni}
\begin{itemize}
    \item \textbf{Nome:} \textit{Realtime Gaming Platform};
    \item \textbf{Proponente:} \textit{Zero12};
    \item \textbf{Committente:} \textit{Prof. Tullio Vardanega, Prof. Riccardo Cardin};
\end{itemize}
\subsection{Descrizione}
Il progetto prevede la realizzazione di un gioco a scorrimento verticale giocabile in due modalità: single player e multiplayer. Il gioco tuttavia non prevede interazione fra gli attori dei giocatori ma semplicemente una sincronizzazione dei nemici, dei power up e di altre componenti in modo che la sfida sia la medesima.
\subsection{Finalità progetto}
Il progetto richiede:
\begin{itemize}
    \item Ricerca della soluzione AWS\textsubscript{\textbf{G}} piu adatta;
    \item Sviluppo della componente server e della sincronizzazione per la modalità multiplayer;
    \item Sviluppo del gioco per mobile, scegliendo una piattaforma tra iOS e Android.
\end{itemize}
Prima di procedere allo sviluppo occorre soddisfare le seguenti richieste:
\begin{itemize}
    \item Scelta servizio AWS;
    \item Schema architettura cloud;
    \item Documentazione dell'API\textsubscript{\textbf{G}};
    \item Piano di test unità.
\end{itemize}
Al termine dello sviluppo si dovrà consegnare:
\begin{itemize}
    \item Report sulle scelte della tecnologia AWS;
    \item Configurazione dell'architettura cloud (con eventualmente codice sorgente);
    \item Codice sorgente dell'app.
\end{itemize}
Il codice dovrà essere consegnato mediante repository Git\textsubscript{\textbf{G}}.
\subsection{Tecnologie impiegate}
\begin{itemize}
    \item \textbf{AWS:} occorre scegliere la soluzione dedicate all'hosting\textsubscript{\textbf{G}} per l'architettura serverless\textsubscript{\textbf{G}} tra:
          \begin{itemize}
              \item \textbf{GameLift:} soluzione di hosting di server di giochi che distribuisce, gestisce e dimensiona i server cloud per giochi multigiocatore;
              \item \textbf{Appsync:} soluzione alternativa generica.
          \end{itemize}
    \item \textbf{NodeJS}: environment per eseguire Javascript\textsubscript{\textbf{G}} esternamente al browser;
    \item Linguaggi di programmazione per lo sviluppo mobile:
          \begin{itemize}
              \item \textbf{Swift:} per iOS;
              \item \textbf{Kotlin:} per Android.
          \end{itemize}
\end{itemize}
\subsection{Aspetti positivi}
\begin{itemize}
    \item Le tecnologie impiegate sono moderne e molto diffuse;
    \item Lo sviluppo mobile è attualmente molto rilevante.
\end{itemize}
\subsection{Criticità e fattori di rischio}
\begin{itemize}
    \item Scelta obbligata fra Android e iOS;
    \item Poca dimestichezza del team con lo sviluppo mobile;
    \item Alcuni aspetti dello sviluppo del gioco, come parti grafiche, regole e attori potrebbero richiedere molto tempo nonostante siano poco rilevanti rispetto agli scopi del progetto.
\end{itemize}
\subsection{Conclusioni}
Il capitolato è stato scartato perchè la parte di sviluppo mobile e di sviluppo del gioco in sè è parsa potenzialmente insidiosa. Inoltre, per quanto riguarda le tecnologia AWS, altri capitolati ne proponevano un uso più approfondito e interessante.