\section{Capitolato C6 - RGP: Realtime Gaming Platform}
\subsection{Informazioni}
\begin{itemize}
    \item \textbf{Nome:} \textit{Realtime Gaming Platform}; 
    \item \textbf{Proponente:} \textit{Zero12};
    \item \textbf{Committente:} \textit{Prof. Tullio Vardanega, Prof. Riccardo Cardin};
\end{itemize}
\subsection{Descrizione}
Il progetto prevede la realizzazione di un gioco a scorrimento verticale giocabile in due modalit\`a: single player e multiplayer. Il gioco tuttavia non prevede interazione fra gli attori dei giocatori ma semplicemente una sincronizzazione dei nemici, power-up e altre componenti in modo che la sfida sia la medesima.
\subsection{Finalità progetto}
Il progetto richiede:
\begin{itemize}
    \item Ricerca della soluzione AWS\textsubscript{\textbf{G}} piu adatta;
    \item Sviluppo della componente server e della sincronizzazione per la modalita multiplayer;
    \item Sviluppo del gioco per mobile, scegliendo una piattaforma tra iOS e Android.
\end{itemize}
Sono inoltre richiesti i seguenti materiali prima di procedere allo sviluppo:
\begin{itemize}
    \item Scelta servizio AWS;
    \item Schema architettura cloud;
    \item Documentazione dell'API;
    \item Piano di test unit\`a.
\end{itemize}
Infine i seguenti materiali sono da consegnare al termine dello svilippo:
\begin{itemize}
    \item Report sulle scelte della tecnologia AWS\textsubscript{\textbf{G}};
    \item Configurazione dell'architettura cloud (con eventualmente il codice sorgente);
    \item codice sorgente dell'app.
\end{itemize}
Le parti di codice sono da consegnare mediante repository GIT.
\subsection{Tecnologie impiegate}
\begin{itemize}
    \item \textbf{AWS}, soluzioni dedicate all'hosting per l'architettura serverless:
    \begin{itemize}
        \item \textbf{GameLift}, apposito per il gaming
        \item \textbf{Appsync}, alternativa generica
    \end{itemize}
    \item \textbf{NodeJS}, environment per eseguire Javascript fuori dal browser, da utilizzare se il servizio AWS\textsubscript{\textbf{G}} impiegato richiede scrittura di codice.
    \item Linguaggi di programmazione per lo sviluppo mobile: 
    \begin{itemize}
        \item \textbf{Swift} per iOS
        \item \textbf{Kotlin} per android
    \end{itemize}
\end{itemize}
\subsection{Aspetti positivi}
\begin{itemize}
    \item Le tecnologie impiegate sono moderne e molto diffuse;
    \item Lo sviluppo mobile \`e attualmente molto rilevante.
\end{itemize}
\subsection{Criticit\`a e fattori di rischio}
\begin{itemize}
    \item Scelta obbligata fra Android e iOS;
    \item Poca dimestichezza del team con lo sviluppo mobile;
    \item Alcuni aspetti dello sviluppo del gioco, come parti grafiche, regole e attori potrebbero essere necessari per svolgere il testing ma dispendiosi da implementare e poco rilevanti rispetto agli scopi del progetto.
\end{itemize}
\subsection{Conclusioni}
Il capitolato \`e stato scartato perch\'e la parte mobile e dello sviluppo del gioco in se \`e parsa poco stimolante e potenzialmente insidiosa. Inoltre per quanto riguarda le tecnologia AWS\textsubscript{\textbf{G}}, altri capitolati ne proponevano un uso più approfondito e interessante.