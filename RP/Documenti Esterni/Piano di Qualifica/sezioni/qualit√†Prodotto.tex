\section{Qualità di prodotto}
Per valutare la qualità di prodotto il gruppo  \textit{TechSWEave} ha selezionato lo standard ISO/IEC 9126\textsubscript{\textbf{G}}.\\
Il modello di qualità scelto espone le caratteristiche di qualità del prodotto con una lista di attributi. Di seguito sono riportati quelli che il 
gruppo ha deciso di selezionare, omettendo quelli ritenuti meno importanti nel contesto del progetto.
\subsection{Funzionalità}
Una funzionalità è la capacità di un prodotto software di offrire gli strumenti necessari allo svolgimento di funzioni prestabilite, definite 
nell'\textit{Analisi dei Requisiti v2.0.0.}
\subsubsection{Obiettivi}
\begin{itemize}
    \item \textbf{Adeguatezza:} capacità del prodotto software di fornire un insieme di funzioni in grado
    di soddisfare gli obiettivi esposti nell’\textit{Analisi dei Requisiti 2.0.0};
    \item \textbf{Accuratezza:} capacità del prodotto software di fornire i risultati desiderati con la precisione richiesta;
    \item \textbf{Conformità:} il prodotto deve aderire agli standard scelti dal gruppo.
\end{itemize}
\subsubsection{Metriche}
\textbf{MPD01 - Completezza dell'implementazione:} misura la completezza del prodotto in percentuale.
\begin{itemize}
    \item \textbf{Codice:} MPD01;
    \item \textbf{Metodo di misura}: valore percentuale: C = $1-\frac{N\textsubscript{FNI}}{N\textsubscript{FI}}*100$ \\
    \\dove N\textsubscript{FNI} indica il numero di funzionalità non implementate e N\textsubscript{FI} indica il numero di funzionalità 
    individuate dall'analisi;
    \item \textbf{Valore preferibile}: 100\%;
    \item \textbf{Valore accettabile}: 100\%.
\end{itemize}
\subsection{Affidabilità}
È la capacità del prodotto software di mantenere il livello di prestazioni
desiderato anche a fronte di anomalie dovute a errori di analisi, progettazione e sviluppo del codice.
\subsubsection{Obiettivi}
\begin{itemize}
    \item \textbf{Maturità:} capacità del prodotto software di evitare errori e risultati non corretti durante l’esecuzione;
    \item \textbf{Tolleranza agli errori:} capacità del prodotto software di conservare il livello di prestazioni 
    anche in caso di malfunzionamenti o di uso inappropriato del prodotto.
\end{itemize}
\subsubsection{Metriche}
\textbf{MPD02 - Densità errori:} indica la capacità del prodotto di resistere a malfunzionamenti.
\begin{itemize}
    \item \textbf{Codice:} MPD02;
    \item \textbf{Metodo di misura}: valore percentuale: M = $\frac{N\textsubscript{ER}}{N\textsubscript{TE}}*100$ \\
    \\dove N\textsubscript{ER} indica il numero di errori rilevati e N\textsubscript{TE} indica il numero di test eseguiti;
    \item \textbf{Valore preferibile}: 0\%;
    \item \textbf{Valore accettabile}: $\leq$ 10\%.
\end{itemize}
\subsection{Usabilità}
È la capacità del prodotto software di essere compreso, appreso, usato e gradito dall’utente quando usato nel contesto per cui è stato progettato.
\subsubsection{Obiettivi}
\begin{itemize}
    \item \textbf{Comprensibilità:} l'utente deve poter comprendere facilmente i concetti base del prodotto software;
    \item \textbf{Apprendibilità:} l'utente deve poter imparare facilmente a usare il software;
    \item \textbf{Attrattività:} il prodotto deve essere piacevole da usare.
\end{itemize}
\subsubsection{Metriche}
\textbf{MPD03 - Facilità di utilizzo:} indica la facilità con cui l'utente riesce a ottenere l'informazione che sta cercando.
\begin{itemize}
    \item \textbf{Codice:} MPD03;
    \item \textbf{Metodo di misura}: numero di click necessari per arrivare alla pagina di checkout;
    \item \textbf{Valore preferibile}: $\leq$ 10;
    \item \textbf{Valore accettabile}: $\leq$ 15.
\end{itemize}
\textbf{MPD04 - Facilità di apprendimento:} indica la facilità con cui l'utente riesce a imparare l'utilizzo delle funzionalità del prodotto.
\begin{itemize}
    \item \textbf{Codice:} MPD04;
    \item \textbf{Metodo di misura}: minuti necessari a raggiungere la pagina di checkout;
    \item \textbf{Valore preferibile}: $\leq$ 3;
    \item \textbf{Valore accettabile}: $\leq$ 5.
\end{itemize}
\textbf{MPD05 - Profondità della gerarchia:} indica la profondità del sito.
\begin{itemize}
    \item \textbf{Codice:} MPD05;
    \item \textbf{Metodo di misura}: livello di profondità delle pagine;
    \item \textbf{Valore preferibile}: $\leq$ 4;
    \item \textbf{Valore accettabile}: $\leq$ 7.
\end{itemize}
\subsection{Manutenibilità}
È la capacità del software di essere modificato mediante correzioni e miglioramenti.
\subsubsection{Obiettivi}
\begin{itemize}
    \item \textbf{Analizzabilità:} indica la difficoltà incontrata nel diagnosticare un errore nel prodotto;
    \item \textbf{Modificabilità:} indica la facilità di apportare modifiche al prodotto;
    \item \textbf{Stabilità:} indica la capacità del software di evitare effetti indesiderati dovuti alle modifiche apportate.
\end{itemize}
\subsubsection{Metriche}
\textbf{MPD06 - Facilità di comprensione:} la facilità con cui l'utente riesce a comprendere il codice può essere rappresentata dal numero di linee di 
commento nel codice.
\begin{itemize}
    \item \textbf{Codice:} MPD06;
    \item \textbf{Metodo di misura}: R = $\frac{N\textsubscript{LCOM}}{N\textsubscript{LCOD}}$ \\
    \\dove N\textsubscript{LCOM} indica il numero di linee di commento e N\textsubscript{LCOD} indica le linee di codice;
    \item \textbf{Valore preferibile}: $\geq$ 0.20;
    \item \textbf{Valore accettabile}: $\geq$ 0.10.
\end{itemize}
\textbf{MPD07 - Semplicità delle funzioni:} indica la semplicità di un metodo in base al numero di parametri passati allo stesso.
\begin{itemize}
    \item \textbf{Codice:} MPD07;
    \item \textbf{Metodo di misura}: numero di parametri del metodo;
    \item \textbf{Valore preferibile}: $\leq$ 3;
    \item \textbf{Valore accettabile}: $\leq$ 6.
\end{itemize}
\textbf{MPD08 - Semplicità delle classi:} indica la semplicità di una classe in base al numero di metodi della stessa.
\begin{itemize}
    \item \textbf{Codice:} MPD08;
    \item \textbf{Metodo di misura}: numero di metodi della classe;
    \item \textbf{Valore preferibile}: $\leq$ 8;
    \item \textbf{Valore accettabile}: $\leq$ 15.
\end{itemize}

