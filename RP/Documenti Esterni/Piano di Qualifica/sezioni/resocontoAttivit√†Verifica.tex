\section{Resoconto delle attività di verifica}
In questa appendice vengono esposti gli esiti delle attività di verifica svolte sui documenti e sul codice consegnati nelle varie revisioni.
\subsection{Revisione dei requisiti}
\subsubsection{Analisi statica dei documenti}
L'analisi statica dei documenti è stata eseguita mediante Walkthrough\textsubscript{\textbf{G}}. In questo modo sarà possibile applicare Inspection\textsubscript{\textbf{G}} successivamente.
\subsection{Revisione di progettazione}
\subsubsection{Analisi statica dei documenti}
L'analisi statica della documentazione ha seguito il procedimento descritto nella sezione §B.1.1., la quale ha garantito un'attività di Inspection\textsubscript{\textbf{G}} più mirata.
\subsection{Metriche}
\subsection{Verifica con indice di Gulpease}

\begin{center}
    \centering
    \rowcolors{2}{logo!10}{logo!40}
    \renewcommand{\arraystretch}{1.8}
    \label{tab:IndiciGulpease}
    \begin{longtable}[!h]{p{150px} p{50px} p{50px} p{50px}}
        \caption{Esiti verifica con Gulpease}                                                  \\
        \rowcolor{logo!70}   \textbf{Documento} & \textbf{RR} & \textbf{RP}  & \textbf{Esito} \\
        \textit{Norme di Progetto}              & 68           & x            & Superato       \\
        \textit{Piano di Progetto}              & 73           & x            & Superato       \\
        \textit{Analisi dei Requisiti}          & 71           & x            & Superato       \\
        \textit{Piano di Qualifica}             & 69           & x            & Superato       \\
        \textit{Studio di Fattibilità}          & 74           & x            & Superato       \\
        \textit{Verbali interni (media)}        & 67           & x            & Superato       \\
        \textit{Verbali esterni (media)}        & 71           & x            & Superato       \\
        \rowcolor{white}
    \end{longtable}
\end{center}