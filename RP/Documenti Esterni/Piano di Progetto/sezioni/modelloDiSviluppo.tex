\section{Modello di Sviluppo}
La scelta del modello di sviluppo è fondamentale per la pianificazione del progetto, per tale scopo è stato adottato il \textbf{modello incrementale.}

\subsection{Modello incrementale}
Il modello di sviluppo incrementale permette di frazionare lo sviluppo del sistema per incrementi, ognuno dei quali rappresenta uno dei principali passi della progettazione software.
L'implementazione di nuovi requisiti, la modifica o cancellazione di quelli preesistenti, sono consentite previa discussione con il proponente e dopo la sua approvazione; in ogni caso non sono permesse durante la fase di sviluppo del processo corrente.
Tale modello di sviluppo si combina bene con il versionamento del sistema, che ne traccia le modifiche rendendole più visibili.
I vantaggi dell'utilizzo del modello incrementale sono i seguenti:
\begin{itemize}
    \item le funzionalità primarie hanno priorità nello sviluppo, in questo modo al proponente è permesso di valutarle subito;
    \item è possibile ricevere il feedback del cliente frequentemente, anche a ogni incremento;
    \item lo sviluppo a incrementi successivi permette di limitare gli errori al singolo incremento;
    \item l'individuazione degli errori, la correzione e l'apporto di modifiche sono più economiche;
    \item vengono facilitate le fasi di verifica e test perché più mirate.
\end{itemize}

\subsection{Incrementi apportati}
Durante i periodi di Progettazione e Codifica per la Technology Baseline sono stati apportati alcuni incrementi. Di seguito ne riportiamo il tracciamento, in modo tale da comprendere meglio quali requisiti vengono soddisfatti in ciascun incremento
\newline
Tutti i requisiti sono reperibili all'interno del documento \textit{Analisi dei Requisiti v.2.0.0}

%Tabella incrementi
\renewcommand{\arraystretch}{1.5}
\rowcolors{2}{logo!40}{logo!10}
\arrayrulecolor{white}
\begin{longtable}{
    >{\centering}p{0.50\textwidth}
    >{\raggedright}p{0.28\textwidth}
    >{\raggedright}p{0.29\textwidth}
    >{\centering}p{0.50\textwidth}
    }

    \caption{Tabella di Tracciamento}                            \\
    \rowcolor{white}                                             \\
    \rowcolor{logo!70}
    \centering\textbf{Incremento} & \centering\textbf{Requisiti}
    \tabularnewline
    \endfirsthead
    \rowcolor{white}\caption[]{(continua)}                       \\
    \rowcolor{logo!70}
    \centering\textbf{Incremento} & \centering\textbf{Requisiti}
    \tabularnewline
    \endhead

    %Incremento 1 ---------------------------------------------------------
    Incremento 1 - Homepage       & R1F                          \\ R1.3F
    \tabularnewline

    %Incremento 2 ---------------------------------------------------------
    Incremento 2 - Login          & R2F                          \\ R3F \\ R3.1F \\ R3.2F \\
    R11                                                          \\ R14F \\ R16F \\ R31F
    \tabularnewline

    %Incremento 3 ---------------------------------------------------------
    Incremento 3 - Prodotto       & R5F                          \\ R5.1F \\ R5.3F \\ R5.4F \\ R5.6F \\ R8F
    \tabularnewline

    %Incremento 4 ---------------------------------------------------------
    Incremento 4 - Carrello       & R13.1F
    \tabularnewline

    %Incremento 5 ---------------------------------------------------------
    Incremento 5 - Checkout       & R19F                         \\ R19.3F \\ R19.4F \\ R20F
    \tabularnewline
\end{longtable}
\counterwithin{table}{subsection}
\renewcommand{\arraystretch}{1}