\section{Consuntivi di periodo}
Di seguito vengono indicate le spese sostenute per ogni ruolo confrontandole con quanto preventivato. Il bilancio potrà essere:
\begin{itemize}
    \item \textbf{Positivo:} se la spesa effettiva è minore di quanto preventivato;
    \item \textbf{Pari:} se la spesa effettiva è uguale a quanto preventivato;
    \item \textbf{Negativo:} se la spesa effettiva è maggiore di quanto preventivato.
\end{itemize}

\subsection{Periodo di analisi}
Le ore di lavoro sostenute in questa fase sono da considerarsi come ore d'investimento per l'approfondimento personale, non vengono quindi rendicontate.

\begin{center}
    \begin{table}[!ht]
        \centering
        \caption{Consuntivo della fase di analisi}
        \vspace{5px}
        \rowcolors{2}{logo!10}{logo!40}
        \renewcommand{\arraystretch}{1.8}
        \begin{tabular}{p{150px} p{110px} p{110px}}
            \rowcolor{logo!70} \textbf{Ruolo} & \textbf{Ore} & \textbf{Costo}                  \\
            Responsabile                      & 21 (+0)      & 630,00\EURdig (+0,00 \EURdig)   \\
            Amministratore                    & 35 (+0)      & 700,00\EURdig (+0,00 \EURdig)   \\
            Analista                          & 98 (+0)      & 2.450,00\EURdig (+0,00 \EURdig) \\
            Progettista                       & 0 (+0)       & 0(+0,00 \EURdig)                \\
            Programmatore                     & 0 (+0)       & 0(+0,00 \EURdig)                \\
            Verificatore                      & 91 (+0)      & 1.365,00\EURdig (+0,00 \EURdig) \\
            \textbf{Totale preventivo}        & 245          & 5.145,00\EURdig                 \\
            \textbf{Totale consuntivo}        & 245          & 5.145,00\EURdig                 \\
            \textbf{Differenza}               & 0            & (+0,00 \EURdig)                 \\
        \end{tabular}
    \end{table}
\end{center}
\subsubsection{Conclusioni}
Avendo deciso con i componenti del gruppo di rispettare la scadenza per la prima consegna disponibile, si è lavorato per rispettare i tempi e si è riusciti a non superare le tempistiche prestabilite.

\subsubsection{Preventivo a finire}
Rispettando le tempistiche preventivate non sono stati aggiunti costi alla fase di analisi.

\newpage
\subsection{Periodo di Consolidamento dei requisiti}
In questo periodo si è svolto il lavoro relativo al consolidamento dei requisiti, successivo a quello di analisi. Sono state dedicate delle ore anche per lo studio e l'approfondimento personale e sono da considerarsi come non rendicontate. Perciò tali ore non sono state riportate nella seguente tabella:

\begin{center}
    \begin{table}[ht!]
        \centering
        \caption{Consuntivo della fase di Consolidamento dei requisiti}
        \vspace{5px}
        \rowcolors{2}{logo!10}{logo!40}
        \renewcommand{\arraystretch}{1.8}
        \begin{tabular}{p{150px} p{110px} p{110px}}
            \rowcolor{logo!70} \textbf{Ruolo} & \textbf{Ore} & \textbf{Costo}                 \\
            Responsabile                      & 5 (+1)       & 150,00\EURdig (+30,00 \EURdig) \\
            Amministratore                    & 6 (+0)       & 120,00\EURdig (+0,00 \EURdig)  \\
            Analista                          & 13 (-3)      & 325,00\EURdig (-75,00 \EURdig) \\
            Progettista                       & 0 (+0)       & 0 (+0,00 \EURdig)              \\
            Programmatore                     & 0 (+0)       & 0 (+0,00 \EURdig)              \\
            Verificatore                      & 16 (+0)      & 240,00\EURdig (+0,00 \EURdig)  \\
            \textbf{Totale preventivo}        & 42           & 880,00\EURdig                  \\
            \textbf{Totale consuntivo}        & 40           & 835,00\EURdig                  \\
            \textbf{Differenza}               & -2           & (-45,00\EURdig)                \\
        \end{tabular}
    \end{table}
\end{center}

\subsubsection{Conclusioni}
Rispetto a quanto preventivato inizialmente sono state necessarie delle modifiche per il lavoro svolto durante questo periodo. Vista la ridotta durata di tale periodo il gruppo è riuscito a ridurre le ore necessarie al completamento del lavoro. Di seguito sono riportate le motivazioni delle variazioni del monte ore di lavoro ricoperto dai diversi ruoli:
\begin{itemize}
    \item \textbf{\textit{Responsabile}} è stata necessaria un'ora aggiuntiva, per la coordinazione dei membri del gruppo inerente alla presentazione e allo studio delle tecnologie, necessarie allo sviluppo software.
    \item \textbf{\textit{Analista}} Viste le non eccessive modifiche nella documentazione sono state necessarie meno ore in questo ruolo.
\end{itemize}

\subsubsection{Preventivo a finire} Il bilancio economico risultante è positivo, ovvero sono stati risparmiati 45,00\EURdig. Tali fondi verranno impiegati nei prossimi periodi per far fronte a eventuali ritardi.

\newpage
\subsection{Periodo di Progettazione architetturale}
Le ore sostenute durante questo periodo sono relative alla progettazione e alla codifica del \textit{Proof of Concept}\textsubscript{\textbf{G}}, necessario al soddisfacimento della \textit{Technology Baseline}\textsubscript{\textbf{G}}. Tale periodo è da considerarsi rendicontato, in quanto il capitolato d'appalto è stato aggiudicato e quindi il lavoro è svolto con lo scopo di sviluppare il prodotto finale.

\begin{center}
    \begin{table}[ht!]
        \centering
        \caption{Consuntivo dei costi per ruolo della fase di Progettazione architetturale}
        \vspace{5px}
        \rowcolors{2}{logo!10}{logo!40}
        \renewcommand{\arraystretch}{1.8}
        \begin{tabular}{p{150px} p{110px} p{110px}}
            \rowcolor{logo!70} \textbf{Ruolo} & \textbf{Ore} & \textbf{Costo}                 \\
            Responsabile                      & 15(+0)       & 450,00\EURdig(+0,00 \EURdig)   \\
            Amministratore                    & 22(+3)       & 440,00\EURdig(+60,00 \EURdig)  \\
            Analista                          & 28(-2)       & 700,00\EURdig(-50,00 \EURdig)  \\
            Progettista                       & 41(-22)      & 902,00\EURdig(-484,00 \EURdig) \\
            Programmatore                     & 50(+24)      & 750,00\EURdig(+360,00 \EURdig) \\
            Verificatore                      & 43(+0)       & 645,00\EURdig(+0,00 \EURdig)   \\
            \textbf{Totale preventivo}        & 196          & 4001,00\EURdig                 \\
            \textbf{Totale consuntivo}        & 199          & 3887,00\EURdig                 \\
            \textbf{Differenza}               & +3           & (-114,00 \EURdig)              \\
        \end{tabular}
    \end{table}
\end{center}

\subsubsection{Conclusioni}
Dai dati riportati nella tabella soprastante si può notare che la progettazione riguardante tale periodo ha subito una consistente modifica. Il gruppo aveva infatti pianificato di redigere una completa progettazione architetturale del prodotto, e di riportarla all’ interno di un documento formale. Tuttavia durante lo svolgimento del lavoro ci si è concentrati maggiormente sul verificare, attraverso la progettazione e la codifica del \textit{Proof of Concept}\textsubscript{\textbf{G}}, che le tecnologie scelte si integrassero efficientemente tra loro, e che con il loro utilizzo i requisiti potessero essere soddisfatti. Di seguito sono riportate le motivazioni delle variazioni del monte ore di lavoro ricoperto dai diversi ruoli:

\begin{itemize}
    \item \textbf{\textit{Amministratore:}} Essendoci stata una maggiore necessità di sistemazione riguardante il versionamento dei documenti, si è presentato un esubero di ore per questo ruolo;
    \item \textbf{\textit{Analista:}} Vista una minore necessità di modifica riguardante la correzione dei casi d'uso e la fase di analisi per la risoluzione dei problemi, vi è stato un decremento delle ore riguardanti gli \textit{Analisti};
    \item \textbf{\textit{Progettista:}} Essendo state scelte preventivamente tutte le tecnologie da parte dei proponenti, non è servito un grande dispendio di ore per la scelta delle stesse.
    \item \textbf{\textit{Programmatore:}} Essendo stato implementato il \textit{Proof of Concept} collegando tutte le tecnologie essenziali allo sviluppo del progetto, si è reso necessario un maggiore dispendio di ore per poter collegare tutte le tecnologie previste.
\end{itemize}

Le ore di lavoro riguardanti ai ruoli che hanno subito delle variazioni rispetto a quanto pianificato sono state distribuite in modo tale che ogni elemento del gruppo svolgesse lo stesso monte ore di lavoro complessivo.

\subsubsection{Preventivo a finire} Il bilancio economico risultante è positivo, ovvero sono stati risparmiati 114,00\EURdig. Tali fondi uniti a quelli già risparmiati nel periodo precedente, per un totale di 159,00\EurDig, saranno impiegati nei prossimi periodi per far fronte a eventuali ritardi o per implementare i requisiti opzionali.
