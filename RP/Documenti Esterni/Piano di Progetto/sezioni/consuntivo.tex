\section{Consuntivo}
Di seguito vengono indicate le spese sostenute per ogni ruolo confrontandole con quanto preventivato. Il bilancio potrà essere:
\begin{itemize}
    \item \textbf{Positivo:} se la spesa effettiva è minore di quanto preventivato;
    \item \textbf{Pari:} se la spesa effettiva è uguale a quanto preventivato;
    \item \textbf{Negativo:} se la spesa effettiva è maggiore di quanto preventivato.
\end{itemize}

\subsection{Periodo di analisi}
Le ore di lavoro sostenute in questa fase sono da considerarsi come ore d'investimento per l’approfondimento personale, non vengono quindi rendicontate.

\begin{center}
    \begin{table}[!ht]
        \centering
        \caption{Consuntivo della fase di analisi}
        \vspace{5px}
        \rowcolors{2}{logo!10}{logo!40}
        \renewcommand{\arraystretch}{1.8}
        \begin{tabular}{p{150px} p{110px} p{110px}}
            \rowcolor{logo!70} \textbf{Ruolo} & \textbf{Ore} & \textbf{Costo}                  \\
            Responsabile                      & 21 (+0)      & 630,00\EURdig (+0,00 \EURdig)   \\
            Amministratore                    & 35 (+0)      & 700,00\EURdig (+0,00 \EURdig)   \\
            Analista                          & 98 (+0)      & 2.450,00\EURdig (+0,00 \EURdig) \\
            Progettista                       & 0 (+0)       & 0(+0,00 \EURdig)                \\
            Programmatore                     & 0 (+0)       & 0(+0,00 \EURdig)                \\
            Verificatore                      & 91 (+0)      & 1.365,00\EURdig (+0,00 \EURdig) \\
            \textbf{Totale preventivo}        & 245          & 5.145,00\EURdig                 \\
            \textbf{Totale consuntivo}        & 245          & 5.145,00\EURdig                 \\
            \textbf{Differenza}               & 0            & (+0,00 \EURdig)                 \\
        \end{tabular}
    \end{table}
\end{center}
\subsubsection{Conclusioni}
Avendo deciso con i componenti del gruppo di rispettare la scadenza per la prima consegna disponibile, si è lavorato per rispettare i tempi e si è riusciti a non superare le tempistiche prestabilite.

\subsubsection{Preventivo a finire}
Rispettando le tempistiche preventivate non sono stati aggiunti costi alla fase di analisi.

\subsection{Periodo di Consolidamento dei requisiti}
In questo periodo si è svolto lavoro relativo al consolidamento dei requisiti, successivo a quello di analisi. Sono state dedicate ore anche per lo studio per l'approfondimento personale e sono da considerarsi come non rendicontate. Perciò tali ore non sono state riportate nella seguente tabella:

\begin{center}
    \begin{table}[ht!]
        \centering
        \caption{Consuntivo della fase di Consolidamento dei requisiti}
        \vspace{5px}
        \rowcolors{2}{logo!10}{logo!40}
        \renewcommand{\arraystretch}{1.8}
        \begin{tabular}{p{150px} p{110px} p{110px}}
            \rowcolor{logo!70} \textbf{Ruolo} & \textbf{Ore} & \textbf{Costo}  \\
            Responsabile                      & 5 (+1)        & 150,00\EURdig (+30,00 \EURdig)  \\
            Amministratore                    & 7 (+1)        & 140,00\EURdig (+20,00 \EURdig)  \\
            Analista                          & 12 (-4)       & 300,00\EURdig (-100,00 \EURdig)  \\
            Progettista                       & 0            & 0 (+0,00 \EURdig)               \\
            Programmatore                     & 0            & 0 (+0,00 \EURdig)              \\
            Verificatore                      & 16           & 240,00\EURdig (+0,00 \EURdig)  \\
            \textbf{Totale preventivo}        & 42           & 880,00\EURdig   \\
            \textbf{Totale consuntivo}        & 40           & 830,00\EURdig   \\
            \textbf{Differenza}               & -2           & (-50,00)\EURdig)\\
        \end{tabular}
    \end{table}
\end{center}
\subsubsection{Conclusioni}
Avendo deciso con i componenti del gruppo di rispettare la scadenza per la prima consegna disponibile, si è lavorato per rispettare i tempi e si è riusciti a non superare le tempistiche prestabilite.

\subsubsection{Preventivo a finire}
Rispettando le tempistiche preventivate non sono stati aggiunti costi alla fase di analisi.


\subsection{Periodo di Progettazione architetturale}


\begin{center}
    \begin{table}[ht!]
        \centering
        \caption{Prospetto dei costi per ruolo della fase di Progettazione architetturale}
        \vspace{5px}
        \rowcolors{2}{logo!10}{logo!40}
        \renewcommand{\arraystretch}{1.8}
        \begin{tabular}{p{150px} p{110px} p{110px}}
            \rowcolor{logo!70} \textbf{Ruolo} & \textbf{Ore} & \textbf{Costo}                     \\
            Responsabile                      & 15(+0)           & 450,00\EURdig(+0,00 \EURdig)   \\
            Amministratore                    & 22(+3)           & 440,00\EURdig(+60,00 \EURdig)  \\
            Analista                          & 28(-2)           & 700,00\EURdig(-50,00 \EURdig)  \\
            Progettista                       & 41(-22)          & 902,00\EURdig(-484,00 \EURdig) \\
            Programmatore                     & 50(+24)          & 750,00\EURdig(+360,00 \EURdig) \\
            Verificatore                      & 43(+0)           & 645,00\EURdig(+0,00 \EURdig)   \\
            \textbf{Totale preventivo}        & 196              & 4001,00\EURdig                 \\
            \textbf{Totale consuntivo}        & 199              & 3887,00\EURdig                 \\
            \textbf{Differenza}               & +3               & (-114,00 \EURdig)              \\
        \end{tabular}
    \end{table}
\end{center}
