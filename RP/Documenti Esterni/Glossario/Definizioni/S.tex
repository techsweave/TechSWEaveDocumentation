\section{}
\section*{SEO (Search Engine Optimization)} Insieme di strategie e pratiche volte ad aumentare la visibilità di un sito internet migliorandone la posizione nelle classifiche dei motori di ricerca.
\subsection*{Serverless} È un framework Web gratuito e open source scritto utilizzando Node.js. È il primo framework sviluppato
per la creazione di applicazioni su AWS Lambda.

\subsection*{Server-side Rendering} È una teconologia di che va a precaricare la pagina HTML nel server ad ogni richiesta.

\subsection*{SQL} Linguaggio specifico del dominio utilizzato nella programmazione e progettato per la gestione dei dati
contenuti in un sistema di gestione di database relazionali.

\subsection*{Staging} È un'area intermedia tra la directory di lavoro e la directory nel VCS.

\subsection*{Stakeholder} Persona interessata alla riuscita del progetto.

\subsection*{Stripe} Fornisce un'infrastruttura software che permette a privati e aziende di inviare e ricevere pagamenti via internet.

\subsection*{Stub} È una porzione di codice utilizzata in sostituzione di altre funzionalità software in quanto può simulare il comportamento di codice esistente