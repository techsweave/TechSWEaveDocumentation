\subsection{UC25: Gestione dei prodotti (\textbf{Aggiungere info specifiche})}
\label{sec:UC25}
\begin{figure}[!ht]
    \caption{Diagramma di UC25: Gestione dei prodotti}
    \vspace{10px}
    \includegraphics[scale=0.5]{../../../Images/AnalisiRequisiti/UC25}
    \centering
\end{figure}
\begin{itemize}
    \item \textbf{Descrizione:} sezione dove il venditore può gestire i prodotti del negozio;
    \item \textbf{Attore Primario:} venditore autenticato;
    \item \textbf{Precondizione:} il venditore si trova in una qualsiasi pagina del negozio;
    \item \textbf{Input:} il venditore clicca il bottone per entrare in questa sezione;
    \item \textbf{Postcondizione:} il venditore ha completato le modifiche ai prodotti desiderati;
    \item \textbf{Scenario Principale:}
          \begin{itemize}
              \item il venditore si trova nella pagina per la gestione del negozio;
              \item il venditore può decidere se aggiungere un nuovo prodotto, modificarne o rimuoverne uno già presente;
              \item una volta terminato applica le modifiche.
          \end{itemize}
\end{itemize}
\subsubsection{UC25.1: Aggiunta di un prodotto}
\label{sec:UC25.1}
\begin{itemize}
    \item \textbf{Descrizione:} sezione per aggiungere un prodotto nel negozio;
    \item \textbf{Attore Primario:} venditore autenticato;
    \item \textbf{Precondizione:} il venditore si trova nella sezione per gestire i prodotti (\hyperref[sec:UC25]{\underline{UC25}});
    \item \textbf{Input:} il venditore inserisce tutti i dati relativi al prodotto da inserire;
    \item \textbf{Postcondizione:} il prodotto viene aggiunto e i clienti possono visualizzarlo nel negozio;
    \item \textbf{Scenario Principale:}
          \begin{itemize}
              \item il venditore si trova nella sezione per gestire i prodotti;
              \item il venditore decide di aggiungere un nuovo prodotto ed entra in questa sezione;
              \item il venditore inserisce tutti i dati richiesti (nome, descrizione, prezzo ecc.);
              \item il venditore conferma l'aggiunta e il prodotto diviene disponibile nel negozio.
          \end{itemize}
\end{itemize}
\subsubsection{UC25.2: Modifica di un prodotto}
\label{sec:UC25.2}
\begin{itemize}
    \item \textbf{Descrizione:} sezione per modificare un prodotto del negozio;
    \item \textbf{Attore Primario:} venditore autenticato;
    \item \textbf{Precondizione:} il venditore si trova nella sezione per gestire i prodotti (\hyperref[sec:UC25]{\underline{UC25}});
    \item \textbf{Input:} il venditore modifica i dati che ritiene opportuni relativi al prodotto da modificare;
    \item \textbf{Postcondizione:} il prodotto è stato modificato;
    \item \textbf{Scenario Principale:}
          \begin{itemize}
              \item il venditore si trova nella sezione per gestire i prodotti;
              \item il venditore decide di modificare un prodotto ed entra in questa sezione;
              \item modifica i dati che preferisce;
              \item conferma le modifiche.
          \end{itemize}
\end{itemize}
\subsubsubsection{UC25.3: Rimozione di un prodotto}
\label{sec:UC25.3}
\begin{itemize}
    \item \textbf{Descrizione:} sezione per rimuovere un prodotto dal negozio;
    \item \textbf{Attore Primario:} venditore autenticato;
    \item \textbf{Precondizione:} il venditore si trova nella sezione per gestire i prodotti (\hyperref[sec:UC25]{\underline{UC25}});
    \item \textbf{Input:} il venditore sceglie il prodotto da rimuovere;
    \item \textbf{Postcondizione:} il prodotto è stato rimosso dal negozio;
    \item \textbf{Scenario Principale:}
          \begin{itemize}
              \item il venditore si trova nella sezione per gestire i prodotti;
              \item il venditore decide di rimuovere un prodotto dal negozio;
              \item il venditore conferma la rimozione del prodotto selezionato;
              \item il prodotto non è più visibile nel negozio.
          \end{itemize}
\end{itemize}