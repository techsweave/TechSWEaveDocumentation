\appendix{Standard di qualità}
\label{sec:A}
\section{ISO/IEC 9126}
È lo standard impiegato ed è composto da quattro parti.
\subsection{Metriche per la qualità interna}
Sono metriche che non riguardano la parte non eseguibile. Vengono rilevate mediante analisi statica. Idealmente determinano la qualità esterna.
\subsection{Metriche per la qualità esterna}
Sono le metriche che riguardano la parte eseguibile. Vengono rilevate tramite l'analisi dinamica. Idealmente determinano la qualità in uso.
\subsection{Metriche per la qualità in uso}
Metriche applicabili al prodotto finito e utilizzato in condizioni reali.
\subsection{Modello per la qualità software}
Questo modello suddivide la qualià in 6 categorie, ognuna con le sue varie sottocategorie così strutturate:
\subsubsection{Funzionalità}
È la capacità del software di soddisfare i requisiti. In maniera più specifica queste sono le caratteristiche che il software deve avere:
\begin{itemize}
    \item \textbf{Appropriatezza:} capacità di fornire funzioni appropriate per svolgere i compiti previsti dagli obiettivi prefissati;
    \item \textbf{Accuratezza:} capacità di fornire i risultati concordati oppure la precisione richiesta;
    \item \textbf{Interoperabilità:} capacità  di interagire con altri sistema;
    \item \textbf{Conformità:} capacità di aderire agli standard;
    \item \textbf{Sicurezza:} capacità di garantire protezione a informazioni e dati.
\end{itemize}
\subsubsection{Affidabilità}
È la capacità del software di mantenere prestazioni specifiche in condizioni specificate.
Si compone di:
\begin{itemize}
    \item \textbf{Maturità: }capacità di evitare errori, malfunzionamenti e risultati non corretti;
    \item \textbf{Tolleranza agli errori: }capacità di mantenere i livelli prefissati di prestazioni anche in caso di errori o di usi scorretti del prodotto;
    \item \textbf{Ricuperabilità: }capacità, a seguito di un malfunzionamento o di un uso scorretto, di ripristinare i livelli di prestazioni;
    \item \textbf{Aderenza: }capacità di aderire a standard e regole riguardanti l'affidabilità.
\end{itemize}
\subsubsection{Efficienza}
\'E la capacita del software di svolgere le proprie funzioni ottimizzando l'uso delle risorse e minimizzando i tempi. Si compone di:
\begin{itemize}
    \item  \textbf{Nello spazio: }capacità di utilizzo di quantità e tipologia di risorse in maniera appropriata;
    \item  \textbf{Nel tempo: }capacità di fornire tempi di risposta ed elaborazione adeguati.
\end{itemize}
\subsubsection{Usabilità}
Capacità del prodotto di essere utilizzato e compreso dall'utente, sotto determinate condizioni. Si compone di:
\begin{itemize}
    \item \textbf{Comprensibilità: }capacità di essere chiaro riguardo le proprie funzionalità e il proprio utilizzo;
    \item \textbf{Apprendibilità: }capacità di essere facilmente appreso dall'utente;
    \item \textbf{Operabilità: }capacità di eseguire gli scopi dell'utente e di essere controllato dall'utente;
    \item \textbf{Attrattività: }capacità di essere piacevole per l'utente.
\end{itemize}
\subsection{Manutenibilità}
Capacità del prodotto di essere modificato, corretto e adattato. Si compone di:
\begin{itemize}
    \item \textbf{Analizzabilità: }capacità di essere analizzato per identificare errori;
    \item \textbf{Modificabilità: }capacità di poter essere modificato nel codice, nella documentazione o nella progettazione;
    \item \textbf{Stabilità: }capacità di evitare effetti collaterali indesiderati a seguito di modifiche;
    \item \textbf{Testabilità: }capacità di essere testato per validare le modifiche.
\end{itemize}
\subsubsection{Portabilità}
\begin{itemize}
    \item \textbf{Adattabiltà: }capacità di essere adattato a vari ambienti operativi senza modifiche;
    \item \textbf{Installabilità: }capacità di essere installato in un ambiente;
    \item \textbf{Conformità: }capacità di coesistere con altre applicazione, condividendo le risorse;
    \item \textbf{Sostituibilità: }capacità di essere impiegato al posto di altre applicazioni per svolgere gli stessi compiti.
\end{itemize}