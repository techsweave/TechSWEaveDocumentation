\appendix{Standard di qualità}
\label{sec:A}
\section{ISO/IEC 9126}
È lo standard impiegato ed è composto da quattro parti.
\subsection{Metriche per la qualità interna}
Sono metriche che non riguardano la parte non eseguibile. Vengono rilevate mediante analisi statica. Idealmente determinano la qualità esterna.
\subsection{Metriche per la qualità esterna}
Sono le metriche che riguardano la parte eseguibile. Vengono rilevate tramite l'analisi dinamica. Idealmente determinano la qualità in uso.
\subsection{Metriche per la qualità in uso}
Metriche applicabili al prodotto finito e utilizzato in condizioni reali.
\subsection{Modello per la qualità software}
Questo modello suddivide la qualità in 6 categorie, ognuna con le sue varie sottocategorie così strutturate:
\subsubsection{Funzionalità}
È la capacità del software di soddisfare i requisiti. In maniera più specifica queste sono le caratteristiche che il software deve avere:
\begin{itemize}
    \item \textbf{Appropriatezza:} capacità di fornire funzioni appropriate per svolgere i compiti previsti dagli obiettivi prefissati;
    \item \textbf{Accuratezza:} capacità di fornire i risultati concordati oppure la precisione richiesta;
    \item \textbf{Interoperabilità:} capacità  d'interagire con altri sistema;
    \item \textbf{Conformità:} capacità di aderire agli standard;
    \item \textbf{Sicurezza:} capacità di garantire protezione a informazioni e dati.
\end{itemize}
\subsubsubsection{Metriche}
\begin{itemize}
    \item \textbf{MPD01 - Completezza dell'implementazione:} misura la completezza del prodotto in percentuale.\\
    \\\textbf{Metodo di misura}: valore percentuale: C = $1-\frac{N\textsubscript{FNI}}{N\textsubscript{FI}}*100$ \\
    \\dove N\textsubscript{FNI} indica il numero di funzionalità non implementate e N\textsubscript{FI} indica il numero di funzionalità 
    individuate dall'analisi.
\end{itemize}
\subsubsection{Affidabilità}
È la capacità del software di mantenere prestazioni specifiche in condizioni specificate.
Si compone di:
\begin{itemize}
    \item \textbf{Maturità: }capacità di evitare errori, malfunzionamenti e risultati non corretti;
    \item \textbf{Tolleranza agli errori: }capacità di mantenere i livelli prefissati di prestazioni anche in caso di errori o di usi scorretti del prodotto;
    \item \textbf{Ricuperabilità: }capacità, a seguito di un malfunzionamento o di un uso scorretto, di ripristinare i livelli di prestazioni;
    \item \textbf{Aderenza: }capacità di aderire a standard e regole riguardanti l'affidabilità.
\end{itemize}
\subsubsubsection{Metriche}
\begin{itemize}
    \item \textbf{MPD02 - Densità errori:} indica la capacità del prodotto di resistere a malfunzionamenti.\\
    \\\textbf{Metodo di misura}: valore percentuale: M = $\frac{N\textsubscript{ER}}{N\textsubscript{TE}}*100$ \\
    \\dove N\textsubscript{ER} indica il numero di errori rilevati e N\textsubscript{TE} indica il numero di test eseguiti.
\end{itemize}
\subsubsection{Efficienza}
\'E la capacita del software di svolgere le proprie funzioni ottimizzando l'uso delle risorse e minimizzando i tempi. Si compone di:
\begin{itemize}
    \item  \textbf{Nello spazio: }capacità di utilizzo di quantità e tipologia di risorse in maniera appropriata;
    \item  \textbf{Nel tempo: }capacità di fornire tempi di risposta ed elaborazione adeguati.
\end{itemize}
\subsubsection{Usabilità}
Capacità del prodotto di essere utilizzato e compreso dall'utente, sotto determinate condizioni. Si compone di:
\begin{itemize}
    \item \textbf{Comprensibilità: }capacità di essere chiaro riguardo le proprie funzionalità e il proprio utilizzo;
    \item \textbf{Apprendibilità: }capacità di essere facilmente appreso dall'utente;
    \item \textbf{Operabilità: }capacità di eseguire gli scopi dell'utente e di essere controllato dall'utente;
    \item \textbf{Attrattività: }capacità di essere piacevole per l'utente.
\end{itemize}
\subsubsubsection{Metriche}
\begin{itemize}
    \item \textbf{MPD03 - Facilità di utilizzo:} indica la facilità con cui l'utente riesce a ottenere l'informazione che sta cercando.\\
    \\\textbf{Metodo di misura}: numero di click necessari per arrivare alla pagina di checkout;
    \item \textbf{MPD04 - Facilità di apprendimento:} indica la facilità con cui l'utente riesce a imparare l'utilizzo delle funzionalità del prodotto.\\
    \\\textbf{Metodo di misura}: minuti necessari a raggiungere la pagina di checkout;
    \item \textbf{MPD05 - Profondità della gerarchia:} indica la profondità del sito.\\
    \\\textbf{Metodo di misura}: livello di profondità delle pagine.
\end{itemize}
\subsection{Manutenibilità}
Capacità del prodotto di essere modificato, corretto e adattato. Si compone di:
\begin{itemize}
    \item \textbf{Analizzabilità: }capacità di essere analizzato per identificare errori;
    \item \textbf{Modificabilità: }capacità di poter essere modificato nel codice, nella documentazione o nella progettazione;
    \item \textbf{Stabilità: }capacità di evitare effetti collaterali indesiderati a seguito di modifiche;
    \item \textbf{Testabilità: }capacità di essere testato per validare le modifiche.
\end{itemize}
\subsubsubsection{Metriche}
\begin{itemize}
    \item \textbf{MPD06 - Facilità di comprensione:} la facilità con cui l'utente riesce a comprendere il codice può essere rappresentata dal numero di linee di commento nel codice.\\
    \\\textbf{Metodo di misura}: R = $\frac{N\textsubscript{LCOM}}{N\textsubscript{LCOD}}$ \\
    \\dove N\textsubscript{LCOM} indica il numero di linee di commento e N\textsubscript{LCOD} indica le linee di codice;
    \item \textbf{MPD07 - Semplicità delle funzioni:} indica la semplicità di un metodo in base al numero di parametri passati allo stesso.\\
    \\\textbf{Metodo di misura}: numero di parametri del metodo;
    \item \textbf{MPD08 - Semplicità delle classi:}  indica la semplicità di una classe in base al numero di metodi della stessa.\\
    \\\textbf{Metodo di misura}: numero di metodi della classe.
\end{itemize}
\subsubsection{Portabilità}
\begin{itemize}
    \item \textbf{Adattabiltà: }capacità di essere adattato a vari ambienti operativi senza modifiche;
    \item \textbf{Installabilità: }capacità di essere installato in un ambiente;
    \item \textbf{Conformità: }capacità di coesistere con altre applicazione, condividendo le risorse;
    \item \textbf{Sostituibilità: }capacità di essere impiegato al posto di altre applicazioni per svolgere gli stessi compiti.
\end{itemize}




% Affidabilità

 
% Usabilità


