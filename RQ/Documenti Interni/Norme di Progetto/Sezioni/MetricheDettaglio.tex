% Qualità di processo
% Processi primari
% Analisi dei requisiti
Di seguito sono descritte le varie metriche impiegate.

\subsubsection{Metriche per il soddisfacimento dei requisiti}
\begin{itemize}
    \item \textbf{MPR01 - PROS (Percentuale di requisiti obbligatori soddisfatti):} indica la percentuale di requisiti obbligatori soddisfatti.\\
    MPR01 - PROS (Percentuale di requisiti obbligatori soddisfatti): indica la percentuale di requisiti obbligatori soddisfatti.
    \\\textbf{Metodo di misura}:\\valore percentuale: $PROS = \frac{requisiti \ obbligatori \ soddisfatti}{requisiti \ obbligatori \ totali} * 100$
    \item \textbf{MPR02 - PRS (Percentuale di requisiti soddisfatti):}indica la percentuale di requisiti soddisfatti.\\
    \\\textbf{Metodo di misura}:\\valore percentuale: $PRS = \frac{requisiti \ soddisfatti}{requisiti \ totali} * 100$
\end{itemize}

% Progettazione di dettaglio
\subsubsection{Metriche per l'accoppiamento tra classi}
\begin{itemize}
    \item \textbf{CBO (accoppiamento tra le classi di oggetti):} una classe è accoppiata ad un'altra se usa metodi o variabili definiti in quest'ultima.\\
    \\\textbf{Metodo di misura}: valore intero.
\end{itemize}

% Processi di supporto
% Pianificazione
\subsubsection{Metriche per la pianificazione}
Per il calcolo delle metriche di questa sezione vengono utilizzati i seguenti valori di utilità:
\begin{itemize}
    \item \textbf{BAC (Budget At Completion):} il valore del preventivo di periodo;
    \item \textbf{EV (Earned Value):}  il valore del lavoro svolto fino al momento del calcolo, cioè il denaro guadagnato fino a quel momento;
    \item \textbf{PV (Planned Value):} il valore del lavoro pianificato al momento del calcolo, cioè il denaro che ci si aspetta di guadagnare in quel momento.
\end{itemize}
\begin{itemize}
    \item \textbf{MPR04 - EAC (Estimate at Completion):} rappresenta il nuovo preventivo sul totale alla fine di un periodo.\\
    \\\textbf{Metodo di misura}: valore intero;
    \item \textbf{MPR05 - VAC (Variance at Completion):} indica la spesa effettivamente sostenuta in percentuale.\\
    \\\textbf{Metodo di misura}: $\frac{BAC - EAC}{100}$;
    \item \textbf{MPR06 - AC (Actual Cost):} somma spesa fino al momento del calcolo.\\
    \item Metodo di calcolo: valore intero;
    \item \textbf{MPR07 - SV (Schedule Variance):} indica l'anticipo o il ritardo nello svolgimento del progetto rispetto alla pianificazione.\\
    \\\textbf{Metodo di misura}: SV = EV - PV;
    \item \textbf{MPR08 - BV (Budget Variance):} differenza tra il costo del lavoro pianificato e quello sostenuto alla data corrente.\\
    \\\textbf{Metodo di misura}: CV = PV - AC.
\end{itemize}

% Verifica
\subsubsection{Metriche per la verifica}
\begin{itemize}
    \item \textbf{MPR09 - CC (Code Coverage):} indica il numero di righe di codice sottoposte ai test di verifica durante la loro esecuzione. Per linee di codice totali
    si intende tutte quelle appartenenti all'unità sottoposta al test.\\
    \\\textbf{Metodo di misura}: valore percentuale: CC = $\frac{linee \ di \ codice \ testate}{linee \ di \ codice \ totali * 100}$
\end{itemize}

% Documentazione
\subsubsection{Metriche per la documentazione}
\begin{itemize}
    \item \textbf{MPR10 - Indice di Gulpease:}  è un indice di leggibilità di un testo tarato sulla lingua italiana. Ha il vantaggio di utilizzare la lunghezza delle parole in lettere anziché in sillabe, semplificandone il calcolo automatico.\\
    \\\textbf{Metodo di misura}: valore intero da 0 a 100: I\textsubscript{G} = 89 $+ \frac{300*numero \ di \ frasi \ - \ 10*numero \ di \ lettere}{numero \ di \ parole}$
    \item \textbf{MPR11 - Correttezza ortografica:}  la documentazione non deve contenere errori ortografici o grammaticali.\\
    \\\textbf{Metodo di misura}: valore intero: numero di errori per documento.
\end{itemize}

% Processi organizzativi
% Gestione della qualità
\subsubsection{PMS}
\begin{itemize}
    \item \textbf{MPR12 - PMS (Percentuale di metriche soddisfatte):} indica la percentuale delle metriche che raggiungono dei valori accettabili.\\
    \\\textbf{Metodo di misura}: valore percentuale: PMS = $\frac{numero \ di \ metriche \ soddisfatte}{numero \ di \ metriche \ totali} * 100$ 
\end{itemize}

% -------------------------------------------------------------------

% Qualità di Prodotto
% Funzionalità
\subsubsection{Metriche per la funzionalit\`a}
\begin{itemize}
    \item \textbf{MPD01 - Completezza dell'implementazione:} misura la completezza del prodotto in percentuale.\\
    \\\textbf{Metodo di misura}: valore percentuale: C = $1-\frac{N\textsubscript{FNI}}{N\textsubscript{FI}}*100$ \\
    \\dove N\textsubscript{FNI} indica il numero di funzionalità non implementate e N\textsubscript{FI} indica il numero di funzionalità 
    individuate dall'analisi.
\end{itemize}

% Affidabilità
\subsubsection{Metriche per l'affidabilit\`a}
\begin{itemize}
    \item \textbf{MPD02 - Densità errori:} indica la capacità del prodotto di resistere a malfunzionamenti.\\
    \\\textbf{Metodo di misura}: valore percentuale: M = $\frac{N\textsubscript{ER}}{N\textsubscript{TE}}*100$ \\
    \\dove N\textsubscript{ER} indica il numero di errori rilevati e N\textsubscript{TE} indica il numero di test eseguiti.
\end{itemize}
 
% Usabilità
\subsubsection{Metriche per l'usabilit\`a}
\begin{itemize}
    \item \textbf{MPD03 - Facilità di utilizzo:} indica la facilità con cui l'utente riesce a ottenere l'informazione che sta cercando.\\
    \\\textbf{Metodo di misura}: numero di click necessari per arrivare alla pagina di checkout;
    \item \textbf{MPD04 - Facilità di apprendimento:} indica la facilità con cui l'utente riesce a imparare l'utilizzo delle funzionalità del prodotto.\\
    \\\textbf{Metodo di misura}: minuti necessari a raggiungere la pagina di checkout;
    \item \textbf{MPD05 - Profondità della gerarchia:} indica la profondità del sito.\\
    \\\textbf{Metodo di misura}: livello di profondità delle pagine.
\end{itemize}

% Manutenibilità
\subsubsection{Metriche per la manutenibilit\`a}
\begin{itemize}
    \item \textbf{MPD06 - Facilità di comprensione:} la facilità con cui l'utente riesce a comprendere il codice può essere rappresentata dal numero di linee di commento nel codice.\\
    \\\textbf{Metodo di misura}: R = $\frac{N\textsubscript{LCOM}}{N\textsubscript{LCOD}}$ \\
    \\dove N\textsubscript{LCOM} indica il numero di linee di commento e N\textsubscript{LCOD} indica le linee di codice;
    \item \textbf{MPD07 - Semplicità delle funzioni:} indica la semplicità di un metodo in base al numero di parametri passati allo stesso.\\
    \\\textbf{Metodo di misura}: numero di parametri del metodo;
    \item \textbf{MPD08 - Semplicità delle classi:}  indica la semplicità di una classe in base al numero di metodi della stessa.\\
    \\\textbf{Metodo di misura}: numero di metodi della classe.
\end{itemize}
