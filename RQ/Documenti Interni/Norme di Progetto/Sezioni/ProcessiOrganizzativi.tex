\section{Processi Organizzativi}
\subsection{Gestione di processo}
\subsubsection{Scopo}
La gestione di processo va a gestire le attività e i compiti generici, da utilizzare per la gestione dei rispettivi processi.\\
Il \textit{Responsabile di progetto} si occupa:
\begin{itemize}
    \item del risultato della gestione;
    \item della gestione del processo stesso;
    \item la gestioni dei compiti applicabili al processo.
\end{itemize}
\subsubsection{Coordinamento}
\subsubsubsection{Scopo}
Questa sezione ha lo scopo di esporre le attività di coordinamento che vanno a gestire sia la comunicazione interna (tra i membri del team) sia quella esterna, che può avvenire nei confronti degli \textit{stakeholders}, del \textit{proponente\textsubscript{\textbf{G}}}, del \textit{committente\textsubscript{\textbf{G}}} o dei \textit{competitors\textsubscript{\textbf{G}}}. Dato il periodo pandemico attuale la maggior parte delle comunicazioni, sia interne che esterne, avverranno in modo telematico salvo evoluzioni della pandemia stessa.
\subsubsubsection{Comunicazione}
\textbf{Comunicazione interna}
Con comunicazione interna si intende quella che avviene tra i vari membri del team, questa è molto frequente e informale. La comunicazione tra i membri del team avviene secondo varie modalità:
\begin{itemize}
    \item \textbf{Discord:}
          è il mezzo principale di comunicazione fra i membri del team, con diversi canali di messaggistica divisi per argomento e varie room per le riunioni, sia per il team intero che per gruppi più piccoli. La scelta è ricaduta piattaforma per la sua estrema versatilità e perché è un sistema multi piattaforma
          I canali impiegati sono i seguenti:
          \begin{itemize}
              \item Generale: canale dedicato a tutte le discussioni che non appartengono agli altri canali. La maggior parte dei problemi e delle domande vengono fatte qui.
              \item Annunci: contiene gli annunci più importanti, riguardanti riunioni, scadenze e simili.
              \item Documentazione: canale per discutere la documentazione, sia negli aspetti teorici, sia in quelli pratici di redazione.
              \item Link: contiene link utili di varia natura, principalmente documenti esterni che vanno letti dai membri del team.
              \item Dati account: contiene i vari dati per l'accesso alle piattaforme con le credenziali TechSWEave.
              \item Pull request: utilizzato per richiedere l'approvazione prima di un merge.
          \end{itemize}
    \item \textbf{Telegram}
          Sistema di messaggistica istantanea, è il principale strumento di comunicazione testuale del team. Attraverso gruppo Telegram vengono comunicate le date e l'orario delle riunioni, e vengono discussi i problemi che non richiedono una tempestiva comunicazione globale. Viene anche usato per invitare alcuni membri del team di entrare in discord per poter risolvere un problema esistente.
    \item \textbf{GitHub}
          Usato, nella sua sezione di task management, per la parte di gestione delle varie attività, attraverso la creazione d'issue, di milestone e dei project.
\end{itemize}
\textbf{Comunicazione Esterna:}
Per quanto riguarda la comunicazione esterna i soggetti con cui il team si relaziona sono i seguenti:
\begin{itemize}
    \item \textbf{Proponente:} l'azienda \textit{Red Babel} rappresentata da \textit{Milo Ertola} e \textit{Alessandro Maccagnan};
    \item \textbf{Committenti:} nella persona del \textit{prof. Tullio Vardanega} e del \textit{prof. Riccardo Cardin};
    \item \textbf{Competitors:} I gruppi \textit{SWException}, \textit{NotOnlyStudents} e \textit{OmicronSWE}. Questi gruppi, appartenenti al primo lotto, lavorano al progetto \textit{Emporio\(\lambda\)ambda}.
\end{itemize}
Per permettere queste comunicazioni il gruppo si avvale dell'utilizzo di due strumenti principali:
\begin{itemize}
    \item \textbf{Email}
          Le e-mail costituiscono il mezzo principale per comunicare con l'esterno, l'indirizzo impiegato è
          \begin{center}
              \textit{\href{mailto:techsweave@gmail.com}{techsweave@gmail.com}}\end{center}. Le e-mail hanno un tono formale ci si rivolge al destinario dando del lei. L'oggetto della mail deve essere conciso e autoesplicativo così da favorire una comunicazione veloce.
    \item \textbf{Slack:}
          è un sistema di messaggistica istantanea. Il proponente ha creato un apposito canale Slack per il gruppo \textit{TechSWEave}. Questo gruppo viene utilizzato dal gruppo per comunicare con i proponenti. Solitamente le comunicazioni vengono effettuate per concordare delle riunioni con i proponenti o per chiedere agli stessi delucidazioni veloci su alcuni aspetti del progetto.
\end{itemize}
\subsubsubsection{Riunioni}
\textbf{Riunioni interne:}
Il gruppo ha deciso di effettuare le riunioni due volte a settimana, il martedì sera e il sabato mattina, attraverso una video chiamata su Discord. Queste riunioni prevedono la presenza della maggior parte del team, se un membro non può essere presente deve comunicarlo in anticipo, e verrà in seguito informato su quanto è stato detto e deciso all'interno della riunione. Per ogni riunione viene assegnato un \textit{segretario} che avrà il compito di redarre il verbale della riunione che dovrà poi essere approvato da il \textit{Responsabile di progetto}.
\textbf{Riunioni esterne}
Le riunioni esterne avverranno con delle modalità specifica seconda dell'interlocutore:
\begin{itemize}
    \item \textbf{Riunione con i proponenti: } avverranno sul \textit{Google Meet} fornito dal proponente, le riunioni saranno decise con qualche giorno d'anticipo con un confronto con i proponenti in base alla disponibilità degli e del team, sullo Slack del gruppo. Anche per queste riunioni viene assegnato un \textit{segretario} che redigerà un verbale che dovrà poi essere approvato dal \textit{Responsabile di progetto};
    \item \textbf{Riunioni con i committenti:} avverranno sullo \textit{Zoom} fornito dagli stessi e saranno concordati preventivamente via e-mail.
\end{itemize}
\subsubsection{Pianificazione}
\subsubsubsection{Scopo}
Lo scopo della seguente sezione è quello di chiarire come il gruppo TechSWEave intenda pianificare il lavoro, partendo dalla scelta dei ruoli, fino all'assegnazione dei compiti ai vari partecipanti.\\
Il processo di pianificazione, in accordo con lo standard ISO/IEC 12207\textsubscript{\textbf{G}}, è strutturato nella seguente maniera:
\begin{itemize}
    \item ruoli di progetto;
    \item assegnazione dei ruoli;
    \item ciclo di vita del ticket.
\end{itemize}
\subsubsubsection{Ruoli di progetto}
Nel corso del progetto, i componenti del gruppo ricopriranno i seguenti ruoli:
\begin{itemize}
    \item \textit{Responsabile di Progetto};
    \item \textit{Amministratore di Progetto};
    \item \textit{Analista};
    \item \textit{Progettista};
    \item \textit{Programmatore};
    \item \textit{Verificatore}.
\end{itemize}

Essi corrispondono alle rispettive figure aziendali, e sarà stabilito un calendario che permetterà a ogni membro di ricoprire almeno una volta ciascun ruolo per un periodo di tempo omogeneo. L'assegnazione di un ruolo comporta lo svolgimento di determinati compiti, così come previsto dal \textit{Piano di progetto\textsubscript{\textbf{G}}}. Inoltre si cercherà di eliminare eventuali conflitti d'interesse: per esempio, un componente non potrà redigere e poi verificare ciò che è stato da lui stesso prodotto.
\textbf{Responsabile di progetto}\\
Il \textit{Responsabile di Progetto}\textsubscript{\textbf{G}} è una figura importante in quanto ricadono su di lui le responsabilità di pianificazione, gestione, controllo e coordinamento. Un altro suo compito è quello di fare da intermediario nella comunicazione tra il gruppo e i soggetti esterni: sono quindi di sua competenza le comunicazioni con committente e proponente.\\
I suoi compiti sono i seguenti:
\begin{itemize}
    \item assegnare i ruoli ai componenti del gruppo;
    \item gestire il coordinamento dei membri del gruppo;
    \item gestire la pianificazione, intesa come attività da svolgere e scadenze da rispettare;
    \item essere responsabile della stima dei costi e dell'analisi dei rischi;
    \item analizzare e gestire le criticità;
    \item approvare la documentazione;
    \item curare le relazioni tra il gruppo e i soggetti esterni.
\end{itemize}
\textbf{Amministratore di progetto}\\
L’\textit{Amministratore di progetto}\textsubscript{\textbf{G}} è incaricato di gestire, controllare e curare gli strumenti che il gruppo utilizza per svolgere il proprio lavoro. È la figura che garantisce l'affidabilità e l'efficacia dei mezzi scelti dal gruppo.\\
I suoi compiti sono i seguenti:
\begin{itemize}
    \item gestire il versionamento e la configurazione dei prodotti;
    \item gestire e salvaguardare la documentazione, controllando che sia corretta, verificata e approvata e semplificando il suo reperimento;
    \item correggere eventuali problemi legati alla gestione dei processi;
    \item amministrare le infrastrutture e i servizi necessari ai processi di supporto;
    \item individuare strumenti utili all'automazione di processi;
    \item redigere e manutenere le norme e procedure che regolano il lavoro.
\end{itemize}
\textbf{Analista}\\
L'\textit{Analista}\textsubscript{\textbf{G}} partecipa al progetto al momento della stesura dell'Analisi dei Requisiti, il suo compito è quello di evidenziare i punti chiave del problema in questione, comprendendone appieno tutte le sue peculiarità. La sua figura è fondamentale per la buona riuscita del lavoro, in quanto errori o mancanze nell'individuazione dei requisiti da soddisfare possono compromettere fortemente l'attività di progettazione.\\
I suoi compiti sono i seguenti:
\begin{itemize}
    \item studiare e definire il problema;
    \item analizzare le richieste e definire quali sono i requisiti in base ai bisogni, impliciti o espliciti;
    \item analizzare il fronte applicativo, gli utenti e i casi d’uso;
    \item redigere lo \textit{Studio di Fattibilità v1.0.0} e l’\textit{Analisi dei Requisiti v3.0.0}.
\end{itemize}
\textbf{Progettista}\\
Il \textit{Progettista}\textsubscript{\textbf{G}} ha il compito di sviluppare una soluzione che soddisfi i bisogni individuati, il suo scopo è quello di produrre un'architettura che modelli il problema a partire da un insieme di requisiti.
\newline
I suoi compiti sono i seguenti:
\begin{itemize}
    \item effettuare scelte efficienti riguardo alle tecnologie da utilizzare per lo sviluppo del progetto;
    \item sviluppare un'architettura che sfrutti tecnologie note e ottimizzate, su cui basare un prodotto stabile e mantenibile;
    \item produrre una soluzione sostenibile e realizzabile, che rientri nei costi stabiliti dal preventivo;
    \item limitare il più possibile il grado di accoppiamento tra le varie componenti.
\end{itemize}
\textbf{Programmatore}\\
Il \textit{Programmatore}\textsubscript{\textbf{G}} è la figura incaricata alla codifica del progetto. Egli deve implementare l’architettura prodotta dal \textit{Progettista} in modo che aderisca alle specifiche, ed è responsabile della manutenzione del codice.
\newline

I suoi compiti sono i seguenti:
\begin{itemize}
    \item codificare secondo le specifiche stabilite dal \textit{Progettista}. Il codice prodotto è documentato, versionabile e strutturato in modo da agevolarne la futura manutenzione;
    \item creare e gestire le componenti per la verifica e validazione del codice;
    \item redigere il manuale utente.
\end{itemize}
\textbf{Verificatore}\\
Il \textit{Verificatore}\textsubscript{\textbf{G}} si occupa di controllare il prodotto del lavoro svolto dagli altri membri del gruppo, sia esso codice o documentazione. Per le correzioni si affida agli standard definiti nelle \textit{Norme di Progetto v3.0.0}.\\
I suoi compiti sono i seguenti:
\begin{itemize}
    \item ispezionare i prodotti in fase di revisione, avvalendosi delle tecniche e degli strumenti definiti nelle \textit{Norme di Progetto v3.0.0};
    \item segnalare eventuali difetti o errori del prodotto in esame;
    \item redigere la parte retrospettiva del \textit{Piano di Qualifica v3.0.0}, il quale descrive e chiarisce le verifiche e le prove effettuate.
\end{itemize}
\subsubsubsection{Assegnazione dei compiti}

La progressione nello svolgimento del progetto può essere vista come il completamento di una serie di compiti, ognuno con la sua scadenza temporale, i quali producono risultati utili alla realizzazione degli obiettivi posti. Tali compiti sono determinati a volte dalla contingenza, altre volte sono legati ai processi in atto. Per l’assegnazione dei compiti si è deciso di utilizzare il servizio di ticketing\textsubscript{\textbf{G}} offerto da \textit{GitHub}\textsubscript{\textbf{G}}, basato sul concetto d'issue\textsubscript{\textbf{G}}. La figura che si occupa della gestione dei compiti è il \textit{Responsabile di Progetto}.\\
Egli:
\begin{itemize}
    \item individua il compito da svolgere;
    \item se ritiene il compito troppo complesso, lo suddivide in più sotto-compiti;
    \item individua uno o più componenti del gruppo al quale assegnare il compito;
    \item apre una issue, assegnando il compito al soggetto/i e definendo una data entro la quale completarlo.

\end{itemize}
Di contro, i membri del gruppo:
\begin{itemize}
    \item devono svolgere il compito entro la data fissata;
    \item devono chiudere la issue una volta terminato il lavoro.
\end{itemize}
\subsubsubsection{Ciclo di vita del ticket}
Il ciclo di vita del ticket si suddivide nei seguenti punti:
\begin{itemize}
    \item \textbf{Individuazione del compito}: il Responsabile rileva determinate necessità, a cui bisogna rispondere con azioni opportune. Concettualizza pertanto quanto identificato in richieste da soddisfare e obiettivi da raggiungere, creando così un compito;

    \item \textbf{Analisi del compito}: il Responsabile stima la complessità del compito appena identificato, dividendolo in più sotto-compiti se lo ritiene troppo oneroso. Individua poi uno o più assegnatari e una data di scadenza entro la quale il lavoro dovrà essere terminato;
    \item \textbf{Creazione issue}: il Responsabile crea una issue utilizzando \textit{GitHub}, definendo l’assegnatario e la data di scadenza decisi precedentemente;
    \item \textbf{Spostamento issue in "To-do"}: una volta creata la issue, il Responsabile la sposta nell’insieme "To-do", indicante tutti i compiti individuati ma non ancora iniziati;
    \item \textbf{Spostamento issue in "In progress"}: una volta che l’assegnatario decide d'iniziare il lavoro, sposta la issue nella categoria "In progress";
    \item \textbf{Svolgimento issue}: in questo periodo di tempo il compito viene portato a termine dall'assegnatario;
    \item \textbf{Spostamento issue in "Review in progress"}: l’assegnatario sposta la issue nella categoria "Review in progress", per indicare che ha completato il suo lavoro;
    \item \textbf{Verifica del lavoro}: il \textit{Verificatore} giudica quanto fatto dall’assegnatario:

          \begin{itemize}
              \item accettando il lavoro svolto e spostando la issue in "Review approved";
              \item rifiutando il lavoro e spostando nuovamente la issue in "To-do".
          \end{itemize}

    \item \textbf{Approvazione del lavoro}: il Responsabile esamina un’ultima volta il lavoro svolto, controllando che rispetti gli obiettivi prefissati, in tal caso lo approva e sposta la issue in "Done".

\end{itemize}
\subsubsubsection{Metriche}
Per il calcolo delle metriche di questa sezione vengono utilizzati i seguenti valori di utilità:
\begin{itemize}
    \item \textbf{BAC (Budget At Completion):} il valore del preventivo di periodo;
    \item \textbf{EV (Earned Value):}  il valore del lavoro svolto fino al momento del calcolo, cioè il denaro guadagnato fino a quel momento;
    \item \textbf{PV (Planned Value):} il valore del lavoro pianificato al momento del calcolo, cioè il denaro che ci si aspetta di guadagnare in quel momento.
\end{itemize}
\begin{itemize}
    \item \textbf{MPR07 - EAC (Estimate at Completion):} rappresenta il nuovo preventivo sul totale alla fine di un periodo.
          \begin{itemize}
              \item
                    \textbf{Metodo di misura}: valore intero;
          \end{itemize}
    \item \textbf{MPR08 - VAC (Variance at Completion):} indica la spesa effettivamente sostenuta in percentuale.
          \begin{itemize}
              \item \textbf{Metodo di misura}: $\frac{BAC - EAC}{100}$;
          \end{itemize}
    \item \textbf{MPR09 - AC (Actual Cost):} somma spesa fino al momento del calcolo.
          \begin{itemize}
              \item
                    \textbf{Metodo di misura}: valore intero;
          \end{itemize}
    \item \textbf{MPR10 - SV (Schedule Variance):} indica l'anticipo o il ritardo nello svolgimento del progetto rispetto alla pianificazione.\begin{itemize}
              \item
                    \textbf{Metodo di misura}: SV = EV - PV;
          \end{itemize}
    \item \textbf{MPR11 - BV (Budget Variance):} differenza tra il costo del lavoro pianificato e quello sostenuto alla data corrente.\begin{itemize}
              \item
                    \textbf{Metodo di misura}: CV = PV - AC.
          \end{itemize}
\end{itemize}
\subsection{Gestione infrastrutturale}
\subsubsection{Scopo}
Questa sezione ha lo scopo di andare a descrivere quella che è la gestione dell'infrastruttura ossia, si vanno a stabilire tutti gli tutti gli strumenti, sia software che hardware, necessari per il mantenimento di qualsiasi processo.
\subsubsection{Coordinamento}
Gli strumenti utilizzati per l'attività di coordinamento sono i seguenti:
\begin{itemize}
    \item \textbf{Discord:} strumento di messaggistica istantanea e di chiamate VOIP, utilizzato per le riunioni interne e per le sessioni di lavoro di gruppo;
    \item \textbf{Telegram:} strumento di messaggistica istantanea utilizzato per la comunicazione veloce tra i membri del team;
    \item \textbf{Slack:} strumento di messaggistica istantanea utilizzato per la comunicazione veloce con i proponenti e per organizzare le riunioni con gli stessi.
    \item \textbf{Google Meet:} strumento di videoconferenze utilizzato per effettuare le riunioni con i proponenti;
    \item \textbf{Zoom:} strumento di videoconferenze utilizzato per effettuare le riunioni con i committenti;
    \item \textbf{Gmail:} strumento di posta elettronica utilizzato per comunicare con i committenti;
    \item \textbf{Google Calendar:} strumento per la gestioni di eventi. Vieni utilizzato dal gruppo per poter organizzare al meglio il lavoro, vengono indicate le riunioni con la loro data e l'orario, scadenze importanti per lo sviluppo del progetto ed eventuali impegni da parte dei membri del team;
    \item \textbf{Google drive:} Strumento di storage cloud per la conservazione di dati. Il gruppo all'interno della proprio cartella drive salva i documenti completanti per ogni revisione di avanzamento. La cartella drive sarà poi inoltrata ai committenti che ne potranno consultare il contenuto.
\end{itemize}
\subsubsection{Pianificazione}
Gli strumenti utilizzati per l'attività di pianificazione sono i seguenti:
\begin{itemize}
    \item \textbf{Gantt project:} software per la realizzazione di diagrammi di Gantt\textsubscript{\textbf{G}}, utilizzati per supportare l'attività di pianificazione;
    \item \textbf{Microsoft Excel:} software utilizzato per la realizzazione d'istogrammi e areogrammi per supportare l'attività di pianificazione.
\end{itemize}
\subsection{Miglioramento del processo}
\subsubsection{Scopo}
Lo scopo di questa sezione viene descritto il processo di miglioramento che permette di stabilire, valutare, misurare, controllare e migliorare il ciclo di vita del software.
\subsubsection{Istituzione di un processo}
Ogni processo deve essere controllato e sviluppato costantemente, durante tutto il ciclo di vita del software così da permettere il raggiungimento degli obbiettivi fissati durante la fase di pianificazione.
\subsubsection{Valutazione di un processo}
Per attuare la valutazione di un processo verranno eseguiti i seguenti compiti:
\begin{itemize}
    \item si valuta il processo attraverso l'applicazione di una procedura di sviluppo precedentemente sviluppata, questa valutazione deve essere documentata per permettere la conservazione dei dati e il successivo miglioramento del processo;
    \item le revisioni devono essere pianificate a intervalli regolari cosicché i processi restino efficaci.
\end{itemize}
Per poter rispettare e attuare quanto appena descritto il gruppo \textit{TechSWEave} controllerà costantemente l'andamento dei processi attraverso il cruscotto di valutazione posto all'interno del \textit{Piano di Qualifica v3.0.0}.
\subsubsection{Miglioramento del processo}
Una volta completato il periodo il team, in base alle valutazioni registrate, provvederà a stabilire e sviluppare eventuali modifiche miglioramenti sui processi così da garantire la miglior qualità possibile.



\subsection{Formazione dei membri del team}
\subsubsection{Scopo}
Questo processo ha lo scopo di mantenere il team competente sulle esigenze tecniche del progetto.
\subsubsection{Formazione personale}
I membri del gruppo devono provvedere alla propria formazione in maniera autonoma, studiando le tecnologie usate e colmando le proprie lacune.
\subsubsubsection{Documentazione}
Il team per la redazione dei documenti ha studiato le seguenti tecnologie
\begin{itemize}
    \item \textbf{\LaTeX\textsubscript{\textbf{G}}}: \url{https://www.latex-project.org/help/documentation/};
    \item \textbf{Visual Studio Code}\textsubscript{\textbf{G}}: \url{https://code.visualstudio.com/docs}.
\end{itemize}
\subsubsubsection{Codifica}
Il gruppo per la parte di codifica ha studiato le seguenti tecnologie:
\begin{itemize}
    \item \textbf{React\textsubscript{\textbf{G}}:}\href{https://reactjs.org/docs/getting-started.html}{https://reactjs.org/docs/getting-started.html};
    \item \textbf{Next.js\textsubscript{\textbf{G}}:} \href{https://nextjs.org/docs/getting-started}{https://nextjs.org/docs/getting-started};
    \item \textbf{Typescript\textsubscript{\textbf{G}}:} \href{https://www.typescriptlang.org/}{https://www.typescriptlang.org/};
    \item \textbf{Node.js\textsubscript{\textbf{G}}:} \href{https://nodejs.org/en/docs/}{https://nodejs.org/en/docs/};
    \item \textbf{Serverless\textsubscript{\textbf{G}}:} \href{https://www.serverless.com/framework/docs/}{https://www.serverless.com/framework/docs/};
\end{itemize}
\subsubsubsection{Servizi esterni:}
I membri del team hanno studiato i seguenti servizi esterni:
\begin{itemize}
    \item \textbf{Stripe\textsubscript{\textbf{G}}:} \href{https://stripe.com/docs}{https://stripe.com/docs};
    \item \textbf{GitHub\textsubscript{\textbf{G}}:} \href{https://docs.github.com/en}{https://docs.github.com/en};
    \item \textbf{Chakra.ui\textsubscript{\textbf{G}}:} \href{https://chakra-ui.com/docs/getting-started}{https://chakra-ui.com/docs/getting-started};
    \item \textbf{AWS\textsubscript{\textbf{G}}:}
          \begin{itemize}
              \item \textbf{Amazon Cognito\textsubscript{\textbf{G}}:} \href{https://docs.aws.amazon.com/cognito/?id=docs_gateway}{https://docs.aws.amazon.com/cognito/?id=docs\_gateway};
              \item \textbf{DynamoDB\textsubscript{\textbf{G}}:} \href{https://docs.aws.amazon.com/dynamodb/?id=docs_gateway}{https://docs.aws.amazon.com/dynamodb/?id=docs\_gateway};
              \item \textbf{API Gateway\textsubscript{\textbf{G}}:} \href{https://docs.aws.amazon.com/apigateway/?id=docs_gateway}{https://docs.aws.amazon.com/apigateway/?id=docs\_gateway};
              \item \textbf{Funzioni Lambda\textsubscript{\textbf{G}}:} \href{https://docs.aws.amazon.com/lambda/?id=docs_gateway}{https://docs.aws.amazon.com/lambda/?id=docs\_gateway};
              \item \textbf{S3\textsubscript{\textbf{G}}:} \href{https://docs.aws.amazon.com/s3/?id=docs_gateway}{https://docs.aws.amazon.com/s3/?id=docs\_gateway}.
          \end{itemize}
\end{itemize}