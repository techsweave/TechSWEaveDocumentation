\section{Qualità di prodotto}
Per valutare la qualità di prodotto il gruppo  \textit{TechSWEave} ha selezionato lo standard ISO/IEC 9126\textsubscript{\textbf{G}}.\\
Il modello di qualità scelto espone le caratteristiche di qualità del prodotto con una lista di attributi. Di seguito sono riportati quelli che il
gruppo ha deciso di selezionare, omettendo quelli ritenuti meno importanti nel contesto del progetto.
\subsection{Funzionalità}
Una funzionalità è la capacità di un prodotto software di offrire gli strumenti necessari allo svolgimento di funzioni prestabilite, definite
nell'\textit{Analisi dei Requisiti v3.0.0.}
\subsubsection{Obiettivi}
\begin{itemize}
    \item \textbf{Adeguatezza:} capacità del prodotto software di fornire un insieme di funzioni in grado
          di soddisfare gli obiettivi esposti nell’\textit{Analisi dei Requisiti v3.0.0};
    \item \textbf{Accuratezza:} capacità del prodotto software di fornire i risultati desiderati con la precisione richiesta;
    \item \textbf{Conformità:} il prodotto deve aderire agli standard scelti dal gruppo.
\end{itemize}
\subsubsection{Metriche}
\textbf{MPD01 - Completezza dell'implementazione:}
\begin{itemize}
    \item \textbf{Codice:} MPD01;
    \item \textbf{Descrizione:} vedi \textit{Norme di Progetto v3.0.0} sezione \S A.4.1.1;
    \item \textbf{Valore preferibile:} 100\%;
    \item \textbf{Valore accettabile:} 100\%.
\end{itemize}
\subsection{Affidabilità}
È la capacità del prodotto software di mantenere il livello di prestazioni
desiderato anche a fronte di anomalie dovute a errori di analisi, progettazione e sviluppo del codice.
\subsubsection{Obiettivi}
\begin{itemize}
    \item \textbf{Maturità:} capacità del prodotto software di evitare errori e risultati non corretti durante l’esecuzione;
    \item \textbf{Tolleranza agli errori:} capacità del prodotto software di conservare il livello di prestazioni
          anche in caso di malfunzionamenti o di uso inappropriato del prodotto.
\end{itemize}
\subsubsection{Metriche}
\textbf{MPD02 - Densità errori:}
\begin{itemize}
    \item \textbf{Codice:} MPD02;
    \item \textbf{Descrizione:} vedi \textit{Norme di Progetto v3.0.0} sezione \S A.4.2.1;
    \item \textbf{Valore preferibile:} 0\%;
    \item \textbf{Valore accettabile:} $\leq$ 10\%.
\end{itemize}
\subsection{Usabilità}
È la capacità del prodotto software di essere compreso, appreso, usato e gradito dall’utente quando usato nel contesto per cui è stato progettato.
\subsubsection{Obiettivi}
\begin{itemize}
    \item \textbf{Comprensibilità:} l'utente deve poter comprendere facilmente i concetti base del prodotto software;
    \item \textbf{Apprendibilità:} l'utente deve poter imparare facilmente a usare il software;
    \item \textbf{Attrattività:} il prodotto deve essere piacevole da usare.
\end{itemize}
\subsubsection{Metriche}
\textbf{MPD03 - Facilità di utilizzo:}
\begin{itemize}
    \item \textbf{Codice:} MPD03;
    \item \textbf{Descrizione:} vedi \textit{Norme di Progetto v3.0.0} sezione \S A.4.4.1;
    \item \textbf{Valore preferibile:} $\leq$ 10;
    \item \textbf{Valore accettabile:} $\leq$ 15.
\end{itemize}
\textbf{MPD04 - Facilità di apprendimento:}
\begin{itemize}
    \item \textbf{Codice:} MPD04;
    \item \textbf{Descrizione:} vedi \textit{Norme di Progetto v3.0.0} sezione \S A.4.4.1;
    \item \textbf{Valore preferibile:} $\leq$ 3;
    \item \textbf{Valore accettabile:} $\leq$ 5.
\end{itemize}
\textbf{MPD05 - Profondità della gerarchia:}
\begin{itemize}
    \item \textbf{Codice:} MPD05;
    \item \textbf{Descrizione:} vedi \textit{Norme di Progetto v3.0.0} sezione \S A.4.4.1;
    \item \textbf{Valore preferibile:} $\leq$ 4;
    \item \textbf{Valore accettabile:} $\leq$ 7.
\end{itemize}
\subsection{Manutenibilità}
È la capacità del software di essere modificato mediante correzioni e miglioramenti.
\subsubsection{Obiettivi}
\begin{itemize}
    \item \textbf{Analizzabilità:} indica la difficoltà incontrata nel diagnosticare un errore nel prodotto;
    \item \textbf{Modificabilità:} indica la facilità di apportare modifiche al prodotto;
    \item \textbf{Stabilità:} indica la capacità del software di evitare effetti indesiderati dovuti alle modifiche apportate.
\end{itemize}
\subsubsection{Metriche}
\textbf{MPD06 - Facilità di comprensione:}
\begin{itemize}
    \item \textbf{Codice:} MPD06;
    \item \textbf{Descrizione:} vedi \textit{Norme di Progetto v3.0.0} sezione \S A.4.5.1;
    \item \textbf{Valore preferibile:} $\geq$ 0.20;
    \item \textbf{Valore accettabile:} $\geq$ 0.10.
\end{itemize}
\textbf{MPD07 - Semplicità delle funzioni:}
\begin{itemize}
    \item \textbf{Codice:} MPD07;
    \item \textbf{Descrizione:} vedi \textit{Norme di Progetto v3.0.0} sezione \S A.4.5.1;
    \item \textbf{Valore preferibile:} $\leq$ 3;
    \item \textbf{Valore accettabile:} $\leq$ 6.
\end{itemize}
\textbf{MPD08 - Semplicità delle classi:}
\begin{itemize}
    \item \textbf{Codice:} MPD08;
    \item \textbf{Descrizione:} vedi \textit{Norme di Progetto v3.0.0} sezione \S A.4.5.1;
    \item \textbf{Valore preferibile:} $\leq$ 8;
    \item \textbf{Valore accettabile:} $\leq$ 15.
\end{itemize}
