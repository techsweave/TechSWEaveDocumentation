\section{Introduzione}
\subsection{Scopo del documento}
Il documento ha lo scopo di descrivere in maniera dettagliata i requisiti individuati per il prodotto. Tali requisiti sono stati identificati dall'analisi del capitolato\textsubscript{\textbf{G}} \textbf{C2} e dagli incontri con il proponente \textit{Red Babel}.
\subsection{Scopo del prodotto}
L'obiettivo del progetto è quello di realizzare una piattaforma e-commerce basata su tecnologie serverless. Il prodotto dovrà poter essere utilizzato da un generico commerciante con la minima interazione tecnica tramite account AWS Merchant\textsubscript{\textbf{G}}. Inoltre dovranno essere implementate delle funzioni irrinunciabili per tutte le categorie di utenti che ne faranno uso:
\begin{itemize}
    \item clienti;
    \item commercianti;
    \item admin.
\end{itemize}
\subsection{Glossario}
All'interno del documento sono presenti termini che potrebbero risultare ambigui a seconda del contesto. Al fine di evitare possibili incomprensioni
e rendere chiari agli stakeholders\textsubscript{\textbf{G}} i termini utilizzati, viene fornito un \textit{Glossario v2.0.0.} contenente i suddetti termini
e la loro spiegazione. Nella seguente documentazione tali termini saranno individuabili tramite una '\textbf{G}' a pedice.
\subsection{Riferimenti}
\subsubsection{Normativi}
\begin{itemize}
    \item \textit{Norme di Progetto v2.0.0};
    \item \textbf{Capitolato d'appalto C2 - EmporioLambda}:\\ \href{https://www.math.unipd.it/~tullio/IS-1/2020/Progetto/C2.pdf}{https://www.math.unipd.it/~tullio/IS-1/2020/Progetto/C2.pdf};
    \item \textit{Verbale esterno 2021-03-17}
    \item \textit{Verbale esterno 2021-03-25}
    \item \textit{Verbale esterno 2021-04-19}
\end{itemize}
\subsubsection{Informativi}
\begin{itemize}
    \item \textit{Studio di fattibilità v2.0.0};
    \item \textbf{Capitolato d'appalto C2 - EmporioLambda}:\\ \href{https://www.math.unipd.it/~tullio/IS-1/2020/Progetto/C2.pdf}{https://www.math.unipd.it/~tullio/IS-1/2020/Progetto/C2.pdf};
    \item \textbf{Slide del corso Ingegneria del Software, A.A. 2020/21}: \\ \href{https://www.math.unipd.it/~tullio/IS-1/2020/Dispense/L07.pdf}{Analisi dei Requisiti}; \\ \href{https://www.math.unipd.it/~rcardin/swea/2021/Diagrammi%20Use%20Case_4x4.pdf}{Diagrammi dei casi d'uso};
    \item \textbf{Software Engineering - Ian Sommerville - 10th Edition 2014}\\
          Chapter 4: Requirements engineering;
    \item \textbf{Sito informativo sull'utilizzo delle tecnologie server-less}:\\ \href{https://aws.amazon.com/training/ramp-up-guides/}{https://aws.amazon.com/training/ramp-up-guides/};
    \item \textbf{Sito informativo sull'utilizzo e l'implementazione di Amazon Cognito}:\\ \href{https://docs.aws.amazon.com/cognito/latest/developerguide/cognito-user-identity-pools.html}{https://docs.aws.amazon.com/cognito/latest/developerguide/cognito-user-identity-pools.html}.
          %Da aggiornare con i siti che useremo per comprendere al meglio le tecnologie usate
\end{itemize}