\section{}

\subsection*{M2M (Machine to Machine)} È una tecnologia in grado di mettere in comunicazione diverse macchine tra loro, permettendo così lo scambio di dati ed informazioni acquisite al fine di migliorare i processi svolti dalle macchine stesse.

\subsection*{Machine learning} È una branca dell'intelligenza artificiale. È un metodo di analisi di dati che automatizza la costruzione di modelli analitici. L'intervento umano è ridotto al minimo.

\subsection*{Metodi di callback} Sono metodi che vengono passati ad altri metodi come parametri.

\subsection*{Metodi ricorsivi} È un algoritmo la cui esecuzione su un insieme di dati comporta la semplificazione o suddivisione dell'insieme di dati e l'applicazione dello stesso agli insiemi di dati semplificati.

\subsection*{Microservices architecture} È un architettura che suddivide un’applicazione in una serie di parti più piccole e specializzate, ciascuna delle quali comunica con le altre attraverso interfacce comuni come API e interfacce REST per creare un’applicazione.

\subsection*{MIT} La Licenza MIT è una licenza di software libero creata dal Massachusetts Institute of Technology.

\subsection*{Modello a V} un processo di sviluppo software che prevede lo sviluppo dei test in parallelo alle attività di analisi e progettazione.

\subsection*{MQTT} È un protocollo ISO standard di messaggistica leggero di tipo \textit{publish-subscribe} posizionato in nel livello superiore di TCP/IP. È stato progettato per le situazioni in cui è richiesto un basso impatto e dove la banda è limitata. Il pattern \textit{publish-subscribe} richiede un message broker. Il broker è responsabile della distribuzione dei messaggi ai client destinatari.

\subsection*{MVC (Model View Controller)} È un design architetturale orientato alla programmazione ad oggetti  e allo sviluppo di applicazioni web. Il suo punto forte è quello di avere una netta separazione tra parte logica e di business e la vista.