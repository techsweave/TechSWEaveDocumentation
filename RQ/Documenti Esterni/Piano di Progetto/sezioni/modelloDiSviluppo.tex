\section{Modello di Sviluppo}
La scelta del modello di sviluppo è fondamentale per la pianificazione del progetto, per tale scopo è stato adottato il \textbf{modello incrementale.}

\subsection{Modello incrementale}
Il modello di sviluppo incrementale permette di frazionare lo sviluppo del sistema per incrementi, ognuno dei quali rappresenta uno dei principali passi della progettazione software.
L'implementazione di nuovi requisiti, la modifica o cancellazione di quelli preesistenti, sono consentite previa discussione con il proponente e dopo la sua approvazione; in ogni caso non sono permesse durante la fase di sviluppo del processo corrente.
Tale modello di sviluppo si combina bene con il versionamento del sistema, che ne traccia le modifiche rendendole più visibili.
I vantaggi dell'utilizzo del modello incrementale sono i seguenti:
\begin{itemize}
    \item le funzionalità primarie hanno priorità nello sviluppo, in questo modo al proponente può valutarle subito;
    \item è possibile ricevere il feedback del cliente frequentemente, anche a ogni incremento;
    \item lo sviluppo a incrementi successivi permette di limitare gli errori al singolo incremento;
    \item l'individuazione degli errori, la correzione e l'apporto di modifiche sono più economiche;
    \item vengono facilitate le fasi di verifica e test perché più mirate.
\end{itemize}

\subsection{Incrementi}
Durante i periodi di Progettazione e Codifica per la Technology Baseline sono stati apportati alcuni incrementi. Di seguito ne riportiamo il tracciamento, in modo tale da comprendere meglio quali requisiti vengono soddisfatti in ciascun incremento
\newline
Tutti i requisiti sono reperibili all'interno del documento \textit{Analisi dei Requisiti v.2.0.0}
\subsubsection{Incrementi per la fase di progettazione architetturale}
%Tabella incrementi
\renewcommand{\arraystretch}{1.5}
\rowcolors{2}{logo!40}{logo!10}
\arrayrulecolor{white}
\begin{longtable}{
    >{\centering}p{0.15\textwidth}
    >{\raggedright}p{0.40\textwidth}
    >{\centering}p{0.15\textwidth}
    >{\centering}p{0.15\textwidth}
    }

    \caption{Tabella di tracciamento progettazione architetturale}                                                                \\
    \rowcolor{white}                                                                                                              \\
    \rowcolor{logo!70}
    \centering\textbf{Incremento} & \centering\textbf{Obbiettivi}  & \centering\textbf{Casi d'uso} & \centering\textbf{Requisiti}
    \tabularnewline
    \endfirsthead
    \rowcolor{white}\caption[]{(continua)}                                                                                        \\
    \rowcolor{logo!70}
    \centering\textbf{Incremento} & \centering\textbf{Obbiettivi}  & \centering\textbf{Casi d'uso} & \centering\textbf{Requisiti}
    \tabularnewline
    \endhead

    %Incremento 1 ---------------------------------------------------------
    Incremento 1                  &
    \vspace{-15px}
    \begin{itemize}
        \renewcommand\labelitemi{-}
        \item creazione della homepage;
        \item aggiunta del bottone per accedere direttamente al carrello.
    \end{itemize}     & Nessun caso d'uso implementato & R1F                                                          \\ R1.3F
    \tabularnewline

    %Incremento 2 ---------------------------------------------------------
    Incremento 2                  &
    \vspace{-15px}
    \begin{itemize}
        \renewcommand\labelitemi{-}
        \item implementazione di registrazione e accesso tramite \textit{Amazon Cognito};
        \item aggiunta del bottone per accedere alla pagina di registrazione e autenticazione.
    \end{itemize}     & UC1                                                                                           \\ UC2 \\  UC9 \\ UC16 \\ UC18 \\ UC33                  & R2F                          \\ R3F \\ R3.1F \\ R3.2F \\
    R11F                                                                                                                          \\ R14F \\ R16F \\ R31F
    \tabularnewline

    %Incremento 3 ---------------------------------------------------------
    Incremento 3                  &
    \vspace{-15px}
    \begin{itemize}
        \renewcommand\labelitemi{-}
        \item creazione della lista dei prodotti;
        \item aggiunta del bottone per accedere alla lista dei prodotti;
        \item aggiunta delle chiamate \textit{Lambda} per prendere i dati dal backend;
        \item aggiunta della pagina di dettaglio per ogni prodotto della piattaforma.
    \end{itemize}     & UC4                                                                                           \\ UC7                              & R5F                          \\ R5.1F \\ R5.3F \\ R5.4F \\ R5.6F \\ R8F
    \tabularnewline

    %Incremento 4 ---------------------------------------------------------
    Incremento 4                  & \vspace{-15px}
    \begin{itemize}
        \renewcommand\labelitemi{-}
        \item creazione del carrello con visualizzazione dei prodotti in esso inseriti.
    \end{itemize}     & UC11                           & R13.1F
    \tabularnewline

    %Incremento 5 ---------------------------------------------------------
    Incremento 5                  & \vspace{-15px}
    \begin{itemize}
        \renewcommand\labelitemi{-}
        \item implementazione del servizio di pagamento \textit{Stripe};
        \item aggiunta del bottone per effettuare il checkout.
    \end{itemize}     & UC21                                                                                          \\ UC22                           & R19F                         \\ R19.3F \\ R19.4F \\ R20F
    \tabularnewline
\end{longtable}
\counterwithin{table}{subsection}
\renewcommand{\arraystretch}{1}
% Non so bene se gli incrementi già effettuati devono essere lasciati oppure non credo però cmq al momento gli lascio
\subsubsection{Incrementi per la fase di progettazione e codifica di dettaglio}
\renewcommand{\arraystretch}{1.5}
\rowcolors{2}{logo!40}{logo!10}
\arrayrulecolor{white}
\begin{longtable}{
    >{\centering}p{0.15\textwidth}
    >{\raggedright}p{0.40\textwidth}
    >{\centering}p{0.15\textwidth}
    >{\centering}p{0.15\textwidth}
    }

    \caption{Tabella di tracciamento per la codifica di dettaglio}                                                                 \\
    \rowcolor{white}                                                                                                               \\
    \rowcolor{logo!70}
    \centering\textbf{Incremento} & \centering\textbf{Obbiettivi}  & \centering\textbf{Casi d'uso}  & \centering\textbf{Requisiti}
    \tabularnewline
    \endfirsthead
    \rowcolor{white}\caption[]{(continua)}                                                                                         \\
    \rowcolor{logo!70}
    \centering\textbf{Incremento} & \centering\textbf{Obbiettivi}  & \centering\textbf{Casi d'uso}  & \centering\textbf{Requisiti}
    \tabularnewline
    \endhead

    %Incremento 6 ---------------------------------------------------------
    Incremento 6                  &
    \vspace{-15px}
    \begin{itemize}
        \renewcommand\labelitemi{-}
        \item Preparazione alle attività di progettazione e codifica di dettaglio;
        \item Incremento della documentazione per verifica e miglioramento continuo.
    \end{itemize}     & Nessun caso d'uso implementato & Nessun caso d'uso implementato
    \tabularnewline

    %Incremento 7 ---------------------------------------------------------
    Incremento 7                  &
    \vspace{-15px}
    \begin{itemize}
        \renewcommand\labelitemi{-}
        \item Miglioramento della struttura delle funzioni Lambda, con l'aggiunta dei filtri di ricerca;
        \item  Miglioramento delle chiamate \textit{API GATEWAY} nel frontend per ottenere le informazioni delle funzioni lambda;
        \item Inizio della stesura della documentazione legata al prodotto software;
        \item Incremento della documentazione per verifica e miglioramento continuo.
    \end{itemize}

                                  & UC15                                                                                           \\ UC35                                                                 & R10F                                                                                                       \\ R10.1F \\ R10.2F \\ R10.3F \\ R10.5F\\ R10.6F
    \tabularnewline

    %Incremento 8 ---------------------------------------------------------
    Incremento 8
                                  &
    \vspace{-15px}
    \begin{itemize}
        \renewcommand\labelitemi{-}
        \item Ristrutturazione dei servizi lato backend;
        \item Riconfigurazione del linter \textit{eslint};
        \item Correzione della documentazione in base alla RP;
        \item Incremento della documentazione per verifica e miglioramento continuo.
    \end{itemize}
                                  & Nessun caso d'uso implementato & R8Q                                                           \\ R1V \\ R2V \\ R4V    \tabularnewline

    %Incremento 9 ---------------------------------------------------------
    Incremento 9                  &
    \vspace{-15px}
    \begin{itemize}
        \renewcommand\labelitemi{-}
        \item Implementazione gestione dell'accesso;
        \item Distinzione tra cliente e venditore;
        \item Implementazione lato frontend;
        \item Incremento della documentazione per verifica e miglioramento continuo.
    \end{itemize}
                                  & UC3                                                                                            \\ UC10                                                                                                                                                                 \\ UC17                                                                                                                                                                                  \\ UC19  \\ UC37
                                  & R4F                                                                                            \\ R12F \\ R15F                                                                                                                                                                                     \\ R15.1F \\ R15.2F \\ R15.3F \\ R17F \\ R33F \\ R33.1F \\ R33.2F
    \tabularnewline

    %Incremento 10 ---------------------------------------------------------
    Incremento 10                 & \vspace{-15px}
    \begin{itemize}
        \renewcommand\labelitemi{-}
        \item Completamento Checkout;
        \item Completamento carrello;
        \item Completamento pagina PDP;
        \item Incremento della documentazione per verifica e miglioramento continuo.
    \end{itemize}    & UC4                                                                                            \\ UC5 \\ UC11                                                                                                                                                                                     \\ UC12 \\ UC21 \\ UC22 \\ UC23 \\ UC24 \\ UC36 & R5F \\ R6F \\ R13F  \\ R13.2F \\ R13.5F \\ R13.6F \\ R19F \\ R19.1F \\ R19.2F \\ R19.3F \\ R19.4F \\ R20F \\ R21F \\ R22F
    \tabularnewline
    %Incremento 11 ---------------------------------------------------------
    Incremento 11                 & \vspace{-15px}
    \begin{itemize}
        \renewcommand\labelitemi{-}
        \item Completamento Homepage;
        \item Completamento PLP;
        \item Implementazione ricerca di un prodotto;
        \item Incremento della documentazione per verifica e miglioramento continuo.
    \end{itemize}    & UC6                                                                                            \\ UC7 & R1.1F \\ R1.2F \\ R7F \\ R8F
    \tabularnewline
    %Incremento 12 ---------------------------------------------------------
    Incremento 12                 & \vspace{-15px}
    \begin{itemize}
        \renewcommand\labelitemi{-}
        \item Completamento pagina venditore;
        \item Aggiunta sezione per la gestione delle categorie;
        \item Aggiunta visualizzazione elenco clienti e ordini ricevuti;
        \item Aggiunta sezione per la gestione di un prodotto;
        \item Incremento della documentazione per verifica e miglioramento continuo.
    \end{itemize}    & UC25                                                                                           \\ UC26 \\ UC28 \\ UC29 \\ UC30 \\ UC31 \\ UC32 & R23F \\ R24F \\ R26F \\ R27F \\ R28F \\ R29F \\ R30F
    \tabularnewline
    %Incremento 13 ---------------------------------------------------------
    Incremento 13                 &
    \vspace{-15px}
    \begin{itemize}
        \renewcommand\labelitemi{-}
        \item Preparazione per la \textit{Revisione di Qualifica};
        \item Correzione del codice secondo le indicazioni ricevute durante la \textit{Product Baseline} e con il proponente;
        \item Correzione dell'\textbf{Allegato Tecnico};
        \item Incremento della documentazione per verifica e miglioramento continuo.
    \end{itemize}    & Nessun caso d'uso implementato & Nessun caso d'uso implementato
    \tabularnewline
\end{longtable}
\counterwithin{table}{subsection}
\renewcommand{\arraystretch}{1}
\subsubsection{Incrementi per la fase di validazione e collaudo}
\renewcommand{\arraystretch}{1.5}
\rowcolors{2}{logo!40}{logo!10}
\arrayrulecolor{white}
\begin{longtable}{
    >{\centering}p{0.15\textwidth}
    >{\raggedright}p{0.40\textwidth}
    >{\centering}p{0.15\textwidth}
    >{\centering}p{0.15\textwidth}
    }

    \caption{Tabella di tracciamento per la validazione}                                                                           \\
    \rowcolor{white}                                                                                                               \\
    \rowcolor{logo!70}
    \centering\textbf{Incremento} & \centering\textbf{Obbiettivi}  & \centering\textbf{Casi d'uso}  & \centering\textbf{Requisiti}
    \tabularnewline
    \endfirsthead
    \rowcolor{white}\caption[]{(continua)}                                                                                         \\
    \rowcolor{logo!70}
    \centering\textbf{Incremento} & \centering\textbf{Obbiettivi}  & \centering\textbf{Casi d'uso}  & \centering\textbf{Requisiti}
    \tabularnewline
    \endhead
    Incremento 14                 & \vspace{-15px}
    \begin{itemize}
        \renewcommand\labelitemi{-}
        \item Aggiunta ordinamento alla lista dei prodotti;
        \item Aggiunti filtri per categoria e marca;
        \item Aggiunto il bottone per aggiungere un prodotto al carrello dalla pagina della lista dei prodotti;
        \item Incremento della documentazione per verifica e miglioramento continuo.
    \end{itemize}    & UC4                                                                                            \\ UC15 \\ UC34 \\ UC35  & R5.9F \\ R9F \\ R9.1F \\ R9.2F \\ R10.3F \\ R10.4F
    \tabularnewline
    Incremento 15                 & \vspace{-15px}
    \begin{itemize}
        \renewcommand\labelitemi{-}
        \item Aggiunta gestione della quantità dei prodotti all'interno del carrello;
        \item Aggiunti dettagli del prodotto;
        \item Correzione della documentazione in base alle indicazioni dei proponenti;
        \item Incremento della documentazione per verifica e miglioramento continuo.
    \end{itemize}    & UC4                                                                                            \\ UC13 \\ UC14   & R5.6F \\ R5.7F \\ R13.3F \\ R13.4F
    \tabularnewline
    Incremento 16                 & \vspace{-15px}
    \begin{itemize}
        \renewcommand\labelitemi{-}
        \item Aggiunto la possibilità d'invio di messaggi tra venditore e cliente;
        \item Incremento della documentazione per verifica e miglioramento continuo.
    \end{itemize}    & UC20                                                                                           \\ UC27  & R18F \\ R25F
    \tabularnewline

    Incremento 17                 & \vspace{-15px}
    \begin{itemize}
        \renewcommand\labelitemi{-}
        \item Preparazione all'esposizione per la \textit{Revisione di accettazione};
        \item Preparazione per il collaudo con il proponente;
        \item Incremento della documentazione per verifica e miglioramento continuo.
    \end{itemize}    & Nessun caso d'uso implementato & Nessun caso d'uso implementato
    \tabularnewline
\end{longtable}
\counterwithin{table}{subsection}
\renewcommand{\arraystretch}{1}