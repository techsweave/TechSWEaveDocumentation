\section{Modello di Sviluppo}
La scelta del modello di sviluppo è fondamentale per la pianificazione del progetto, per tale scopo è stato adottato il \textbf{modello incrementale.}

\subsection{Modello incrementale}
Il modello di sviluppo incrementale permette di frazionare lo sviluppo del sistema per incrementi, ognuno dei quali rappresenta uno dei principali passi della progettazione software.
L'implementazione di nuovi requisiti, la modifica o cancellazione di quelli preesistenti, sono consentite previa discussione con il proponente e dopo la sua approvazione; in ogni caso non sono permesse durante la fase di sviluppo del processo corrente.
Tale modello di sviluppo si combina bene con il versionamento del sistema, che ne traccia le modifiche rendendole più visibili.
I vantaggi dell'utilizzo del modello incrementale sono i seguenti:
\begin{itemize}
    \item le funzionalità primarie hanno priorità nello sviluppo, in questo modo al proponente può valutarle subito;
    \item è possibile ricevere il feedback del cliente frequentemente, anche a ogni incremento;
    \item lo sviluppo a incrementi successivi permette di limitare gli errori al singolo incremento;
    \item l'individuazione degli errori, la correzione e l'apporto di modifiche sono più economiche;
    \item vengono facilitate le fasi di verifica e test perché più mirate.
\end{itemize}

\subsection{Incrementi}
Durante i periodi di Progettazione e Codifica per la Technology Baseline sono stati apportati alcuni incrementi. Di seguito ne riportiamo il tracciamento, in modo tale da comprendere meglio quali requisiti vengono soddisfatti in ciascun incremento
\newline
Tutti i requisiti sono reperibili all'interno del documento \textit{Analisi dei Requisiti v.2.0.0}
\subsubsection{Incrementi per la fase di progettazione architetturale}
%Tabella incrementi
\renewcommand{\arraystretch}{1.5}
\rowcolors{2}{logo!40}{logo!10}
\arrayrulecolor{white}
\begin{longtable}{
    >{\centering}p{0.50\textwidth}
    >{\raggedright}p{0.28\textwidth}
    >{\raggedright}p{0.29\textwidth}
    >{\centering}p{0.50\textwidth}
    }

    \caption{Tabella di Tracciamento}                            \\
    \rowcolor{white}                                             \\
    \rowcolor{logo!70}
    \centering\textbf{Incremento} & \centering\textbf{Requisiti}
    \tabularnewline
    \endfirsthead
    \rowcolor{white}\caption[]{(continua)}                       \\
    \rowcolor{logo!70}
    \centering\textbf{Incremento} & \centering\textbf{Requisiti}
    \tabularnewline
    \endhead

    %Incremento 1 ---------------------------------------------------------
    Incremento 1 - Homepage       & R1F                          \\ R1.3F
    \tabularnewline

    %Incremento 2 ---------------------------------------------------------
    Incremento 2 - Login          & R2F                          \\ R3F \\ R3.1F \\ R3.2F \\
    R11                                                          \\ R14F \\ R16F \\ R31F
    \tabularnewline

    %Incremento 3 ---------------------------------------------------------
    Incremento 3 - Prodotto       & R5F                          \\ R5.1F \\ R5.3F \\ R5.4F \\ R5.6F \\ R8F
    \tabularnewline

    %Incremento 4 ---------------------------------------------------------
    Incremento 4 - Carrello       & R13.1F
    \tabularnewline

    %Incremento 5 ---------------------------------------------------------
    Incremento 5 - Checkout       & R19F                         \\ R19.3F \\ R19.4F \\ R20F
    \tabularnewline
\end{longtable}
\counterwithin{table}{subsection}
\renewcommand{\arraystretch}{1}
% Non so bene se gli incrementi già effettuati devono essere lasciati oppure non credo però cmq al momento gli lascio
\subsubsection{Incrementi per la fase di progettazione e codifica di dettaglio}
\renewcommand{\arraystretch}{1.5}
\rowcolors{2}{logo!40}{logo!10}
\arrayrulecolor{white}
\begin{longtable}{
    >{\centering}p{0.15\textwidth}
    >{\raggedright}p{0.40\textwidth}
    >{\centering}p{0.15\textwidth}
    >{\centering}p{0.15\textwidth}
    }

    \caption{Tabella di Tracciamento}                                                                                                                                                                                        \\
    \rowcolor{white}                                                                                                                                                                                                         \\
    \rowcolor{logo!70}
    \centering\textbf{Incremento} & \centering\textbf{Obbiettivi}                                               & \centering\textbf{Casi d'uso}                                               & \centering\textbf{Requisiti}
    \tabularnewline
    \endfirsthead
    \rowcolor{white}\caption[]{(continua)}                                                                                                                                                                                   \\
    \rowcolor{logo!70}
    \centering\textbf{Incremento} & \centering\textbf{Obbiettivi}                                               & \centering\textbf{Casi d'uso}                                               & \centering\textbf{Requisiti}
    \tabularnewline
    \endhead

    %Incremento 6 ---------------------------------------------------------
    Incremento 6                  &
    \vspace{-15px}
    \begin{itemize}
        \renewcommand\labelitemi{-}
        \item Preparazione alle attività di progettazione e codifica di dettaglio;
        \item Incremento della documentazione per verifica e miglioramento continuo.
    \end{itemize}     & Durante questo incremento non verranno aggiunte nuove funzionalità software & Durante questo incremento non verranno aggiunte nuove funzionalità software
    \tabularnewline

    %Incremento 7 ---------------------------------------------------------
    Incremento 7                  &
    \vspace{-15px}
    \begin{itemize}
        \renewcommand\labelitemi{-}
        \item Miglioramento della struttura delle funzioni Lambda, con l'aggiunta dei filtri di ricerca;
        \item  Miglioramento delle chiamate \textit{API GATEWAY} nel frontend per ottenere le informazioni delle funzioni lambda;
        \item Inizio della stesura della documentazione legata al prodotto software;
        \item Incremento della documentazione per verifica e miglioramento continuo.
    \end{itemize}

                                  & UC15                                                                                                                                                                                     \\ UC35                                                                 & R10F                                                                                                       \\ R10.1F \\ R10.2F \\ R10.3F \\ R10.4F \\ R10.5F\\ R10.6F
    \tabularnewline

    %Incremento 8 ---------------------------------------------------------
    Incremento 8
                                  &
    \vspace{-15px}
    \begin{itemize}
        \renewcommand\labelitemi{-}
        \item Ristrutturazione dei servizi lato backend;
        \item Riconfigurazione del linter \textit{eslint};
        \item Correzione della documentazione in base alla RP;
        \item Incremento della documentazione per verifica e miglioramento continuo.
    \end{itemize}
                                  & Nessun caso d'uso implementato                                              & R8Q                                                                                                        \\ R1V \\ R2V \\ R4V    \tabularnewline

    %Incremento 9 ---------------------------------------------------------
    Incremento 9                  &
    \vspace{-15px}
    \begin{itemize}
        \renewcommand\labelitemi{-}
        \item Implementazione gestione dell'accesso;
        \item Distinzione tra cliente e venditore;
        \item Implementazione lato frontend;
        \item Incremento della documentazione per verifica e miglioramento continuo.
    \end{itemize}
                                  & UC1                                                                                                                                                                                      \\ UC2 \\ UC3 \\ UC8 \\ UC9 \\ UC10                                                                                                                                                                                 \\ UC16 \\ UC17  \\ UC18                                                                                                                                                                                   \\ UC19 \\ UC33 \\ UC37
                                  & R15F                                                                                                                                                                                     \\ R15.1F \\ R15.2F \\ R15.3F \\ R17F \\ R33F \\ R33.1F \\ R33.2F
    \tabularnewline

    %Incremento 10 ---------------------------------------------------------
    Incremento 10                 & \vspace{-15px}
    \begin{itemize}
        \renewcommand\labelitemi{-}
        \item Completamento Checkout;
        \item Completamento carrello;
        \item Completamento pagina PDP;
        \item Incremento della documentazione per verifica e miglioramento continuo.
    \end{itemize}     & UC4                                                                                                                                                                                      \\ UC5 UC11                                                                                                                                                                                     \\ UC12 \\ UC21 \\ UC22 \\ UC23 \\ UC24 \\ UC36 & R13F  \\ R13.2F \\ R13.5F \\ R13.6F
    \tabularnewline
    %Incremento 11 ---------------------------------------------------------
    Incremento 11                 & \vspace{-15px}
    \begin{itemize}
        \renewcommand\labelitemi{-}
        \item Completamento Homepage;
        \item Completamento PLP;
        \item Implementazione ricerca di un prodotto;
        \item Incremento della documentazione per verifica e miglioramento continuo.
    \end{itemize}     & UC11                                                                                                                                                                                     \\ UC12 \\ UC21 \\ UC22 \\ UC23 \\ UC34 & R1.1F \\ R1.2F
    \tabularnewline
    %Incremento 12 ---------------------------------------------------------
    Incremento 12                 & \vspace{-15px}
    \begin{itemize}
        \renewcommand\labelitemi{-}
        \item Completamento pagina venditore;
        \item Aggiunta sezione per la gestione delle categorie;
        \item Aggiunta visualizzazione elenco clienti e ordini ricevuti;
        \item Aggiunta sezione per la gestione di un prodotto;
        \item Incremento della documentazione per verifica e miglioramento continuo.
    \end{itemize}     & UC25                                                                                                                                                                                     \\ UC16 \\ UC28 \\ UC29 \\ UC30 \\ UC31 \\ UC32 &
    \tabularnewline
\end{longtable}
\counterwithin{table}{subsection}
\renewcommand{\arraystretch}{1}
