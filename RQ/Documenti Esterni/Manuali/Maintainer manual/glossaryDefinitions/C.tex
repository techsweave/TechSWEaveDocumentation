\section{}

% \subsection*{Callback} È un metodologia di programmazione in cui una funzione viene passata come parametro ad un'altra funzione.

% \subsection*{camelCase} È la pratica di scrivere le parole unite tra loro, con le iniziali maiuscole, a differenza della prima parola che va con l'iniziale minuscola(es. camelCase, metodoDiJava).

% \subsection*{Capitolato} Documento tecnico proposto da un committente in cui si definiscono i vincoli da utilizzare per sviluppare un prodotto.

% \subsection*{Casi d'uso} Sono le modalità in cui si può interagire con il sistema, sono le funzionalità che il sistema mette a disposizione.

% \subsection*{Checkout} Sono tutti i passaggi che un cliente deve fare per completare l'acquisto di un prodotto.

% \subsection*{Ciclo di vita del software} Sono gli stati che un prodotto software assume tra il concepimento e il ritiro, in conseguenza delle attività svolte su di esso.

% \subsection*{Coesione} È una misura di quanto correlate siano le varie funzionalità messe a disposizione da una unità. Queste funzionalità hanno uno scopo in comune.

% \subsection*{CSS (Cascading Style Sheets)} È un linguaggio usato per definire la formattazione di documenti HTML, XHTML e XML ad esempio di siti web.
