\section{Common technologies and libraries}
\subsection{Typescript}
TypeScript is a programming language developed and maintained by Microsoft. It is a strict syntactical
superset of JavaScript and adds optional static typing to the language. TypeScript is designed for the
development of large applications and transcompiles to JavaScript. As TypeScript is a superset of JavaScript,
existing JavaScript programs are also valid TypeScript programs.
\begin{itemize}
  \item \textbf{Used Version: 4.3.4}
  \item \textbf{Link: \url{https://aws.amazon.com/sqs/?nc1=h_ls}}
\end{itemize}
\subsection{JSON}
JSON (JavaScript Object Notation) is an open standard file format, and data interchange format, that
uses human-readable text to store and transmit data objects consisting of attribute–value pairs and
array data types (or any other serializable value).
JSON is a language-independent data format. It was derived from JavaScript,
but many modern programming languages include code to generate and parse JSON-format data.
\begin{itemize}
  \item \textbf{Link: \url{https://www.json.org/json-en.html}}
\end{itemize}
\subsection{npm}
npm is the default package manager\textsubscript{\textbf{G}} for the JavaScript runtime environment Node.js.
It consists of a command line client, also called npm, and an online database of public and paid-for private packages,
called the npm registry. The registry is accessed via the client, and the available packages can be browsed and
searched via the npm website.
\begin{itemize}
  \item \textbf{Used Version: 6.14.13}
  \item \textbf{Link: \url{https://www.npmjs.com/}}
\end{itemize}
\subsection{yarn}
Yarn is a new package manager that replaces the existing workflow for the npm client or other package managers while
remaining compatible with the npm registry.
\begin{itemize}
  \item \textbf{Used Version: 1.22.10}
  \item \textbf{Link: \url{https://yarnpkg.com/}}
\end{itemize}
\subsection{ESLint}
ESLint is a tool for identifying and reporting on patterns found in ECMAScript/JavaScript code, with the goal of making code more consistent and avoiding bugs. 
\begin{itemize}
  \item \textbf{Used Version: 7.26.0}
  \item \textbf{Link: \url{https://eslint.org/}}
\end{itemize}