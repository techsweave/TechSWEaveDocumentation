\section{Setup}
\subsection{Minimum requirements}
To deploy the platform, you must have a valid AWS account to obtain the credentials to be included in the configuration of the serverless framework profile.\newline
The system where the deployment will be carried out must have the following technologies installed:
\begin{itemize}
    \item aws-cli;
    \item serverless;
    \item nodeJS;
    \item npm.
\end{itemize}
In order to access the platform, it will be necessary to use a browser, preferably updated to the latest version.
\subsection{Deploy}
To deploy the platform, follow these steps:
\begin{itemize}
    \item go to the folder downloaded from github;
    \item run the ./deploy.sh command;
    \item enter the data through which all aws services, the aws and serverless account will be configured.
\end{itemize}
\subsection{Creation of a vendor account}
To create a vendor account, follow these steps:
\begin{itemize}
    \item go to the folder downloaded from github;
    \item run the ./createVendor.sh command;
    \item Enter the data through which the vendor account will be created.
\end{itemize}
\subsection{Testing}
For static testing of the code, ESLint is used to check that the code meets certain standards.\newline
For dynamic testing of the code chai is used, a library that provides functions for making assertions on the code.
Jest was used for frontend testing.