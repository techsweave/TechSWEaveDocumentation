\section{}
% \subsection*{Git}
% È un sistema di controllo della versione distribuito per tenere traccia delle modifiche al codice sorgente durante lo sviluppo del software. Progettato per coordinare il lavoro tra i \textit{Programmatori}, ma utilizzabile per tenere traccia delle modifiche in qualsiasi set di file. I suoi obiettivi sono velocità, integrità dei dati e supporto per flussi di lavoro distribuiti e non lineari.

% \subsection*{GitHub}
% È un servizio di hosting per progetti software. È una implementazione dello strumento di controllo versione distribuito Git.

% \subsection*{GitLab} È una piattaforma web open source che permette la gestione di repository Git e di funzioni trouble ticket.

% \subsection*{gRPC} È un moderno sistema di \textit{remote procedure call} che, grazie a tecniche di processo innovative, rende possibile una comunicazione particolarmente efficiente all’interno delle architetture client-server distribuite.

% \subsection*{GUI (Graphical User Interface)} È un tipo di interfaccia utente che consente l'interazione uomo macchina attraverso rappresentazioni grafiche.
