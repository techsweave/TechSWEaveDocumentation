\section{}
% \subsection*{PAAS (Platform As A Service)} È una tipologia di architettura a servizi in cui si mettono a disposizione del cliente piattaforme di elaborazione.

% \subsection*{PascalCase} È la pratica di scrivere le parole unite tra loro, con le iniziali maiuscole (es. PascalCase, ClasseDiJava).

% \subsection*{Predictive Analytics} Sono delle tecniche statistiche della modellazione predittiva, apprendimento automatico e data mining per analizzare fatti storici e fornire predizioni su eventi futuri o sconosciuti.

% \subsection*{Product baseline} È una descrizione concordata degli attributi del prodotto in un determinato momento, che funge da base per la definizione del cambiamento.

% \subsection*{Proof of Concept} È una realizzazione incompleta o abbozzata di un determinato progetto o metodo, allo scopo di provarne la fattibilità o dimostrare la fondatezza di alcuni principi o concetti costituenti. Un esempio tipico è quello di un prototipo.

% \subsection*{Provisioning} È il processo di configurazione di un'infrastruttura IT.
% Lo stesso termine è utilizzato per definire anche le procedure necessarie per gestire l'accesso ai dati e alle risorse e per renderle disponibili a utenti e sistemi.

% \subsection*{Pull Request} È una richiesta che un contributore effettua al Responsabile di Progetto per includere nel progetto le modifiche che ha effettuato

% \subsection*{Python} È un linguaggio di programmazione di più alto livello rispetto alla maggior parte degli altri linguaggi, orientato a oggetti, adatto a sviluppare applicazioni distribuite, scripting, computazione numerica e system testing.
\subsection*{Package manager}
A package manager or package-management system is a collection of software tools that automates the process of installing, upgrading, configuring, and removing computer programs for a computer's operating system in a consistent manner.