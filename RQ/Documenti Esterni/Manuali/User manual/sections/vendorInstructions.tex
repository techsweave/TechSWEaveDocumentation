\section{Vendor instructions}
\subsection{Orders view}
By clicking on the \textit{Orders} button in the header, the seller reaches the orders page, where he can view all the orders placed by customers.
\newline
From this page the vendor can view for each order:
\begin {itemize}
\item the order code;
\item the status of the order;
\item the total and subtotal price;
\item the quantity of ordered items;
\item the checkout date;
\item the name of the customer who placed the order.
\end {itemize}
\begin{figure}[!ht]
    \caption{Orders page viewed from the vendor}
    \vspace{5px}
    \includegraphics[scale=0.25]{../../../../Images/userManual/ordersVendor.png}
    \centering
\end{figure}
\pagebreak
\subsection{Shop management}
By clicking on the \textit{Manage shop} button in the header, the vendor reaches the page to manage the shop, where he can create a new product, create a new category or delete a category.
\subsubsection{Create new product}
By clicking on \textit{Create New Product}, a screen will be displayed where the vendor can insert a new product in the system. Here the vendor can:
\begin {itemize}
\item enter the product title;
\item enter the price of the product;
\item enter the description;
\item enter the discount;
\item enter the availability;
\item insert an image of the product;
\item select the product category;
\item add notes.
\end {itemize}
After having entered the necessary data for the new product, it will be possible to insert the product by clicking on the \textit{Submit} button at the bottom of the page.
\begin{figure}[!ht]
    \caption{Create new product section viewed from the vendor}
    \vspace{5px}
    \includegraphics[scale=0.25]{../../../../Images/userManual/createNewProductVendor.png}
    \centering
\end{figure}
\pagebreak
\subsubsection{Create new category}
By clicking on \textit{Create New Category}, a screen will be displayed where the vendor can insert a new category of products in the system. Here the vendor can:
\begin {itemize}
\item enter the category name;
\item enter the description;
\item enter the taxation for the category;
\item add additional customizable specifications based on the category.
\end {itemize} 
After entering the necessary data for the new category, it will be possible to insert the category by clicking on the \textit{Submit} button at the bottom of the page.
\begin{figure}[!ht]
    \caption{Create new category section viewed from the vendor}
    \vspace{5px}
    \includegraphics[scale=0.25]{../../../../Images/userManual/createNewCategoryVendor.png}
    \centering
\end{figure}
\subsubsubsection{Add new template specification}
By clicking on the \textit{Add New Template Specification} button displayed at the top right of the screen for adding a new category, the vendor can add one or more additional customizable specifications related to the category.
\linebreak
A row will be displayed for each new specification, in which the vendor can enter:
\begin {itemize}
\item the name of the specification;
\item the unit of measurement of the specification.
\end {itemize}
The specifications will finally be added to the category by clicking on the \textit{Submit} button to create the category itself.
\begin{figure}[!ht]
    \caption{Create new category section viewed from the vendor}
    \vspace{5px}
    \includegraphics[scale=0.25]{../../../../Images/userManual/addSpecificationCategoryVendor.png}
    \centering
\end{figure}
\pagebreak
\subsubsection{Delete categories}
By clicking on \textit{Delete Categories}, a screen will be displayed where the vendor can delete the categories.\linebreak
To remove a specific category the vendor will have to click on the button \textit{DELETE CATEGORY}, displayed on the bottom of the category that he wants to delete.
\begin{figure}[!ht]
    \caption{Delete categories section viewed from the vendor}
    \vspace{5px}
    \includegraphics[scale=0.25]{../../../../Images/userManual/deleteCategoriesVendor.png}
    \centering
\end{figure}