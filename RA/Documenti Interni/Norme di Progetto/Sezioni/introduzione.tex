\section{Introduzione}
\subsection{Scopo del documento}
Questo documento ha lo scopo di stabilire tutte le regole per uniformare il modo di lavorare di tutto il team, per ottenere un'organizzazione efficiente dei file prodotti.\\Durante lo svolgimento del progetto tutti i membri sono tenuti a visionare questo documento per rispettare le norme definite.
Le norme possono subire modifiche, che dovranno essere tempestivamente comunicate a ciascun membro del gruppo.

\subsection{Scopo del prodotto}
L'obiettivo del progetto è quello di realizzare una piattaforma e-commerce basata su tecnologie serverless. Dovranno essere implementate delle funzioni irrinunciabili per tutte le categorie di utenti che ne faranno uso:
\begin{itemize}
    \item clienti;
    \item commercianti;
    \item admin.
\end{itemize}
\subsection{Glossario}
All'interno del documento sono presenti termini che potrebbero risultare ambigui a seconda del contesto. Al fine di evitare possibili incomprensioni
e rendere chiari agli stakeholders\textsubscript{\textbf{G}} i termini utilizzati, viene fornito un \textit{Glossario v3.0.0.} contenente i suddetti termini
e la loro spiegazione. Nella seguente documentazione tali termini saranno individuabili tramite una '\textbf{G}' a pedice.


\subsubsection{Riferimenti informativi}
\begin{itemize}
    \item ISO 12207:1995\\
          \href{https://www.math.unipd.it/~tullio/IS-1/2009/Approfondimenti/ISO\_12207-1995.pdf}{https://www.math.unipd.it/~tullio/IS-1/2009/Approfondimenti/ISO\_12207-1995.pdf}
    \item Slide lezione L03\\ \href{https://www.math.unipd.it/~tullio/IS-1/2020/Dispense/L03.pdf}{https://www.math.unipd.it/~tullio/IS-1/2020/Dispense/L03.pdf}
    \item Libro di testo: Ingegneria del Software - 10\textsuperscript{A} edizione - Ian Sommerville
\end{itemize}