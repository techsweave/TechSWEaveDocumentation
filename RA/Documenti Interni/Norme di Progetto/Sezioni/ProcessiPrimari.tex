\section{Processi Primari}
\subsection{Fornitura}
\subsubsection{Scopo}
Il processo di Fornitura, come stabilito dallo standard ISO/IEC 12207:1995\textsubscript{\textbf{G}}, descrive le attività e i compiti che il fornitore deve rispettare per soddisfare le richieste del proponente. Dopo una prima attività di analisi per comprendere le richieste del proponente, il fornitore redige lo \textit{Studio di Fattibilità} al fine d'individuare gli aspetti positivi e le criticità in base alle informazioni acquisite. Occorre quindi stipulare un contratto con cui regolare i rapporti e le scadenze con il proponente e definire un \textit{Piano di Progetto} da seguire fino alla consegna del prodotto finale.
\\Il processo di fornitura è composto dalle fasi seguenti:
\begin{itemize}
    \item avvio;
    \item approntamento di risposte alle richieste;
    \item contrattazione;
    \item pianificazione;
    \item esecuzione e controllo;
    \item revisione e valutazione;
    \item consegna e completamento.
\end{itemize}
\subsubsection{Descrizione}
L'obiettivo del processo di Fornitura consiste nell'individuare e formalizzare norme e procedure a cui il gruppo \textit{TechSWEave} deve attenersi durante tutte le fasi di realizzazione del prodotto per diventare fornitore del proponente \textit{RedBabel} e dei committenti Prof. Tullio Vardanega e Prof. Riccardo Cardin.
\subsubsection{Aspettative}
Per ottenere un riscontro efficace sul lavoro svolto, il gruppo \textit{TechSWEave} si propone di
instaurare e mantenere un dialogo continuo e un costante rapporto collaborativo con l'azienda \textit{RedBabel}
\subsubsection{Rapporti con l'azienda RedBabel}
Tale rapporto permette di arrivare ad un contratto fra le parti per:
\begin{itemize}
    \item determinare i bisogni che il prodotto finale dovrà soddisfare;
    \item fissare i vincoli e i requisiti sui processi;
    \item stimare le tempistiche di lavoro;
    \item chiarire eventuali dubbi;
    \item assicurare una verifica continua;
    \item accordarsi sulla qualifica del prodotto.
\end{itemize}
\subsubsection{Studio di Fattibilità}
Il \textit{Responsabile di Progetto}\textsubscript{\textbf{G}} ha il compito di convocare i membri del gruppo per discutere dei capitolati d'appalto disponibili, effettuando una prima analisi del materiale proposto. Per ogni capitolato gli \textit{Analisti}\textsubscript{\textbf{G}} provvedono poi a redigere uno \textit{Studio di Fattibilità} in cui viene analizzato ulteriormente il materiale disponibile.
\\Lo \textit{Studio di Fattibiltà} comprende i seguenti punti:
\begin{itemize}
    \item \textbf{Informazioni generali}: informazioni di base quali titolo del capitolato, nome del proponente e del committente;
    \item \textbf{Descrizione del capitolato}: breve sintesi delle caratteristiche del prodotto da sviluppare;
    \item \textbf{Finalità del progetto}: le finalità richieste dal capitolato d'appalto;
    \item \textbf{Tecnologie interessate}: elenco delle tecnologie da utilizzare per la realizzazione del progetto;
    \item \textbf{Aspetti positivi}: elenco di motivi per cui il gruppo reputa interessante il progetto proposto;
    \item \textbf{Aspetti critici}: elenco di motivi per cui il gruppo reputa rischioso o di difficile attuazione il progetto proposto;
    \item \textbf{Conclusioni}: valutazione finale con cui viene dichiarata l'intenzione del gruppo di accettare o rifiutare il capitolato.
\end{itemize}
\subsubsection{Materiale fornito}
Al fine di conseguire un metodo di tracciamento in modo da assicurarne la trasparenza durante tutto il ciclo di vita del progetto, è necessario fornire al proponente \textit{RedBabel} e ai committenti Prof. Tullio Vardanega e Prof. Riccardo Cardin i documenti a seguire.
\subsubsubsection{Analisi dei Requisiti}
Contiene l'analisi dei casi d'uso e dei requisiti (diretti ed indiretti, espliciti e non). Tale documento punta ad evitare
l'insorgere di ambiguità riguardanti il capitolato e definire nel dettaglio le funzionalità del prodotto.
\subsubsubsection{Piano di Progetto}
Il \textit{Responsabile}\textsubscript{\textbf{G}}, con il supporto degli \textit{Amministratori}\textsubscript{\textbf{G}}, redige il \textit{Piano di Progetto} da seguire durante il progetto.
\\Questo documento è strutturato nel modo seguente:
\begin{itemize}
    \item \textbf{Analisi dei rischi}: vengono analizzati i possibili rischi riscontrabili durante la realizzazione del progetto, la probabilità con cui possono presentarsi e il rispettivo livello di gravità;
    \item \textbf{Modello di sviluppo}: descrizione del modello di sviluppo scelto, indispensabile per la progettazione;
    \item \textbf{Pianificazione}: vengono pianificate le attività e le rispettive scadenze temporali che si susseguiranno durante il progetto;
    \item \textbf{Preventivo e consuntivo}: viene calcolato un preventivo per il costo totale del progetto ed esposto un consuntivo di periodo per evidenziare eventuali discostamenti da quanto preventivato.
\end{itemize}
\subsubsubsection{Piano di Qualifica}
I \textit{Verificatori} redigono un documento chiamato \textit{Piano di Qualifica} che stabilisce le strategie da rispettare per garantire la qualità dei processi\textsubscript{\textbf{G}} attuati e del materiale prodotto dal gruppo.
\\Questo documento è strutturato nel modo seguente:
\begin{itemize}
    \item \textbf{Qualità di processo}: vengono individuati dei processi dagli standard e fissate delle metriche\textsubscript{\textbf{G}} per misurarli;
    \item \textbf{Qualità di prodotto}: vengono individuati gli attributi principali del prodotto, definiti degli obiettivi per raggiungerli e delle metriche\textsubscript{\textbf{G}} per misurarli;
    \item \textbf{Specifiche dei test}: definiscono dei test che il prodotto deve passare per garantire il soddisfacimento dei requisiti;
    \item \textbf{Resoconto delle attività di verifica}: vengono riportati e descritti i risultati delle metriche\textsubscript{\textbf{G}} calcolate in forma di resoconto;
\end{itemize}
\subsubsubsection{Glossario}
Questo documento contiene una definizione dei termini utilizzati nella documentazione che potrebbero risultare ambigui a seconda del contesto in cui si trovano.
\subsubsubsection{Proof of Concept e Technology Baseline}
Nell'ambito della Progettazione architetturale, definiscono una panoramica ad alto livello dell'applicazione che il gruppo
TechSWEave realizzerà e delle tecnologie impiegate (vedi \ref{Periodi della progettazione}).
\subsubsubsection{Product Baseline}
Nell'ambito dell'attività di Progettazione di dettaglio, definisce l'insieme di classi, metodi, attributi e scelte implementative a livello tecnico (vedi \ref{Periodi della progettazione}).
\subsubsection{Strumenti}
Di seguito è riportato l'elenco degli strumenti impiegati dal gruppo per il processo di fornitura.
\subsubsubsection{Gantt Project}
Software utile alla realizzazione dei diagrammi di Gantt\textsubscript{\textbf{G}}, utili nella pianificazione, gestione e assegnazione delle risorse;
\subsubsubsection{Microsoft Excel}
Software utile alla produzione e alla gestione di fogli elettronici. Utilizzato per la creazione di diagrammi, tabelle e per il calcolo matematico.
\subsection{Sviluppo}
\subsubsection{Scopo}
Lo scopo del processo di sviluppo, come stabilito dallo standard ISO/IEC 12207:1995\textsubscript{\textbf{G}}, è descrivere i compiti e le attività da svolgere relative al prodotto software da sviluppare.
\subsubsection{Descrizione}
Di seguito sono elencate le attività che compongono il processo di sviluppo:
\begin{itemize}
    \item Analisi dei requisiti;
    \item Progettazione;
    \item Codifica del software.
\end{itemize}
\subsubsection{Aspettative}
Le aspettative sono le seguenti:
\begin{itemize}
    \item determinare gli obiettivi di sviluppo;
    \item determinare i vincoli tecnologici;
    \item determinare i vincoli di design;
    \item realizzare un prodotto finale che supera i test, che soddisfa i requisiti e le richieste del proponente.
\end{itemize}
\subsubsection{Analisi dei requisiti}
\subsubsubsection{Scopo}
Il documento di \textit{Analisi dei Requisiti} è redatto dagli \textit{Analisti}\textsubscript{\textbf{G}} ed ha come scopo quello d'individuare i requisiti diretti e indiretti, espliciti e impliciti ed ha come scopo quello di:
\begin{itemize}
    \item definire gli obiettivi di sviluppo;
    \item definire le funzionalità del prodotto concordate con il proponente;
    \item definire i vincoli tecnologici;
    \item definire i vincoli di design;
    \item fornire ai verificatori dei riferimenti per l'attività di controllo;
    \item fornire una stima del quantitativo di lavoro da svolgere per tracciare una stima dei costi;
    \item realizzare un prodotto finale che superi i test, che soddisfi i requisiti e le richieste del proponente.
\end{itemize}
\subsubsubsection{Descrizione}
I requisiti si possono trarre da diverse fonti:
\begin{itemize}
    \item \textbf{Capitolati d'Appalto}: il requisito è esplicitato nel capitolato del proponente;
    \item \textbf{Verbali Interni}: il requisito è individuato durante le riunioni del gruppo \textit{TechSWEave};
    \item \textbf{Verbali Esterni}: il requisito è individuato durante le riunioni o in seguito a comunicazioni con il proponente;
    \item \textbf{Casi d'Uso}: il requisito è ricavato da uno o più casi d'uso.
\end{itemize}
\subsubsubsection{Aspettative}
L'obiettivo è quello di creare un documento formale contenente tutti i requisiti richiesti e concordati con il proponente.

\subsubsubsection{Classificazione dei requisiti}
Per la classificazione dei requisiti si è deciso per la seguente convenzione:
\begin{itemize}
    \item \textbf{Codice identificativo:} ogni codice identificativo è univoco e conforme alla seguente codifica:
          \begin{center}
              \textbf{R[ID][Tipologia]}
          \end{center}
          Il significato delle voci è:
          \begin{itemize}
              \item \textbf{ID:} identificatore univoco del requisito in forma gerarchica.
              \item \textbf{Tipologia:} ogni requisito può assumere uno dei seguenti valori:
                    \begin{itemize}
                        \item \textit{F}: Funzionale;
                        \item \textit{Q}: Qualitativo;
                        \item \textit{V}: Vincolo;
                        \item \textit{P}: Prestazionale.
                    \end{itemize}
          \end{itemize}

    \item \textbf{Importanza:} riporta l'importanza del requisito:
          \begin{itemize}
              \item obbligatorio: requisito irrinunciabile per gli stakeholders\textsubscript{\textbf{G}};
              \item desiderabile: requisito non strettamente necessario che però apporta un valore aggiunto al prodotto;
              \item opzionale: requisito relativamente utile oppure contrattabile in un secondo momento durante lo sviluppo;
          \end{itemize}
    \item \textbf{Descrizione:} descrizione del requisito, meno ambigua possibile;
    \item \textbf{Fonti:} ogni requisito deriva da una di queste fonti:
          \begin{itemize}
              \item \textit{capitolato:} è un requisito individuato dalle condizioni imposte dal capitolato;
              \item \textit{interno:} è un requisito aggiunto dagli analisti;
              \item \textit{caso d'uso:} è un requisito estratto da uno o più casi d'uso, è riportato il codice del caso d'uso a cui ci si riferisce;
              \item \textit{verbale:} è un requisito individuato attraverso una riunione con i proponenti.
          \end{itemize}
\end{itemize}
\subsubsubsection{Classificazione dei casi d'uso}
Un caso d'uso è un diagramma, accompagnato da una descrizione testuale aggiuntiva, che rappresenta un comportamento, offerto o desiderato, sulla base di risultati osservabili.
\\La struttura dei casi d'uso è così suddivisa:
\begin{itemize}
    \item \textbf{Codice identificativo:}
          \begin{center}
              \textbf{UC[CodiceBase](.[CodiceSottoCaso])}
          \end{center}
          dove:
          \begin{itemize}
              \item \textbf{UC}: acronimo di “use case”;
              \item \textbf{CodiceBase}: codice che identifica il caso d'uso generico;
              \item \textbf{CodiceSottoCaso}: codice opzionale che identifica gli eventuali sottocasi.
          \end{itemize}
    \item \textbf{Titolo}: titolo testuale del caso d'uso;
    \item \textbf{Diagramma UML}: rappresentazione grafica del caso d'uso realizzata impiegando il formalismo dell'UML 2.0\textsubscript{\textbf{G}}.
    \item \textbf{Attori}: rappresentano qualsiasi entità esterna al sistema e con il quale interagiscono. Esistono due tipologie di attori:\begin{itemize}
              \item \textbf{Attori primari}: interagiscono attivamente con il sistema per soddisfare le proprie esigenze;
              \item \textbf{Attori secondari}: attori il cui scopo è aiutare l'attore primario.
          \end{itemize}
    \item \textbf{Descrizione}: breve descrizione testuale del caso d'uso;
    \item \textbf{Scenario principale}: elenco puntato dei passi che compongono il caso d'uso;
    \item \textbf{Scenario alternativo (se presente)}: elenco puntato degli eventi che possono manifestarsi in seguito a un evento imprevisto che causa una deviazione dallo scenario principale;
    \item \textbf{Inclusioni (se presenti)}: utilizzate quando due casi d'uso sono collegati e il secondo è incluso nel primo;
    \item \textbf{Estensioni (se presenti)}: utilizzate per modellare scenari alternativi. Al verificarsi di una determinata condizione il caso d'uso a essa collegata viene interrotto;
    \item \textbf{Generalizzazioni (se presenti)}: utilizzate per modellare specializzazioni dei casi d'uso;
    \item \textbf{Precondizione}: espone le condizioni in cui si trova il sistema prima del verificarsi dell'evento degli eventi del caso d'uso;
    \item \textbf{Postcondizione}: espone le condizioni in cui si trova il sistema al termine degli eventi del caso d'uso;
    \item \textbf{Input (se presenti)}: valore o oggetto che l'attore porta nel sistema;
    \item \textbf{Output (se presenti)}: valore o oggetto risultato di un'azione di un attore.
\end{itemize}
\subsubsubsection{Metriche}
\begin{itemize}
    \item \textbf{MPR01 - PROS (Percentuale di requisiti obbligatori soddisfatti):} indica la percentuale di requisiti obbligatori soddisfatti.\\
          MPR01 - PROS (Percentuale di requisiti obbligatori soddisfatti): indica la percentuale di requisiti obbligatori soddisfatti.
          \\\textbf{Metodo di misura}:\\valore percentuale: $PROS = \frac{requisiti \ obbligatori \ soddisfatti}{requisiti \ obbligatori \ totali} * 100$
    \item \textbf{MPR02 - PRS (Percentuale di requisiti soddisfatti):}indica la percentuale di requisiti soddisfatti.\\
          \\\textbf{Metodo di misura}:\\valore percentuale: $PRS = \frac{requisiti \ soddisfatti}{requisiti \ totali} * 100$
\end{itemize}
\subsubsection{Progettazione}
\subsubsubsection{Scopo}
L'attività di progettazione ha come scopo la ricerca di una soluzione architetturale in grado di soddisfare i requisiti degli \textit{stakeholders}\textsubscript{\textbf{G}}, esposti nel documento \textit{Analisi dei Requisiti v1.0.0}.
I \textit{progettisti} devono tenere conto che questa fase deve garantire un approccio sistematico\textsubscript{\textbf{G}} ai problemi, suddividendoli, se necessario, in sottoproblemi più semplici da risolvere.
In questo modo la parte di codifica del codice è semplificata e ottimizzata.
\subsubsubsection{Descrizione}
Tale attività punta a realizzare l'architettura del sistema. Inizialmente realizzata dal
\textit{Proof of Concept} della \textit{Technology Baseline}, poi approfondita e descritta nel documento
tecnico allegato alla \textit{Product Baseline}.
\subsubsubsection{Aspettative}
Il lavoro dei \textit{progettisti} deve produrre una soluzione architetturale che presenti le seguenti caratteristiche:
\begin{itemize}
    \item soddisfare a pieno i requisiti degli \textit{stackeholders};
    \item identificare varie componenti distinte, facilmente comprensibili e coese;
    \item essere affidabile, garantendo il corretto funzionamento del prodotto;
    \item avere una gestione degli errori robusta;
    \item avere una gestione delle risorse ottimizzata;
    \item essere facilmente manutenibile.
\end{itemize}
\subsubsubsection{Qualità dell'architettura}
Un'architettura di buona qualità ha come caratteristiche misurabili e osservabili oggettivamente:
\begin{itemize}
    \item \textbf{Sufficienza:} capace di soddisfare tutti i requisiti indicati nell'\textit{Analisi dei Requisiti};
    \item \textbf{Comprensibilità:} capace di essere capita dagli \textit{stakeholders};
    \item \textbf{Robustezza:} capace di gestire eventi non previsti causati dall'utente e dall'ambiente;
    \item \textbf{Modularità:} suddivisa in parti ben distinte;
    \item \textbf{Flessibilità:} permette modifiche a costo contenuto al variare dei requisti;
    \item \textbf{Riusabilità:} le sue parti possono essere utilizzate in altre applicazioni;
    \item \textbf{Affidabilità:} svolge quanto previsto;
    \item \textbf{Sicura:} capace di resistere a malfunzionamenti ed evitare possibili intrusioni.
\end{itemize}
Le sue componenti devono essere semplici, coese nel raggiungimento degli obiettivi, incapsulate e con un basso livello di accoppiamento.
\subsubsubsection{Periodi della progettazione}
\label{Periodi della progettazione}
L'attività di progettazione si divide in due parti:\\\\
\textbf{Progettazione architetturale:} vengono scelte le tecnologie e definite le specifiche dell'architettura e le componenti del sistema. Al termine di questo periodo
si ottiene la \textit{Technology Baseline}\textsubscript{\textbf{G}}, che contiene:
a termine durante questo periodo sono:
\begin{itemize}
    \item \textbf{Proof of Concept:} un primo eseguibile del sistema che dimostri che le tecnologie scelte sono adatte allo sviluppo del prodotto atteso;
    \item \textbf{Tecnologie utilizzate:} descrizione e motivazione dettagliata della scelta effettuata sulle tecnologie impiegate;
    \item \textbf{Test di integrazione:} definizione dei test eseguiti per verificare che le componenti del sistema interagiscano in modo corretto.
\end{itemize}
\textbf{Progettazione di dettaglio:} si scompongono le componenti individuate precedentemente fino ad
arrivare a singole unità ben definite e facilmente realizzabili da un singolo \textit{Programmatore}. Al termine di questo periodo si
ottiene la \textit{Product Baseline}\textsubscript{\textbf{G}}, che contiene:
\begin{itemize}
    \item \textbf{Design Pattern:} descrizione dei design pattern\textsubscript{\textbf{G}} e delle scelte progettuali adottati per l'architettura del sistema; ogni design pattern deve essere accompagnato da una spiegazione e
          da un diagramma che ne mostri la struttura;
    \item \textbf{Diagrammi UML:} diagrammi realizzati utilizzando UML\textsubscript{\textbf{G}} 2.0, utilizzati per rendere più chiare le soluzioni progettuali adottate;
    \item \textbf{Test di unità:} definizione dei test eseguiti per verificare che le classi e i metodi che implementano il sistema software siano corretti e conformi ai requisiti.
\end{itemize}
\subsubsubsection{Design Pattern}
Una volta stilata l'\textit{Analisi dei Requisiti}, ai \textit{progettisti} verrà affidato il compiti di trovare soluzioni architetturali ai problemi più frequenti.
Tutti i design pattern\textsubscript{\textbf{G}} non devono essere troppo specifici, in modo da permettere ai \textit{programmatori} di essere abbastanza flessibili nel momento della stesura del codice.
Il significato e la struttura di un design pattern\textsubscript{\textbf{G}} vengono rappresentati con diagrammi e una descrizione.
\subsubsubsection{Diagrammi UML 2.0}
Per facilitare la comprensione delle soluzioni architetturali pensate dai progettisti, questi devono utilizzare diagrammi UML\textsubscript{\textbf{G}} 2.0.
Nello specifico:
\begin{itemize}
    \item \textbf{Diagrammi di classi}: illustra le classi, comprensive di metodi e membri;
    \item \textbf{Diagrammi delle attività}: illustra il flusso di un attività;
    \item \textbf{Diagrammi di sequenza}: illustra sequenza di azioni attraverso l'impiego di scelte definite;
    \item \textbf{Diagrammi dei moduli}: illustra il raggruppamento di classi nei moduli dell'applicazione.
\end{itemize}
\subsubsubsection{Test}
Durante la fase di progettazione devono essere definiti test opportuni, in coerenza con le richieste dei proponenti.
Tali test devono poter individuare errori o anomalie nell'architettura, logica o codice dell'applicazione.\\
I test progettati dovranno rispettare le convenzioni di nomenclature specificate in \ref{Codice test}.
\subsubsection{Codifica del software}
\subsubsubsection{Scopo}
L'attività di codifica del software ha come scopo la stesura del codice stesso, e quindi dell'effettiva realizzazione del prodotto.
Vengono quindi implementate le specifiche esposte nella fase di progettazione, trasformandole in codice eseguibile.
\subsubsubsection{Descrizione}
Ogni \textit{programmatore} deve rispettare un insieme di regole stilistiche di stesura del codice, in modo da renderlo uniforme e fruibile.
\subsubsubsection{Aspettative}
L'obiettivo di questa attività è la scrittura del codice del prodotto in modo che questo sia conforme alle richieste del proponente.
Stabilire delle regole stilistiche in questa fase ha lo scopo di migliorare la qualità del prodotto finito, nonché di renderne più semplice la manutenzione e verifica.
\subsubsubsection{Qualità della codifica software}
La codifica è di qualità se:
\begin{itemize}
    \item il codice è facilmente leggibile;
    \item i costrutti del linguaggio sono utilizzati in modo chiaro e coerente;
    \item la compilazione non presenta errori fatali o potenziali.
\end{itemize}
Queste caratteristiche sono in grado di agevolare manutenzione, verifica e validazione e di conseguenza migliorare
la qualità di prodotto.
\subsubsubsection{Regole di Stile}
Ogni \textit{Programmatore} deve attenersi a queste regole stilistiche:
\begin{itemize}
    \item \textbf{Lingua}: ogni riga di codice deve obbligatoriamente essere scritta in inglese. Quindi anche commenti, nomi di variabili e classi;
    \item \textbf{Classi}: le classi devono avere queste caratteristiche:
          \begin{itemize}
              \item nome in PascalCase\textsubscript{\textbf{G}};
              \item gli attributi vengono scritti prima dei metodi, e gli attributi privati prima di quelli pubblici;
              \item i metodi privati devono essere dichiarati prima di quelli pubblici;
              \item i nomi dei metodi e attributi non devono contenere un riferimento alla classe alla quale appartengono. Esempio sbagliato: nome classe: Persona, nome attributo: CognomePersona.
          \end{itemize}
    \item \textbf{Variabili}: le variabili vanno scritte in camelCase\textsubscript{\textbf{G}}, precedute da un \_ se sono attributi di una classe;
    \item \textbf{Costanti}: le costanti vanno scritte in maiuscoletto, precedute da un \_ se sono attributi di una classe;
    \item \textbf{Metodi}: i metodi vanno scritti in camelCase\textsubscript{\textbf{G}}, con l'apertura della prima parentesi graffa in linea con la firma. Il codice all'interno deve avere una sola tabulazione.
    \item Altre regole:
          \begin{itemize}
              \item \textbf{Un solo livello d'indentazione per metodo}: ogni metodo deve contenere righe di codice con massimo una tabulazione. Se il codice ne richiede più di una, allora tale parte va estratta in un nuovo metodo;
              \item \textbf{Non abbreviare}: non abbreviare il nome di variabili, metodi e classi;
              \item \textbf{Callback}: non utilizzare mai metodi di callback\textsubscript{\textbf{G}};
              \item \textbf{Un solo punto per linea}: se possibile, ogni riga di codice deve contenere al massimo un punto;
              \item \textbf{Ricorsione}: se possibile, evitare di utilizzare metodi ricorsivi\textsubscript{\textbf{G}}.
          \end{itemize}
\end{itemize}
\subsubsubsection{Strumenti}
Di seguito gli strumenti utilizzati:
\begin{itemize}
    \item \textbf{Visual Studio Code:} un \textit{IDE}\textsubscript{\textbf{G}} \textit{(Integrated Development Environment)} adatto alla stesura del codice di diversi linguaggi di programmazione, oltre che all'ottima integrazione con Git\textsubscript{\textbf{G}};
    \item \textbf{AWS Lambda:} servizio \textit{Serverless}\textsubscript{\textbf{G}} offerto da \textit{Amazon}, il quale permette di eseguire codice senza effettuare il \textit{provisioning}\textsubscript{\textbf{G}} o gestire i server;
    \item \textbf{AWS DynamoDB:} servizio di \textit{database}\textsubscript{\textbf{G}} NoSql gestito da \textit{Amazon} come parte di \textit{Amazon Web Service}\textsubscript{\textbf{G}};
    \item \textbf{Amazon CloudWatch:} offre un servizio gestito da \textit{Amazon} di monitoraggio e osservabilità in grado di monitorare le applicazioni, rispondere ai cambiamenti di prestazioni a livello di sistema, ottimizzare l'utilizzo delle risorse e ottenere una visualizzazione unificata dello stato d'integrità operativa;
    \item \textbf{Stripe:} servizio che fornisce un'infrastruttura software che permette a privati e aziende d'inviare e ricevere pagamenti via Internet;
    \item \textbf{Next.js:} Next.js è un framework web di sviluppo frontend React\textsubscript{\textbf{G}} open source che abilita funzionalità come il rendering lato server e la generazione di siti web statici per applicazioni web basate su React. Fornisce possibilità di integrazione con Typescript;
    \item \textbf{Chakra UI:} libreria di componenti React;
    \item \textbf{TypeScript:} TypeScript è un Super-set di JavaScript\textsubscript{\textbf{G}}. Il linguaggio estende la sintassi di JavaScript in modo che qualunque programma scritto in JavaScript sia anche in grado di funzionare con TypeScript senza nessuna modifica. Nel progetto è richiesto l'impiego dell'ultima versione di TypeScript;
    \item \textbf{ESLint:} strumento che permette di verificare che il codice TypeScript scritto rispetti correttamente la sintassi di linguaggio;
    \item \textbf{AWS API Gateway:} Amazon API Gateway è un servizio che semplifica per gli sviluppatori la creazione, la pubblicazione, la manutenzione, il monitoraggio e la protezione delle API su qualsiasi scala;
    \item \textbf{AWS Amazon Cognito:} Amazon Cognito permette di aggiungere strumenti di registrazione degli utenti, accesso e controllo degli accessi alle app Web e per dispositivi mobili;
    \item \textbf{AWS S3:} AWS S3, acronimo di Amazon Web Services Amazon Simple Storage, è un servizio Web di memorizzazione offerto da Amazon Web Services;
    \item \textbf{Npm:} npm è un gestore di pacchetti per il linguaggio di programmazione JavaScript. È il gestore di pacchetti predefinito per l'ambiente di runtime JavaScript Node.js;
    \item \textbf{Amazon SNS:} è un servizio di messaggistica completamente gestito per la comunicazione application-to-person (A2P) e application-to-application (A2A);
    \item \textbf{Amazon SQS:}  è un servizio di accodamento messaggi completamente gestito che consente la separazione e la scalabilità di microservizi, sistemi distribuiti e applicazioni serverless;
    \item \textbf{Amazon SES:} è un servizio di e-mail che consente agli sviluppatori di inviare e-mail da qualsiasi applicazione;
    \item \textbf{Chai:} Chai è una libreria di testing per Node.js e browser. Può essere utilizzata con ogni framework di testing Javascript ed è molto diffuso il suo utilizzo con Mocha\textsubscript{\textbf{G}};
    \item \textbf{Jest:} Jest è un framework di unit test JavaScript.
\end{itemize}
