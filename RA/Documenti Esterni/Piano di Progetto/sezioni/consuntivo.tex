\section{Consuntivi di periodo}
Di seguito vengono indicate le spese sostenute per ogni ruolo confrontandole con quanto preventivato. Il bilancio potrà essere:
\begin{itemize}
    \item \textbf{Positivo:} se la spesa effettiva è minore di quanto preventivato;
    \item \textbf{Pari:} se la spesa effettiva è uguale a quanto preventivato;
    \item \textbf{Negativo:} se la spesa effettiva è maggiore di quanto preventivato.
\end{itemize}

\subsection{Periodo di analisi}
Le ore di lavoro sostenute in questa fase sono da considerarsi come ore d'investimento per l'approfondimento personale, non vengono quindi rendicontate.

\begin{center}
    \begin{table}[!ht]
        \centering
        \caption{Consuntivo della fase di analisi}
        \vspace{5px}
        \rowcolors{2}{logo!10}{logo!40}
        \renewcommand{\arraystretch}{1.8}
        \begin{tabular}{p{150px} p{110px} p{110px}}
            \rowcolor{logo!70} \textbf{Ruolo} & \textbf{Ore} & \textbf{Costo}                  \\
            Responsabile                      & 21 (+0)      & 630,00\EURdig (+0,00 \EURdig)   \\
            Amministratore                    & 35 (+0)      & 700,00\EURdig (+0,00 \EURdig)   \\
            Analista                          & 98 (+0)      & 2.450,00\EURdig (+0,00 \EURdig) \\
            Progettista                       & 0 (+0)       & 0(+0,00 \EURdig)                \\
            Programmatore                     & 0 (+0)       & 0(+0,00 \EURdig)                \\
            Verificatore                      & 91 (+0)      & 1.365,00\EURdig (+0,00 \EURdig) \\
            \textbf{Totale preventivo}        & 245          & 5.145,00\EURdig                 \\
            \textbf{Totale consuntivo}        & 245          & 5.145,00\EURdig                 \\
            \textbf{Differenza}               & 0            & (+0,00 \EURdig)                 \\
        \end{tabular}
    \end{table}
\end{center}
\subsubsection{Conclusioni}
Avendo deciso con i componenti del gruppo di rispettare la scadenza per la prima consegna disponibile, si è lavorato per rispettare i tempi e si è riusciti a non superare le tempistiche prestabilite.

\subsubsection{Preventivo a finire}
Rispettando le tempistiche preventivate non sono stati aggiunti costi alla fase di analisi.

\newpage
\subsection{Periodo di Consolidamento dei requisiti}
In questo periodo si è svolto il lavoro relativo al consolidamento dei requisiti, successivo a quello di analisi. Sono state dedicate delle ore anche per lo studio e l'approfondimento personale e sono da considerarsi come non rendicontate. Perciò tali ore non sono state riportate nella seguente tabella:

\begin{center}
    \begin{table}[ht!]
        \centering
        \caption{Consuntivo della fase di Consolidamento dei requisiti}
        \vspace{5px}
        \rowcolors{2}{logo!10}{logo!40}
        \renewcommand{\arraystretch}{1.8}
        \begin{tabular}{p{150px} p{110px} p{110px}}
            \rowcolor{logo!70} \textbf{Ruolo} & \textbf{Ore} & \textbf{Costo}                 \\
            Responsabile                      & 5 (+1)       & 150,00\EURdig (+30,00 \EURdig) \\
            Amministratore                    & 6 (+0)       & 120,00\EURdig (+0,00 \EURdig)  \\
            Analista                          & 13 (-3)      & 325,00\EURdig (-75,00 \EURdig) \\
            Progettista                       & 0 (+0)       & 0 (+0,00 \EURdig)              \\
            Programmatore                     & 0 (+0)       & 0 (+0,00 \EURdig)              \\
            Verificatore                      & 16 (+0)      & 240,00\EURdig (+0,00 \EURdig)  \\
            \textbf{Totale preventivo}        & 42           & 880,00\EURdig                  \\
            \textbf{Totale consuntivo}        & 40           & 835,00\EURdig                  \\
            \textbf{Differenza}               & -2           & (-45,00\EURdig)                \\
        \end{tabular}
    \end{table}
\end{center}

\subsubsection{Conclusioni}
Rispetto a quanto preventivato inizialmente sono state necessarie delle modifiche per il lavoro svolto durante questo periodo. Vista la ridotta durata di tale periodo il gruppo è riuscito a ridurre le ore necessarie al completamento del lavoro. Di seguito sono riportate le motivazioni delle variazioni del monte ore di lavoro ricoperto dai diversi ruoli:
\begin{itemize}
    \item \textbf{\textit{Responsabile}} è stata necessaria un'ora aggiuntiva, per la coordinazione dei membri del gruppo inerente alla presentazione e allo studio delle tecnologie, necessarie allo sviluppo software;
    \item \textbf{\textit{Analista}} Viste le non eccessive modifiche nella documentazione sono state necessarie meno ore in questo ruolo.
\end{itemize}

\subsubsection{Preventivo a finire} Il bilancio economico risultante è positivo, ovvero sono stati risparmiati 45,00\EURdig. Tali fondi verranno impiegati nei prossimi periodi per far fronte a eventuali ritardi.

\newpage
\subsection{Periodo di Progettazione architetturale}
Le ore sostenute durante questo periodo sono relative alla progettazione e alla codifica del \textit{Proof of Concept}\textsubscript{\textbf{G}}, necessario al soddisfacimento della \textit{Technology Baseline}\textsubscript{\textbf{G}}. Tale periodo è da considerarsi rendicontato, in quanto il capitolato d'appalto è stato aggiudicato e quindi il lavoro è svolto con lo scopo di sviluppare il prodotto finale.

\begin{center}
    \begin{table}[ht!]
        \centering
        \caption{Consuntivo dei costi per ruolo della fase di Progettazione architetturale}
        \vspace{5px}
        \rowcolors{2}{logo!10}{logo!40}
        \renewcommand{\arraystretch}{1.8}
        \begin{tabular}{p{150px} p{110px} p{110px}}
            \rowcolor{logo!70} \textbf{Ruolo} & \textbf{Ore} & \textbf{Costo}                 \\
            Responsabile                      & 15(+0)       & 450,00\EURdig(+0,00 \EURdig)   \\
            Amministratore                    & 22(+3)       & 440,00\EURdig(+60,00 \EURdig)  \\
            Analista                          & 28(-2)       & 700,00\EURdig(-50,00 \EURdig)  \\
            Progettista                       & 41(-22)      & 902,00\EURdig(-484,00 \EURdig) \\
            Programmatore                     & 50(+24)      & 750,00\EURdig(+360,00 \EURdig) \\
            Verificatore                      & 43(+0)       & 645,00\EURdig(+0,00 \EURdig)   \\
            \textbf{Totale preventivo}        & 196          & 4001,00\EURdig                 \\
            \textbf{Totale consuntivo}        & 199          & 3887,00\EURdig                 \\
            \textbf{Differenza}               & +3           & (-114,00 \EURdig)              \\
        \end{tabular}
    \end{table}
\end{center}

\subsubsection{Conclusioni}
Dai dati riportati nella tabella soprastante si può notare che la progettazione riguardante tale periodo ha subito una consistente modifica. Il gruppo aveva infatti pianificato di redigere una completa progettazione architetturale del prodotto, e di riportarla all’ interno di un documento formale. Tuttavia durante lo svolgimento del lavoro ci si è concentrati maggiormente sul verificare, attraverso la progettazione e la codifica del \textit{Proof of Concept}\textsubscript{\textbf{G}}, che le tecnologie scelte si integrassero efficientemente tra loro, e che con il loro utilizzo i requisiti potessero essere soddisfatti. Di seguito sono riportate le motivazioni delle variazioni del monte ore di lavoro ricoperto dai diversi ruoli:

\begin{itemize}
    \item \textbf{\textit{Amministratore:}} Essendoci stata una maggiore necessità di sistemazione riguardante il versionamento dei documenti, si è presentato un esubero di ore per questo ruolo;
    \item \textbf{\textit{Analista:}} Vista una minore necessità di modifica riguardante la correzione dei casi d'uso e la fase di analisi per la risoluzione dei problemi, vi è stato un decremento delle ore riguardanti gli \textit{Analisti};
    \item \textbf{\textit{Progettista:}} Essendo state scelte preventivamente tutte le tecnologie da parte dei proponenti, non è servito un grande dispendio di ore per la scelta delle stesse;
    \item \textbf{\textit{Programmatore:}} Essendo stato implementato il \textit{Proof of Concept} collegando tutte le tecnologie essenziali allo sviluppo del progetto, si è reso necessario un maggiore dispendio di ore per poter collegare tutte le tecnologie previste.
\end{itemize}

Le ore di lavoro riguardanti ai ruoli che hanno subito delle variazioni rispetto a quanto pianificato sono state distribuite in modo tale che ogni elemento del gruppo svolgesse lo stesso monte ore di lavoro complessivo.

\subsubsection{Preventivo a finire} Il bilancio economico risultante è positivo, ovvero sono stati risparmiati 114,00\EURdig. Tali fondi uniti a quelli già risparmiati nel periodo precedente, per un totale di 159,00\EurDig, saranno impiegati nei prossimi periodi per far fronte a eventuali ritardi o per implementare i requisiti opzionali.


\newpage
\subsection{Periodo di Progettazione di dettaglio e codifica}
\subsubsection{Incremento 6}
\begin{center}
    \begin{table}[ht!]
        \centering
        \caption{Consuntivo dei costi per ruolo dell'incremento 6}
        \vspace{5px}
        \rowcolors{2}{logo!10}{logo!40}
        \renewcommand{\arraystretch}{1.8}
        \begin{tabular}{p{150px} p{110px} p{110px}}
            \rowcolor{logo!70} \textbf{Ruolo} & \textbf{Ore} & \textbf{Costo}                 \\
            Responsabile                      & 5(+0)        & 150,00\EURdig(+0,00 \EURdig)   \\
            Amministratore                    & 13(+4)       & 260,00\EURdig(+80,00 \EURdig)   \\
            Analista                          & 0(+0)        & 0,00\EURdig(+0,00 \EURdig)     \\
            Progettista                       & 30(+0)       & 660,00\EURdig(+0,00 \EURdig)   \\
            Programmatore                     & 0(+0)        & 0,00\EURdig(+0,00 \EURdig)     \\
            Verificatore                      & 12(+0)       & 180,00\EURdig(+0,00 \EURdig)   \\
            \textbf{Totale preventivo}        & 56           & 1.170,00\EURdig                \\
            \textbf{Totale consuntivo}        & 60           & 1.250,00\EURdig                \\
            \textbf{Differenza}               & +4           & (+80,00 \EURdig)                \\
        \end{tabular}
    \end{table}
\end{center}
\subsubsubsection{Conclusioni}
Avendo dovuto modificare in maniera più approfondita del previsto le \textit{Norme di Progetto}, si sono rese necessarie più ore di \textit{Amministratore} rispetto a quanto preventivato.
\subsubsubsection{Preventivo a finire rispetto alla fase}
L'aumento delle ore di \textit{Amministratore} ha portato ad un incremento del costo per un totale di 80,00 \EURdig.
\subsubsubsection{Preventivo a finire complessivo}
La spesa aggiuntiva, sottratta al risparmio accumulato nella fase precedente, mantiene comunque il bilancio positivo di 79,00\EurDig, che saranno impiegati nei prossimi periodi per far fronte a eventuali ritardi o per implementare i requisiti opzionali.

\pagebreak
\subsubsection{Incremento 7}
\begin{center}
    \begin{table}[ht!]
        \centering
        \caption{Consuntivo dei costi per ruolo dell'incremento 7}
        \vspace{5px}
        \rowcolors{2}{logo!10}{logo!40}
        \renewcommand{\arraystretch}{1.8}
        \begin{tabular}{p{150px} p{110px} p{110px}}
            \rowcolor{logo!70} \textbf{Ruolo} & \textbf{Ore}  & \textbf{Costo}                     \\
            Responsabile                      & 8(+2)         & 240,00\EURdig(+0,00 \EURdig)       \\
            Amministratore                    & 6(+0)         & 120,00\EURdig(+0,00 \EURdig)       \\
            Analista                          & 0(+0)         & 0,00\EURdig(+0,00 \EURdig)         \\
            Progettista                       & 20(-3)        & 440,00\EURdig(+0,00 \EURdig)       \\
            Programmatore                     & 10(+0)        & 150,00\EURdig(+0,00 \EURdig)       \\
            Verificatore                      & 11(+0)        & 165,00\EURdig(+0,00 \EURdig)       \\
            \textbf{Totale preventivo}        & 56            & 1.121,00\EURdig                    \\
            \textbf{Totale consuntivo}        & 55            & 1.115,00\EURdig                    \\
            \textbf{Differenza}               & -1            & (-6,00 \EURdig)                    \\
        \end{tabular}
    \end{table}
\end{center}
\subsubsubsection{Conclusioni}
A causa di imprevisti ed impegni personali dei membri del gruppo, sono risultate necessarie più ore di \textit{Responsabile} rispetto a quanto preventivato, per organizzare il lavoro dei componenti del gruppo in maniera efficiente. Avendo effettuato scelte ponderate riguardo alle tecnologie da utilizzare per lo sviluppo del progetto, in questo incremento sono risultate necessarie meno ore di \textit{Progettista} rispetto a quanto preventivato.
\subsubsubsection{Preventivo a finire rispetto alla fase}
La differenza tra le ore di \textit{Responsabile} Aggiunte e quelle di \textit{Progettista} rimosse ha portato ad un risparmio di 6,00 \EURdig in questo incremento.
\subsubsubsection{Preventivo a finire complessivo}
I soldi risparmiati, sommati a quelli risultanti alla fine dell'incremento precedente, portano ad un bilancio positivo di 85,00\EurDig, che saranno impiegati nei prossimi periodi per far fronte a eventuali ritardi o per implementare i requisiti opzionali.

\pagebreak
\subsubsection{Incremento 8}
\begin{center}
    \begin{table}[ht!]
        \centering
        \caption{Consuntivo dei costi per ruolo dell'incremento 8}
        \vspace{5px}
        \rowcolors{2}{logo!10}{logo!40}
        \renewcommand{\arraystretch}{1.8}
        \begin{tabular}{p{150px} p{110px} p{110px}}
            \rowcolor{logo!70} \textbf{Ruolo} & \textbf{Ore}  & \textbf{Costo}                    \\
            Responsabile                      & 3(+0)         & 90,00\EURdig(+0,00 \EURdig)       \\
            Amministratore                    & 5(+1)         & 100,00\EURdig(+0,00 \EURdig)      \\
            Analista                          & 0(+0)         & 0,00\EURdig(+0,00 \EURdig)        \\
            Progettista                       & 14(+0)        & 308,00\EURdig(+0,00 \EURdig)      \\
            Programmatore                     & 18(+5)        & 270,00\EURdig(+0,00 \EURdig)      \\
            Verificatore                      & 8(+0)         & 120,00\EURdig(+0,00 \EURdig)      \\
            \textbf{Totale preventivo}        & 42            & 793,00\EURdig                     \\
            \textbf{Totale consuntivo}        & 48            & 888,00\EURdig                     \\
            \textbf{Differenza}               & +6            & (+95,00 \EURdig)                  \\
        \end{tabular}
    \end{table}
\end{center}
\subsubsubsection{Conclusioni}
Avendo avuto difficoltà nella configurazione del linter \textit{Eslint} e con le modifiche apportate per la suddivisione dei repository in maniera più appropriata alla filosofia dei microservizi, si è presentato un esubero delle ore di \textit{Programmatore}. La conseguente modifica delle \textit{Norme di Progetto}, legata alla modifica della struttura dei repository e alla conseguente descrizione della nuova struttura, ha portato ad un esubero delle ore di \textit{Amministratore} rispetto a quanto preventivato.
\subsubsubsection{Preventivo a finire rispetto alla fase}
L'aumento delle ore di codifica ha portato ad un esubero di 75,00 \EURdig rispetto al bilancio dell'incremento.
\subsubsubsection{Preventivo a finire complessivo}
La spesa aggiuntiva, sottratta al risparmio accumulato negli incrementi precedenti, porta ad un bilancio negativo di -10,00\EurDig, questi soldi verranno recuperati con un'oculata gestione delle risorse.

\pagebreak
\subsubsection{Incremento 9}
\begin{center}
    \begin{table}[ht!]
        \centering
        \caption{Consuntivo dei costi per ruolo dell'incremento 9}
        \vspace{5px}
        \rowcolors{2}{logo!10}{logo!40}
        \renewcommand{\arraystretch}{1.8}
        \begin{tabular}{p{150px} p{110px} p{110px}}
            \rowcolor{logo!70} \textbf{Ruolo} & \textbf{Ore}  & \textbf{Costo}                   \\
            Responsabile                      & 2(+0)         & 60,00\EURdig(+0,00 \EURdig)      \\
            Amministratore                    & 2(+0)         & 40,00\EURdig(+0,00 \EURdig)      \\
            Analista                          & 0(+0)         & 0,00\EURdig(+0,00 \EURdig)       \\
            Progettista                       & 10(+0)        & 220,00\EURdig(+0,00 \EURdig)     \\
            Programmatore                     & 8(+0)         & 120,00\EURdig(+0,00 \EURdig)     \\
            Verificatore                      & 6(+0)         & 90,00\EURdig(+0,00 \EURdig)      \\
            \textbf{Totale preventivo}        & 28            & 530,00\EURdig                    \\
            \textbf{Totale consuntivo}        & 28            & 530,00\EURdig                    \\
            \textbf{Differenza}               & +0            & (+0,00\EURdig)                   \\
        \end{tabular}
    \end{table}
\end{center}
\subsubsubsection{Conclusioni}
Durante questo incremento è stato rispettato il numero di ore disponibili come risorse per implementare gli obiettivi, questo ha portato in pari le spese da sostenere rispetto a quanto preventivato.
\subsubsubsection{Preventivo a finire rispetto alla fase}
Tutti gli obiettivi dell’incremento sono stati raggiunti con successo, nel pieno rispetto del consumo di risorse preventivato. Il bilancio dell’incremento è quindi in pari.
\subsubsubsection{Preventivo a finire complessivo}
Date le considerazioni precedenti su costi e obiettivi, il preventivo complessivo resta invariato.

\pagebreak
\subsubsection{Incremento 10}
\begin{center}
    \begin{table}[ht!]
        \centering
        \caption{Consuntivo dei costi per ruolo dell'incremento 10}
        \vspace{5px}
        \rowcolors{2}{logo!10}{logo!40}
        \renewcommand{\arraystretch}{1.8}
        \begin{tabular}{p{150px} p{110px} p{110px}}
            \rowcolor{logo!70} \textbf{Ruolo} & \textbf{Ore}  & \textbf{Costo}                   \\
            Responsabile                      & 4(+0)         & 120,00\EURdig(+0,00 \EURdig)     \\
            Amministratore                    & 3(+0)         & 60,00\EURdig(+0,00 \EURdig)      \\
            Analista                          & 0(+0)         & 0,00\EURdig(+0,00 \EURdig)       \\
            Progettista                       & 3(+0)         & 66,00\EURdig(+0,00 \EURdig)      \\
            Programmatore                     & 22(+3)        & 285,00\EURdig(+0,00 \EURdig)     \\
            Verificatore                      & 9(-4)         & 195,00\EURdig(+0,00 \EURdig)     \\
            \textbf{Totale preventivo}        & 42            & 726,00\EURdig                    \\
            \textbf{Totale consuntivo}        & 41            & 711,00\EURdig                    \\
            \textbf{Differenza}               & -1            & (-15,00 \EURdig)                 \\
        \end{tabular}
    \end{table}
\end{center}
\subsubsubsection{Conclusioni}
Avendo avuto dei problemi con l'implementazione del carrello e del checkout nell’integrazione tra lambda e struttura dei servizi, si sono rese necessarie più ore di \textit{Programmatore} rispetto a quanto preventivato. La minor necessità di verifica dei documenti rispetto a quanto previsto ha portato ad un calo delle ore di \textit{Verificatore}.
\subsubsubsection{Preventivo a finire rispetto alla fase}
L'aggiunta delle ore di \textit{Programmatore} sottratte a quelle da \textit{Verificatore} portano ad un risparmio durante la fase di 15,00\EurDig.
\subsubsubsection{Preventivo a finire complessivo}
i 15,00\EurDig risparmiati, sottratti sommati alla spesa in esubero di 10,00\EurDig, porta il bilancio in positivo di 5,00\EurDig, che saranno impiegati nei prossimi periodi per far fronte a eventuali ritardi o per implementare i requisiti opzionali.

\pagebreak
\subsubsection{Incremento 11}
\begin{center}
    \begin{table}[ht!]
        \centering
        \caption{Consuntivo dei costi per ruolo dell'incremento 11}
        \vspace{5px}
        \rowcolors{2}{logo!10}{logo!40}
        \renewcommand{\arraystretch}{1.8}
        \begin{tabular}{p{150px} p{110px} p{110px}}
            \rowcolor{logo!70} \textbf{Ruolo} & \textbf{Ore}  & \textbf{Costo}                   \\
            Responsabile                      & 3(+0)         & 90,00\EURdig(+0,00 \EURdig)      \\
            Amministratore                    & 2(+0)         & 40,00\EURdig(+0,00 \EURdig)      \\
            Analista                          & 0(+0)         & 0,00\EURdig(+0,00 \EURdig)       \\
            Progettista                       & 4(+0)         & 88,00\EURdig(+0,00 \EURdig)      \\
            Programmatore                     & 14(+2)        & 180,00\EURdig(+0,00 \EURdig)     \\
            Verificatore                      & 7(+0)         & 105,00\EURdig(+0,00 \EURdig)     \\
            \textbf{Totale preventivo}        & 28            & 503,00\EURdig                    \\
            \textbf{Totale consuntivo}        & 30            & 533,00\EURdig                    \\
            \textbf{Differenza}               & +2            & (+30,00 \EURdig)                  \\
        \end{tabular}
    \end{table}
\end{center}
\subsubsubsection{Conclusioni}
Dopo aver parlato coi proponenti si è deciso di implementare le tecnologie \textit{SNS} ed \textit{SQS}, questo ha portato ad un incremento delle ore di \textit{Programmatore} necessarie.
\subsubsubsection{Preventivo a finire rispetto alla fase}
L'aggiunta delle ore di \textit{Programmatore} ha portato ad un aumento dei costi durante l'incremento di 30,00\EurDig.
\subsubsubsection{Preventivo a finire complessivo}
La spesa aggiuntiva porta ad un bilancio negativo di -25,00\EurDig, questi soldi verranno recuperati con un'oculata gestione delle risorse.

\pagebreak
\subsubsection{Incremento 12}
\begin{center}
    \begin{table}[ht!]
        \centering
        \caption{Consuntivo dei costi per ruolo dell'incremento 12}
        \vspace{5px}
        \rowcolors{2}{logo!10}{logo!40}
        \renewcommand{\arraystretch}{1.8}
        \begin{tabular}{p{150px} p{110px} p{110px}}
            \rowcolor{logo!70} \textbf{Ruolo} & \textbf{Ore}  & \textbf{Costo}                   \\
            Responsabile                      & 2(-1)         & 60,00\EURdig(+0,00 \EURdig)      \\
            Amministratore                    & 3(+0)         & 60,00\EURdig(+0,00 \EURdig)      \\
            Analista                          & 0(+0)         & 0,00\EURdig(+0,00 \EURdig)       \\
            Progettista                       & 6(+0)         & 132,00\EURdig(+0,00 \EURdig)     \\
            Programmatore                     & 15(+2)        & 225,00\EURdig(+0,00 \EURdig)     \\
            Verificatore                      & 17(+0)        & 255,00\EURdig(+0,00 \EURdig)     \\
            \textbf{Totale preventivo}        & 42            & 732,00\EURdig                    \\
            \textbf{Totale consuntivo}        & 43            & 732,00\EURdig                    \\
            \textbf{Differenza}               & +1            & (+0,00 \EURdig)                  \\
        \end{tabular}
    \end{table}
\end{center}
\subsubsubsection{Conclusioni}
Avendo avuto difficoltà nell'implementazione dei filtri dal lato frontend, si sono rese necessarie più ore di \textit{Programmatore} rispetto a quanto preventivato. Essendo stato organizzato precedentemente il lavoro, si è reso necessario meno tempo del previsto da responsabile.
\subsubsubsection{Preventivo a finire rispetto alla fase}
L'aggiunta delle ore di \textit{Programmatore} sottratte a quelle da \textit{Responsabile} portano in pari il costo del bilancio dell'incremento corrente.
\subsubsubsection{Preventivo a finire complessivo}
Date le considerazioni precedenti su costi e obiettivi, il preventivo complessivo resta invariato.

\pagebreak
\subsubsection{Incremento 13}
\begin{center}
    \begin{table}[ht!]
        \centering
        \caption{Consuntivo dei costi per ruolo dell'incremento 13}
        \vspace{5px}
        \rowcolors{2}{logo!10}{logo!40}
        \renewcommand{\arraystretch}{1.8}
        \begin{tabular}{p{150px} p{110px} p{110px}}
            \rowcolor{logo!70} \textbf{Ruolo} & \textbf{Ore}  & \textbf{Costo}                   \\
            Responsabile                      & 5(+0)         & 150,00\EURdig(+0,00 \EURdig)     \\
            Amministratore                    & 5(+0)         & 100,00\EURdig(+0,00 \EURdig)     \\
            Analista                          & 0(+0)         & 0,00\EURdig(+0,00 \EURdig)       \\
            Progettista                       & 0(+0)         & 0,00\EURdig(+0,00 \EURdig)       \\
            Programmatore                     & 25(+0)        & 375,00\EURdig(+0,00 \EURdig)     \\
            Verificatore                      & 21(+0)        & 315,00\EURdig(+0,00 \EURdig)     \\
            \textbf{Totale preventivo}        & 56            & 940,00\EURdig                    \\
            \textbf{Totale consuntivo}        & 56            & 940,00\EURdig                    \\
            \textbf{Differenza}               & +0            & (+0,00 \EURdig)                  \\
        \end{tabular}
    \end{table}
\end{center}
\subsubsubsection{Conclusioni}
Durante questo incremento è stato rispettato il numero di ore disponibili come risorse per implementare gli obiettivi, questo ha portato in pari le spese da sostenere rispetto a quanto preventivato.
\subsubsubsection{Preventivo a finire rispetto alla fase}
Tutti gli obiettivi dell’incremento sono stati raggiunti con successo, nel pieno rispetto del consumo di risorse preventivato. Il bilancio dell’incremento è quindi in pari.
\subsubsubsection{Preventivo a finire complessivo}
Date le considerazioni precedenti su costi e obiettivi, il preventivo complessivo resta invariato.
\pagebreak

\subsubsection{Consuntivo complessivo del periodo di Progettazione di dettaglio e codifica}
\begin{center}
    \begin{table}[ht!]
        \centering
        \caption{Consuntivo dei costi per ruolo della fase di Progettazione di dettaglio e codifica}
        \vspace{5px}
        \rowcolors{2}{logo!10}{logo!40}
        \renewcommand{\arraystretch}{1.8}
        \begin{tabular}{p{150px} p{110px} p{110px}}
            \rowcolor{logo!70} \textbf{Ruolo} & \textbf{Ore} & \textbf{Costo}                  \\
            Responsabile                      & 32(+1)       & 960,00\EURdig(+30,00 \EURdig)   \\
            Amministratore                    & 39(+5)       & 780,00\EURdig(+100,00 \EURdig)  \\
            Analista                          & 0(+0)        & 0,00\EURdig(+0,00 \EURdig)      \\
            Progettista                       & 87(-3)       & 1914,00\EURdig(-66,00 \EURdig)  \\
            Programmatore                     & 112(+12)     & 1680,00\EURdig(+180,00 \EURdig) \\
            Verificatore                      & 91(-4)       & 1365,00\EURdig(-60,00 \EURdig)  \\
            \textbf{Totale preventivo}        & 350          & 6515,00\EURdig                  \\
            \textbf{Totale consuntivo}        & 361          & 6699,00\EURdig                  \\
            \textbf{Differenza}               & +0           & (+184,00 \EURdig)               \\
        \end{tabular}
    \end{table}
\end{center}
\subsubsubsection{Conclusioni}
\begin{itemize}
    \item \textbf{\textit{Responsabile:}} A causa di imprevisti ed impegni personali dei membri del gruppo, sono risultate necessarie più ore di \textit{Responsabile} rispetto a quanto preventivato, per organizzare il lavoro dei componenti del gruppo in maniera efficiente;
    \item \textbf{\textit{Amministratore:}} Essendoci stata una maggiore necessità di sistemazione riguardante le \textit{Norme di Progetto}, si è presentato un esubero di ore per questo ruolo;
    \item \textbf{\textit{Progettista:}} Avendo effettuato scelte ponderate riguardo alle tecnologie da utilizzare per lo sviluppo del progetto, sono risultate necessarie meno ore di \textit{Progettista} rispetto a quanto preventivato;
    \item \textbf{\textit{Programmatore:}} Avendo dovuto configurare diverse tecnologie e riorganizzare i repository, si è presentato un esubero delle ore necessarie per questo ruolo.
\end{itemize}
Gli impegni personali e universitari dei membri del gruppo legati principalmente alla sessione di esami hanno portato ad un ritardo nella consegna di questa fase, per questi motivi la fase di validazione e collaudo è stata riorganizzata di conseguenza.
\subsubsubsection{Preventivo a finire complessivo}
Il bilancio economico risultante sottraendo i costi della fase ai soldi risparmiati in quelle precedenti, è negativo di 25,00\EURdig. Questi soldi verranno recuperati con un'oculata gestione delle risorse nella fase successiva.
\pagebreak

%INIZIO CONSUNTIVI RA
\subsection{Periodo di Progettazione di Validazione e Collaudo}
\subsubsection{Incremento 14}
\begin{center}
    \begin{table}[ht!]
        \centering
        \caption{Consuntivo dei costi per ruolo dell'incremento 14}
        \vspace{5px}
        \rowcolors{2}{logo!10}{logo!40}
        \renewcommand{\arraystretch}{1.8}
        \begin{tabular}{p{150px} p{110px} p{110px}}
            \rowcolor{logo!70} \textbf{Ruolo} & \textbf{Ore}  & \textbf{Costo}                   \\
            Responsabile                      & 4(+0)         & 120,00\EURdig(+0,00 \EURdig)     \\
            Amministratore                    & 5(+1)         & 100,00\EURdig(+20,00 \EURdig)    \\
            Analista                          & 0(+0)         & 0,00\EURdig(+0,00 \EURdig)       \\
            Progettista                       & 4(+0)         & 88,00\EURdig(+0,00 \EURdig)      \\
            Programmatore                     & 11(-3)        & 165,00\EURdig(-45,00 \EURdig)    \\
            Verificatore                      & 7(+0)         & 105,00\EURdig(+0,00 \EURdig)     \\
            \textbf{Totale preventivo}        & 33            & 603,00\EURdig                    \\
            \textbf{Totale consuntivo}        & 31            & 578,00\EURdig                    \\
            \textbf{Differenza}               & -2            & (-25,00 \EURdig)                 \\
        \end{tabular}
    \end{table}
\end{center}
\subsubsubsection{Conclusioni}
Dopo aver parlato coi proponenti si è deciso di non implementare alcune funzionalità, questo ha portato ad una diminuzione delle ore di \textit{Programmatore} necessarie. Essendoci stata una maggiore necessità di sistemazione riguardante le \textit{Norme di Progetto} a causa di correzioni al documento, si è presentato un esubero di ore per questo ruolo.
\subsubsubsection{Preventivo a finire rispetto alla fase}
La minore necessità di ore per il ruolo di \textit{Programmatore} ha portato ad un risparmio di 45,00\EURdig. La necessità di modifica delle \textit{Norme di Progetto} per il ruolo di \textit{Amministratore} ha portato ad una spesa aggiuntiva di 20,00\EURdig.
\subsubsubsection{Preventivo a finire complessivo}
Il risparmio ottenuto sommato alla spesa in eccesso della fase precedente e dell'incremento attuale porta il bilancio in pari.
\pagebreak

\pagebreak
\subsubsection{Incremento 15}
\begin{center}
    \begin{table}[ht!]
        \centering
        \caption{Consuntivo dei costi per ruolo dell'incremento 15}
        \vspace{5px}
        \rowcolors{2}{logo!10}{logo!40}
        \renewcommand{\arraystretch}{1.8}
        \begin{tabular}{p{150px} p{110px} p{110px}}
            \rowcolor{logo!70} \textbf{Ruolo} & \textbf{Ore}  & \textbf{Costo}                   \\
            Responsabile                      & 4(+0)         & 120,00\EURdig(+0,00 \EURdig)     \\
            Amministratore                    & 4(+0)         & 80,00\EURdig(+0,00 \EURdig)      \\
            Analista                          & 0(+0)         & 0,00\EURdig(+0,00 \EURdig)       \\
            Progettista                       & 5(+0)         & 110,00\EURdig(+0,00 \EURdig)     \\
            Programmatore                     & 12(+0)        & 180,00\EURdig(+0,00 \EURdig)     \\
            Verificatore                      & 8(+0)         & 120,00\EURdig(+0,00 \EURdig)     \\
            \textbf{Totale preventivo}        & 33            & 610,00\EURdig                    \\
            \textbf{Totale consuntivo}        & 33            & 610,00\EURdig                    \\
            \textbf{Differenza}               & +0            & (+0,00 \EURdig)                  \\
        \end{tabular}
    \end{table}
\end{center}
\subsubsubsection{Conclusioni}
Durante questo incremento è stata realizzata una pagina per poter contattare il venditore e si è trovata una soluzione per fare in modo che le pagine relative al venditore non fossero accessibili ai clienti tramite url. In questo periodo è stato rispettato il numero di ore disponibili come risorse per implementare gli obiettivi, questo ha portato in pari le spese da sostenere rispetto a quanto preventivato.
\subsubsubsection{Preventivo a finire rispetto alla fase}
Tutti gli obiettivi dell’incremento sono stati raggiunti con successo, nel pieno rispetto del consumo di risorse preventivato. Il bilancio dell’incremento è quindi in pari.
\subsubsubsection{Preventivo a finire complessivo}
Date le considerazioni precedenti su costi e obiettivi, il preventivo complessivo resta invariato.
\pagebreak

\pagebreak
\subsubsection{Incremento 16}
\begin{center}
    \begin{table}[ht!]
        \centering
        \caption{Consuntivo dei costi per ruolo dell'incremento 16}
        \vspace{5px}
        \rowcolors{2}{logo!10}{logo!40}
        \renewcommand{\arraystretch}{1.8}
        \begin{tabular}{p{150px} p{110px} p{110px}}
            \rowcolor{logo!70} \textbf{Ruolo} & \textbf{Ore}  & \textbf{Costo}                   \\
            Responsabile                      & 6(+0)         & 180,00\EURdig(+0,00 \EURdig)     \\
            Amministratore                    & 5(+0)         & 100,00\EURdig(+0,00 \EURdig)     \\
            Analista                          & 0(+0)         & 0,00\EURdig(+0,00 \EURdig)       \\
            Progettista                       & 6(+0)         & 132,00\EURdig(+0,00 \EURdig)     \\
            Programmatore                     & 10(+0)        & 150,00\EURdig(+0,00 \EURdig)     \\
            Verificatore                      & 9(+0)         & 135,00\EURdig(+0,00 \EURdig)     \\
            \textbf{Totale preventivo}        & 36            & 697,00\EURdig                    \\
            \textbf{Totale consuntivo}        & 36            & 697,00\EURdig                    \\
            \textbf{Differenza}               & +0            & (+0,00 \EURdig)                  \\
        \end{tabular}
    \end{table}
\end{center}
\subsubsubsection{Conclusioni}
Durante questo incremento è stato implementato un filtro per categoria, in modo da visualizzare solamente gli elementi appartenenti alla categoria filtrata. Sono state inoltre sistemate le carenze presenti nel \textit{Maintainer Manual} e nello \textit{User Manual}. In questo incremento è stato rispettato il numero di ore disponibili come risorse per implementare gli obiettivi, questo ha portato in pari le spese da sostenere rispetto a quanto preventivato.
\subsubsubsection{Preventivo a finire rispetto alla fase}
Tutti gli obiettivi dell’incremento sono stati raggiunti con successo, nel pieno rispetto del consumo di risorse preventivato. Il bilancio dell’incremento è quindi in pari.
\subsubsubsection{Preventivo a finire complessivo}
Date le considerazioni precedenti su costi e obiettivi, il preventivo complessivo resta invariato.
\pagebreak

\pagebreak
\subsubsection{Incremento 17}
\begin{center}
    \begin{table}[ht!]
        \centering
        \caption{Consuntivo dei costi per ruolo dell'incremento 17}
        \vspace{5px}
        \rowcolors{2}{logo!10}{logo!40}
        \renewcommand{\arraystretch}{1.8}
        \begin{tabular}{p{150px} p{110px} p{110px}}
            \rowcolor{logo!70} \textbf{Ruolo} & \textbf{Ore}  & \textbf{Costo}                   \\
            Responsabile                      & 6(+0)         & 180,00\EURdig(+0,00 \EURdig)     \\
            Amministratore                    & 6(+0)         & 120,00\EURdig(+0,00 \EURdig)     \\
            Analista                          & 0(+0)         & 0,00\EURdig(+0,00 \EURdig)       \\
            Progettista                       & 5(+0)         & 110,00\EURdig(+0,00 \EURdig)     \\
            Programmatore                     & 5(+0)         & 75,00\EURdig(+0,00 \EURdig)      \\
            Verificatore                      & 16(+0)        & 240,00\EURdig(+0,00 \EURdig)     \\
            \textbf{Totale preventivo}        & 38            & 725,00\EURdig                    \\
            \textbf{Totale consuntivo}        & 38            & 725,00\EURdig                    \\
            \textbf{Differenza}               & +0            & (+0,00 \EURdig)                  \\
        \end{tabular}
    \end{table}
\end{center}
\subsubsubsection{Conclusioni}
Durante questo incremento è stato preparato il materiale per la presentazione della RA, sono stati decisi gli argomenti da preparare ed è stato ricontrollato il codice. In questo incremento è stato rispettato il numero di ore disponibili come risorse, questo ha portato in pari le spese da sostenere rispetto a quanto preventivato.
\subsubsubsection{Preventivo a finire rispetto alla fase}
Tutti gli obiettivi dell’incremento sono stati raggiunti con successo, nel pieno rispetto del consumo di risorse preventivato. Il bilancio dell’incremento è quindi in pari.
\subsubsubsection{Preventivo a finire complessivo}
Date le considerazioni precedenti su costi e obiettivi, il preventivo complessivo resta invariato.
\pagebreak

\subsubsection{Consuntivo complessivo del periodo di Validazione e Collaudo}
\begin{center}
    \begin{table}[ht!]
        \centering
        \caption{Consuntivo dei costi per ruolo della fase di Validazione e Collaudo}
        \vspace{5px}
        \rowcolors{2}{logo!10}{logo!40}
        \renewcommand{\arraystretch}{1.8}
        \begin{tabular}{p{150px} p{110px} p{110px}}
            \rowcolor{logo!70} \textbf{Ruolo} & \textbf{Ore} & \textbf{Costo}                  \\
            Responsabile                      & 20(+0)       & 600,00\EURdig(+0,00 \EURdig)    \\
            Amministratore                    & 20(+1)       & 400,00\EURdig(+20,00 \EURdig)   \\
            Analista                          & 0(+0)        & 0,00\EURdig(+0,00 \EURdig)      \\
            Progettista                       & 20(+0)       & 440,00\EURdig(+0,00 \EURdig)    \\
            Programmatore                     & 38(-3)       & 570,00\EURdig(-45,00 \EURdig)   \\
            Verificatore                      & 40(+0)       & 600,00\EURdig(+0,00 \EURdig)    \\
            \textbf{Totale preventivo}        & 140          & 2635,00\EURdig                  \\
            \textbf{Totale consuntivo}        & 138          & 2610,00\EURdig                  \\
            \textbf{Differenza}               & -2           & (-25,00 \EURdig)                \\
        \end{tabular}
    \end{table}
\end{center}
\subsubsubsection{Conclusioni}
\begin{itemize}
    \item \textbf{\textit{Amministratore:}} Essendoci stata una maggiore necessità di sistemazione riguardante le \textit{Norme di Progetto}, si è presentato un esubero di ore per questo ruolo;
    \item \textbf{\textit{Programmatore:}} Avendo dovuto configurare un minor numero di funzionalità in accordo coi proponenti, si è presentata una minore necessità di ore per questo ruolo.
\end{itemize}
\subsubsubsection{Preventivo a finire complessivo}
Il bilancio economico risultante sottraendo i costi della fase ai soldi risparmiati in quelle precedenti, è in pari. Si ricorda che, trovandoci nell’ultima fase del progetto, il preventivo a finire rispetto alla fase e quello complessivo coincidono.

\pagebreak
\subsection{Consuntivo totale complessivo}
\subsubsection{Prospetto orario}
La seguente tabella mostra la suddivisione delle ore complessive suddivise per membro del gruppo. Le ore di differenza rispetto al preventivo sono state redistribuite in modo da mantenere al meglio possibile lo stesso numero di ore totali di lavoro per ogni membro del gruppo.
\begin{center}
    \begin{table}[ht!]
        \centering
        \caption{Distribuzione delle ore complessive per membro del gruppo}
        \vspace{5px}
        \rowcolors{2}{logo!10}{logo!40}
        \renewcommand{\arraystretch}{1.8}
        \begin{tabular}{p{100px} p{30px} p{30px} p{35px} p{40px} p{40px} p{35px} p{50px} }
            \rowcolor{logo!70} \textbf{Nominativo} & \textbf{Re} & \textbf{Am} & \textbf{An} & \textbf{Pt} & \textbf{Pr} & \textbf{Ve} & \textbf{Ore totali} \\
            Marco Barbaresco                       & 10(+1)      & 19(+1)          & 17(-1)      & 25(-4)      & 22(+5)      & 47(-1)      & 140                 \\
            Samuele De Simone                      & 16          & 10(+2)          & 26(-1)      & 14(-4)      & 31(+5)      & 43(-1)      & 140                 \\
            Nicolò Giaccone                        & 8           & 20(+1)          & 27          & 26(-3)      & 30(+4)      & 30          & 141                 \\
            Amedeo Meggiolaro                      & 11          & 7(+2)           & 18          & 30(-3)      & 27(+4)      & 47(-2)      & 140                 \\
            Tito Scutari                           & 16          & 25(+1)          & 17(-1)      & 22(-3)      & 35(+5)      & 26          & 141                 \\
            Simone Urbani                          & 14          & 23(+1)          & 14          & 22(-4)      & 31(+5)      & 37          & 141                 \\
            Manuel Varo                            & 18(+1)      & 18(+1)          & 20(-2)      & 9(-4)       & 24(+5)      & 51          & 140                 \\
            Ore totali di ruolo                    & 93(+2)      & 122(+9)         & 139(-5)     & 148(-25)    & 200(+33)    & 281(-4)     & 983(+10)            \\
        \end{tabular}
    \end{table}
\end{center}

\pagebreak
\subsubsection{Prospetto economico}
La seguente tabella riporta i costi derivati dalle ore impiegate per ogni ruolo durante il progetto, inclusa la fase di \textit{Analisi dei requisiti}.
\begin{center}
    \begin{table}[ht!]
        \centering
        \caption{Consuntivo totale complessivo dei costi per ruolo}
        \vspace{5px}
        \rowcolors{2}{logo!10}{logo!40}
        \renewcommand{\arraystretch}{1.8}
        \begin{tabular}{p{150px} p{110px} p{110px}}
            \rowcolor{logo!70} \textbf{Ruolo} & \textbf{Ore}  & \textbf{Costo}                    \\
            Responsabile                      & 93(+2)        & 2790,00\EURdig(+60,00 \EURdig)    \\
            Amministratore                    & 122(+9)       & 2440,00\EURdig(+180,00 \EURdig)   \\
            Analista                          & 139(-5)       & 3475,00\EURdig(-125,00 \EURdig)   \\
            Progettista                       & 148(-25)      & 3256,00\EURdig(-550,00 \EURdig)   \\
            Programmatore                     & 200(+33)      & 3000,00\EURdig(+495,00 \EURdig)   \\
            Verificatore                      & 281(-4)       & 4215,00\EURdig(-60,00 \EURdig)    \\
            \textbf{Totale preventivo}        & 973           & 19176,00\EURdig                   \\
            \textbf{Totale consuntivo}        & 983           & 19176,00\EURdig                   \\
            \textbf{Differenza}               & +0            & (+0,00 \EURdig)                   \\
        \end{tabular}
    \end{table}
\end{center}

\pagebreak
\subsection{Consuntivo totale rendicontato}
\subsubsection{Prospetto orario}
La seguente tabella mostra la suddivisione delle ore totali rendicontate suddivise per membro del gruppo. Le ore di differenza rispetto al preventivo sono state redistribuite in modo da mantenere al meglio possibile lo stesso numero di ore totali di lavoro per ogni membro del gruppo.
\begin{center}
    \begin{table}[ht!]
        \centering
        \caption{Distribuzione delle ore complessive per membro del gruppo}
        \vspace{5px}
        \rowcolors{2}{logo!10}{logo!40}
        \renewcommand{\arraystretch}{1.8}
        \begin{tabular}{p{100px} p{30px} p{30px} p{35px} p{40px} p{40px} p{35px} p{50px} }
            \rowcolor{logo!70} \textbf{Nominativo} & \textbf{Re} & \textbf{Am} & \textbf{An} & \textbf{Pt} & \textbf{Pr} & \textbf{Ve} & \textbf{Ore totali} \\
            Marco Barbaresco                       & 10(+1)      & 9(+1)           & 7(-1)      & 25(-4)      & 22(+5)      & 32(-1)      & 105                 \\
            Samuele De Simone                      & 8           & 10(+2)          & 9(-1)      & 14(-4)      & 31(+5)      & 33(-1)      & 105                 \\
            Nicolò Giaccone                        & 8           & 13(+1)          & 6          & 26(-3)      & 30(+4)      & 23          & 106                 \\
            Amedeo Meggiolaro                      & 11          & 7(+2)           & 0          & 30(-3)      & 27(+4)      & 30(-2)      & 105                 \\
            Tito Scutari                           & 16          & 17(+1)          & 5(-1)      & 22(-3)      & 35(+5)      & 11          & 106                 \\
            Simone Urbani                          & 8           & 13(+1)          & 6          & 22(-4)      & 31(+5)      & 26          & 106                 \\
            Manuel Varo                            & 11(+1)      & 18(+1)          & 8(-2)      & 9(-4)       & 24(+5)      & 35          & 105                 \\
            Ore totali di ruolo                    & 72(+2)      & 87(+9)          & 41(-5)     & 148(-25)    & 200(+33)    & 190(-4)     & 738(+10)            \\
        \end{tabular}
    \end{table}
\end{center}

\pagebreak
\subsubsection{Prospetto economico}
La seguente tabella riporta i costi derivati dalle ore rendicontate impiegate per ogni ruolo, partendo dalla fase di \textit{Progettazione architetturale} inclusa.
\begin{center}
    \begin{table}[ht!]
        \centering
        \caption{Consuntivo totale rendicontato dei costi per ruolo}
        \vspace{5px}
        \rowcolors{2}{logo!10}{logo!40}
        \renewcommand{\arraystretch}{1.8}
        \begin{tabular}{p{150px} p{110px} p{110px}}
            \rowcolor{logo!70} \textbf{Ruolo} & \textbf{Ore}  & \textbf{Costo}                   \\
            Responsabile                      & 72(+2)        & 2160,00\EURdig(+60,00 \EURdig)   \\
            Amministratore                    & 87(+9)        & 1740,00\EURdig(+180,00 \EURdig)  \\
            Analista                          & 41(-5)        & 1025,00\EURdig(-125,00 \EURdig)  \\
            Progettista                       & 148(-25)      & 3256,00\EURdig(-550,00 \EURdig)  \\
            Programmatore                     & 200(+33)      & 3000,00\EURdig(+495,00 \EURdig)  \\
            Verificatore                      & 190(-4)       & 2850,00\EURdig(-60,00 \EURdig)   \\
            \textbf{Totale preventivo}        & 728           & 14031,00\EURdig                  \\
            \textbf{Totale consuntivo}        & 738           & 14031,00\EURdig                  \\
            \textbf{Differenza}               & +10           & (+0,00 \EURdig)                  \\
        \end{tabular}
    \end{table}
\end{center}
