\section{}
\subsection* {AI} Acronimo di intelligenza artificiale. L'intelligenza artificiale è una disciplina appartenente all'informatica che studia i fondamenti teorici, le metodologie e le tecniche che consentono la progettazione di sistemi hardware e sistemi di programmi software capaci di fornire all'elaboratore elettronico prestazioni che, a un osservatore comune, sembrerebbero essere di pertinenza esclusiva dell’intelligenza umana.

\subsection*{Amazon CloudWatch} Amazon CloudWatch è un servizio, fornito da AmazonWebServices, di monitoraggio e gestione che fornisce dati e informazioni concrete per risorse di infrastruttura e applicazioni locali, su AWS e ibride. Con CloudWatch è possibile raccogliere e visualizzare tutti i dati relativi a prestazioni e operatività sotto forma di log e parametri, in un'unica piattaforma.

\subsection*{Amministratore di progetto} È incaricato di gestire, controllare e curare gli strumenti che il gruppo utilizzaper svolgere il proprio lavoro. È la figura che garantisce l’affidabilità e l’efficacia dei mezzi scelti dal gruppo.I suoi compiti sono i seguenti:
\begin{itemize}
    \item gestire il versionamento e la configurazione dei prodotti;
    \item gestire e salvaguardare la documentazione, controllando che sia corretta, verificata ed approvata e sem-plificando il suo reperimento;
    \item Correggere eventuali problemi legati alla gestione dei processi;
    \item Amministrare le infrastrutture e i servizi necessari ai processi di supporto;
    \item individuare strumenti utili all’automazione di processi;
    \item redigere e manutenere le norme e procedure che regolano il lavoro.
\end{itemize}

\subsection*{Analista} L’Analista partecipa al progetto al momento della stesura dell’Analisi dei Requisiti, il suo compito è quello di evidenziare i punti chiave del problema in questione, comprendendone appieno tutte le sue peculiarità. La sua figura è fondamentale per la buona riuscita del lavoro, in quanto errori o mancanze nell’individuazione dei requisiti da soddisfare possono compromettere fortemente l’attività di progettazione. I suoi compiti sono i seguenti:
\begin{itemize}
    \item studiare e definire il problema;
    \item analizzare le richieste e definire quali sono i requisiti in base ai bisogni, impliciti o espliciti;
    \item analizzare il fronte applicativo, gli utenti e i casi d’uso;
    \item redigere lo \textit{Studio di Fattibilità} e l’\textit{Analisi dei Requisiti}.
\end{itemize}

\subsection*{Angular} È un framework open source, scritto in Typescript, per la realizzazione di applicazioini web con licenza MIT.

\subsection*{API} Un'interfaccia di programmazione delle applicazioni (API) è un set di definizioni e protocolli per la compilazione e l'integrazione di software applicativi. Spesso chiamata libreria software.

\subsection*{API Rest} È un \textit{API} che rispetta i vincoli imposti dall'architettura Rest acronimo di \textit{REpresentational State Transfer}.

\subsection*{Approccio sistematico} Approccio in cui si vanno ad eseguire i vari compiti in modo rigoroso e seguendo uno standard.

\subsection*{Awayt/async} È una metodologia di programmazione asincrona che permette di mantenere una struttura del codice sincrona e si basa sul meccanismo delle \textit{Promise}.

\subsection*{AWS} Acronimo di AmazonWebServices, una piattaforma che offre servizi di elaborazione, storage di dati, distribuzione di contenuti ed altre funzionalità per la creazione di applicazioni.

\subsection*{AWS Lambda} È un servizio di elaborazione serverless, fornito da AWS, che permette di eseguire codice senza gestire i server. Permette inoltre di creare una logica di dimensionamento dei cluster in funzione dei carichi di lavoro. AWS Lambda permette quindi di eseguire il codice per qualsiasi applicazione o servizi di back-end, senza alcuna amministrazione.

\subsection*{AWS Merchant} È una tipologia di account fornita da AmazonWebServices, dedicata ai venditori.