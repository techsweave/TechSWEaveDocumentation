\section{}
% \section*{SEO (Search Engine Optimization)} Insieme di strategie e pratiche volte ad aumentare la visibilità di un sito internet migliorandone la posizione nelle classifiche dei motori di ricerca.
% \subsection*{Serverless} È un framework Web gratuito e open source scritto utilizzando Node.js. È il primo framework sviluppato
% per la creazione di applicazioni su AWS Lambda.

% \subsection*{Server-side Rendering} È una teconologia di che va a precaricare la pagina HTML nel server ad ogni richiesta.

% \subsection*{SQL} Linguaggio specifico del dominio utilizzato nella programmazione e progettato per la gestione dei dati
% contenuti in un sistema di gestione di database relazionali.

% \subsection*{Staging} È un'area intermedia tra la directory di lavoro e la directory nel VCS.

% \subsection*{Stakeholder} Persona interessata alla riuscita del progetto.

% \subsection*{Stripe} Fornisce un'infrastruttura software che permette a privati e aziende di inviare e ricevere pagamenti via internet.

% \subsection*{Stub} È una porzione di codice utilizzata in sostituzione di altre funzionalità software in quanto può simulare il comportamento di codice esistente
\subsection*{Serverless}
Serverless computing is a cloud computing execution model in which the cloud provider allocates machine resources on demand,
taking care of the servers on behalf of their customers. Serverless computing does not hold resources in volatile memory;
computing is rather done in short bursts with the results persisted to storage. When an app is not in use,
there are no computing resources allocated to the app. Pricing is based on the actual amount of resources consumed
by an application.
\subsection*{Server-side rendering}
Server-side rendering (SSR) is the process of rendering web pages on a server and passing them to the browser (client-side), instead of rendering them in the browser.
\subsection*{Snippet}
A snippet is a small section of text or source code that can be inserted into the code of a program or Web page. Snippets provide an easy way to implement commonly used code or functions into a larger section of code.
\subsection*{Software architecture}
Software architecture refers to the fundamental structures of a software system and the discipline of creating such structures and systems. Each structure comprises software elements, relations among them, and properties of both elements and relations.
\subsection*{Static generation}
It's the process of generating the HTML at build time so it will be reused on each request.